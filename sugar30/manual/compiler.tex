
% Compiler side of things

(Taken from implementor.pdf -- out of date)

In order to easily support a more powerful netlist language, the parser in
SUGAR 2.0 was re-written using the UNIX tools flex and bison. flex and
bison are GNU versions of the classic lex and yacc compiler construction tools.
For those who wish to understand or modify the parser code, a good source of
information on the tools used can be found in the book lex \& yacc by John
Levine, Tony Mason, and Doug Brown, published by O'Reilly and Associates.
Online resources on flex and bison include the GNU man pages and info pages.
   In the current version of SUGAR, the cho load command to load netlists
basically proceeds in two phases. In the first phase, the yacc translator, which
is compiled into an external Matlab routine (MEX file) converts the user netlist
into a Matlab function (called nettemp.m by default). In the second phase, the
Matlab function is executed to partially build the final data structure. After
that, additional Matlab routines process the data structure to assign global
variable indices, compute the undisplaced positions of the mechanical nodes,
and sanity check the model function parameters.

\section{Translator structure}


\subsection{Scanning}

The scanner, sugar.lex, is responsible for tokenizing the input file and for
managing the inclusion of files via the uses statement. The token description
is straightforward, and the curious reader is referred to the source code for
further details. The handling of uses statements is slightly more complicated.

A SUGAR uses statement is a combination of the uses statement in a lan- guage
like Delphi (Object Pascal) and the C \#include statement. The text of a file
included via uses is processed only at the first place it is encountered. So,
for example, if subnets.net uses mumps.net, and foo.net uses both subnets.net
and mumps.net, the text of mumps.net will only be used once. Consequently, the
scanner file keeps two structures to keep track of uses statements: a stack
which keeps track of nested uses, and a list of files which have been included
already.

Before attempting to open a file for inclusion, the scanner calls the function
which file, defined in the general library file sugar lib.c. When called from
within a MEX file, which file scans the Matlab path for files of the given
name.

\subsection{Parsing and intermediate representation}

The parser file sugar.y is little more than a copy of the formal grammar for
the SUGAR netlist language. The parser actions call routines in parse.c in
order to build an intermediate representation of the netlist parse tree. The
intermediate representation is described in parse.h.

The error checking done in the current version of the parser is very rudi-
mentary. Besides the automatic detection of parse errors, the parsing routines
check only for undefined variables and invalid process layers.

\subsection{Matlab code generation}

The final output of the SUGAR netlist translator is a Matlab script. The
task of the code generation routines in codegen.c is to recursively traverse
the intermediate representation of the netlist structure built by sugar.y and
parse.c and output Matlab code that will create appropriate corresponding
data structures.

Perhaps the most confusing aspect of the current code generation code is its
treatment of subnets. When the code generator starts on a subnet instance, it
stores relevant state, such as the name and coordinate system associated with
the subnet, or the assignment of local node names to global node indices, into
several different places. Some such information is kept with the parse tree data
structure; other information is kept in local variables in the run-time stack of
the generator code. Work is underway on a version of the code generator which
keeps a separate stack for subnet state, similar to the runtime stack kept by
most modern languages.

\section{Matlab post-translation processing}

\subsection{Argument sanity checks}

The first step after the generated Matlab script generates a partial version of
the netlist data structure is to sanity check the input arguments. This is done
by the check netlist routine, which calls the 'check' clause of each model
function in turn. The model functions are responsible for returning a diagnostic
message if they lack some piece of information needed for later analysis, or if
the arguments they receive are inconsistent or out of range.

\subsection{Index assignment}

Global variable index assignment proceeds in two steps. First, grounded
variables are identified and marked. Then, nodal and branch variables which
were not identified as grounded in the first phase are assigned indices. Index
assignment is done in the file parse enrich2.m.

In the current implementation, index assignment requires some string
comparisons, which slows it down substantially. Index assignment and node
positioning take substantially more time than other parts of the netlist
loading process. This is likely to change in future versions, as more of the
phases that are currently done by post-processing move into the code generator.

\subsection{Node positioning}

Before determining where nodes should be located, the node positioning code
determines which nodes should be located. This is done by scanning through the
list of elements and determining which nodes are assigned a relative position
by the 'pos' clause of some element. It is possible for a single model function
to contain some mechanical nodes (which have positions) and some nodes which do
not have an associated position. In this case, the number of columns returned
by the 'relpos' clause of the model will be smaller than the number of nodes.

The node positioning routine then does a breadth-first traversal of the
position graph. The first node visited is arbitrarily assigned to be positioned
at the origin, and subsequently visited nodes are assigned locations based on
the relative position information contained in the model functions and on the
already-determined locations of their neighboring nodes. If the mechanical
nodes are not all in a single connected component, the positioning routine will
issue a warning message and position at the origin the first node it encounters
in each component.

Ideally, an additional pass after node positions were determined would check
that the locations were consistent with the relative position information for
every element. Such a geometry check is not yet implemented.

Node positioning is implemented in find pos.m.


