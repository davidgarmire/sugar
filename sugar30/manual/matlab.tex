% Matlab side of things

\section{The netlist data structure}

The parsed representation of the netlist returned by 
\texttt{cho\_load} is one of SUGAR's central data structures. 
The netlist structure contains the following fields:

\begin{itemize}

\item{elements(i)} Information structure for the ith element.
\begin{itemize}
     \item{name} name of element
     \item{model} model function name
     \item{node\_ids} indices of nodes involving this element
     \item{parameter} structure mapping model parameters to value
     \item{var\_ids} list of indices of variables for this element. 
                     Grounded variables are assigned index 0.
\end{itemize}

\item{nodes(j)} Information for the jth node.
\begin{itemize}
     \item{name} name of node. An element with branch variables has a dummy
           node with the same name as the element, and any branch variables
           are assigned to that node.
     \item{elt\_ids} indices of elements involving this node
     \item{vars} structure mapping variable names to indices
     \item{pos} coordinates of the undisplaced node (mechanical nodes only)
\end{itemize}

\item{dof} number of (ungrounded) degrees of freedom

\item{scales(k)} characteristic size of the kth ungrounded variable

\end{itemize}

   The final netlist data structure does not have the same heirarchy as the
original design, which might contain subnets and array constructs. The only
way that the structure of the original netlist is portrayed in the final data
structure is in the node and element name fields. Element names have the form
\begin{verbatim}
            subnet element.subnet element. ... .element.
\end{verbatim}
For example, an element named bar in a subnet instance named foo would be
called foo.bar. Similarly, the structure of a node name is
\begin{verbatim}
              subnet element.subnet element. ... .node.
\end{verbatim}
A single node may have several aliases if it is used as an argument to a subnet.
For instance, in the netlist fragment
\begin{verbatim}
  subnet silly_anchor [x] [l=* w=*]
  [ anchor parent [x] [l=l w=w]
  ]

  silly_anchor an_anchor p1 [A] [l=5u w=5u]
\end{verbatim}
the node A could also be called an anchor.x. In such cases, the name assigned
to the node is the name at the highest scoping level; in this example, A. Elements
which are not explicitly named are assigned unique names of the form anon plus
a number. Anonymous elements use the same scoping rules as normal elements,
so that an anonymous element inside an anonymous subnet instance has a name
like anon5.anon10.

\section{Assembly and Display}

The assembly and display routines have very simple (and very similar) structure.
Each routine loops through all the elements in turn, calls the model function
to get a local contribution, and then merges the local contribution into a global
structure. The body of assemble system is a typical example. Here net is the
netlist data structure and mflag is a flag describing whether the mass ('M'),
damping ('D'), or stiffness ('K') matrix should be assembled:

\begin{verbatim}
for i = 1:length(net.elements)
  elt = net.elements(i);

  % Get the local contribution
  [Mlocal] = feval( elt.model, mflag, elt.R, elt.parameter );

  % If this element has anything to contribute, incorporate it
  if (~isempty(Mlocal))

       j = find(elt.var_ids ~= 0);            % Find ungrounded variables
       jdx = elt.var_ids(j);                  % Get associated global indices

       M(jdx,jdx) = M(jdx,jdx) + Mlocal(j,j); % Add local contribution

  end
end
\end{verbatim}

Note the use of two different indexing systems. The element "stamps" returned
by the model functions are ordered according to a local variable ordering cor-
responding to the order in which variable names appear in the output of the
'vars' case. Entry i in the var ids field then gives the global index for the ith
local variable. However, some local variables may be grounded, and therefore
will not have a corresponding global index. The matrix rows and columns cor-
responding to grounded variables, represented in var ids by zero entries, are
not added into the global matrix.
   The matrix assembly routines take a flag is sp to indicate whether the
matrices should be assembled using Matlab's sparse data structures. If the
is sp flag is ommitted, sparse output is assumed by default.
   There are only a few assembly functions:
\begin{itemize}
\item{assemble\_system} assembles the linear mass, damping, and stiffness matrices

\item{assemble\_F} assembles the forcing term

\item{assemble\_dFdx} assembles the Jacobian with respect to the position variables

\item{assemble\_dFdxdot} assembles the Jacobian with respect to velocity variables

\item{cho\_display} assembles local display output into a complete device picture

\end{itemize}

   Additionally, the structure of the netlist checking function check netlist
is similar to the structure of the assembly routines.


