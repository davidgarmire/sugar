
\chapter{Code architecture}

\section{Overview}

% SUGAR 3 vs old versions
%  - core implementation in C
%  - netlist generation language is Lua
% Advantages of new implementation

This document describes the newest version of SUGAR.  SUGAR 3.0
differs substantially from earlier versions: the core functionality
is implemented in C rather than in Matlab; netlists are described
in the Lua extension language instead of in the old SUGAR netlist
language; and the documentation is evolving alongside the
code itself.


\section{Coding techniques}

\subsection{Literate programming}

% Idea of literate programming
% Brief history and references
% The noweb system

Donald Knuth coined the term \emph{literate programming} for his
document-centered programming style.  Though the first literate
programming system, WEB, was built around the Pascal language,
the concepts apply to other languages, too.
A literate program (or ``web'')
consists of descriptive text intermixed with ``chunks'' of code.
Code chunks are presented in an order suitable for human readers,
rather than necessarily in the order in which they must appear for
use by the machine.  For example, code to check for errors or handle
exceptional cases can be placed after the code and text describing
behavior in the common case.

Knuth's musings on literate programming (and related topics) are
collected in a volume titled ``Literate Programming,'' published by
Stanford Press~\cite{Knut92}.  Perhaps the most famous literate program, TeX,
is published separately.  The books 
\emph{C Interfaces and Implementations}~\cite{Hans96} by Hanson and
\emph{Building a Compiler in C}~\cite{FrasHans95} by Fraser and Hanson 
are also literate programs.

The literate programming system used in this implementation is
Norman Ramsey's \verb|noweb| package \cite{Noweb-url, Rams94}.


\subsection{Classes and interfaces in C}

% Opaque objects and data hiding in C
% Implementing interfaces / virtual methods

Even C code can be object-oriented, though the style requires more work
than it would in languages like C++ and Java.  After all, the original
C++ compiler, CFRONT, translated C++ code to C.

We maintain privacy of object data by using opaque pointers.  In C,
the declaration
\begin{verbatim}
    typedef struct foo_struct* foo_t;
\end{verbatim}
is valid even if \verb|foo_struct| is not defined.  Consequently,
we can pass pointers to object data without defining the object
structure in the header file.  We sometimes fall back to a more
basic technique of using void pointers, but the compiler does
not generally catch as many errors involving void pointers as it
does errors involving typed pointers.

Generic interfaces are implemented using tables of function pointers.
An object supporting a particular interface is usually represented
by a structure containing a pointer to a function table of object
methods and a pointer to some object data.  By convention, the first
argument to an object method is the data for that object (corresponding
to the implicit \verb|this| pointer in C++).  We use generic interfaces
in a variety of places in the SUGAR architecture: materials, elements,
drawing contexts, and output contexts are all object types with a single
generic interface to multiple implementations.


\section{Core modules}

% Meshes
% Nodes, elements, and materials
% Assembly objects

% Mesh = nodes + elements + materials + some auxiliaries
% Builders
%  - Provide some mechanism for adding local contributions
%  - Assembly loop runs through all the elements and calls
%    one of their cases.  Those cases call back to make
%    local contribution.

The central data structure in SUGAR is the mesh object.
A mesh object builds and manages lists of nodes, elements,
and materials that form the device.  The mesh object
interacts with various assembly objects, which put together
some global construct from local contributions.  An assembly 
object provides a set of methods by which the elements 
(or materials) can make their local contributions.  The main
assembly loop then calls a method for each element; the elements
in turn call the methods of the assembly object to make their
contributions.

The assembly objects include:
\begin{itemize}

  \item The \emph{variable manager} object, which is responsible
        for keeping track of the types of nodal variables,
        and for assigning global variable indices.

  \item The \emph{assembler} object, which is responsible for
        building the residual vector and tangent matrix.
        The assembler object also keeps track of which variables
        are subject to displacement boundary conditions.

  \item The \emph{mesh output} objects, which are responsible
        for generating an external representation of the mesh
        components.  This external representation may be text,
        or a Matlab data structure.

  \item The \emph{mesh drawing} objects, which are responsible
        for generating a drawing of the device.  Depending on
        the mesh drawing object, the device might be drawn
        to the screen, or a representation might be written to
        file for later viewing.

\end{itemize}


\section{Lua and Matlab}

% Extension and interface language
% Lua -- brief history and application
% MEX interface
% Handles and tags

\emph{Lua}~\cite{Lua01} is an extension language designed at TecGRAF 
in the early 90s.
Like LISP, Lua is centered around a single basic data structure.  In LISP,
the data structure is a list; in Lua, it is an associative array, or table.
The language has
\begin{itemize}
  \item simple syntax -- the Pascal-like grammar fits on a single page,
  \item flexibility -- the set of types and functions can be extended 
        almost arbitrarily, and a ``tag method'' system allows existing 
        syntax to be given new semantics,
  \item portability -- the implementation is in ANSI C, and 
  \item speed -- Lua uses a bytecode interpreter.
\end{itemize}
Lua has become popular as a scripting language for computer games, and
has also been used as an input language for several codes,
including a finite element code.  The SUGAR netlist input language is a
slight extension of Lua, and there is a Lua-based command-line interactive
interface to SUGAR (\verb|sugar-lua|).

Mathworks' Matlab environment is well-known in scientific, engineering,
and computational mathematics for its intuitive, interactive problem-solving
environment.  The Matlab EXternal interface system (MEX) allows Matlab
to interact with codes written in C and Fortran.  The SUGAR MEX file
interface (\verb|sugarmex|) allows Matlab code to access the core C objects.
Using the MEX interface, SUGAR can be extended with net Matlab models 
and analysis routines.

While Lua provides a built-in mechanism for referring to C objects,
Matlab does not.  To refer to SUGAR C objects from Matlab code,
we use a system of integer handles.  Like Lua user data objects,
SUGAR MEX handles have associated tags to identify the associated
data types.  New tags (in Lua or in the MEX handle system) can be
allocated at any time, and the semantics of a tag are known only
to the C code.  The MEX handle system is implemented in the \verb|mex-handle|
module.

New functionality can be added to a Lua registering C functions with
the interpreter object.  The \verb|sugarmex| interface implements a
similar system for Matlab, so that once a function ``foo'' has been
registered with the Matlab interface, a Matlab call to
\begin{verbatim}
  sugarmex('foo', 1, 2)
\end{verbatim}
will be converted to a call to the C function ``foo'' with two numeric
Matlab arguments.  Because of this registration system, the Matlab
interface can be extended with new commands with only minimal modifications
to the main \verb|sugarmex| file.
