
\section{Introduction}

The {\tt{}assemble{\char95}mex} module provides the Matlab interface to
the assembly routines.


\section{Interface}

\nwfilename{assemble-mex.nw}\nwbegincode{1}\sublabel{NWassF-assE-1}\nwmargintag{{\nwtagstyle{}\subpageref{NWassF-assE-1}}}\moddef{assemble-mex.h~{\nwtagstyle{}\subpageref{NWassF-assE-1}}}\endmoddef
#ifndef ASSEMBLE_MEX_H
#define ASSEMBLE_MEX_H

#include <mex.h>
#include "assemble.h"

void assemble_mex_register();
int  assemble_mex_tag();

assemble_matrix_t* mx_to_assemble(const mxArray* array);

#endif /* ASSEMBLE_MEX_H */
\nwnotused{assemble-mex.h}\nwendcode{}\nwbegindocs{2}\nwdocspar

The {\tt{}assemble{\char95}mex{\char95}register} function registers the commands
{\tt{}assemble{\char95}R}, {\tt{}assemble{\char95}bc}, {\tt{}add{\char95}displace}, {\tt{}add{\char95}R},
{\tt{}add{\char95}dR}, and {\tt{}x{\char95}local}.


\section{Implementation}

\nwenddocs{}\nwbegincode{3}\sublabel{NWassF-assE.2-1}\nwmargintag{{\nwtagstyle{}\subpageref{NWassF-assE.2-1}}}\moddef{assemble-mex.c~{\nwtagstyle{}\subpageref{NWassF-assE.2-1}}}\endmoddef
#include <stdio.h>
#include <stdlib.h>
#include <string.h>

#include "mex-handle.h"
#include "sugarmex.h"
#include "mesh-mex.h"
#include "assemble-mex.h"

\LA{}static data~{\nwtagstyle{}\subpageref{NWassF-staB-1}}\RA{}
\LA{}mex functions~{\nwtagstyle{}\subpageref{NWassF-mexD-1}}\RA{}
\LA{}functions~{\nwtagstyle{}\subpageref{NWassF-fun9-1}}\RA{}
\nwnotused{assemble-mex.c}\nwendcode{}\nwbegindocs{4}\nwdocspar


\subsection{Assembler data structures}

\nwenddocs{}\nwbegincode{5}\sublabel{NWassF-staB-1}\nwmargintag{{\nwtagstyle{}\subpageref{NWassF-staB-1}}}\moddef{static data~{\nwtagstyle{}\subpageref{NWassF-staB-1}}}\endmoddef
static int assemble_handle_tag;

\nwused{\\{NWassF-assE.2-1}}\nwendcode{}\nwbegindocs{6}\nwdocspar

\nwenddocs{}\nwbegincode{7}\sublabel{NWassF-fun9-1}\nwmargintag{{\nwtagstyle{}\subpageref{NWassF-fun9-1}}}\moddef{functions~{\nwtagstyle{}\subpageref{NWassF-fun9-1}}}\endmoddef
void assemble_mex_register()
\{
    assemble_handle_tag = mex_handle_new_tag();
    \LA{}register mex functions~{\nwtagstyle{}\subpageref{NWassF-regM-1}}\RA{}
\}

void assemble_mex_shutdown()
\{
\}

int assemble_mex_handle_tag()
\{
    return assemble_handle_tag;
\}

\nwalsodefined{\\{NWassF-fun9-2}}\nwused{\\{NWassF-assE.2-1}}\nwendcode{}\nwbegindocs{8}\nwdocspar


\subsection{Conversion functions}

\nwenddocs{}\nwbegincode{9}\sublabel{NWassF-fun9-2}\nwmargintag{{\nwtagstyle{}\subpageref{NWassF-fun9-2}}}\moddef{functions~{\nwtagstyle{}\subpageref{NWassF-fun9-1}}}\plusendmoddef
assemble_matrix_t* mx_to_assemble(const mxArray* array)
\{
    return mx_to_handle(array, assemble_handle_tag, 
                        "Invalid assembler handle");
\}

\nwendcode{}\nwbegindocs{10}\nwdocspar


\subsection{{\tt{}mesh{\char95}nactive} command}

The {\tt{}mesh{\char95}nactive(mesh)} function returns the number of active
degrees of freedom in the system.

\nwenddocs{}\nwbegincode{11}\sublabel{NWassF-choI-1}\nwmargintag{{\nwtagstyle{}\subpageref{NWassF-choI-1}}}\moddef{cho-mesh-nactive.m~{\nwtagstyle{}\subpageref{NWassF-choI-1}}}\endmoddef
function [n] = cho_mesh_nactive(mesh)
n = sugarmex('mesh_nactive', mesh);
\nwnotused{cho-mesh-nactive.m}\nwendcode{}\nwbegindocs{12}\nwdocspar

\nwenddocs{}\nwbegincode{13}\sublabel{NWassF-regM-1}\nwmargintag{{\nwtagstyle{}\subpageref{NWassF-regM-1}}}\moddef{register mex functions~{\nwtagstyle{}\subpageref{NWassF-regM-1}}}\endmoddef
mexmgr_add("mesh_nactive", mesh_nactive);
\nwalsodefined{\\{NWassF-regM-2}\\{NWassF-regM-3}\\{NWassF-regM-4}\\{NWassF-regM-5}\\{NWassF-regM-6}\\{NWassF-regM-7}\\{NWassF-regM-8}\\{NWassF-regM-9}}\nwused{\\{NWassF-fun9-1}}\nwendcode{}\nwbegindocs{14}\nwdocspar

\nwenddocs{}\nwbegincode{15}\sublabel{NWassF-mexD-1}\nwmargintag{{\nwtagstyle{}\subpageref{NWassF-mexD-1}}}\moddef{mex functions~{\nwtagstyle{}\subpageref{NWassF-mexD-1}}}\endmoddef
static void mesh_nactive(int nlhs, mxArray* plhs[],
                         int nrhs, const mxArray* prhs[])
\{
    mesh_t mesh;

    mex_check_num_args(nrhs, 1);
    mesh = mx_to_mesh(prhs[0]);

    plhs[0] = mx_from_double(assemble_get_active(mesh_assembler(mesh)));
\}

\nwalsodefined{\\{NWassF-mexD-2}\\{NWassF-mexD-3}\\{NWassF-mexD-4}\\{NWassF-mexD-5}\\{NWassF-mexD-6}\\{NWassF-mexD-7}\\{NWassF-mexD-8}\\{NWassF-mexD-9}}\nwused{\\{NWassF-assE.2-1}}\nwendcode{}\nwbegindocs{16}\nwdocspar


\subsection{{\tt{}assemble{\char95}R} and {\tt{}assemble{\char95}R{\char95}full} commands}

The {\tt{}[R]\ =\ assemble{\char95}R(mesh,\ x)} function computes the
residual vector at $x$.  If $x$ is omitted, then a zero 
vector is used.  The {\tt{}[R]\ =\ assemble{\char95}R{\char95}full(mesh,\ x)} function
computes the unreduced residual vector (including components
corresponding to reaction forces at nodes subject to
displacement boundary conditions).

\nwenddocs{}\nwbegincode{17}\sublabel{NWassF-choG-1}\nwmargintag{{\nwtagstyle{}\subpageref{NWassF-choG-1}}}\moddef{cho-assemble-R.m~{\nwtagstyle{}\subpageref{NWassF-choG-1}}}\endmoddef
function [R] = cho_assemble_R(mesh, x, v, a)

if nargin == 1
    [R] = sugarmex('assemble_R', mesh, 0);
else
    [R] = sugarmex('assemble_R', mesh, 0, x);
end
\nwnotused{cho-assemble-R.m}\nwendcode{}\nwbegindocs{18}\nwdocspar

\nwenddocs{}\nwbegincode{19}\sublabel{NWassF-choL-1}\nwmargintag{{\nwtagstyle{}\subpageref{NWassF-choL-1}}}\moddef{cho-assemble-R-full.m~{\nwtagstyle{}\subpageref{NWassF-choL-1}}}\endmoddef
function [R] = cho_assemble_R_full(mesh, x, v, a)

if nargin == 1
    [R] = sugarmex('assemble_R', mesh, 1);
else
    [R] = sugarmex('assemble_R', mesh, 1, x);
end
\nwnotused{cho-assemble-R-full.m}\nwendcode{}\nwbegindocs{20}\nwdocspar

\nwenddocs{}\nwbegincode{21}\sublabel{NWassF-regM-2}\nwmargintag{{\nwtagstyle{}\subpageref{NWassF-regM-2}}}\moddef{register mex functions~{\nwtagstyle{}\subpageref{NWassF-regM-1}}}\plusendmoddef
mexmgr_add("assemble_R", mex_assemble_R);
\nwendcode{}\nwbegindocs{22}\nwdocspar

\nwenddocs{}\nwbegincode{23}\sublabel{NWassF-mexD-2}\nwmargintag{{\nwtagstyle{}\subpageref{NWassF-mexD-2}}}\moddef{mex functions~{\nwtagstyle{}\subpageref{NWassF-mexD-1}}}\plusendmoddef
static void mex_assemble_R(int nlhs, mxArray* plhs[],
                           int nrhs, const mxArray* prhs[])
\{
    mesh_t      mesh;
    int         full_size;
    assembler_t assembler;
    double*     x = NULL;
    double*     v = NULL;
    double*     a = NULL;
    double*     R;
    int         Rsize;

    int nvars;
    int nactive;

    mex_check_num_args(nrhs, 2);
    mesh      = mx_to_mesh(prhs[0]);
    full_size = mx_to_double(prhs[1]);
    assembler = mesh_assembler(mesh);

    nvars   = vars_count(mesh_vars_mgr(mesh));
    nactive = assemble_get_active(assembler);

    Rsize   = (full_size ? nvars : nactive);

    x = mxMalloc(nvars * sizeof(double));
    memcpy(x, assemble_get_displacements(assembler), nvars * sizeof(double));
    if (nrhs > 2) \{
        if (!mxIsNumeric(prhs[2]))
            mexErrMsgTxt("Argument must be a numeric vector");
        if (mxGetM(prhs[2]) != nactive || mxGetN(prhs[2]) != 1)
            mexErrMsgTxt("x vector is wrong size");
        memcpy(x, mxGetPr(prhs[2]), nactive * sizeof(double));
    \}

    if (nrhs > 3) \{
        v = mxCalloc(nvars, sizeof(double));
        if (!mxIsNumeric(prhs[3]))
            mexErrMsgTxt("Argument must be a numeric vector");
        if (mxGetM(prhs[3]) != nactive || mxGetN(prhs[3]) != 1)
            mexErrMsgTxt("v vector is wrong size");
        memcpy(v, mxGetPr(prhs[3]), nactive * sizeof(double));
    \}

    if (nrhs > 4) \{
        a = mxCalloc(nvars, sizeof(double));
        if (!mxIsNumeric(prhs[4]))
            mexErrMsgTxt("Argument must be a numeric vector");
        if (mxGetM(prhs[4]) != nactive || mxGetN(prhs[4]) != 1)
            mexErrMsgTxt("a vector is wrong size");
        memcpy(a, mxGetPr(prhs[4]), nactive * sizeof(double));
    \}

    plhs[0] = mxCreateDoubleMatrix(Rsize, 1, mxREAL);
    R       = mxGetPr(plhs[0]);

    assemble_R(assembler, x, v, a, R, Rsize); 

    if (a) mxFree(a);
    if (v) mxFree(v);
    if (x) mxFree(x);
\}

\nwendcode{}\nwbegindocs{24}\nwdocspar


\subsection{{\tt{}assemble{\char95}dR} command}

The {\tt{}[dR]\ =\ assemble{\char95}dR(mesh,\ x)} function computes the
residual tangent matrix at $x$.  If $x$ is omitted,
then a zero vector is used.

\nwenddocs{}\nwbegincode{25}\sublabel{NWassF-choH-1}\nwmargintag{{\nwtagstyle{}\subpageref{NWassF-choH-1}}}\moddef{cho-assemble-dR.m~{\nwtagstyle{}\subpageref{NWassF-choH-1}}}\endmoddef
function [dR] = cho_assemble_dR(mesh, c, x, v, a)

if nargin == 1
    [dR] = sugarmex('assemble_dR', mesh, [1 0 0]);
elseif nargin == 2
    [dR] = sugarmex('assemble_dR', mesh, c);
elseif nargin == 3
    [dR] = sugarmex('assemble_dR', mesh, c, x);
elseif nargin == 4
    [dR] = sugarmex('assemble_dR', mesh, c, x, v);
elseif nargin == 5
    [dR] = sugarmex('assemble_dR', mesh, c, x, v, a);
end
\nwnotused{cho-assemble-dR.m}\nwendcode{}\nwbegindocs{26}\nwdocspar

\nwenddocs{}\nwbegincode{27}\sublabel{NWassF-regM-3}\nwmargintag{{\nwtagstyle{}\subpageref{NWassF-regM-3}}}\moddef{register mex functions~{\nwtagstyle{}\subpageref{NWassF-regM-1}}}\plusendmoddef
mexmgr_add("assemble_dR", mex_assemble_dR);
\nwendcode{}\nwbegindocs{28}\nwdocspar

\nwenddocs{}\nwbegincode{29}\sublabel{NWassF-mexD-3}\nwmargintag{{\nwtagstyle{}\subpageref{NWassF-mexD-3}}}\moddef{mex functions~{\nwtagstyle{}\subpageref{NWassF-mexD-1}}}\plusendmoddef
static void mex_assemble_dR(int nlhs, mxArray* plhs[],
                            int nrhs, const mxArray* prhs[])
\{
    mesh_t      mesh;
    assembler_t assembler;
    double      c[3] = \{1, 0, 0\};
    double*     x = NULL;
    double*     v = NULL;
    double*     a = NULL;
    double*     dR;

    int nvars;
    int nactive;

    mex_check_num_args(nrhs, 2);
    mesh      = mx_to_mesh(prhs[0]);
    assembler = mesh_assembler(mesh);

    nvars   = vars_count(mesh_vars_mgr(mesh));
    nactive = assemble_get_active(assembler);

    if (!mxIsNumeric(prhs[1])) \{
        mexErrMsgTxt("c argument must be a numeric vector");
    \} else if (mxGetM(prhs[1]) * mxGetN(prhs[1]) > 3) \{
        mexErrMsgTxt("c vector is wrong size");
    \} else \{
        memcpy(c, mxGetPr(prhs[1]), 
               mxGetM(prhs[1]) * mxGetN(prhs[1]) * sizeof(double));
    \}

    x = mxMalloc(nvars * sizeof(double));
    memcpy(x, assemble_get_displacements(assembler), nvars * sizeof(double));
    if (nrhs > 2) \{
        if (!mxIsNumeric(prhs[2]))
            mexErrMsgTxt("Argument must be a numeric vector");
        if (mxGetM(prhs[2]) != nactive || mxGetN(prhs[2]) != 1)
            mexErrMsgTxt("x vector is wrong size");
        memcpy(x, mxGetPr(prhs[2]), nactive * sizeof(double));
    \}

    if (nrhs > 3) \{
        v = mxCalloc(nvars, sizeof(double));
        if (!mxIsNumeric(prhs[3]))
            mexErrMsgTxt("Argument must be a numeric vector");
        if (mxGetM(prhs[3]) != nactive || mxGetN(prhs[3]) != 1)
            mexErrMsgTxt("v vector is wrong size");
        memcpy(v, mxGetPr(prhs[3]), nactive * sizeof(double));
    \}

    if (nrhs > 4) \{
        a = mxCalloc(nvars, sizeof(double));
        if (!mxIsNumeric(prhs[4]))
            mexErrMsgTxt("Argument must be a numeric vector");
        if (mxGetM(prhs[4]) != nactive || mxGetN(prhs[4]) != 1)
            mexErrMsgTxt("a vector is wrong size");
        memcpy(a, mxGetPr(prhs[4]), nactive * sizeof(double));
    \}

    plhs[0] = mxCreateDoubleMatrix(nactive, nactive, mxREAL);
    dR      = mxGetPr(plhs[0]);

    assemble_dR(assembler, c[0], x, c[1], v, c[2], a, dR, nactive);

    if (a) mxFree(a);
    if (v) mxFree(v);
    if (x) mxFree(x);
\}

\nwendcode{}\nwbegindocs{30}\nwdocspar


\subsection{{\tt{}assemble{\char95}dR{\char95}sparse} command}

The {\tt{}[dR]\ =\ assemble{\char95}dR{\char95}sparse(mesh,\ x)} function computes the
residual vector and tangent matrix at $x$.  If $x$ is omitted,
then a zero vector is used.  The tangent matrix is assembled 
in sparse form.

\nwenddocs{}\nwbegincode{31}\sublabel{NWassF-choO-1}\nwmargintag{{\nwtagstyle{}\subpageref{NWassF-choO-1}}}\moddef{cho-assemble-dR-sparse.m~{\nwtagstyle{}\subpageref{NWassF-choO-1}}}\endmoddef
function [dR] = cho_assemble_dR_sparse(mesh, c, x, v, a)

if nargin == 1
    [dR] = sugarmex('assemble_dR_sparse', mesh, [1 0 0]);
elseif nargin == 2
    [dR] = sugarmex('assemble_dR_sparse', mesh, c);
elseif nargin == 3
    [dR] = sugarmex('assemble_dR_sparse', mesh, c, x);
elseif nargin == 4
    [dR] = sugarmex('assemble_dR_sparse', mesh, c, x, v);
elseif nargin == 5
    [dR] = sugarmex('assemble_dR_sparse', mesh, c, x, v, a);
end
\nwnotused{cho-assemble-dR-sparse.m}\nwendcode{}\nwbegindocs{32}\nwdocspar

\nwenddocs{}\nwbegincode{33}\sublabel{NWassF-regM-4}\nwmargintag{{\nwtagstyle{}\subpageref{NWassF-regM-4}}}\moddef{register mex functions~{\nwtagstyle{}\subpageref{NWassF-regM-1}}}\plusendmoddef
mexmgr_add("assemble_dR_sparse", mex_assemble_dR_sparse);
\nwendcode{}\nwbegindocs{34}\nwdocspar

\nwenddocs{}\nwbegincode{35}\sublabel{NWassF-mexD-4}\nwmargintag{{\nwtagstyle{}\subpageref{NWassF-mexD-4}}}\moddef{mex functions~{\nwtagstyle{}\subpageref{NWassF-mexD-1}}}\plusendmoddef
static void mex_assemble_dR_sparse(int nlhs, mxArray* plhs[],
                                   int nrhs, const mxArray* prhs[])
\{
    mesh_t      mesh;
    assembler_t assembler;
    double      c[3] = \{1, 0, 0\};
    double*     x = NULL;
    double*     v = NULL;
    double*     a = NULL;
    int*        pi;

    assemble_matrix_t* assembler_x = NULL;
    assemble_matrix_t* assembler_v = NULL;
    assemble_matrix_t* assembler_a = NULL;
    assemble_matrix_t* assembler_dR;

    int i, n;
    int nvars;
    int nactive;

    mex_check_num_args(nrhs, 1);
    mesh      = mx_to_mesh(prhs[0]);
    assembler = mesh_assembler(mesh);

    nvars   = vars_count(mesh_vars_mgr(mesh));
    nactive = assemble_get_active(assembler);

    if (!mxIsNumeric(prhs[1])) \{
        mexErrMsgTxt("c argument must be a numeric vector");
    \} else if (mxGetM(prhs[1]) * mxGetN(prhs[1]) > 3) \{
        mexErrMsgTxt("c vector is wrong size");
    \} else \{
        memcpy(c, mxGetPr(prhs[1]), 
               mxGetM(prhs[1]) * mxGetN(prhs[1]) * sizeof(double));
    \}

    x = mxMalloc(nvars * sizeof(double));
    memcpy(x, assemble_get_displacements(assembler), nvars * sizeof(double));
    if (nrhs > 2) \{
        if (!mxIsNumeric(prhs[2]))
            mexErrMsgTxt("Argument must be a numeric vector");
        if (mxGetM(prhs[2]) != nactive || mxGetN(prhs[2]) != 1)
            mexErrMsgTxt("x vector is wrong size");
        memcpy(x, mxGetPr(prhs[2]), nactive * sizeof(double));
    \}

    if (nrhs > 3) \{
        v = mxCalloc(nvars, sizeof(double));
        if (!mxIsNumeric(prhs[3]))
            mexErrMsgTxt("Argument must be a numeric vector");
        if (mxGetM(prhs[3]) != nactive || mxGetN(prhs[3]) != 1)
            mexErrMsgTxt("v vector is wrong size");
        memcpy(v, mxGetPr(prhs[3]), nactive * sizeof(double));
    \}

    if (nrhs > 4) \{
        a = mxCalloc(nvars, sizeof(double));
        if (!mxIsNumeric(prhs[4]))
            mexErrMsgTxt("Argument must be a numeric vector");
        if (mxGetM(prhs[4]) != nactive || mxGetN(prhs[4]) != 1)
            mexErrMsgTxt("a vector is wrong size");
        memcpy(a, mxGetPr(prhs[4]), nactive * sizeof(double));
    \}

    if (x) assembler_x = assembler_dense_getvector(x, nvars, nvars);
    if (v) assembler_v = assembler_dense_getvector(v, nvars, nvars);
    if (a) assembler_a = assembler_dense_getvector(a, nvars, nvars);
    assembler_dR = assembler_sparse_matrix(nactive, nactive);

    n = mesh_num_elements(mesh);
    for (i = 1; i <= n; ++i) \{
        element_dR( mesh_element(mesh, i), assembler_dR, 
                    c[0], assembler_x,
                    c[1], assembler_v,
                    c[2], assembler_a);
    \}

    assembler_sparse_csc(assembler_dR);

    n = assembler_sparse_nnz(assembler_dR);
    plhs[0] = mxCreateSparse(nactive, nactive, n, mxREAL);
    memcpy(mxGetPr(plhs[0]), assembler_sparse_nzval(assembler_dR), 
           n * sizeof(double));
    memcpy(mxGetIr(plhs[0]), assembler_sparse_rowind(assembler_dR), 
           n * sizeof(int));
    memcpy(mxGetJc(plhs[0]), assembler_sparse_colptr(assembler_dR), 
           (nactive+1) * sizeof(int));

    assembler_matrix_free(assembler_dR);
    if (assembler_a) assembler_matrix_free(assembler_a);
    if (assembler_v) assembler_matrix_free(assembler_v);
    if (assembler_x) assembler_matrix_free(assembler_x);

    if (a) mxFree(a);
    if (v) mxFree(v);
    if (x) mxFree(x);
\}

\nwendcode{}\nwbegindocs{36}\nwdocspar


\subsection{{\tt{}assemble{\char95}bc} command}

The {\tt{}assemble{\char95}bc(mesh)} function applies displacement boundary
conditions from the mesh.

\nwenddocs{}\nwbegincode{37}\sublabel{NWassF-choH.2-1}\nwmargintag{{\nwtagstyle{}\subpageref{NWassF-choH.2-1}}}\moddef{cho-assemble-bc.m~{\nwtagstyle{}\subpageref{NWassF-choH.2-1}}}\endmoddef
function cho_assemble_bc(mesh)
sugarmex('assemble_bc', mesh);
\nwnotused{cho-assemble-bc.m}\nwendcode{}\nwbegindocs{38}\nwdocspar

\nwenddocs{}\nwbegincode{39}\sublabel{NWassF-regM-5}\nwmargintag{{\nwtagstyle{}\subpageref{NWassF-regM-5}}}\moddef{register mex functions~{\nwtagstyle{}\subpageref{NWassF-regM-1}}}\plusendmoddef
mexmgr_add("assemble_bc", mex_assemble_bc);
\nwendcode{}\nwbegindocs{40}\nwdocspar

\nwenddocs{}\nwbegincode{41}\sublabel{NWassF-mexD-5}\nwmargintag{{\nwtagstyle{}\subpageref{NWassF-mexD-5}}}\moddef{mex functions~{\nwtagstyle{}\subpageref{NWassF-mexD-1}}}\plusendmoddef
static void mex_assemble_bc(int nlhs, mxArray* plhs[],
                            int nrhs, const mxArray* prhs[])
\{
    mesh_t      mesh;
    assembler_t assembler;

    mex_check_num_args(nrhs, 1);
    mesh      = mx_to_mesh(prhs[0]);
    assembler = mesh_assembler(mesh);

    assemble_displace(assembler);
\}

\nwendcode{}\nwbegindocs{42}\nwdocspar


\subsection{{\tt{}add{\char95}displace} command}

The {\tt{}add{\char95}displace(assembler,\ var{\char95}id,\ value)} function applies displacement
boundary conditions to an indicated degree of freedom in the mesh.

\nwenddocs{}\nwbegincode{43}\sublabel{NWassF-choI.2-1}\nwmargintag{{\nwtagstyle{}\subpageref{NWassF-choI.2-1}}}\moddef{cho-add-displace.m~{\nwtagstyle{}\subpageref{NWassF-choI.2-1}}}\endmoddef
function cho_add_displace(assembler, var_id, value)

if nargin == 2
    value = 0;
end
sugarmex('add_displace', assembler, var_id, value);
\nwnotused{cho-add-displace.m}\nwendcode{}\nwbegindocs{44}\nwdocspar

\nwenddocs{}\nwbegincode{45}\sublabel{NWassF-regM-6}\nwmargintag{{\nwtagstyle{}\subpageref{NWassF-regM-6}}}\moddef{register mex functions~{\nwtagstyle{}\subpageref{NWassF-regM-1}}}\plusendmoddef
mexmgr_add("add_displace", mex_add_displace);
\nwendcode{}\nwbegindocs{46}\nwdocspar

\nwenddocs{}\nwbegincode{47}\sublabel{NWassF-mexD-6}\nwmargintag{{\nwtagstyle{}\subpageref{NWassF-mexD-6}}}\moddef{mex functions~{\nwtagstyle{}\subpageref{NWassF-mexD-1}}}\plusendmoddef
static void mex_add_displace(int nlhs, mxArray* plhs[],
                             int nrhs, const mxArray* prhs[])
\{
    assemble_matrix_t* assembler;

    int     var_id;
    double  value;

    mex_check_num_args(nrhs, 3);
    assembler = mx_to_assemble(prhs[0]);
    var_id    = (int) mx_to_double(prhs[1]);
    value     = mx_to_double(prhs[2]);

    assemble_matrix_add1(assembler, var_id, value);
\}

\nwendcode{}\nwbegindocs{48}\nwdocspar


\subsection{{\tt{}add{\char95}R} command}

The {\tt{}add{\char95}R(assembler,\ var{\char95}ids,\ Rlocal)} function assembles the local
residual contribution into the global vector.

\nwenddocs{}\nwbegincode{49}\sublabel{NWassF-choB-1}\nwmargintag{{\nwtagstyle{}\subpageref{NWassF-choB-1}}}\moddef{cho-add-R.m~{\nwtagstyle{}\subpageref{NWassF-choB-1}}}\endmoddef
function cho_add_R(assembler, var_ids, Rlocal)
sugarmex('add_R', assembler, var_ids, Rlocal);
\nwnotused{cho-add-R.m}\nwendcode{}\nwbegindocs{50}\nwdocspar

\nwenddocs{}\nwbegincode{51}\sublabel{NWassF-regM-7}\nwmargintag{{\nwtagstyle{}\subpageref{NWassF-regM-7}}}\moddef{register mex functions~{\nwtagstyle{}\subpageref{NWassF-regM-1}}}\plusendmoddef
mexmgr_add("add_R", mex_add_R);
\nwendcode{}\nwbegindocs{52}\nwdocspar

\nwenddocs{}\nwbegincode{53}\sublabel{NWassF-mexD-7}\nwmargintag{{\nwtagstyle{}\subpageref{NWassF-mexD-7}}}\moddef{mex functions~{\nwtagstyle{}\subpageref{NWassF-mexD-1}}}\plusendmoddef
static void mex_add_R(int nlhs, mxArray* plhs[],
                      int nrhs, const mxArray* prhs[])
\{
    assemble_matrix_t* assembler;

    int*    var_ids;
    double* value;
    int     num_vars;
    int     i;

    mex_check_num_args(nrhs, 3);
    assembler = mx_to_assemble(prhs[0]);

    if (!mxIsNumeric(prhs[1]) || !mxIsNumeric(prhs[2]))
        mexErrMsgTxt("Arguments must be numeric");

    num_vars  = mxGetM(prhs[1]) * mxGetN(prhs[1]);
    if (mxGetM(prhs[2]) != num_vars || mxGetN(prhs[2]) != 1)
        mexErrMsgTxt("Mismatch between number of arguments and Rlocal");

    value     = mxGetPr(prhs[2]);
    var_ids   = mxCalloc(num_vars, sizeof(int));
    for (i = 0; i < num_vars; ++i)
        var_ids[i] = (int) (mxGetPr(prhs[1])[i]);

    assemble_matrix_add(assembler, var_ids, num_vars, value);
    mxFree(var_ids);
\}

\nwendcode{}\nwbegindocs{54}\nwdocspar


\subsection{{\tt{}add{\char95}dR} command}

The {\tt{}add{\char95}dR(assembler,\ var{\char95}ids,\ dRlocal)} function assembles the local
residual tangent contribution into the global matrix.

\nwenddocs{}\nwbegincode{55}\sublabel{NWassF-choC-1}\nwmargintag{{\nwtagstyle{}\subpageref{NWassF-choC-1}}}\moddef{cho-add-dR.m~{\nwtagstyle{}\subpageref{NWassF-choC-1}}}\endmoddef
function cho_add_dR(assembler, var_ids, dRlocal)
sugarmex('add_dR', assembler, var_ids, dRlocal);
\nwnotused{cho-add-dR.m}\nwendcode{}\nwbegindocs{56}\nwdocspar

\nwenddocs{}\nwbegincode{57}\sublabel{NWassF-regM-8}\nwmargintag{{\nwtagstyle{}\subpageref{NWassF-regM-8}}}\moddef{register mex functions~{\nwtagstyle{}\subpageref{NWassF-regM-1}}}\plusendmoddef
mexmgr_add("add_dR", mex_add_dR);
\nwendcode{}\nwbegindocs{58}\nwdocspar

\nwenddocs{}\nwbegincode{59}\sublabel{NWassF-mexD-8}\nwmargintag{{\nwtagstyle{}\subpageref{NWassF-mexD-8}}}\moddef{mex functions~{\nwtagstyle{}\subpageref{NWassF-mexD-1}}}\plusendmoddef
static void mex_add_dR(int nlhs, mxArray* plhs[],
                       int nrhs, const mxArray* prhs[])
\{
    assemble_matrix_t* assembler;

    int*    var_ids;
    double* value;
    int     num_vars;
    int     i;

    mex_check_num_args(nrhs, 3);
    assembler = mx_to_assemble(prhs[0]);

    if (!mxIsNumeric(prhs[1]) || !mxIsNumeric(prhs[2]))
        mexErrMsgTxt("Arguments must be numeric");

    num_vars  = mxGetM(prhs[1]) * mxGetN(prhs[1]);
    if (mxGetM(prhs[2]) != num_vars || mxGetN(prhs[2]) != num_vars)
        mexErrMsgTxt("Mismatch between number of arguments and dRlocal");

    value     = mxGetPr(prhs[2]);
    var_ids   = mxCalloc(num_vars, sizeof(int));
    for (i = 0; i < num_vars; ++i)
        var_ids[i] = (int) (mxGetPr(prhs[1])[i]);

    assemble_matrix_add(assembler, var_ids, num_vars, value);
    mxFree(var_ids);
\}

\nwendcode{}\nwbegindocs{60}\nwdocspar


\subsection{{\tt{}x{\char95}local} command}

The {\tt{}x{\char95}local(assembler,\ var{\char95}ids)} function assembles the local
residual contribution into the global vector.

\nwenddocs{}\nwbegincode{61}\sublabel{NWassF-choD-1}\nwmargintag{{\nwtagstyle{}\subpageref{NWassF-choD-1}}}\moddef{cho-x-local.m~{\nwtagstyle{}\subpageref{NWassF-choD-1}}}\endmoddef
function [xlocal] = cho_x_local(assembler, var_ids)
xlocal = sugarmex('x_local', assembler, var_ids);
\nwnotused{cho-x-local.m}\nwendcode{}\nwbegindocs{62}\nwdocspar

\nwenddocs{}\nwbegincode{63}\sublabel{NWassF-regM-9}\nwmargintag{{\nwtagstyle{}\subpageref{NWassF-regM-9}}}\moddef{register mex functions~{\nwtagstyle{}\subpageref{NWassF-regM-1}}}\plusendmoddef
mexmgr_add("x_local", mex_x_local);
\nwendcode{}\nwbegindocs{64}\nwdocspar

\nwenddocs{}\nwbegincode{65}\sublabel{NWassF-mexD-9}\nwmargintag{{\nwtagstyle{}\subpageref{NWassF-mexD-9}}}\moddef{mex functions~{\nwtagstyle{}\subpageref{NWassF-mexD-1}}}\plusendmoddef
static void mex_x_local(int nlhs, mxArray* plhs[],
                        int nrhs, const mxArray* prhs[])
\{
    assemble_matrix_t* assembler;

    int*    var_ids;
    double* value;
    int     num_vars;
    int     i;

    mex_check_num_args(nrhs, 2);
    assembler = mx_to_assemble(prhs[0]);

    if (!mxIsNumeric(prhs[1]))
        mexErrMsgTxt("Arguments must be numeric");

    num_vars  = mxGetM(prhs[1]) * mxGetN(prhs[1]);
    plhs[0]   = mxCreateDoubleMatrix(mxGetM(prhs[1]) * mxGetN(prhs[1]), 1, 
                                     mxREAL);
    value     = mxGetPr(plhs[0]);

    var_ids   = mxCalloc(num_vars, sizeof(int));
    for (i = 0; i < num_vars; ++i)
        var_ids[i] = (int) (mxGetPr(prhs[1])[i]);

    assemble_matrix_add(assembler, var_ids, num_vars, value);
    mxFree(var_ids);
\}

\nwendcode{}

\nwixlogsorted{c}{{assemble-mex.c}{NWassF-assE.2-1}{\nwixd{NWassF-assE.2-1}}}%
\nwixlogsorted{c}{{assemble-mex.h}{NWassF-assE-1}{\nwixd{NWassF-assE-1}}}%
\nwixlogsorted{c}{{cho-add-R.m}{NWassF-choB-1}{\nwixd{NWassF-choB-1}}}%
\nwixlogsorted{c}{{cho-add-dR.m}{NWassF-choC-1}{\nwixd{NWassF-choC-1}}}%
\nwixlogsorted{c}{{cho-add-displace.m}{NWassF-choI.2-1}{\nwixd{NWassF-choI.2-1}}}%
\nwixlogsorted{c}{{cho-assemble-R-full.m}{NWassF-choL-1}{\nwixd{NWassF-choL-1}}}%
\nwixlogsorted{c}{{cho-assemble-R.m}{NWassF-choG-1}{\nwixd{NWassF-choG-1}}}%
\nwixlogsorted{c}{{cho-assemble-bc.m}{NWassF-choH.2-1}{\nwixd{NWassF-choH.2-1}}}%
\nwixlogsorted{c}{{cho-assemble-dR.m}{NWassF-choH-1}{\nwixd{NWassF-choH-1}}}%
\nwixlogsorted{c}{{cho-assemble-dR-sparse.m}{NWassF-choO-1}{\nwixd{NWassF-choO-1}}}%
\nwixlogsorted{c}{{cho-mesh-nactive.m}{NWassF-choI-1}{\nwixd{NWassF-choI-1}}}%
\nwixlogsorted{c}{{cho-x-local.m}{NWassF-choD-1}{\nwixd{NWassF-choD-1}}}%
\nwixlogsorted{c}{{functions}{NWassF-fun9-1}{\nwixu{NWassF-assE.2-1}\nwixd{NWassF-fun9-1}\nwixd{NWassF-fun9-2}}}%
\nwixlogsorted{c}{{mex functions}{NWassF-mexD-1}{\nwixu{NWassF-assE.2-1}\nwixd{NWassF-mexD-1}\nwixd{NWassF-mexD-2}\nwixd{NWassF-mexD-3}\nwixd{NWassF-mexD-4}\nwixd{NWassF-mexD-5}\nwixd{NWassF-mexD-6}\nwixd{NWassF-mexD-7}\nwixd{NWassF-mexD-8}\nwixd{NWassF-mexD-9}}}%
\nwixlogsorted{c}{{register mex functions}{NWassF-regM-1}{\nwixu{NWassF-fun9-1}\nwixd{NWassF-regM-1}\nwixd{NWassF-regM-2}\nwixd{NWassF-regM-3}\nwixd{NWassF-regM-4}\nwixd{NWassF-regM-5}\nwixd{NWassF-regM-6}\nwixd{NWassF-regM-7}\nwixd{NWassF-regM-8}\nwixd{NWassF-regM-9}}}%
\nwixlogsorted{c}{{static data}{NWassF-staB-1}{\nwixu{NWassF-assE.2-1}\nwixd{NWassF-staB-1}}}%
\nwbegindocs{66}\nwdocspar
\nwenddocs{}
