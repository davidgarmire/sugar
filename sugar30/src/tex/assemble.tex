
\section{Introduction}

The assembly module is responsible for building global entities
from local contributions, and for extracting the local part of
global structures.  In particular, this module builds the tangent
stiffness matrix and load vector.


\section{Interface}

\nwfilename{assemble.nw}\nwbegincode{1}\sublabel{NWassB-assA-1}\nwmargintag{{\nwtagstyle{}\subpageref{NWassB-assA-1}}}\moddef{assemble.h~{\nwtagstyle{}\subpageref{NWassB-assA-1}}}\endmoddef
#ifndef ASSEMBLE_H
#define ASSEMBLE_H

#include "mempool.h"
#include "mesh.h"
#include "vars.h"

#include "assemblem.h"
#include "assemblem-sparse.h"
#include "assemblem-dense.h"
#include "superlu.h"

\LA{}exported types~{\nwtagstyle{}\subpageref{NWassB-expE-1}}\RA{}
\LA{}exported functions~{\nwtagstyle{}\subpageref{NWassB-expI-1}}\RA{}

#endif /* ASSEMBLE_H */
\nwnotused{assemble.h}\nwendcode{}\nwbegindocs{2}\nwdocspar

An assembler object requires a mesh and a variable manager.

\nwenddocs{}\nwbegincode{3}\sublabel{NWassB-expE-1}\nwmargintag{{\nwtagstyle{}\subpageref{NWassB-expE-1}}}\moddef{exported types~{\nwtagstyle{}\subpageref{NWassB-expE-1}}}\endmoddef
typedef struct assembler_struct* assembler_t;

\nwused{\\{NWassB-assA-1}}\nwendcode{}\nwbegindocs{4}\nwdocspar

\nwenddocs{}\nwbegincode{5}\sublabel{NWassB-expI-1}\nwmargintag{{\nwtagstyle{}\subpageref{NWassB-expI-1}}}\moddef{exported functions~{\nwtagstyle{}\subpageref{NWassB-expI-1}}}\endmoddef
assembler_t assemble_pcreate(mempool_t pool, mesh_t mesh);
assembler_t assemble_create (mesh_t mesh);
void        assemble_destroy(assembler_t self);

\nwalsodefined{\\{NWassB-expI-2}\\{NWassB-expI-3}}\nwused{\\{NWassB-assA-1}}\nwendcode{}\nwbegindocs{6}\nwdocspar

The {\tt{}add{\char95}displace} method applies displacement boundary conditions
to a specified degree of freedom.  The {\tt{}displace} method builds
a global list of displacement boundary conditions.  Inactive degrees
of freedom are permuted to the end of the list.
The displacement vector and the number of active degrees of 
freedom after boundary conditions can be retrieved using 
{\tt{}get{\char95}displacements} and {\tt{}get{\char95}active}, respectively.

\nwenddocs{}\nwbegincode{7}\sublabel{NWassB-expI-2}\nwmargintag{{\nwtagstyle{}\subpageref{NWassB-expI-2}}}\moddef{exported functions~{\nwtagstyle{}\subpageref{NWassB-expI-1}}}\plusendmoddef
void    assemble_add_displace(assembler_t self, int var_id, double value);
void    assemble_displace    (assembler_t self);

double* assemble_get_displacements(assembler_t self);
int     assemble_get_active       (assembler_t self);

\nwendcode{}\nwbegindocs{8}\nwdocspar

{\tt{}assemble{\char95}R} and {\tt{}assemble{\char95}dR} put together the global version
of the (reduced) residual and stiffness.   The {\tt{}assemble{\char95}dR{\char95}csc} function
builds a compressed sparse column representation of the matrix.

\nwenddocs{}\nwbegincode{9}\sublabel{NWassB-expI-3}\nwmargintag{{\nwtagstyle{}\subpageref{NWassB-expI-3}}}\moddef{exported functions~{\nwtagstyle{}\subpageref{NWassB-expI-1}}}\plusendmoddef
void assemble_R   (assembler_t self, double* x, double* v, double* a,
                   double* R, int Rsize);
void assemble_dR  (assembler_t self,
                   double cx, double* x, 
                   double cv, double* v, 
                   double ca, double* a, 
                   double* dR,
                   int dRsize);
superlu_t assemble_dR_csc(assembler_t self,
                          double cx, double* x, 
                          double cv, double* v, 
                          double ca, double* a,
                          int dRsize);

\nwendcode{}\nwbegindocs{10}\nwdocspar


\section{Implementation}

\nwenddocs{}\nwbegincode{11}\sublabel{NWassB-assA.2-1}\nwmargintag{{\nwtagstyle{}\subpageref{NWassB-assA.2-1}}}\moddef{assemble.c~{\nwtagstyle{}\subpageref{NWassB-assA.2-1}}}\endmoddef
#include <stdlib.h>
#include <string.h>
#include <assert.h>

#include "modelmgr.h"
#include "assemble.h"

\LA{}types~{\nwtagstyle{}\subpageref{NWassB-typ5-1}}\RA{}
\LA{}functions~{\nwtagstyle{}\subpageref{NWassB-fun9-1}}\RA{}
\nwnotused{assemble.c}\nwendcode{}\nwbegindocs{12}\nwdocspar

\nwenddocs{}\nwbegincode{13}\sublabel{NWassB-typ5-1}\nwmargintag{{\nwtagstyle{}\subpageref{NWassB-typ5-1}}}\moddef{types~{\nwtagstyle{}\subpageref{NWassB-typ5-1}}}\endmoddef
struct assembler_struct \{
    \LA{}\code{}assembler{\char95}t\edoc{} entries~{\nwtagstyle{}\subpageref{NWassB-**aN-1}}\RA{}
\};

\nwused{\\{NWassB-assA.2-1}}\nwendcode{}\nwbegindocs{14}\nwdocspar

Right now, the ``by pool'' allocation option is pretty useless,
since we end up allocating all our temporary space using {\tt{}malloc}.
The only piece that is actually allocated on the pool is the
{\tt{}assembler{\char95}t} structure itself.

\nwenddocs{}\nwbegincode{15}\sublabel{NWassB-**aN-1}\nwmargintag{{\nwtagstyle{}\subpageref{NWassB-**aN-1}}}\moddef{\code{}assembler{\char95}t\edoc{} entries~{\nwtagstyle{}\subpageref{NWassB-**aN-1}}}\endmoddef
mesh_t     mesh;
vars_mgr_t vars_mgr;
int        via_pool;
\nwalsodefined{\\{NWassB-**aN-2}}\nwused{\\{NWassB-typ5-1}}\nwendcode{}\nwbegindocs{16}\nwdocspar

\nwenddocs{}\nwbegincode{17}\sublabel{NWassB-fun9-1}\nwmargintag{{\nwtagstyle{}\subpageref{NWassB-fun9-1}}}\moddef{functions~{\nwtagstyle{}\subpageref{NWassB-fun9-1}}}\endmoddef
assembler_t assemble_create(mesh_t mesh)
\{
    assembler_t self = calloc(1, sizeof(*self));
    self->mesh       = mesh;
    self->vars_mgr   = mesh_vars_mgr(mesh);
    self->via_pool   = 0;
    return self;
\}

\nwalsodefined{\\{NWassB-fun9-2}\\{NWassB-fun9-3}\\{NWassB-fun9-4}\\{NWassB-fun9-5}\\{NWassB-fun9-6}\\{NWassB-fun9-7}\\{NWassB-fun9-8}}\nwused{\\{NWassB-assA.2-1}}\nwendcode{}\nwbegindocs{18}\nwdocspar

\nwenddocs{}\nwbegincode{19}\sublabel{NWassB-fun9-2}\nwmargintag{{\nwtagstyle{}\subpageref{NWassB-fun9-2}}}\moddef{functions~{\nwtagstyle{}\subpageref{NWassB-fun9-1}}}\plusendmoddef
assembler_t assemble_pcreate(mempool_t pool, mesh_t mesh)
\{
    assembler_t self = (assembler_t) mempool_cget(pool, sizeof(*self));
    self->mesh       = mesh;
    self->vars_mgr   = mesh_vars_mgr(mesh);
    self->via_pool   = 1;
    return self;
\}

\nwendcode{}\nwbegindocs{20}\nwdocspar

\nwenddocs{}\nwbegincode{21}\sublabel{NWassB-fun9-3}\nwmargintag{{\nwtagstyle{}\subpageref{NWassB-fun9-3}}}\moddef{functions~{\nwtagstyle{}\subpageref{NWassB-fun9-1}}}\plusendmoddef
void assemble_destroy(assembler_t self)
\{
    \LA{}destroy assembler variables~{\nwtagstyle{}\subpageref{NWassB-desR-1}}\RA{}

    if (!self->via_pool) \{
        free(self);
    \}
\}

\nwendcode{}\nwbegindocs{22}\nwdocspar

We need some internal storage when we assemble displacement boundary
conditions.  We store the number of variables, total and active,
in {\tt{}num{\char95}vars} and {\tt{}num{\char95}active}.  We keep track of which variables
have been fixed by displacement boundary conditions using the
logical {\tt{}is{\char95}displaced} vector, and we keep track of the
displacement values in the {\tt{}displacements} vector.  The
{\tt{}is{\char95}displaced} vector becomes redundant after we generate
a {\tt{}permutation} vector that moves all the inactive degrees
of freedom to the end of the list.

\nwenddocs{}\nwbegincode{23}\sublabel{NWassB-**aN-2}\nwmargintag{{\nwtagstyle{}\subpageref{NWassB-**aN-2}}}\moddef{\code{}assembler{\char95}t\edoc{} entries~{\nwtagstyle{}\subpageref{NWassB-**aN-1}}}\plusendmoddef
int        num_vars;
int        num_active;
double*    displacements;
\nwendcode{}\nwbegindocs{24}\nwdocspar

\nwenddocs{}\nwbegincode{25}\sublabel{NWassB-desd-1}\nwmargintag{{\nwtagstyle{}\subpageref{NWassB-desd-1}}}\moddef{destroy displacement assembly variables~{\nwtagstyle{}\subpageref{NWassB-desd-1}}}\endmoddef
if (self->displacements) free(self->displacements);
\nwused{\\{NWassB-desR-1}\\{NWassB-fun9-4}}\nwendcode{}\nwbegindocs{26}\nwdocspar

\nwenddocs{}\nwbegincode{27}\sublabel{NWassB-desR-1}\nwmargintag{{\nwtagstyle{}\subpageref{NWassB-desR-1}}}\moddef{destroy assembler variables~{\nwtagstyle{}\subpageref{NWassB-desR-1}}}\endmoddef
\LA{}destroy displacement assembly variables~{\nwtagstyle{}\subpageref{NWassB-desd-1}}\RA{}
\nwused{\\{NWassB-fun9-3}}\nwendcode{}\nwbegindocs{28}\nwdocspar

\nwenddocs{}\nwbegincode{29}\sublabel{NWassB-fun9-4}\nwmargintag{{\nwtagstyle{}\subpageref{NWassB-fun9-4}}}\moddef{functions~{\nwtagstyle{}\subpageref{NWassB-fun9-1}}}\plusendmoddef
void assemble_displace(assembler_t self)
\{
    int i, n, active_id, inactive_id;
    assemble_matrix_t* bc_assembler;
    int*  permutation;
    char* has_bc;

    \LA{}destroy displacement assembly variables~{\nwtagstyle{}\subpageref{NWassB-desd-1}}\RA{}

    self->num_vars      = vars_count(self->vars_mgr);
    self->displacements = calloc(self->num_vars, sizeof(double));
    permutation         = calloc(self->num_vars, sizeof(int));
    has_bc              = calloc(self->num_vars, 1);
    bc_assembler = assembler_bc(self->displacements, has_bc, self->num_vars);

    \LA{}call element \code{}displace\edoc{} cases~{\nwtagstyle{}\subpageref{NWassB-calV-1}}\RA{}
    \LA{}count active degrees of freedom~{\nwtagstyle{}\subpageref{NWassB-couV-1}}\RA{}
    \LA{}bucket sort inactive variables to the end~{\nwtagstyle{}\subpageref{NWassB-bucf-1}}\RA{}
    \LA{}rearrange entries of \code{}displacements\edoc{}~{\nwtagstyle{}\subpageref{NWassB-reac-1}}\RA{}

    vars_permute(self->vars_mgr, permutation);
    assembler_matrix_free(bc_assembler);
    free(permutation);
    free(has_bc);
\}

\nwendcode{}\nwbegindocs{30}\nwdocspar

The main assembly loop calls the {\tt{}displace} method for each
element that has one.  Those routines in turn call {\tt{}add{\char95}displace},
which records the boundary condition in the {\tt{}is{\char95}displaced} and
{\tt{}displacements} vectors.

\nwenddocs{}\nwbegincode{31}\sublabel{NWassB-calV-1}\nwmargintag{{\nwtagstyle{}\subpageref{NWassB-calV-1}}}\moddef{call element \code{}displace\edoc{} cases~{\nwtagstyle{}\subpageref{NWassB-calV-1}}}\endmoddef
n = mesh_num_elements(self->mesh);
for (i = 1; i <= n; ++i)
    element_displace( mesh_element(self->mesh, i), bc_assembler );

\nwused{\\{NWassB-fun9-4}}\nwendcode{}\nwbegindocs{32}\nwdocspar

We build the permutation vector using a simple bucket sort.  Among
other things, that means the permutation retains the relative ordering
of the active variables among themselves and of the inactive variables
among themselves.

\nwenddocs{}\nwbegincode{33}\sublabel{NWassB-couV-1}\nwmargintag{{\nwtagstyle{}\subpageref{NWassB-couV-1}}}\moddef{count active degrees of freedom~{\nwtagstyle{}\subpageref{NWassB-couV-1}}}\endmoddef
self->num_active = 0;
for (i = 0; i < self->num_vars; ++i)
    if (!has_bc[i])
        self->num_active++;
\nwused{\\{NWassB-fun9-4}}\nwendcode{}\nwbegindocs{34}\nwdocspar

\nwenddocs{}\nwbegincode{35}\sublabel{NWassB-bucf-1}\nwmargintag{{\nwtagstyle{}\subpageref{NWassB-bucf-1}}}\moddef{bucket sort inactive variables to the end~{\nwtagstyle{}\subpageref{NWassB-bucf-1}}}\endmoddef
active_id   = 0;
inactive_id = self->num_active;
for (i = 0; i < self->num_vars; ++i) \{
    if (has_bc[i])
        permutation[i] = inactive_id++;
    else
        permutation[i] = active_id++;
\}
\nwused{\\{NWassB-fun9-4}}\nwendcode{}\nwbegindocs{36}\nwdocspar

Why can we do this?  Because we know that the inactive degrees of freedom
are shifted to the end, and that their relative order does not change.
That means that for all the inactive variables, $\pi(i) >= i$.

\nwenddocs{}\nwbegincode{37}\sublabel{NWassB-reac-1}\nwmargintag{{\nwtagstyle{}\subpageref{NWassB-reac-1}}}\moddef{rearrange entries of \code{}displacements\edoc{}~{\nwtagstyle{}\subpageref{NWassB-reac-1}}}\endmoddef
for (i = self->num_vars - 1; i >= 0; --i) \{
    if (permutation[i] >= self->num_active) \{
        self->displacements[permutation[i]] = self->displacements[i];
    \}
\}
memset(self->displacements, 0, self->num_active * sizeof(double));
\nwused{\\{NWassB-fun9-4}}\nwendcode{}\nwbegindocs{38}\nwdocspar

\nwenddocs{}\nwbegincode{39}\sublabel{NWassB-fun9-5}\nwmargintag{{\nwtagstyle{}\subpageref{NWassB-fun9-5}}}\moddef{functions~{\nwtagstyle{}\subpageref{NWassB-fun9-1}}}\plusendmoddef
double* assemble_get_displacements(assembler_t self)
\{
    return self->displacements;
\}

int assemble_get_active(assembler_t self)
\{
    return self->num_active;
\}

\nwendcode{}\nwbegindocs{40}\nwdocspar

The {\tt{}R} and {\tt{}dR} routines loop through the element model
functions and call their {\tt{}R} and {\tt{}dR} model cases.  Those functions
in turn call {\tt{}add{\char95}R} and {\tt{}add{\char95}dR} to add their
local contributions to the global matrix.  In the process, they
may need to get information from the global $x$ vector using {\tt{}xlocal}.
While we are doing the assembly, we keep pointers to the global $x$
vector and to the output matrix in fields in the assembler object,
so that we can get at them readily from {\tt{}add{\char95}R} and {\tt{}add{\char95}dR}.

\nwenddocs{}\nwbegincode{41}\sublabel{NWassB-fun9-6}\nwmargintag{{\nwtagstyle{}\subpageref{NWassB-fun9-6}}}\moddef{functions~{\nwtagstyle{}\subpageref{NWassB-fun9-1}}}\plusendmoddef
void assemble_R(assembler_t self, double* x, double* v, double* a, double* R,
                int Rsize)
\{
    int i, n;

    assemble_matrix_t* assembler_x;
    assemble_matrix_t* assembler_v;
    assemble_matrix_t* assembler_a;
    assemble_matrix_t* assembler_R;

    assembler_x = (x == NULL) ? NULL : 
        assembler_dense_getvector(x, self->num_vars, self->num_vars);

    assembler_v = (v == NULL) ? NULL : 
        assembler_dense_getvector(v, self->num_vars, self->num_vars);

    assembler_a = (a == NULL) ? NULL : 
        assembler_dense_getvector(a, self->num_vars, self->num_vars);

    assembler_R = 
        assembler_dense_vector(R, Rsize, Rsize);

    memset(R, 0, self->num_active * sizeof(double));
    n = mesh_num_elements(self->mesh);
    for (i = 1; i <= n; ++i)
        element_R( mesh_element(self->mesh, i), assembler_R, 
                   assembler_x, assembler_v, assembler_a );

    assembler_matrix_free(assembler_R);
    if (assembler_a) assembler_matrix_free(assembler_a);
    if (assembler_v) assembler_matrix_free(assembler_v);
    if (assembler_x) assembler_matrix_free(assembler_x);
\}

\nwendcode{}\nwbegindocs{42}\nwdocspar

\nwenddocs{}\nwbegincode{43}\sublabel{NWassB-fun9-7}\nwmargintag{{\nwtagstyle{}\subpageref{NWassB-fun9-7}}}\moddef{functions~{\nwtagstyle{}\subpageref{NWassB-fun9-1}}}\plusendmoddef
void assemble_dR(assembler_t self, 
                 double cx, double* x, 
                 double cv, double* v, 
                 double ca, double* a, 
                 double* dR,
                 int dRsize)
\{
    int i, n;

    assemble_matrix_t* assembler_x;
    assemble_matrix_t* assembler_v;
    assemble_matrix_t* assembler_a;
    assemble_matrix_t* assembler_dR;

    assembler_x = (x == NULL) ? NULL : 
        assembler_dense_getvector(x, self->num_vars, self->num_vars);

    assembler_v = (v == NULL) ? NULL : 
        assembler_dense_getvector(v, self->num_vars, self->num_vars);

    assembler_a = (a == NULL) ? NULL : 
        assembler_dense_getvector(a, self->num_vars, self->num_vars);

    assembler_dR = 
        assembler_dense_matrix(dR, dRsize, dRsize);

    memset(dR, 0, self->num_active * self->num_active * sizeof(double));
    n = mesh_num_elements(self->mesh);
    for (i = 1; i <= n; ++i)
        element_dR( mesh_element(self->mesh, i), assembler_dR, 
                    cx, assembler_x,
                    cv, assembler_v,
                    ca, assembler_a );

    assembler_matrix_free(assembler_dR);
    if (assembler_a) assembler_matrix_free(assembler_a);
    if (assembler_v) assembler_matrix_free(assembler_v);
    if (assembler_x) assembler_matrix_free(assembler_x);
\}

\nwendcode{}\nwbegindocs{44}\nwdocspar

\nwenddocs{}\nwbegincode{45}\sublabel{NWassB-fun9-8}\nwmargintag{{\nwtagstyle{}\subpageref{NWassB-fun9-8}}}\moddef{functions~{\nwtagstyle{}\subpageref{NWassB-fun9-1}}}\plusendmoddef
superlu_t assemble_dR_csc(assembler_t self,
                          double cx, double* x, 
                          double cv, double* v, 
                          double ca, double* a,
                          int dRsize)
\{
#ifdef HAS_SUPERLU
    int i, n;
    superlu_t dRsuper;

    assemble_matrix_t* assembler_x;
    assemble_matrix_t* assembler_v;
    assemble_matrix_t* assembler_a;
    assemble_matrix_t* assembler_dR;

    assembler_x = (x == NULL) ? NULL : 
        assembler_dense_getvector(x, self->num_vars, self->num_vars);

    assembler_v = (v == NULL) ? NULL : 
        assembler_dense_getvector(v, self->num_vars, self->num_vars);

    assembler_a = (a == NULL) ? NULL : 
        assembler_dense_getvector(a, self->num_vars, self->num_vars);

    assembler_dR = 
        assembler_sparse_matrix(dRsize, dRsize);

    n = mesh_num_elements(self->mesh);
    for (i = 1; i <= n; ++i)
        element_dR( mesh_element(self->mesh, i), assembler_dR, 
                    cx, assembler_x,
                    cv, assembler_v,
                    ca, assembler_a);

    assembler_sparse_csc(assembler_dR);
    dRsuper = superlu_create(assembler_dR, dRsize);

    assembler_matrix_free(assembler_dR);
    if (assembler_a) assembler_matrix_free(assembler_a);
    if (assembler_v) assembler_matrix_free(assembler_v);
    if (assembler_x) assembler_matrix_free(assembler_x);

    return dRsuper;
#else
    printf("Library not compiled with SuperLU support\\n");
    assert(0);
#endif
\}

\nwendcode{}

\nwixlogsorted{c}{{assemble.c}{NWassB-assA.2-1}{\nwixd{NWassB-assA.2-1}}}%
\nwixlogsorted{c}{{assemble.h}{NWassB-assA-1}{\nwixd{NWassB-assA-1}}}%
\nwixlogsorted{c}{{\code{}assembler{\char95}t\edoc{} entries}{NWassB-**aN-1}{\nwixu{NWassB-typ5-1}\nwixd{NWassB-**aN-1}\nwixd{NWassB-**aN-2}}}%
\nwixlogsorted{c}{{bucket sort inactive variables to the end}{NWassB-bucf-1}{\nwixu{NWassB-fun9-4}\nwixd{NWassB-bucf-1}}}%
\nwixlogsorted{c}{{call element \code{}displace\edoc{} cases}{NWassB-calV-1}{\nwixu{NWassB-fun9-4}\nwixd{NWassB-calV-1}}}%
\nwixlogsorted{c}{{count active degrees of freedom}{NWassB-couV-1}{\nwixu{NWassB-fun9-4}\nwixd{NWassB-couV-1}}}%
\nwixlogsorted{c}{{destroy assembler variables}{NWassB-desR-1}{\nwixu{NWassB-fun9-3}\nwixd{NWassB-desR-1}}}%
\nwixlogsorted{c}{{destroy displacement assembly variables}{NWassB-desd-1}{\nwixd{NWassB-desd-1}\nwixu{NWassB-desR-1}\nwixu{NWassB-fun9-4}}}%
\nwixlogsorted{c}{{exported functions}{NWassB-expI-1}{\nwixu{NWassB-assA-1}\nwixd{NWassB-expI-1}\nwixd{NWassB-expI-2}\nwixd{NWassB-expI-3}}}%
\nwixlogsorted{c}{{exported types}{NWassB-expE-1}{\nwixu{NWassB-assA-1}\nwixd{NWassB-expE-1}}}%
\nwixlogsorted{c}{{functions}{NWassB-fun9-1}{\nwixu{NWassB-assA.2-1}\nwixd{NWassB-fun9-1}\nwixd{NWassB-fun9-2}\nwixd{NWassB-fun9-3}\nwixd{NWassB-fun9-4}\nwixd{NWassB-fun9-5}\nwixd{NWassB-fun9-6}\nwixd{NWassB-fun9-7}\nwixd{NWassB-fun9-8}}}%
\nwixlogsorted{c}{{rearrange entries of \code{}displacements\edoc{}}{NWassB-reac-1}{\nwixu{NWassB-fun9-4}\nwixd{NWassB-reac-1}}}%
\nwixlogsorted{c}{{types}{NWassB-typ5-1}{\nwixu{NWassB-assA.2-1}\nwixd{NWassB-typ5-1}}}%
\nwbegindocs{46}\nwdocspar
\nwenddocs{}
