
\section{Introduction}

The {\tt{}mempool} module manages a region-based allocator,
or memory pool.  A user can create a region associated with
some task or data structure and allocate from it freely;
when the time comes to finish the program task or destroy
the data structure, the region can be deallocated.
See Dave Hanson's book ``C Interfaces and Implementations'' %'
for some further information on region-based allocators.

This particular allocator is not very fancy.  It \emph{does}
allow us to mark entire data structures as persistent across
MeX files in Matlab, which is convenient.  It also has special
support for stack-order allocation / deallocation patterns.


\section{Interface}

\nwfilename{mempool.nw}\nwbegincode{1}\sublabel{NWmemA-mem9-1}\nwmargintag{{\nwtagstyle{}\subpageref{NWmemA-mem9-1}}}\moddef{mempool.h~{\nwtagstyle{}\subpageref{NWmemA-mem9-1}}}\endmoddef
#ifndef MEMPOOL_H
#define MEMPOOL_H

typedef struct mempool_struct* mempool_t;

\LA{}exported types~{\nwtagstyle{}\subpageref{NWmemA-expE-1}}\RA{}
\LA{}exported macros~{\nwtagstyle{}\subpageref{NWmemA-expF-1}}\RA{}
\LA{}exported functions~{\nwtagstyle{}\subpageref{NWmemA-expI-1}}\RA{}

#endif /* MEMPOOL_H */
\nwnotused{mempool.h}\nwendcode{}\nwbegindocs{2}\nwdocspar

The {\tt{}mempool{\char95}create} and {\tt{}mempool{\char95}destroy}
functions create and destroy regions.  When creating a new
region, you must specify the typical size of a \emph{span},
or chunk of memory to be allocated contiguously.  New spans
are allocated as needed, and requests to allocate a contiguous
chunk of memory larger than the ordinary span size are accomodated
by special extra-large spans.  The default span size,
{\tt{}MEMPOOL{\char95}DEFAULT{\char95}SPAN} is 4K.

\nwenddocs{}\nwbegincode{3}\sublabel{NWmemA-expI-1}\nwmargintag{{\nwtagstyle{}\subpageref{NWmemA-expI-1}}}\moddef{exported functions~{\nwtagstyle{}\subpageref{NWmemA-expI-1}}}\endmoddef
mempool_t mempool_create (int span_size);
void      mempool_destroy(mempool_t pool);

\nwalsodefined{\\{NWmemA-expI-2}\\{NWmemA-expI-3}\\{NWmemA-expI-4}\\{NWmemA-expI-5}}\nwused{\\{NWmemA-mem9-1}}\nwendcode{}\nwbegindocs{4}\nwdocspar

\nwenddocs{}\nwbegincode{5}\sublabel{NWmemA-expF-1}\nwmargintag{{\nwtagstyle{}\subpageref{NWmemA-expF-1}}}\moddef{exported macros~{\nwtagstyle{}\subpageref{NWmemA-expF-1}}}\endmoddef
#define MEMPOOL_DEFAULT_SPAN (4096)

\nwused{\\{NWmemA-mem9-1}}\nwendcode{}\nwbegindocs{6}\nwdocspar

The basic operation on a memory pool is (obviously enough)
getting memory.  The {\tt{}mempool{\char95}get} function gets a chunk
of memory from the pool; the {\tt{}mempool{\char95}cget} function gets
a zero-filled chunk of memory.

It is sometimes convenient to access memory pools using a
\emph{stack discipline}.  That is, allocation and deletion
requests can be interlaced in stack order.  The beginning
of a ``stack frame'' in our memory pool module is marked
by a \emph{handle}.  The functions {\tt{}mempool{\char95}geth} and
{\tt{}mempool{\char95}cgeth} are much like {\tt{}mempool{\char95}get} and
{\tt{}mempool{\char95}cget}, except that the memory returned is associated
with a handle in the memory pool.  If {\tt{}mempool{\char95}freeh} is
called on a pointer returned from {\tt{}mempool{\char95}geth}, the memory
pool is rolled back to the state it was in immediately before
the corresponding {\tt{}mempool{\char95}geth} call.  This means that not
only is the memory allocated in that {\tt{}mempool{\char95}geth} call
freed, but so also is all the memory allocated from the pool
\emph{after} the call.

\nwenddocs{}\nwbegincode{7}\sublabel{NWmemA-expI-2}\nwmargintag{{\nwtagstyle{}\subpageref{NWmemA-expI-2}}}\moddef{exported functions~{\nwtagstyle{}\subpageref{NWmemA-expI-1}}}\plusendmoddef
void* mempool_get  (mempool_t pool, int size);
void* mempool_cget (mempool_t pool, int size);
void* mempool_geth (mempool_t pool, int size);
void* mempool_cgeth(mempool_t pool, int size);
void  mempool_freeh(void* mem);

\nwendcode{}\nwbegindocs{8}\nwdocspar

The standard function {\tt{}strdup} mallocs a buffer and copies
into it an argument string.  It's a convenient function. %'
It's also often convenient to copy some more general object %'
from a temporary buffer into dynamically allocated memory.
The mempool module supports this functionality with
{\tt{}mempool{\char95}strdup} and {\tt{}mempool{\char95}memdup}.

Note that if a NULL argument is passed to {\tt{}mempool{\char95}strdup}
or {\tt{}mempool{\char95}memdup}, the return value will be NULL.

\nwenddocs{}\nwbegincode{9}\sublabel{NWmemA-expI-3}\nwmargintag{{\nwtagstyle{}\subpageref{NWmemA-expI-3}}}\moddef{exported functions~{\nwtagstyle{}\subpageref{NWmemA-expI-1}}}\plusendmoddef
void* mempool_memdup(mempool_t pool, const void* src, int size);
void* mempool_strdup(mempool_t pool, const char* src);

\nwendcode{}\nwbegindocs{10}\nwdocspar

For debugging purposes, the mempool class supports logging.
The mempool log file is a global object, not associated with
any particular pool instance.

\nwenddocs{}\nwbegincode{11}\sublabel{NWmemA-expI-4}\nwmargintag{{\nwtagstyle{}\subpageref{NWmemA-expI-4}}}\moddef{exported functions~{\nwtagstyle{}\subpageref{NWmemA-expI-1}}}\plusendmoddef
void mempool_start_log(char* filename);
void mempool_end_log  ();
void mempool_flush_log();

\nwendcode{}\nwbegindocs{12}\nwdocspar

In the Matlab environment, by default, memory allocated during an
invocation of a MEX file is automatically freed after
the call finishes.  It is possible to override this default
by explicitly requesting that Matlab make the allocated memory
persistent across calls.  The {\tt{}mempool{\char95}persist} function
causes Matlab to mark all memory in a pool as persistent.

\nwenddocs{}\nwbegincode{13}\sublabel{NWmemA-expI-5}\nwmargintag{{\nwtagstyle{}\subpageref{NWmemA-expI-5}}}\moddef{exported functions~{\nwtagstyle{}\subpageref{NWmemA-expI-1}}}\plusendmoddef
void mempool_persist(mempool_t pool);

\nwendcode{}\nwbegindocs{14}\nwdocspar

Memory pool allocations must be properly \emph{aligned} on most
systems.  There are some other modules that also care about
alignment.  See the implementation section for details.

\nwenddocs{}\nwbegincode{15}\sublabel{NWmemA-expE-1}\nwmargintag{{\nwtagstyle{}\subpageref{NWmemA-expE-1}}}\moddef{exported types~{\nwtagstyle{}\subpageref{NWmemA-expE-1}}}\endmoddef
\LA{}alignment definitions~{\nwtagstyle{}\subpageref{NWmemA-aliL-1}}\RA{}
\nwused{\\{NWmemA-mem9-1}}\nwendcode{}\nwbegindocs{16}\nwdocspar


\section{Implementation}

Besides standard library and memory manipulation routines,
we will include the SUGAR main header file, which defines
macros to alias {\tt{}malloc} and {\tt{}free} to the Matlab
functions {\tt{}mxMalloc} and {\tt{}mxFree}.

\nwenddocs{}\nwbegincode{17}\sublabel{NWmemA-mem9.2-1}\nwmargintag{{\nwtagstyle{}\subpageref{NWmemA-mem9.2-1}}}\moddef{mempool.c~{\nwtagstyle{}\subpageref{NWmemA-mem9.2-1}}}\endmoddef
#include <stdio.h>
#include <stdlib.h>
#include <string.h>
#include <assert.h>

#include "mempool.h"

\LA{}types~{\nwtagstyle{}\subpageref{NWmemA-typ5-1}}\RA{}
\LA{}static data~{\nwtagstyle{}\subpageref{NWmemA-staB-1}}\RA{}

\LA{}static functions~{\nwtagstyle{}\subpageref{NWmemA-staG-1}}\RA{}
\LA{}functions~{\nwtagstyle{}\subpageref{NWmemA-fun9-1}}\RA{}
\nwnotused{mempool.c}\nwendcode{}\nwbegindocs{18}\nwdocspar


\subsection{Alignment}

In order to accomodate arbitrary objects, the memory returned
by allocation requests to a mempool must be properly aligned.
The {\tt{}align{\char95}t} union contains the possible basic types;
the size of an {\tt{}align{\char95}t} structure, indicated by the macro
{\tt{}align{\char95}size}, determines exactly how little memory can be
allocated.

\nwenddocs{}\nwbegincode{19}\sublabel{NWmemA-aliL-1}\nwmargintag{{\nwtagstyle{}\subpageref{NWmemA-aliL-1}}}\moddef{alignment definitions~{\nwtagstyle{}\subpageref{NWmemA-aliL-1}}}\endmoddef
typedef union \{
    int i;
    long l;
    long* lp;
    void* p;
    void (*fp)();
    float f;
    double d;
    long double ld;
\} align_t;

#define align_size (sizeof(align_t))

\nwused{\\{NWmemA-expE-1}}\nwendcode{}\nwbegindocs{20}\nwdocspar


\subsection{Creating and destroying pools}

As mentioned in the interface section, a memory pool's space
consists of a linked list of \emph{spans}.  Spans are further
subdivided into \emph{blocks}, which are the smallest units
guaranteed to start at, and end just before, alignment boundaries.
The first block in each span is reserved for a link to
the next span.

The mempool structure tracks the typical number of blocks
to be allocated in ordinary new spans, the number of blocks
used in the current span, and the locations of the first and
current span.  In addition, the first field of the mempool
structure is a ``magic number'' used for debugging purposes.
The magic number is used for debugging purposes;
functions expecting a pool argument can use it to check that
the user did not accidentally pass in a bogus pointer.

\nwenddocs{}\nwbegincode{21}\sublabel{NWmemA-typ5-1}\nwmargintag{{\nwtagstyle{}\subpageref{NWmemA-typ5-1}}}\moddef{types~{\nwtagstyle{}\subpageref{NWmemA-typ5-1}}}\endmoddef
#define MEMPOOL_MAGIC_NUMBER 0x1234
#define ASSERT_VALID_POOL \\
    assert(pool != NULL && pool->magic_number == MEMPOOL_MAGIC_NUMBER)

struct mempool_struct \{
    int magic_number;
    int blocks_per_span;
    int blocks_used;
    align_t* first_span;
    align_t* current_span;
\};

\nwalsodefined{\\{NWmemA-typ5-2}}\nwused{\\{NWmemA-mem9.2-1}}\nwendcode{}\nwbegindocs{22}\nwdocspar

The local {\tt{}allocate{\char95}span} function allocates and initializes new
spans to go into pools list.  The {\tt{}blocks{\char95}in{\char95}span} argument
gives the number of blocks required to accomodate an incoming
request (recall that requests larger than the typical span size
sometimes happen)

\nwenddocs{}\nwbegincode{23}\sublabel{NWmemA-staG-1}\nwmargintag{{\nwtagstyle{}\subpageref{NWmemA-staG-1}}}\moddef{static functions~{\nwtagstyle{}\subpageref{NWmemA-staG-1}}}\endmoddef
static align_t* allocate_span(mempool_t pool, int blocks_in_span)
\{
    align_t* span;

    \LA{}allocate memory for the new span~{\nwtagstyle{}\subpageref{NWmemA-allW-1}}\RA{}
    \LA{}optionally clear the new span~{\nwtagstyle{}\subpageref{NWmemA-optT-1}}\RA{}
    \LA{}initialize the span~{\nwtagstyle{}\subpageref{NWmemA-iniJ-1}}\RA{}

    \LA{}log span allocation~{\nwtagstyle{}\subpageref{NWmemA-logJ-1}}\RA{}

    return span;
\}

\nwused{\\{NWmemA-mem9.2-1}}\nwendcode{}\nwbegindocs{24}\nwdocspar

When we allocate the memory for the new span, we want to make
sure that it is sufficient to accomodate extra-large requests.
For smaller requests, we default to the standard {\tt{}blocks{\char95}per{\char95}span}.
Note that there will be one extra block in each span to keep
the ``next'' pointer.

\nwenddocs{}\nwbegincode{25}\sublabel{NWmemA-allW-1}\nwmargintag{{\nwtagstyle{}\subpageref{NWmemA-allW-1}}}\moddef{allocate memory for the new span~{\nwtagstyle{}\subpageref{NWmemA-allW-1}}}\endmoddef
if (blocks_in_span < pool->blocks_per_span)
    blocks_in_span = pool->blocks_per_span;
span = (align_t*) malloc((1+blocks_in_span) * align_size);
\nwused{\\{NWmemA-staG-1}}\nwendcode{}\nwbegindocs{26}\nwdocspar

If the {\tt{}MEMPOOL{\char95}CLEAR{\char95}DEBUG} macro is defined, then we zero out
the new span before proceeding.

\nwenddocs{}\nwbegincode{27}\sublabel{NWmemA-optT-1}\nwmargintag{{\nwtagstyle{}\subpageref{NWmemA-optT-1}}}\moddef{optionally clear the new span~{\nwtagstyle{}\subpageref{NWmemA-optT-1}}}\endmoddef
#ifdef MEMPOOL_CLEAR
memset(span, 0, (1+blocks_in_span) * align_size);
#endif
\nwused{\\{NWmemA-staG-1}}\nwendcode{}\nwbegindocs{28}\nwdocspar

Finally, we check to make sure that the allocation request succeeded,
and then we initialize block 0 (the next pointer) to {\tt{}NULL}.

\nwenddocs{}\nwbegincode{29}\sublabel{NWmemA-iniJ-1}\nwmargintag{{\nwtagstyle{}\subpageref{NWmemA-iniJ-1}}}\moddef{initialize the span~{\nwtagstyle{}\subpageref{NWmemA-iniJ-1}}}\endmoddef
assert(span != NULL);
span[0].p = NULL;
\nwused{\\{NWmemA-staG-1}}\nwendcode{}\nwbegindocs{30}\nwdocspar

{\tt{}mempool{\char95}create} allocates and initializes a new pool

\nwenddocs{}\nwbegincode{31}\sublabel{NWmemA-fun9-1}\nwmargintag{{\nwtagstyle{}\subpageref{NWmemA-fun9-1}}}\moddef{functions~{\nwtagstyle{}\subpageref{NWmemA-fun9-1}}}\endmoddef
mempool_t mempool_create(int span_size)
\{
    mempool_t pool;

    assert( span_size > 0 );
    assert( pool = (mempool_t) malloc(sizeof(struct mempool_struct)) );

    \LA{}initialize new pool~{\nwtagstyle{}\subpageref{NWmemA-iniJ.2-1}}\RA{}
    \LA{}log pool creation~{\nwtagstyle{}\subpageref{NWmemA-logH-1}}\RA{}

    return pool;
\}

\nwalsodefined{\\{NWmemA-fun9-2}\\{NWmemA-fun9-3}\\{NWmemA-fun9-4}\\{NWmemA-fun9-5}\\{NWmemA-fun9-6}\\{NWmemA-fun9-7}\\{NWmemA-fun9-8}\\{NWmemA-fun9-9}\\{NWmemA-fun9-A}\\{NWmemA-fun9-B}\\{NWmemA-fun9-C}}\nwused{\\{NWmemA-mem9.2-1}}\nwendcode{}\nwbegindocs{32}\nwdocspar

Note that we start off by allocating an initial span, rather than
allocating a new span on the first memory request.  That makes it
so that our list logic doesn't have to deal with the ``add to an
empty list'' case.  Note that the allocation of the new span
\emph{must} occur after the {\tt{}blocks{\char95}per{\char95}span} field has been set.

\nwenddocs{}\nwbegincode{33}\sublabel{NWmemA-iniJ.2-1}\nwmargintag{{\nwtagstyle{}\subpageref{NWmemA-iniJ.2-1}}}\moddef{initialize new pool~{\nwtagstyle{}\subpageref{NWmemA-iniJ.2-1}}}\endmoddef
pool->magic_number = MEMPOOL_MAGIC_NUMBER;
pool->blocks_per_span = (span_size + align_size-1)/align_size;
pool->first_span = allocate_span(pool, 0);
pool->current_span = pool->first_span;
pool->blocks_used = 0;
\nwused{\\{NWmemA-fun9-1}}\nwendcode{}\nwbegindocs{34}\nwdocspar

The {\tt{}mempool{\char95}destroy} function frees the pool structure and all spans.
It also clears the magic number field, so that future (invalid)
references to this pool will be caught.
Additionally, if the {\tt{}MEMPOOL{\char95}CLEAR{\char95}DEBUG} macro is defined, 
{\tt{}mempool{\char95}destroy} also overwrites all the spans with zeros 
before deleting them.

\nwenddocs{}\nwbegincode{35}\sublabel{NWmemA-fun9-2}\nwmargintag{{\nwtagstyle{}\subpageref{NWmemA-fun9-2}}}\moddef{functions~{\nwtagstyle{}\subpageref{NWmemA-fun9-1}}}\plusendmoddef
void mempool_destroy(mempool_t pool)
\{
    align_t* next_span;

    ASSERT_VALID_POOL;
    \LA{}delete all spans in the pool~{\nwtagstyle{}\subpageref{NWmemA-delS-1}}\RA{}
    pool->magic_number = 0;
    free(pool);

    \LA{}log pool deletion~{\nwtagstyle{}\subpageref{NWmemA-logH.2-1}}\RA{}
\}

\nwendcode{}\nwbegindocs{36}\nwdocspar

\nwenddocs{}\nwbegincode{37}\sublabel{NWmemA-delS-1}\nwmargintag{{\nwtagstyle{}\subpageref{NWmemA-delS-1}}}\moddef{delete all spans in the pool~{\nwtagstyle{}\subpageref{NWmemA-delS-1}}}\endmoddef
next_span = pool->first_span;
while (next_span != NULL) \{
    align_t* dead_span = next_span;
    next_span = (align_t*) next_span[0].p;
    \LA{}optionally clear dead span~{\nwtagstyle{}\subpageref{NWmemA-optQ-1}}\RA{}
    free(dead_span);
\}
\nwused{\\{NWmemA-fun9-2}}\nwendcode{}\nwbegindocs{38}\nwdocspar

Recall that there may be spans longer than the standard size of
{\tt{}1\ +\ pool->blocks{\char95}per{\char95}span}.  In such cases, only the leading
blocks are cleared.

\nwenddocs{}\nwbegincode{39}\sublabel{NWmemA-optQ-1}\nwmargintag{{\nwtagstyle{}\subpageref{NWmemA-optQ-1}}}\moddef{optionally clear dead span~{\nwtagstyle{}\subpageref{NWmemA-optQ-1}}}\endmoddef
#ifdef MEMPOOL_CLEAR_DEBUG
memset(dead_span, 0, (1 + pool->blocks_per_span)*align_size);
#endif
\nwused{\\{NWmemA-delS-1}\\{NWmemA-freT-1}}\nwendcode{}\nwbegindocs{40}\nwdocspar


\subsubsection{Basic allocation from the pool}

The most basic allocation function is {\tt{}mempool{\char95}get}.

\nwenddocs{}\nwbegincode{41}\sublabel{NWmemA-fun9-3}\nwmargintag{{\nwtagstyle{}\subpageref{NWmemA-fun9-3}}}\moddef{functions~{\nwtagstyle{}\subpageref{NWmemA-fun9-1}}}\plusendmoddef
void* mempool_get(mempool_t pool, int size)
\{
    int blocks_requested = (size + align_size-1) / align_size;
    void* mem;

    ASSERT_VALID_POOL;
    assert(blocks_requested >= 0);

    \LA{}allocate a new span if necessary~{\nwtagstyle{}\subpageref{NWmemA-allW.2-1}}\RA{}
    \LA{}allocate blocks from the current span~{\nwtagstyle{}\subpageref{NWmemA-allb-1}}\RA{}

    \LA{}log allocation from span~{\nwtagstyle{}\subpageref{NWmemA-logO-1}}\RA{}

    return mem;
\}

\nwendcode{}\nwbegindocs{42}\nwdocspar

The {\tt{}mempool{\char95}cget} function is just a convenience function
that allocates using {\tt{}mempool{\char95}get} and then clears the
returned space.

\nwenddocs{}\nwbegincode{43}\sublabel{NWmemA-fun9-4}\nwmargintag{{\nwtagstyle{}\subpageref{NWmemA-fun9-4}}}\moddef{functions~{\nwtagstyle{}\subpageref{NWmemA-fun9-1}}}\plusendmoddef
void* mempool_cget(mempool_t pool, int size)
\{
    void* mem = mempool_get(pool, size);
    memset(mem, 0, size);
    return mem;
\}

\nwendcode{}\nwbegindocs{44}\nwdocspar

The first free block in a span is always at {\tt{}1\ +\ pool->blocks{\char95}used}.
The extra 1 is to accomodate the initial block containing the link
to the next span.

\nwenddocs{}\nwbegincode{45}\sublabel{NWmemA-allb-1}\nwmargintag{{\nwtagstyle{}\subpageref{NWmemA-allb-1}}}\moddef{allocate blocks from the current span~{\nwtagstyle{}\subpageref{NWmemA-allb-1}}}\endmoddef
mem = &(pool->current_span[1 + pool->blocks_used]);
pool->blocks_used += blocks_requested;
\nwused{\\{NWmemA-fun9-3}}\nwendcode{}\nwbegindocs{46}\nwdocspar

We allocate a new span only if the request is too big
to fit into the remaining blocks in our current span.
Note that if the last span allocated was extra large to accomodate
a big allocation request, then {\tt{}pool->blocks{\char95}used\ >\ pool->blocks{\char95}per{\char95}span},
and we automatically end up allocating a new span.

\nwenddocs{}\nwbegincode{47}\sublabel{NWmemA-allW.2-1}\nwmargintag{{\nwtagstyle{}\subpageref{NWmemA-allW.2-1}}}\moddef{allocate a new span if necessary~{\nwtagstyle{}\subpageref{NWmemA-allW.2-1}}}\endmoddef
if (pool->blocks_used + blocks_requested > pool->blocks_per_span) \{
    align_t* new_span = allocate_span(pool, blocks_requested);
    \LA{}link in new span~{\nwtagstyle{}\subpageref{NWmemA-linG-1}}\RA{}
\}
\nwused{\\{NWmemA-fun9-3}}\nwendcode{}\nwbegindocs{48}\nwdocspar

\nwenddocs{}\nwbegincode{49}\sublabel{NWmemA-linG-1}\nwmargintag{{\nwtagstyle{}\subpageref{NWmemA-linG-1}}}\moddef{link in new span~{\nwtagstyle{}\subpageref{NWmemA-linG-1}}}\endmoddef
pool->current_span[0].p = new_span;
pool->current_span = new_span;
pool->blocks_used = 0;
\nwused{\\{NWmemA-allW.2-1}}\nwendcode{}\nwbegindocs{50}\nwdocspar


\subsection{Allocating with handles}

A memory pool handle contains the fields needed to save the state
of a memory pool (the {\tt{}current{\char95}span} and current {\tt{}blocks{\char95}used}),
and also a pointer to the pool with which it is associated.
The memory pool handle structure also contains a magic number field.
During a handle allocation, the handle goes in the blocks
immediately \emph{preceding} the returned pointer.
The {\tt{}MEMPOOL{\char95}HANDLE{\char95}SIZE} macro gives the size of the handle
rounded up to the nearest multiple of the alignment block size.

\nwenddocs{}\nwbegincode{51}\sublabel{NWmemA-typ5-2}\nwmargintag{{\nwtagstyle{}\subpageref{NWmemA-typ5-2}}}\moddef{types~{\nwtagstyle{}\subpageref{NWmemA-typ5-1}}}\plusendmoddef
#define MEMPOOL_HANDLE_MAGIC_NUMBER 0xA911

#define MEMPOOL_HANDLE_SIZE \\
   (((sizeof(mempool_handle_t) + align_size-1) / align_size) * align_size)

typedef struct mempool_handle_t \{
    int magic_number;
    int blocks_used;
    align_t* current_span;
    mempool_t pool;
\} mempool_handle_t;

\nwendcode{}\nwbegindocs{52}\nwdocspar

\nwenddocs{}\nwbegincode{53}\sublabel{NWmemA-inij-1}\nwmargintag{{\nwtagstyle{}\subpageref{NWmemA-inij-1}}}\moddef{initialize new handle from current pool state~{\nwtagstyle{}\subpageref{NWmemA-inij-1}}}\endmoddef
new_handle.magic_number = MEMPOOL_HANDLE_MAGIC_NUMBER;
new_handle.blocks_used  = pool->blocks_used;
new_handle.current_span = pool->current_span;
new_handle.pool         = pool;
\nwused{\\{NWmemA-fun9-5}}\nwendcode{}\nwbegindocs{54}\nwdocspar

The basic function to allocate with a handle simply allocates
a somewhat larger chunk of memory than requested using
{\tt{}mempool{\char95}get}, puts a handle in the leading blocks of
that memory, and returns a pointer to the remainder

\nwenddocs{}\nwbegincode{55}\sublabel{NWmemA-fun9-5}\nwmargintag{{\nwtagstyle{}\subpageref{NWmemA-fun9-5}}}\moddef{functions~{\nwtagstyle{}\subpageref{NWmemA-fun9-1}}}\plusendmoddef
void* mempool_geth(mempool_t pool, int size)
\{
    mempool_handle_t new_handle;
    char* mem;

    ASSERT_VALID_POOL;

    \LA{}initialize new handle from current pool state~{\nwtagstyle{}\subpageref{NWmemA-inij-1}}\RA{}
    \LA{}allocate memory and tuck handle into leading blocks~{\nwtagstyle{}\subpageref{NWmemA-allp-1}}\RA{}

    \LA{}log handle allocation~{\nwtagstyle{}\subpageref{NWmemA-logL-1}}\RA{}

    return mem;
\}

\nwendcode{}\nwbegindocs{56}\nwdocspar

\nwenddocs{}\nwbegincode{57}\sublabel{NWmemA-allp-1}\nwmargintag{{\nwtagstyle{}\subpageref{NWmemA-allp-1}}}\moddef{allocate memory and tuck handle into leading blocks~{\nwtagstyle{}\subpageref{NWmemA-allp-1}}}\endmoddef
mem = (char*) mempool_get(pool, size + MEMPOOL_HANDLE_SIZE);
*((mempool_handle_t*) mem) = new_handle;
mem += MEMPOOL_HANDLE_SIZE;
\nwused{\\{NWmemA-fun9-5}}\nwendcode{}\nwbegindocs{58}\nwdocspar

As in the basic allocation case, {\tt{}mempool{\char95}cgeth} is
a small convenience wrapper around {\tt{}mempool{\char95}geth}.

\nwenddocs{}\nwbegincode{59}\sublabel{NWmemA-fun9-6}\nwmargintag{{\nwtagstyle{}\subpageref{NWmemA-fun9-6}}}\moddef{functions~{\nwtagstyle{}\subpageref{NWmemA-fun9-1}}}\plusendmoddef
void* mempool_cgeth(mempool_t pool, int size)
\{
    void* mem = mempool_geth(pool, size);
    memset(mem, 0, size);
    return mem;
\}

\nwendcode{}\nwbegindocs{60}\nwdocspar

The whole point of allocating with a handle is that you
should be able to roll back the pool to the state recorded
in the handle.  The {\tt{}mempool{\char95}freeh} function performs
this rollback.

\nwenddocs{}\nwbegincode{61}\sublabel{NWmemA-fun9-7}\nwmargintag{{\nwtagstyle{}\subpageref{NWmemA-fun9-7}}}\moddef{functions~{\nwtagstyle{}\subpageref{NWmemA-fun9-1}}}\plusendmoddef
void mempool_freeh(void* mem)
\{
    mempool_handle_t handle;
    mempool_t pool;
    align_t* next_span;

    \LA{}extract and validate handle~{\nwtagstyle{}\subpageref{NWmemA-extR-1}}\RA{}
    \LA{}free spans from handle onward~{\nwtagstyle{}\subpageref{NWmemA-freT-1}}\RA{}
    \LA{}restore pool state~{\nwtagstyle{}\subpageref{NWmemA-resI-1}}\RA{}

    \LA{}optionally clear freed space~{\nwtagstyle{}\subpageref{NWmemA-optS-1}}\RA{}

    \LA{}log handle deallocation~{\nwtagstyle{}\subpageref{NWmemA-logN-1}}\RA{}
\}

\nwendcode{}\nwbegindocs{62}\nwdocspar

Remember that the handle is stored at {\tt{}mem\ -\ MEMPOOL{\char95}HANDLE{\char95}SIZE}.
We want a copy of the handle, rather than just a pointer to it,
since it may live in a span that will be freed during the rollback
process.

\nwenddocs{}\nwbegincode{63}\sublabel{NWmemA-extR-1}\nwmargintag{{\nwtagstyle{}\subpageref{NWmemA-extR-1}}}\moddef{extract and validate handle~{\nwtagstyle{}\subpageref{NWmemA-extR-1}}}\endmoddef
assert(mem != NULL);
handle = *(mempool_handle_t*) ((char*) mem - MEMPOOL_HANDLE_SIZE);
pool = handle.pool;
assert(handle.magic_number == MEMPOOL_HANDLE_MAGIC_NUMBER);
ASSERT_VALID_POOL;
\nwused{\\{NWmemA-fun9-7}}\nwendcode{}\nwbegindocs{64}\nwdocspar

\nwenddocs{}\nwbegincode{65}\sublabel{NWmemA-freT-1}\nwmargintag{{\nwtagstyle{}\subpageref{NWmemA-freT-1}}}\moddef{free spans from handle onward~{\nwtagstyle{}\subpageref{NWmemA-freT-1}}}\endmoddef
next_span = handle.current_span[0].p;
while (next_span != NULL) \{
    align_t* dead_span = next_span;
    next_span = next_span[0].p;
    \LA{}optionally clear dead span~{\nwtagstyle{}\subpageref{NWmemA-optQ-1}}\RA{}
    free(dead_span);
\}
\nwused{\\{NWmemA-fun9-7}}\nwendcode{}\nwbegindocs{66}\nwdocspar

\nwenddocs{}\nwbegincode{67}\sublabel{NWmemA-resI-1}\nwmargintag{{\nwtagstyle{}\subpageref{NWmemA-resI-1}}}\moddef{restore pool state~{\nwtagstyle{}\subpageref{NWmemA-resI-1}}}\endmoddef
pool->blocks_used  = handle.blocks_used;
pool->current_span = handle.current_span;
pool->current_span[0].p = NULL;
\nwused{\\{NWmemA-fun9-7}}\nwendcode{}\nwbegindocs{68}\nwdocspar

In addition to all the spans we might have freed,
there is probably space just freed at the end of the current
span after rollback.  If the {\tt{}MEMPOOL{\char95}CLEAR{\char95}DEBUG} flag is
set, we want to clear that, too.

\nwenddocs{}\nwbegincode{69}\sublabel{NWmemA-optS-1}\nwmargintag{{\nwtagstyle{}\subpageref{NWmemA-optS-1}}}\moddef{optionally clear freed space~{\nwtagstyle{}\subpageref{NWmemA-optS-1}}}\endmoddef
#ifdef MEMPOOL_CLEAR_DEBUG
memset(&(pool->current_span[1 + pool->blocks_used]), 0,
       (pool->blocks_per_span - pool->blocks_used) * align_size);
#endif
\nwused{\\{NWmemA-fun9-7}}\nwendcode{}\nwbegindocs{70}\nwdocspar


\subsection{String and buffer duplication}

The {\tt{}mempool{\char95}memdup} function consists only of a {\tt{}mempool{\char95}get}
and a {\tt{}memcpy}.

\nwenddocs{}\nwbegincode{71}\sublabel{NWmemA-fun9-8}\nwmargintag{{\nwtagstyle{}\subpageref{NWmemA-fun9-8}}}\moddef{functions~{\nwtagstyle{}\subpageref{NWmemA-fun9-1}}}\plusendmoddef
void* mempool_memdup(mempool_t pool, const void* src, int size)
\{
    if (src == NULL)
        return NULL;
    else \{
        void* dest = mempool_get(pool, size);
        memcpy(dest, src, size);
        return dest;
    \}
\}

\nwendcode{}\nwbegindocs{72}\nwdocspar

The {\tt{}mempool{\char95}strdup} function is a wrapper around {\tt{}mempool{\char95}memdup}.

\nwenddocs{}\nwbegincode{73}\sublabel{NWmemA-fun9-9}\nwmargintag{{\nwtagstyle{}\subpageref{NWmemA-fun9-9}}}\moddef{functions~{\nwtagstyle{}\subpageref{NWmemA-fun9-1}}}\plusendmoddef
void* mempool_strdup(mempool_t pool, const char* src)
\{
    if (src == NULL)
        return NULL;
    return mempool_memdup(pool, src, strlen(src) + 1);
\}

\nwendcode{}\nwbegindocs{74}\nwdocspar


\subsection{Logging}

The {\tt{}mempool{\char95}log{\char95}fp} pointer tracks the current log file.  By default,
logging is turned off, and {\tt{}mempool{\char95}log{\char95}fp} is {\tt{}NULL}.

\nwenddocs{}\nwbegincode{75}\sublabel{NWmemA-staB-1}\nwmargintag{{\nwtagstyle{}\subpageref{NWmemA-staB-1}}}\moddef{static data~{\nwtagstyle{}\subpageref{NWmemA-staB-1}}}\endmoddef
static FILE* mempool_log_fp = NULL;
\nwused{\\{NWmemA-mem9.2-1}}\nwendcode{}\nwbegindocs{76}\nwdocspar

We are only allowed to start a log if one is not already open,
and we can only end a log if there is a log to end.  Killing
the program with an assertion when the log can't be opened
may seem harsh, but remember that mempool logging will probably
only be used when debugging.

\nwenddocs{}\nwbegincode{77}\sublabel{NWmemA-fun9-A}\nwmargintag{{\nwtagstyle{}\subpageref{NWmemA-fun9-A}}}\moddef{functions~{\nwtagstyle{}\subpageref{NWmemA-fun9-1}}}\plusendmoddef
void mempool_start_log(char* filename)
\{
    assert(mempool_log_fp == NULL);
    mempool_log_fp = fopen(filename, "a");
    assert(mempool_log_fp != NULL);
    fprintf(mempool_log_fp, "Memory log started (%s)\\n", filename);
\}

\nwendcode{}\nwbegindocs{78}\nwdocspar

\nwenddocs{}\nwbegincode{79}\sublabel{NWmemA-fun9-B}\nwmargintag{{\nwtagstyle{}\subpageref{NWmemA-fun9-B}}}\moddef{functions~{\nwtagstyle{}\subpageref{NWmemA-fun9-1}}}\plusendmoddef
void mempool_end_log()
\{
    assert(mempool_log_fp != NULL);
    fprintf(mempool_log_fp, "Closing memory log\\n");
    fclose(mempool_log_fp);
    mempool_log_fp = NULL;
\}

\nwendcode{}\nwbegindocs{80}\nwdocspar

We may periodically want to ensure the mempool log has been flushed
to disk, particularly since we're most likely to use logging when
the program is crashing (and so not getting to the {\tt{}mempool{\char95}end{\char95}log}
call.

\nwenddocs{}\nwbegincode{81}\sublabel{NWmemA-fun9-C}\nwmargintag{{\nwtagstyle{}\subpageref{NWmemA-fun9-C}}}\moddef{functions~{\nwtagstyle{}\subpageref{NWmemA-fun9-1}}}\plusendmoddef
void mempool_flush_log()
\{
    assert(mempool_log_fp != NULL);
    fflush(mempool_log_fp);
\}

\nwendcode{}\nwbegindocs{82}\nwdocspar

Part of the reason that we don't associate logging with a particular
pool is that we would also like to log the creation and destruction
of pool objects.

\nwenddocs{}\nwbegincode{83}\sublabel{NWmemA-logH-1}\nwmargintag{{\nwtagstyle{}\subpageref{NWmemA-logH-1}}}\moddef{log pool creation~{\nwtagstyle{}\subpageref{NWmemA-logH-1}}}\endmoddef
if (mempool_log_fp)
    fprintf(mempool_log_fp, "Allocated new mempool (%p), span size %d\\n",
            (void*) pool, span_size);
\nwused{\\{NWmemA-fun9-1}}\nwendcode{}\nwbegindocs{84}\nwdocspar

\nwenddocs{}\nwbegincode{85}\sublabel{NWmemA-logH.2-1}\nwmargintag{{\nwtagstyle{}\subpageref{NWmemA-logH.2-1}}}\moddef{log pool deletion~{\nwtagstyle{}\subpageref{NWmemA-logH.2-1}}}\endmoddef
if (mempool_log_fp)
    fprintf(mempool_log_fp, "Destroyed mempool (%p)\\n", (void*) pool);
\nwused{\\{NWmemA-fun9-2}}\nwendcode{}\nwbegindocs{86}\nwdocspar

\nwenddocs{}\nwbegincode{87}\sublabel{NWmemA-logJ-1}\nwmargintag{{\nwtagstyle{}\subpageref{NWmemA-logJ-1}}}\moddef{log span allocation~{\nwtagstyle{}\subpageref{NWmemA-logJ-1}}}\endmoddef
if (mempool_log_fp)
    fprintf(mempool_log_fp, "%p: Created new span at %p\\n",
            (void*) pool, (void*) span);
\nwused{\\{NWmemA-staG-1}}\nwendcode{}\nwbegindocs{88}\nwdocspar

\nwenddocs{}\nwbegincode{89}\sublabel{NWmemA-logO-1}\nwmargintag{{\nwtagstyle{}\subpageref{NWmemA-logO-1}}}\moddef{log allocation from span~{\nwtagstyle{}\subpageref{NWmemA-logO-1}}}\endmoddef
if (mempool_log_fp)
    fprintf(mempool_log_fp,
            "%p: Allocated %d bytes / %d blocks at %p\\n",
            (void*) pool, size, blocks_requested * align_size,
            mem);
\nwused{\\{NWmemA-fun9-3}}\nwendcode{}\nwbegindocs{90}\nwdocspar

\nwenddocs{}\nwbegincode{91}\sublabel{NWmemA-logL-1}\nwmargintag{{\nwtagstyle{}\subpageref{NWmemA-logL-1}}}\moddef{log handle allocation~{\nwtagstyle{}\subpageref{NWmemA-logL-1}}}\endmoddef
if (mempool_log_fp)
    fprintf(mempool_log_fp,
            "%p: Handle at %p, memory at %p\\n",
            (void*) pool, (void*) (mem - MEMPOOL_HANDLE_SIZE), (void*) mem);
\nwused{\\{NWmemA-fun9-5}}\nwendcode{}\nwbegindocs{92}\nwdocspar

\nwenddocs{}\nwbegincode{93}\sublabel{NWmemA-logN-1}\nwmargintag{{\nwtagstyle{}\subpageref{NWmemA-logN-1}}}\moddef{log handle deallocation~{\nwtagstyle{}\subpageref{NWmemA-logN-1}}}\endmoddef
if (mempool_log_fp)
    fprintf(mempool_log_fp,
            "%p: Rollback to handle at %p\\n", (void*) pool,
            (void*) ((char*) mem - MEMPOOL_HANDLE_SIZE));
\nwused{\\{NWmemA-fun9-7}}\nwendcode{}

\nwixlogsorted{c}{{alignment definitions}{NWmemA-aliL-1}{\nwixu{NWmemA-expE-1}\nwixd{NWmemA-aliL-1}}}%
\nwixlogsorted{c}{{allocate a new span if necessary}{NWmemA-allW.2-1}{\nwixu{NWmemA-fun9-3}\nwixd{NWmemA-allW.2-1}}}%
\nwixlogsorted{c}{{allocate blocks from the current span}{NWmemA-allb-1}{\nwixu{NWmemA-fun9-3}\nwixd{NWmemA-allb-1}}}%
\nwixlogsorted{c}{{allocate memory and tuck handle into leading blocks}{NWmemA-allp-1}{\nwixu{NWmemA-fun9-5}\nwixd{NWmemA-allp-1}}}%
\nwixlogsorted{c}{{allocate memory for the new span}{NWmemA-allW-1}{\nwixu{NWmemA-staG-1}\nwixd{NWmemA-allW-1}}}%
\nwixlogsorted{c}{{delete all spans in the pool}{NWmemA-delS-1}{\nwixu{NWmemA-fun9-2}\nwixd{NWmemA-delS-1}}}%
\nwixlogsorted{c}{{exported functions}{NWmemA-expI-1}{\nwixu{NWmemA-mem9-1}\nwixd{NWmemA-expI-1}\nwixd{NWmemA-expI-2}\nwixd{NWmemA-expI-3}\nwixd{NWmemA-expI-4}\nwixd{NWmemA-expI-5}}}%
\nwixlogsorted{c}{{exported macros}{NWmemA-expF-1}{\nwixu{NWmemA-mem9-1}\nwixd{NWmemA-expF-1}}}%
\nwixlogsorted{c}{{exported types}{NWmemA-expE-1}{\nwixu{NWmemA-mem9-1}\nwixd{NWmemA-expE-1}}}%
\nwixlogsorted{c}{{extract and validate handle}{NWmemA-extR-1}{\nwixu{NWmemA-fun9-7}\nwixd{NWmemA-extR-1}}}%
\nwixlogsorted{c}{{free spans from handle onward}{NWmemA-freT-1}{\nwixu{NWmemA-fun9-7}\nwixd{NWmemA-freT-1}}}%
\nwixlogsorted{c}{{functions}{NWmemA-fun9-1}{\nwixu{NWmemA-mem9.2-1}\nwixd{NWmemA-fun9-1}\nwixd{NWmemA-fun9-2}\nwixd{NWmemA-fun9-3}\nwixd{NWmemA-fun9-4}\nwixd{NWmemA-fun9-5}\nwixd{NWmemA-fun9-6}\nwixd{NWmemA-fun9-7}\nwixd{NWmemA-fun9-8}\nwixd{NWmemA-fun9-9}\nwixd{NWmemA-fun9-A}\nwixd{NWmemA-fun9-B}\nwixd{NWmemA-fun9-C}}}%
\nwixlogsorted{c}{{initialize new handle from current pool state}{NWmemA-inij-1}{\nwixd{NWmemA-inij-1}\nwixu{NWmemA-fun9-5}}}%
\nwixlogsorted{c}{{initialize new pool}{NWmemA-iniJ.2-1}{\nwixu{NWmemA-fun9-1}\nwixd{NWmemA-iniJ.2-1}}}%
\nwixlogsorted{c}{{initialize the span}{NWmemA-iniJ-1}{\nwixu{NWmemA-staG-1}\nwixd{NWmemA-iniJ-1}}}%
\nwixlogsorted{c}{{link in new span}{NWmemA-linG-1}{\nwixu{NWmemA-allW.2-1}\nwixd{NWmemA-linG-1}}}%
\nwixlogsorted{c}{{log allocation from span}{NWmemA-logO-1}{\nwixu{NWmemA-fun9-3}\nwixd{NWmemA-logO-1}}}%
\nwixlogsorted{c}{{log handle allocation}{NWmemA-logL-1}{\nwixu{NWmemA-fun9-5}\nwixd{NWmemA-logL-1}}}%
\nwixlogsorted{c}{{log handle deallocation}{NWmemA-logN-1}{\nwixu{NWmemA-fun9-7}\nwixd{NWmemA-logN-1}}}%
\nwixlogsorted{c}{{log pool creation}{NWmemA-logH-1}{\nwixu{NWmemA-fun9-1}\nwixd{NWmemA-logH-1}}}%
\nwixlogsorted{c}{{log pool deletion}{NWmemA-logH.2-1}{\nwixu{NWmemA-fun9-2}\nwixd{NWmemA-logH.2-1}}}%
\nwixlogsorted{c}{{log span allocation}{NWmemA-logJ-1}{\nwixu{NWmemA-staG-1}\nwixd{NWmemA-logJ-1}}}%
\nwixlogsorted{c}{{mempool.c}{NWmemA-mem9.2-1}{\nwixd{NWmemA-mem9.2-1}}}%
\nwixlogsorted{c}{{mempool.h}{NWmemA-mem9-1}{\nwixd{NWmemA-mem9-1}}}%
\nwixlogsorted{c}{{optionally clear dead span}{NWmemA-optQ-1}{\nwixu{NWmemA-delS-1}\nwixd{NWmemA-optQ-1}\nwixu{NWmemA-freT-1}}}%
\nwixlogsorted{c}{{optionally clear freed space}{NWmemA-optS-1}{\nwixu{NWmemA-fun9-7}\nwixd{NWmemA-optS-1}}}%
\nwixlogsorted{c}{{optionally clear the new span}{NWmemA-optT-1}{\nwixu{NWmemA-staG-1}\nwixd{NWmemA-optT-1}}}%
\nwixlogsorted{c}{{restore pool state}{NWmemA-resI-1}{\nwixu{NWmemA-fun9-7}\nwixd{NWmemA-resI-1}}}%
\nwixlogsorted{c}{{static data}{NWmemA-staB-1}{\nwixu{NWmemA-mem9.2-1}\nwixd{NWmemA-staB-1}}}%
\nwixlogsorted{c}{{static functions}{NWmemA-staG-1}{\nwixu{NWmemA-mem9.2-1}\nwixd{NWmemA-staG-1}}}%
\nwixlogsorted{c}{{types}{NWmemA-typ5-1}{\nwixu{NWmemA-mem9.2-1}\nwixd{NWmemA-typ5-1}\nwixd{NWmemA-typ5-2}}}%
\nwbegindocs{94}\nwdocspar

\nwenddocs{}
