
\section{Introduction}

The mesh module describes and manipulates the basic mesh data
structure.  This module is only responsible for keeping track
of and providing access to the data structures, and does
\emph{not} handle any particularly sophisticated functions.

There are three main users of this module:
\begin{itemize}
 \item Mesh generation routines
 \item Model initialization routines
 \item Routines to process the constructed mesh
\end{itemize}


\section{Interface}

\nwfilename{mesh.nw}\nwbegincode{1}\sublabel{NWmes7-mes6-1}\nwmargintag{{\nwtagstyle{}\subpageref{NWmes7-mes6-1}}}\moddef{mesh.h~{\nwtagstyle{}\subpageref{NWmes7-mes6-1}}}\endmoddef
#ifndef MESH_H
#define MESH_H

#include "mempool.h"
#include "modelmgr.h"

typedef struct mesh_struct* mesh_t;

\LA{}exported types~{\nwtagstyle{}\subpageref{NWmes7-expE-1}}\RA{}
\LA{}exported functions~{\nwtagstyle{}\subpageref{NWmes7-expI-1}}\RA{}

#endif /* MESH_H */
\nwnotused{mesh.h}\nwendcode{}\nwbegindocs{2}\nwdocspar


\subsection{Types}

A node in the mesh is described by a {\tt{}mesh{\char95}node{\char95}t} structure.
Nodes have names and positions, but that's about it.  A node may %'
be anonymous (when the {\tt{}name} field is NULL).  There may also
be nodes whose position is unimportant.  In that case, the default
location of the node is at the origin.

\nwenddocs{}\nwbegincode{3}\sublabel{NWmes7-expE-1}\nwmargintag{{\nwtagstyle{}\subpageref{NWmes7-expE-1}}}\moddef{exported types~{\nwtagstyle{}\subpageref{NWmes7-expE-1}}}\endmoddef
typedef struct mesh_node_t \{
    const char* name;
    double x[3];
\} mesh_node_t;

\nwalsodefined{\\{NWmes7-expE-2}}\nwused{\\{NWmes7-mes6-1}}\nwendcode{}\nwbegindocs{4}\nwdocspar

In C++, ``material'' and ``element'' would be abstract base classes.
In Java, they would probably be interfaces.  In C, we require a bit
more typing to achieve the same effect.  Material and element models are
defined by function tables (see {\tt{}modelmgr} for details).  Particular
instances of materials and elements are defined by a function table plus
some data for the particular instance.  The semantics of the data
describing each instance is known only to the model writer.

In addition to the basic mesh object types, we define one more type
for convenience: a tagged union type with a label to represent
named parameters.  Types are characters strings ({\tt{}type\ ==\ 's'});
numbers ({\tt{}type\ ==\ 'd'}); and matrices ({\tt{}type\ ==\ 'm'}).

\nwenddocs{}\nwbegincode{5}\sublabel{NWmes7-expE-2}\nwmargintag{{\nwtagstyle{}\subpageref{NWmes7-expE-2}}}\moddef{exported types~{\nwtagstyle{}\subpageref{NWmes7-expE-1}}}\plusendmoddef
enum \{
    MESH_PARAM_NUMBER,
    MESH_PARAM_STRING,
    MESH_PARAM_MATRIX
\};

typedef struct mesh_param_t \{
    int tag;
    const char* name;
    union \{
        double d;
        const char* s;
        struct \{
            int m, n;
            double* data;
        \} m;
    \} val;
\} mesh_param_t;

\nwendcode{}\nwbegindocs{6}\nwdocspar


\subsection{Functions}

The {\tt{}mesh{\char95}create} and {\tt{}mesh{\char95}destroy} functions respectively
create and destroy mesh objects.
The {\tt{}mesh{\char95}create} function takes a pointer to a model manager
(see the {\tt{}modelmgr} file).

\nwenddocs{}\nwbegincode{7}\sublabel{NWmes7-expI-1}\nwmargintag{{\nwtagstyle{}\subpageref{NWmes7-expI-1}}}\moddef{exported functions~{\nwtagstyle{}\subpageref{NWmes7-expI-1}}}\endmoddef
mesh_t mesh_create(model_mgr_t model_mgr);
void   mesh_destroy(mesh_t self);

\nwalsodefined{\\{NWmes7-expI-2}\\{NWmes7-expI-3}\\{NWmes7-expI-4}\\{NWmes7-expI-5}\\{NWmes7-expI-6}\\{NWmes7-expI-7}\\{NWmes7-expI-8}\\{NWmes7-expI-9}\\{NWmes7-expI-A}\\{NWmes7-expI-B}\\{NWmes7-expI-C}}\nwused{\\{NWmes7-mes6-1}}\nwendcode{}\nwbegindocs{8}\nwdocspar

Each mesh is allocated on a memory pool, which can be accessed via
the {\tt{}mesh{\char95}pool} function.  Model functions should allocate memory
for local storage from this pool.

\nwenddocs{}\nwbegincode{9}\sublabel{NWmes7-expI-2}\nwmargintag{{\nwtagstyle{}\subpageref{NWmes7-expI-2}}}\moddef{exported functions~{\nwtagstyle{}\subpageref{NWmes7-expI-1}}}\plusendmoddef
mempool_t mesh_pool(mesh_t self);

\nwendcode{}\nwbegindocs{10}\nwdocspar

A mesh can have associated handlers for errors and warnings.
A handler consists of a function pointer and a pointer
to data needed by the handler.  The default handler for a
warning simply prints an error message to the standard output;
the default error handler is simpler, but it additionally
aborts the program.  The default behavior can be stored by
passing in a NULL function.

\nwenddocs{}\nwbegincode{11}\sublabel{NWmes7-expI-3}\nwmargintag{{\nwtagstyle{}\subpageref{NWmes7-expI-3}}}\moddef{exported functions~{\nwtagstyle{}\subpageref{NWmes7-expI-1}}}\plusendmoddef
void mesh_set_handlers(mesh_t self,
                       void (*error_handler)  (void* arg, const char* msg),
                       void (*warning_handler)(void* arg, const char* msg),
                       void* arg);
void mesh_error       (mesh_t self, const char* msg);
void mesh_warning     (mesh_t self, const char* msg);


\nwendcode{}\nwbegindocs{12}\nwdocspar

The {\tt{}mesh{\char95}add{\char95}param{\char95}string} and {\tt{}mesh{\char95}add{\char95}param{\char95}number} functions
add named string and number parameters to the list of parameters
used by the next material or element.  The parameter list is
cleared after any call to {\tt{}mesh{\char95}add{\char95}element} or {\tt{}mesh{\char95}add{\char95}material}.

The {\tt{}mesh{\char95}add{\char95}param{\char95}node} adds to the list of nodes used by the next
element.  The node list is cleared after any call to {\tt{}mesh{\char95}add{\char95}element}.

\nwenddocs{}\nwbegincode{13}\sublabel{NWmes7-expI-4}\nwmargintag{{\nwtagstyle{}\subpageref{NWmes7-expI-4}}}\moddef{exported functions~{\nwtagstyle{}\subpageref{NWmes7-expI-1}}}\plusendmoddef
void mesh_add_param_string(mesh_t self, const char* name, const char* s);
void mesh_add_param_number(mesh_t self, const char* name, double d);
void mesh_add_param_matrix(mesh_t self, const char* name, int m, int n,
                           double* data);
void mesh_add_param_node  (mesh_t self, int node_id);

\nwendcode{}\nwbegindocs{14}\nwdocspar

The current list of named parameters and node identifiers can
later be accessed through {\tt{}mesh{\char95}param}, {\tt{}mesh{\char95}param{\char95}byname},
and {\tt{}mesh{\char95}param{\char95}node}.  {\tt{}mesh{\char95}num{\char95}params} and {\tt{}mesh{\char95}num{\char95}param{\char95}nodes}
will give the number of named parameters and nodes, respectively.
The arrays of named parameters and element nodes use zero-based
indexing.  The functions to access the current parameter lists will
primarily be of interest to model function writers.

\nwenddocs{}\nwbegincode{15}\sublabel{NWmes7-expI-5}\nwmargintag{{\nwtagstyle{}\subpageref{NWmes7-expI-5}}}\moddef{exported functions~{\nwtagstyle{}\subpageref{NWmes7-expI-1}}}\plusendmoddef
mesh_param_t* mesh_param       (mesh_t self, int index);
mesh_param_t* mesh_param_byname(mesh_t self, const char* name);
int           mesh_param_node  (mesh_t self, int index);

\nwendcode{}\nwbegindocs{16}\nwdocspar

\nwenddocs{}\nwbegincode{17}\sublabel{NWmes7-expI-6}\nwmargintag{{\nwtagstyle{}\subpageref{NWmes7-expI-6}}}\moddef{exported functions~{\nwtagstyle{}\subpageref{NWmes7-expI-1}}}\plusendmoddef
int mesh_num_params     (mesh_t self);
int mesh_num_param_nodes(mesh_t self);

\nwendcode{}\nwbegindocs{18}\nwdocspar

After a mesh generation error, we may be left with unprocessed
arguments not cleared from the stack.  The current solution is
the {\tt{}mesh{\char95}params{\char95}clear} routine; a better future solution might
be to separate the argument stack handler into its own module
and independent data structure.

\nwenddocs{}\nwbegincode{19}\sublabel{NWmes7-expI-7}\nwmargintag{{\nwtagstyle{}\subpageref{NWmes7-expI-7}}}\moddef{exported functions~{\nwtagstyle{}\subpageref{NWmes7-expI-1}}}\plusendmoddef
void mesh_params_clear(mesh_t self);

\nwendcode{}\nwbegindocs{20}\nwdocspar

Often, we know what type the parameters \emph{should} be,
and it is an error if we see anything else.  The
{\tt{}mesh{\char95}param{\char95}string}, {\tt{}mesh{\char95}param{\char95}number}, and
{\tt{}mesh{\char95}param{\char95}int} functions report a ``parameter missing''
error if the requested parameter is unavailable and a
``wrong type'' error if the parameter is invalid.  If
the parameter is legitimate, they return the value.

The {\tt{}mesh{\char95}param{\char95}material} function takes a material identifier,
checks its validity (is it an in-range integer?) and returns a
pointer to the material.  Note that the pointer is \emph{only} valid
until a new material is added, since the data structure in which
the materials are stored may be shifted around dynamically.

The mesh argument is provided only for its error handling
routines.

\nwenddocs{}\nwbegincode{21}\sublabel{NWmes7-expI-8}\nwmargintag{{\nwtagstyle{}\subpageref{NWmes7-expI-8}}}\moddef{exported functions~{\nwtagstyle{}\subpageref{NWmes7-expI-1}}}\plusendmoddef
const char*      mesh_param_string  (mesh_t self, mesh_param_t* param);
double           mesh_param_number  (mesh_t self, mesh_param_t* param);
int              mesh_param_int     (mesh_t self, mesh_param_t* param);
material_t*      mesh_param_material(mesh_t self, mesh_param_t* param);

\nwendcode{}\nwbegindocs{22}\nwdocspar

The {\tt{}mesh{\char95}add{\char95}node} function adds a node to the global list.
The return value is the identifier for that node.
The {\tt{}mesh{\char95}add{\char95}element} and {\tt{}mesh{\char95}add{\char95}material} add to the list of
elements and materials used in the mesh.  Both functions return
identifiers for the newly created objects.

\nwenddocs{}\nwbegincode{23}\sublabel{NWmes7-expI-9}\nwmargintag{{\nwtagstyle{}\subpageref{NWmes7-expI-9}}}\moddef{exported functions~{\nwtagstyle{}\subpageref{NWmes7-expI-1}}}\plusendmoddef
int mesh_add_node    (mesh_t self, const char* name, 
                      double x, double y, double z);
int mesh_add_element (mesh_t self, const char* model);
int mesh_add_material(mesh_t self, const char* model);

\nwendcode{}\nwbegindocs{24}\nwdocspar

The {\tt{}mesh{\char95}node}, {\tt{}mesh{\char95}element}, and {\tt{}mesh{\char95}material} functions
retrieve the nodes, elements, and material objects by their
identifiers.  Note that the pointers returned are \emph{not} guaranteed
to remain valid as new entries are added into the mesh.

\nwenddocs{}\nwbegincode{25}\sublabel{NWmes7-expI-A}\nwmargintag{{\nwtagstyle{}\subpageref{NWmes7-expI-A}}}\moddef{exported functions~{\nwtagstyle{}\subpageref{NWmes7-expI-1}}}\plusendmoddef
mesh_node_t*  mesh_node    (mesh_t self, int id);
element_t*    mesh_element (mesh_t self, int id);
material_t*   mesh_material(mesh_t self, int id);

\nwendcode{}\nwbegindocs{26}\nwdocspar

The {\tt{}mesh{\char95}num{\char95}nodes}, {\tt{}mesh{\char95}num{\char95}elements}, and {\tt{}mesh{\char95}num{\char95}materials}
functions return the number of nodes, elements, and materials present
in the mesh.

\nwenddocs{}\nwbegincode{27}\sublabel{NWmes7-expI-B}\nwmargintag{{\nwtagstyle{}\subpageref{NWmes7-expI-B}}}\moddef{exported functions~{\nwtagstyle{}\subpageref{NWmes7-expI-1}}}\plusendmoddef
int mesh_num_nodes    (mesh_t self);
int mesh_num_elements (mesh_t self);
int mesh_num_materials(mesh_t self);

\nwendcode{}\nwbegindocs{28}\nwdocspar

The {\tt{}mesh{\char95}vars{\char95}mgr} and {\tt{}mesh{\char95}assembler} getter functions return
the variable manager and assembler objects owned by the mesh.

\nwenddocs{}\nwbegincode{29}\sublabel{NWmes7-expI-C}\nwmargintag{{\nwtagstyle{}\subpageref{NWmes7-expI-C}}}\moddef{exported functions~{\nwtagstyle{}\subpageref{NWmes7-expI-1}}}\plusendmoddef
struct vars_mgr_struct* mesh_vars_mgr(mesh_t self);
struct assembler_struct* mesh_assembler(mesh_t self);
\nwendcode{}\nwbegindocs{30}\nwdocspar


\section{Implementation}

\nwenddocs{}\nwbegincode{31}\sublabel{NWmes7-mes6.2-1}\nwmargintag{{\nwtagstyle{}\subpageref{NWmes7-mes6.2-1}}}\moddef{mesh.c~{\nwtagstyle{}\subpageref{NWmes7-mes6.2-1}}}\endmoddef
#include <stdio.h>
#include <stdlib.h>
#include <string.h>
#include <assert.h>

#include "mempool.h"
#include "hash.h"
#include "mesh.h"
#include "vars.h"
#include "assemble.h"
#include "dynarray.h"

\LA{}types~{\nwtagstyle{}\subpageref{NWmes7-typ5-1}}\RA{}
\LA{}constants~{\nwtagstyle{}\subpageref{NWmes7-con9-1}}\RA{}
\LA{}static functions~{\nwtagstyle{}\subpageref{NWmes7-staG-1}}\RA{}
\LA{}functions~{\nwtagstyle{}\subpageref{NWmes7-fun9-1}}\RA{}
\nwnotused{mesh.c}\nwendcode{}\nwbegindocs{32}\nwdocspar

\nwenddocs{}\nwbegincode{33}\sublabel{NWmes7-typ5-1}\nwmargintag{{\nwtagstyle{}\subpageref{NWmes7-typ5-1}}}\moddef{types~{\nwtagstyle{}\subpageref{NWmes7-typ5-1}}}\endmoddef
struct mesh_struct \{
    \LA{}\code{}mesh{\char95}t\edoc{} data~{\nwtagstyle{}\subpageref{NWmes7-**mF-1}}\RA{}
\};

\nwused{\\{NWmes7-mes6.2-1}}\nwendcode{}\nwbegindocs{34}\nwdocspar


\subsection{Mesh data structure}

The main point of the mesh object is to track the nodes, elements,
and materials that make up the device description.

\nwenddocs{}\nwbegincode{35}\sublabel{NWmes7-**mF-1}\nwmargintag{{\nwtagstyle{}\subpageref{NWmes7-**mF-1}}}\moddef{\code{}mesh{\char95}t\edoc{} data~{\nwtagstyle{}\subpageref{NWmes7-**mF-1}}}\endmoddef
dynarray_t nodes;
dynarray_t elements;
dynarray_t materials;
\nwalsodefined{\\{NWmes7-**mF-2}\\{NWmes7-**mF-3}\\{NWmes7-**mF-4}\\{NWmes7-**mF-5}\\{NWmes7-**mF-6}}\nwused{\\{NWmes7-typ5-1}}\nwendcode{}\nwbegindocs{36}\nwdocspar

The {\tt{}param{\char95}nodes} and {\tt{}params} entries are used as storage
for the list of node identifiers and named parameters to be passed
to an element or a material.

\nwenddocs{}\nwbegincode{37}\sublabel{NWmes7-**mF-2}\nwmargintag{{\nwtagstyle{}\subpageref{NWmes7-**mF-2}}}\moddef{\code{}mesh{\char95}t\edoc{} data~{\nwtagstyle{}\subpageref{NWmes7-**mF-1}}}\plusendmoddef
dynarray_t param_nodes;
dynarray_t params;
\nwendcode{}\nwbegindocs{38}\nwdocspar

We keep around a pool object to give the models a place to allocate
their local data.  We also keep around a string tables so that we
don't have to store ``Youngsmodulus'' more than once. %'

\nwenddocs{}\nwbegincode{39}\sublabel{NWmes7-**mF-3}\nwmargintag{{\nwtagstyle{}\subpageref{NWmes7-**mF-3}}}\moddef{\code{}mesh{\char95}t\edoc{} data~{\nwtagstyle{}\subpageref{NWmes7-**mF-1}}}\plusendmoddef
mempool_t pool;
hash_t strings;
\nwendcode{}\nwbegindocs{40}\nwdocspar

\nwenddocs{}\nwbegincode{41}\sublabel{NWmes7-**mF-4}\nwmargintag{{\nwtagstyle{}\subpageref{NWmes7-**mF-4}}}\moddef{\code{}mesh{\char95}t\edoc{} data~{\nwtagstyle{}\subpageref{NWmes7-**mF-1}}}\plusendmoddef
model_mgr_t model_mgr;
\nwendcode{}\nwbegindocs{42}\nwdocspar

The information for the current error handler consists of three
variables: a function for reporting errors, another for reporting 
warnings, and an opaque data pointer so that the error handlers
have something to do when they report.

\nwenddocs{}\nwbegincode{43}\sublabel{NWmes7-**mF-5}\nwmargintag{{\nwtagstyle{}\subpageref{NWmes7-**mF-5}}}\moddef{\code{}mesh{\char95}t\edoc{} data~{\nwtagstyle{}\subpageref{NWmes7-**mF-1}}}\plusendmoddef
void (*error_handler)(void* arg, const char* msg);
void (*warning_handler)(void* arg, const char* msg);
void* handler_arg;
\nwendcode{}\nwbegindocs{44}\nwdocspar

Variable index assignment is handled by a separate object (a
{\tt{}vars{\char95}mgr{\char95}t}), but we keep a pointer to that object with the
mesh information.  We can access the object with a simple
getter function.

\nwenddocs{}\nwbegincode{45}\sublabel{NWmes7-**mF-6}\nwmargintag{{\nwtagstyle{}\subpageref{NWmes7-**mF-6}}}\moddef{\code{}mesh{\char95}t\edoc{} data~{\nwtagstyle{}\subpageref{NWmes7-**mF-1}}}\plusendmoddef
vars_mgr_t vars_mgr;
assembler_t assembler;
\nwendcode{}\nwbegindocs{46}\nwdocspar

\nwenddocs{}\nwbegincode{47}\sublabel{NWmes7-fun9-1}\nwmargintag{{\nwtagstyle{}\subpageref{NWmes7-fun9-1}}}\moddef{functions~{\nwtagstyle{}\subpageref{NWmes7-fun9-1}}}\endmoddef
vars_mgr_t mesh_vars_mgr(mesh_t self)
\{
    return self->vars_mgr;
\}

assembler_t mesh_assembler(mesh_t self)
\{
    return self->assembler;
\}

\nwalsodefined{\\{NWmes7-fun9-2}\\{NWmes7-fun9-3}\\{NWmes7-fun9-4}\\{NWmes7-fun9-5}\\{NWmes7-fun9-6}\\{NWmes7-fun9-7}\\{NWmes7-fun9-8}\\{NWmes7-fun9-9}\\{NWmes7-fun9-A}\\{NWmes7-fun9-B}\\{NWmes7-fun9-C}\\{NWmes7-fun9-D}\\{NWmes7-fun9-E}\\{NWmes7-fun9-F}\\{NWmes7-fun9-G}\\{NWmes7-fun9-H}\\{NWmes7-fun9-I}\\{NWmes7-fun9-J}\\{NWmes7-fun9-K}\\{NWmes7-fun9-L}\\{NWmes7-fun9-M}\\{NWmes7-fun9-N}}\nwused{\\{NWmes7-mes6.2-1}}\nwendcode{}\nwbegindocs{48}\nwdocspar


The default sizes of the various tables and initial array capacities
are off the top of my head.  A mesh with 128 nodes and elements sounds
like a reasonable size for something small; ditto with 16 materials.
And we probably won't have many single elements with nearly 27 %'
nodes (the number of nodes in a 3-d second-order Lagrangian brick),
nor anything with 64 parameters.  Note, though, that the data structures
we're using will work fine even if something larger than the specified %'
capacity is used.  They will just have to grow.

\nwenddocs{}\nwbegincode{49}\sublabel{NWmes7-con9-1}\nwmargintag{{\nwtagstyle{}\subpageref{NWmes7-con9-1}}}\moddef{constants~{\nwtagstyle{}\subpageref{NWmes7-con9-1}}}\endmoddef
#define NODES_SIZ      128
#define ELEMENT_SIZ    128
#define MATERIAL_SIZ   16
#define ELT_NODES_SIZ  27
#define PARAMS_SIZ     64
#define STRINGS_SIZ    500

\nwused{\\{NWmes7-mes6.2-1}}\nwendcode{}\nwbegindocs{50}\nwdocspar

\nwenddocs{}\nwbegincode{51}\sublabel{NWmes7-fun9-2}\nwmargintag{{\nwtagstyle{}\subpageref{NWmes7-fun9-2}}}\moddef{functions~{\nwtagstyle{}\subpageref{NWmes7-fun9-1}}}\plusendmoddef
mesh_t mesh_create(model_mgr_t model_mgr)
\{
    mempool_t pool = mempool_create(MEMPOOL_DEFAULT_SPAN);
    mesh_t    self = (mesh_t) mempool_get(pool, sizeof(struct mesh_struct));

    self->nodes     = dynarray_create(sizeof(mesh_node_t), NODES_SIZ);
    self->elements  = dynarray_create(sizeof(element_t*),  ELEMENT_SIZ);
    self->materials = dynarray_create(sizeof(material_t*), MATERIAL_SIZ);

    self->param_nodes = dynarray_create(sizeof(int), ELT_NODES_SIZ);
    self->params = dynarray_create(sizeof(mesh_param_t), PARAMS_SIZ);

    self->pool      = pool;
    self->strings   = hstrings_pcreate(pool, STRINGS_SIZ);
    self->model_mgr = model_mgr;

    mesh_set_handlers(self, NULL, NULL, NULL);

    self->vars_mgr  = vars_create(self);
    self->assembler = assemble_create(self);

    return self;
\}

\nwendcode{}\nwbegindocs{52}\nwdocspar

\nwenddocs{}\nwbegincode{53}\sublabel{NWmes7-fun9-3}\nwmargintag{{\nwtagstyle{}\subpageref{NWmes7-fun9-3}}}\moddef{functions~{\nwtagstyle{}\subpageref{NWmes7-fun9-1}}}\plusendmoddef
void mesh_destroy(mesh_t self)
\{
    int i, n;

    assemble_destroy(self->assembler);
    vars_destroy(self->vars_mgr);

    \LA{}destroy element data~{\nwtagstyle{}\subpageref{NWmes7-desK-1}}\RA{}

    \LA{}destroy material data~{\nwtagstyle{}\subpageref{NWmes7-desL-1}}\RA{}

    dynarray_destroy(self->params);
    dynarray_destroy(self->param_nodes);

    dynarray_destroy(self->materials);
    dynarray_destroy(self->elements);
    dynarray_destroy(self->nodes);

    mempool_destroy(self->pool);
\}

\nwendcode{}\nwbegindocs{54}\nwdocspar

\nwenddocs{}\nwbegincode{55}\sublabel{NWmes7-desK-1}\nwmargintag{{\nwtagstyle{}\subpageref{NWmes7-desK-1}}}\moddef{destroy element data~{\nwtagstyle{}\subpageref{NWmes7-desK-1}}}\endmoddef
n = mesh_num_elements(self);
for (i = 1; i <= n; ++i)
    element_destroy( mesh_element(self, i) );
\nwused{\\{NWmes7-fun9-3}}\nwendcode{}\nwbegindocs{56}\nwdocspar

\nwenddocs{}\nwbegincode{57}\sublabel{NWmes7-desL-1}\nwmargintag{{\nwtagstyle{}\subpageref{NWmes7-desL-1}}}\moddef{destroy material data~{\nwtagstyle{}\subpageref{NWmes7-desL-1}}}\endmoddef
n = mesh_num_materials(self);
for (i = 1; i <= n; ++i)
    material_destroy( mesh_material(self, i) );
\nwused{\\{NWmes7-fun9-3}}\nwendcode{}\nwbegindocs{58}\nwdocspar

\nwenddocs{}\nwbegincode{59}\sublabel{NWmes7-fun9-4}\nwmargintag{{\nwtagstyle{}\subpageref{NWmes7-fun9-4}}}\moddef{functions~{\nwtagstyle{}\subpageref{NWmes7-fun9-1}}}\plusendmoddef
mempool_t mesh_pool(mesh_t self)
\{
    return self->pool;
\}

\nwendcode{}\nwbegindocs{60}\nwdocspar


\subsection{Error handlers}

The only really noteworthy thing about setting the handlers is the
case when a NULL argument is passed in to indicate that we want
to use the default handler.

\nwenddocs{}\nwbegincode{61}\sublabel{NWmes7-fun9-5}\nwmargintag{{\nwtagstyle{}\subpageref{NWmes7-fun9-5}}}\moddef{functions~{\nwtagstyle{}\subpageref{NWmes7-fun9-1}}}\plusendmoddef
void mesh_set_handlers(mesh_t self,
                       void (*error_handler)(void* arg, const char* msg),
                       void (*warning_handler)(void* arg, const char* msg),
                       void* arg)
\{
    self->error_handler = 
        (error_handler == NULL) ? default_error_handler : error_handler;
    self->warning_handler = 
        (error_handler == NULL) ? default_warning_handler : warning_handler;
    self->handler_arg = arg;
\}

\nwendcode{}\nwbegindocs{62}\nwdocspar

Generating a warning or an error is simply a matter of calling the
current handler.

\nwenddocs{}\nwbegincode{63}\sublabel{NWmes7-fun9-6}\nwmargintag{{\nwtagstyle{}\subpageref{NWmes7-fun9-6}}}\moddef{functions~{\nwtagstyle{}\subpageref{NWmes7-fun9-1}}}\plusendmoddef
void mesh_error(mesh_t self, const char* msg)
\{
    (self->error_handler)(self->handler_arg, msg);
\}

void mesh_warning(mesh_t self, const char* msg)
\{
    (self->warning_handler)(self->handler_arg, msg);
\}

\nwendcode{}\nwbegindocs{64}\nwdocspar

By default, we print the error or warning message to the standard
error stream, then abort on error or return on warning.

\nwenddocs{}\nwbegincode{65}\sublabel{NWmes7-staG-1}\nwmargintag{{\nwtagstyle{}\subpageref{NWmes7-staG-1}}}\moddef{static functions~{\nwtagstyle{}\subpageref{NWmes7-staG-1}}}\endmoddef
static void default_error_handler(void* arg, const char* msg)
\{
    fprintf(stderr, "%s\\n", msg);
    abort();
\}

static void default_warning_handler(void* arg, const char* msg)
\{
    fprintf(stderr, "%s\\n", msg);
\}

\nwused{\\{NWmes7-mes6.2-1}}\nwendcode{}\nwbegindocs{66}\nwdocspar


\subsection{Parameters}

The {\tt{}mesh{\char95}add{\char95}param{\char95}string} and {\tt{}mesh{\char95}add{\char95}param{\char95}number} functions simply
append new parameters to the {\tt{}params} array.  Note that any new strings
that we might need to keep around (like names of parameters or string
parameter values) are immediately copied into the string table.

\nwenddocs{}\nwbegincode{67}\sublabel{NWmes7-fun9-7}\nwmargintag{{\nwtagstyle{}\subpageref{NWmes7-fun9-7}}}\moddef{functions~{\nwtagstyle{}\subpageref{NWmes7-fun9-1}}}\plusendmoddef
void mesh_add_param_number(mesh_t self, const char* name, double d)
\{
    mesh_param_t param;
    param.name  = hstrings_add(self->strings, name);
    param.tag   = MESH_PARAM_NUMBER;
    param.val.d = d;
    dynarray_append(self->params, &param);
\}

\nwendcode{}\nwbegindocs{68}\nwdocspar

\nwenddocs{}\nwbegincode{69}\sublabel{NWmes7-fun9-8}\nwmargintag{{\nwtagstyle{}\subpageref{NWmes7-fun9-8}}}\moddef{functions~{\nwtagstyle{}\subpageref{NWmes7-fun9-1}}}\plusendmoddef
void mesh_add_param_string(mesh_t self, const char* name, const char* s)
\{
    mesh_param_t param;
    param.name  = hstrings_add(self->strings, name);
    param.tag   = MESH_PARAM_STRING;
    param.val.s = hstrings_add(self->strings, s);
    dynarray_append(self->params, &param);
\}

\nwendcode{}\nwbegindocs{70}\nwdocspar

\nwenddocs{}\nwbegincode{71}\sublabel{NWmes7-fun9-9}\nwmargintag{{\nwtagstyle{}\subpageref{NWmes7-fun9-9}}}\moddef{functions~{\nwtagstyle{}\subpageref{NWmes7-fun9-1}}}\plusendmoddef
void mesh_add_param_matrix(mesh_t self, const char* name, int m, int n,
                           double* data)
\{
    mesh_param_t param;
    param.name = hstrings_add(self->strings, name);
    param.tag  = MESH_PARAM_MATRIX;
    param.val.m.m = m;
    param.val.m.n = n;
    param.val.m.data = mempool_memdup(self->pool, data, m*n * sizeof(double));
    dynarray_append(self->params, &param);
\}

\nwendcode{}\nwbegindocs{72}\nwdocspar

The {\tt{}mesh{\char95}add{\char95}param{\char95}node} function is very similar in nature to the
other {\tt{}mesh{\char95}add{\char95}param} functions, but it addes nodes rather than named
parameters.  The order in which nodes are added matters, obviously.

\nwenddocs{}\nwbegincode{73}\sublabel{NWmes7-fun9-A}\nwmargintag{{\nwtagstyle{}\subpageref{NWmes7-fun9-A}}}\moddef{functions~{\nwtagstyle{}\subpageref{NWmes7-fun9-1}}}\plusendmoddef
void mesh_add_param_node(mesh_t self, int node_id)
\{
    if (node_id <= 0 || node_id > mesh_num_nodes(self))
        mesh_error(self, "Invalid node identifier");
    dynarray_append(self->param_nodes, &node_id);
\}

\nwendcode{}\nwbegindocs{74}\nwdocspar

The {\tt{}mesh{\char95}param} and {\tt{}mesh{\char95}param{\char95}node} functions are little
more than wrappers around {\tt{}dynarray{\char95}get}.

\nwenddocs{}\nwbegincode{75}\sublabel{NWmes7-fun9-B}\nwmargintag{{\nwtagstyle{}\subpageref{NWmes7-fun9-B}}}\moddef{functions~{\nwtagstyle{}\subpageref{NWmes7-fun9-1}}}\plusendmoddef
mesh_param_t* mesh_param(mesh_t self, int index)
\{
    return dynarray_get(self->params, index);
\}

int mesh_param_node(mesh_t self, int index)
\{
    int* n = (int*) dynarray_get(self->param_nodes, index);
    return (n == NULL) ? 0 : *n;
\}

\nwendcode{}\nwbegindocs{76}\nwdocspar

The {\tt{}mesh{\char95}param{\char95}byname} runs through the parameters and checks if
any have matching name.  We do not currently check if either the 
name argument to the function or the names of the parameters in the 
array are NULL.  If there is no match for the name, we return NULL.

\nwenddocs{}\nwbegincode{77}\sublabel{NWmes7-fun9-C}\nwmargintag{{\nwtagstyle{}\subpageref{NWmes7-fun9-C}}}\moddef{functions~{\nwtagstyle{}\subpageref{NWmes7-fun9-1}}}\plusendmoddef
mesh_param_t* mesh_param_byname(mesh_t self, const char* name)
\{
    mesh_param_t* params = (mesh_param_t*) dynarray_data(self->params);
    int n = dynarray_count(self->params);
    int i;
    for (i = 0; i < n; ++i) \{
        if (strcmp(params[i].name, name) == 0)
            return &(params[i]);
    \}
    return NULL;
\}

\nwendcode{}\nwbegindocs{78}\nwdocspar

\nwenddocs{}\nwbegincode{79}\sublabel{NWmes7-fun9-D}\nwmargintag{{\nwtagstyle{}\subpageref{NWmes7-fun9-D}}}\moddef{functions~{\nwtagstyle{}\subpageref{NWmes7-fun9-1}}}\plusendmoddef
int mesh_num_params(mesh_t self)
\{
    return dynarray_count(self->params);
\}

int mesh_num_param_nodes(mesh_t self)
\{
    return dynarray_count(self->param_nodes);
\}

\nwendcode{}\nwbegindocs{80}\nwdocspar

\nwenddocs{}\nwbegincode{81}\sublabel{NWmes7-fun9-E}\nwmargintag{{\nwtagstyle{}\subpageref{NWmes7-fun9-E}}}\moddef{functions~{\nwtagstyle{}\subpageref{NWmes7-fun9-1}}}\plusendmoddef
void mesh_params_clear(mesh_t self)
\{
    dynarray_set_count(self->params, 0);
    dynarray_set_count(self->param_nodes, 0);
\}

\nwendcode{}\nwbegindocs{82}\nwdocspar

The parameter conversion routines {\tt{}mesh{\char95}param{\char95}string},
{\tt{}mesh{\char95}param{\char95}number}, and {\tt{}mesh{\char95}param{\char95}int} check parameter
validity and then pull out the desired value.  Ideally, this
routine would provide a little more information in its error
messages (like what the name of the parameter is).  I should
probably switch over to using variable argument list error
reporting routines that call {\tt{}vfprintf} (for instance),
or add a {\tt{}mesh{\char95}message} routine in addition to {\tt{}mesh{\char95}error}
and {\tt{}mesh{\char95}warning} so that I could assemble the error messages
before calling {\tt{}mesh{\char95}error}.  Either way, it would then be
possible to have a little more flexibility in the error messages.

\nwenddocs{}\nwbegincode{83}\sublabel{NWmes7-fun9-F}\nwmargintag{{\nwtagstyle{}\subpageref{NWmes7-fun9-F}}}\moddef{functions~{\nwtagstyle{}\subpageref{NWmes7-fun9-1}}}\plusendmoddef
const char* mesh_param_string(mesh_t self, mesh_param_t* param)
\{
    if (param == NULL) \{
        mesh_error(self, "Missing parameter");
        return NULL;
    \} else if (param->tag != MESH_PARAM_STRING) \{
        mesh_error(self, "Parameter should be a string");
        return NULL;
    \} else
        return param->val.s;
\}

\nwendcode{}\nwbegindocs{84}\nwdocspar

\nwenddocs{}\nwbegincode{85}\sublabel{NWmes7-fun9-G}\nwmargintag{{\nwtagstyle{}\subpageref{NWmes7-fun9-G}}}\moddef{functions~{\nwtagstyle{}\subpageref{NWmes7-fun9-1}}}\plusendmoddef
double mesh_param_number(mesh_t self, mesh_param_t* param)
\{
    if (param == NULL) \{
        mesh_error(self, "Missing parameter");
        return 0;
    \} else if (param->tag != MESH_PARAM_NUMBER) \{
        mesh_error(self, "Parameter should be numeric");
        return 0;
    \} else 
        return param->val.d;
\}

\nwendcode{}\nwbegindocs{86}\nwdocspar

\nwenddocs{}\nwbegincode{87}\sublabel{NWmes7-fun9-H}\nwmargintag{{\nwtagstyle{}\subpageref{NWmes7-fun9-H}}}\moddef{functions~{\nwtagstyle{}\subpageref{NWmes7-fun9-1}}}\plusendmoddef
int mesh_param_int(mesh_t self, mesh_param_t* param)
\{
    double v = mesh_param_number(self, param);
    if (v != (int) v)
        mesh_error(self, "Number is not an integer");
    return (int) v;
\}

\nwendcode{}\nwbegindocs{88}\nwdocspar

The {\tt{}mesh{\char95}param{\char95}material} function just checks that the material
identifier is in range and, if so, passes the pointer onward.
We might be friendly here and allow the possibility of a name lookup
as well; that is, if the parameter was a string, we would check
to see if any of the available materials had a name that matched that
string.

\nwenddocs{}\nwbegincode{89}\sublabel{NWmes7-fun9-I}\nwmargintag{{\nwtagstyle{}\subpageref{NWmes7-fun9-I}}}\moddef{functions~{\nwtagstyle{}\subpageref{NWmes7-fun9-1}}}\plusendmoddef
material_t* mesh_param_material(mesh_t self, mesh_param_t* param)
\{
    int material_id = mesh_param_int(self, param);
    if (material_id <= 0 || material_id > mesh_num_materials(self)) \{
        mesh_error(self, "Invalid material identifier");
        return NULL;
    \} else
        return mesh_material(self, material_id);
\}

\nwendcode{}\nwbegindocs{90}\nwdocspar


\subsection{Accessing nodes, elements, and materials}

The functions to add to the list of nodes, elements, and materials
also only do a little more than append to dynamic arrays.  The identifiers
are the array indices plus one.  We clear the parameter array
after initializing a new element or material, and we clear the element
nodes array after initializing a new element.  Note also that we make
a duplicate of element and material model names to keep in the string table.

\nwenddocs{}\nwbegincode{91}\sublabel{NWmes7-fun9-J}\nwmargintag{{\nwtagstyle{}\subpageref{NWmes7-fun9-J}}}\moddef{functions~{\nwtagstyle{}\subpageref{NWmes7-fun9-1}}}\plusendmoddef
int mesh_add_node(mesh_t self, const char* name, double x, double y, double z)
\{
    mesh_node_t node;
    node.name = hstrings_add(self->strings, name);
    node.x[0] = x;
    node.x[1] = y;
    node.x[2] = z;
    dynarray_append(self->nodes, &node);
    return dynarray_count(self->nodes);
\}

\nwendcode{}\nwbegindocs{92}\nwdocspar

\nwenddocs{}\nwbegincode{93}\sublabel{NWmes7-fun9-K}\nwmargintag{{\nwtagstyle{}\subpageref{NWmes7-fun9-K}}}\moddef{functions~{\nwtagstyle{}\subpageref{NWmes7-fun9-1}}}\plusendmoddef
int mesh_add_element(mesh_t self, const char* model)
\{
    element_t* element;
    model = hstrings_add(self->strings, model);
    element = element_init(self->model_mgr, self, model);
    dynarray_set_count(self->params, 0);
    dynarray_set_count(self->param_nodes, 0);
    dynarray_append(self->elements, &element);
    return dynarray_count(self->elements);
\}

\nwendcode{}\nwbegindocs{94}\nwdocspar

\nwenddocs{}\nwbegincode{95}\sublabel{NWmes7-fun9-L}\nwmargintag{{\nwtagstyle{}\subpageref{NWmes7-fun9-L}}}\moddef{functions~{\nwtagstyle{}\subpageref{NWmes7-fun9-1}}}\plusendmoddef
int mesh_add_material(mesh_t self, const char* model)
\{
    material_t* material;
    model = hstrings_add(self->strings, model);
    material = material_init(self->model_mgr, self, model);
    dynarray_set_count(self->params, 0);
    dynarray_append(self->materials, &material);
    return dynarray_count(self->materials);
\}

\nwendcode{}\nwbegindocs{96}\nwdocspar

The functions {\tt{}mesh{\char95}node}, {\tt{}mesh{\char95}element}, and {\tt{}mesh{\char95}material}
are little more than wrappers around the corresponding {\tt{}dynarray}
functions.  Note that a zero or negative index in any case will trigger
a failed assertion, since the {\tt{}dynarray} module does not permit
negative indices.

\nwenddocs{}\nwbegincode{97}\sublabel{NWmes7-fun9-M}\nwmargintag{{\nwtagstyle{}\subpageref{NWmes7-fun9-M}}}\moddef{functions~{\nwtagstyle{}\subpageref{NWmes7-fun9-1}}}\plusendmoddef
mesh_node_t* mesh_node(mesh_t self, int node_id)
\{
    return dynarray_get(self->nodes, node_id-1);
\}

element_t* mesh_element(mesh_t self, int element_id)
\{
    return *(element_t**) dynarray_get(self->elements, element_id-1);
\}

material_t* mesh_material(mesh_t self, int material_id)
\{
    return *(material_t**) dynarray_get(self->materials, material_id-1);
\}

\nwendcode{}\nwbegindocs{98}\nwdocspar

Similarly, the {\tt{}mesh{\char95}num{\char95}nodes}, {\tt{}mesh{\char95}num{\char95}elements}, and
{\tt{}mesh{\char95}num{\char95}materials} functions are simply wrappers around
the {\tt{}dynarray{\char95}count} function.

\nwenddocs{}\nwbegincode{99}\sublabel{NWmes7-fun9-N}\nwmargintag{{\nwtagstyle{}\subpageref{NWmes7-fun9-N}}}\moddef{functions~{\nwtagstyle{}\subpageref{NWmes7-fun9-1}}}\plusendmoddef
int mesh_num_nodes(mesh_t self)
\{
    return dynarray_count(self->nodes);
\}

int mesh_num_elements(mesh_t self)
\{
    return dynarray_count(self->elements);
\}

int mesh_num_materials(mesh_t self)
\{
    return dynarray_count(self->materials);
\}

\nwendcode{}

\nwixlogsorted{c}{{constants}{NWmes7-con9-1}{\nwixu{NWmes7-mes6.2-1}\nwixd{NWmes7-con9-1}}}%
\nwixlogsorted{c}{{destroy element data}{NWmes7-desK-1}{\nwixu{NWmes7-fun9-3}\nwixd{NWmes7-desK-1}}}%
\nwixlogsorted{c}{{destroy material data}{NWmes7-desL-1}{\nwixu{NWmes7-fun9-3}\nwixd{NWmes7-desL-1}}}%
\nwixlogsorted{c}{{exported functions}{NWmes7-expI-1}{\nwixu{NWmes7-mes6-1}\nwixd{NWmes7-expI-1}\nwixd{NWmes7-expI-2}\nwixd{NWmes7-expI-3}\nwixd{NWmes7-expI-4}\nwixd{NWmes7-expI-5}\nwixd{NWmes7-expI-6}\nwixd{NWmes7-expI-7}\nwixd{NWmes7-expI-8}\nwixd{NWmes7-expI-9}\nwixd{NWmes7-expI-A}\nwixd{NWmes7-expI-B}\nwixd{NWmes7-expI-C}}}%
\nwixlogsorted{c}{{exported types}{NWmes7-expE-1}{\nwixu{NWmes7-mes6-1}\nwixd{NWmes7-expE-1}\nwixd{NWmes7-expE-2}}}%
\nwixlogsorted{c}{{functions}{NWmes7-fun9-1}{\nwixu{NWmes7-mes6.2-1}\nwixd{NWmes7-fun9-1}\nwixd{NWmes7-fun9-2}\nwixd{NWmes7-fun9-3}\nwixd{NWmes7-fun9-4}\nwixd{NWmes7-fun9-5}\nwixd{NWmes7-fun9-6}\nwixd{NWmes7-fun9-7}\nwixd{NWmes7-fun9-8}\nwixd{NWmes7-fun9-9}\nwixd{NWmes7-fun9-A}\nwixd{NWmes7-fun9-B}\nwixd{NWmes7-fun9-C}\nwixd{NWmes7-fun9-D}\nwixd{NWmes7-fun9-E}\nwixd{NWmes7-fun9-F}\nwixd{NWmes7-fun9-G}\nwixd{NWmes7-fun9-H}\nwixd{NWmes7-fun9-I}\nwixd{NWmes7-fun9-J}\nwixd{NWmes7-fun9-K}\nwixd{NWmes7-fun9-L}\nwixd{NWmes7-fun9-M}\nwixd{NWmes7-fun9-N}}}%
\nwixlogsorted{c}{{mesh.c}{NWmes7-mes6.2-1}{\nwixd{NWmes7-mes6.2-1}}}%
\nwixlogsorted{c}{{mesh.h}{NWmes7-mes6-1}{\nwixd{NWmes7-mes6-1}}}%
\nwixlogsorted{c}{{\code{}mesh{\char95}t\edoc{} data}{NWmes7-**mF-1}{\nwixu{NWmes7-typ5-1}\nwixd{NWmes7-**mF-1}\nwixd{NWmes7-**mF-2}\nwixd{NWmes7-**mF-3}\nwixd{NWmes7-**mF-4}\nwixd{NWmes7-**mF-5}\nwixd{NWmes7-**mF-6}}}%
\nwixlogsorted{c}{{static functions}{NWmes7-staG-1}{\nwixu{NWmes7-mes6.2-1}\nwixd{NWmes7-staG-1}}}%
\nwixlogsorted{c}{{types}{NWmes7-typ5-1}{\nwixu{NWmes7-mes6.2-1}\nwixd{NWmes7-typ5-1}}}%
\nwbegindocs{100}\nwdocspar
\nwenddocs{}
