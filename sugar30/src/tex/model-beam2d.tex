
\section{Introduction}

\subsection{Interface}

The two-dimensional beam model {\tt{}beam2d} is a small-deflection 
Euler-Bernoulli beam model with damping from the fluid layer between
the fabrication layer and the substrate.

The model is characterized by
\begin{itemize}
 \item {\tt{}w} = beam width (in-plane)
 \item {\tt{}h} = beam height (out-of-plane)
 \item {\tt{}density} = mass density of the fabrication layer
 \item {\tt{}Youngsmodulus} = Young's modulus of the material %'
\end{itemize}

The parameters {\tt{}h}, {\tt{}density}, 
and {\tt{}Youngsmodulus} are typically specified by the material.

The nodal variables are displacements in the $x$ and $y$ directions
and rotation about the $z$ axis.


\subsection{Model description}

The derivation of the Euler-Bernoulli beam model is described in 
Reddy~\cite{Redd93}, section 4, or in various other finite element 
or civil/mechanical engineering structures books.  
The axial and transverse deflections
are decoupled (which is part of why this is appropriate only for
small deflections).  The displacement field is approximated using
a standard Galerkin approximation.  The only place our model deviates
from the standard description is in the inclusion of a damping term
from the fluid film between the beam and the substrate.

The axial part is modeled as a simple spring system:
\[
  \rho A u_{,tt} - (E A u_{,x})_{,x} = F_{a}
\]
and the transverse deflection is governed by a fourth order
equation
\[
  \rho A v_{,tt} + (E I v_{,xx})_{,xx} = F_{t}
\]
where $\rho$ is the density, $A$ is the cross-section area,
$w$ is the beam width, 
and $E$ and $I$ are the
Young's modulus and moment of inertia for the beam, respectively. %'


\section{Interface}

\nwfilename{model-beam2d.nw}\nwbegincode{1}\sublabel{NWmodF-modE-1}\nwmargintag{{\nwtagstyle{}\subpageref{NWmodF-modE-1}}}\moddef{model-beam2d.h~{\nwtagstyle{}\subpageref{NWmodF-modE-1}}}\endmoddef
#ifndef MODEL_BEAM2D_H
#define MODEL_BEAM2D_H

#include "modelmgr.h"

void model_beam2d_register(model_mgr_t model_mgr);

#endif /* MODEL_BEAM2D_H */
\nwnotused{model-beam2d.h}\nwendcode{}\nwbegindocs{2}\nwdocspar


\section{Implementation}

\nwenddocs{}\nwbegincode{3}\sublabel{NWmodF-modE.2-1}\nwmargintag{{\nwtagstyle{}\subpageref{NWmodF-modE.2-1}}}\moddef{model-beam2d.c~{\nwtagstyle{}\subpageref{NWmodF-modE.2-1}}}\endmoddef
#include <stdio.h>
#include <stdlib.h>
#include <string.h>
#include <math.h>
#include <assert.h>

#include "model-beam2d.h"
#include "affine.h"
#include "mesh.h"
#include "vars.h"
#include "assemble.h"
#include "netdraw.h"
#include "netout.h"

\LA{}types~{\nwtagstyle{}\subpageref{NWmodF-typ5-1}}\RA{}
\LA{}static functions~{\nwtagstyle{}\subpageref{NWmodF-staG-1}}\RA{}
\LA{}model functions~{\nwtagstyle{}\subpageref{NWmodF-modF.2-1}}\RA{}
\LA{}registration function~{\nwtagstyle{}\subpageref{NWmodF-regL-1}}\RA{}
\nwnotused{model-beam2d.c}\nwendcode{}\nwbegindocs{4}\nwdocspar


For the moment, the beam model keeps track of the density per unit length,
the axial stiffness, the product $EI$, the beam length, and a transformation
from the local to global coordinates for the beam.  It also tracks the
identifiers for the end nodes, of course.

\nwenddocs{}\nwbegincode{5}\sublabel{NWmodF-typ5-1}\nwmargintag{{\nwtagstyle{}\subpageref{NWmodF-typ5-1}}}\moddef{types~{\nwtagstyle{}\subpageref{NWmodF-typ5-1}}}\endmoddef
typedef struct beam2d_t \{
    element_t element;
    int node[2];
    int vars[6];
    double density;
    double Kaxial;
    double EI, EA;
    double l, w, h;
    double transform[12];
\} beam2d_t;

\nwused{\\{NWmodF-modE.2-1}}\nwendcode{}\nwbegindocs{6}\nwdocspar

\nwenddocs{}\nwbegincode{7}\sublabel{NWmodF-modF.2-1}\nwmargintag{{\nwtagstyle{}\subpageref{NWmodF-modF.2-1}}}\moddef{model functions~{\nwtagstyle{}\subpageref{NWmodF-modF.2-1}}}\endmoddef
static element_t* beam2d_init(mesh_t mesh, const char* model,
                              model_element_t* modelfunc)
\{
    beam2d_t* self = (beam2d_t*) 
        mempool_cget(mesh_pool(mesh), sizeof(beam2d_t));

    double rho, A, E;
    material_t* material;

    self->element.data = self;
    self->element.model = modelfunc;

    \LA{}get nodes~{\nwtagstyle{}\subpageref{NWmodF-get9-1}}\RA{}
    \LA{}get material~{\nwtagstyle{}\subpageref{NWmodF-getC-1}}\RA{}
    \LA{}get process parameters~{\nwtagstyle{}\subpageref{NWmodF-getM-1}}\RA{}

    return &(self->element);
\}

\nwalsodefined{\\{NWmodF-modF.2-2}\\{NWmodF-modF.2-3}\\{NWmodF-modF.2-4}\\{NWmodF-modF.2-5}\\{NWmodF-modF.2-6}\\{NWmodF-modF.2-7}}\nwused{\\{NWmodF-modE.2-1}}\nwendcode{}\nwbegindocs{8}\nwdocspar

Once we have the nodes, we can get the node positions and (ta-da!)
the lengths.  We cross our fingers and hope that the third coordinate
of both nodes is the same -- perhaps it should be a checked error if
that is not the case?

\nwenddocs{}\nwbegincode{9}\sublabel{NWmodF-modF.2-2}\nwmargintag{{\nwtagstyle{}\subpageref{NWmodF-modF.2-2}}}\moddef{model functions~{\nwtagstyle{}\subpageref{NWmodF-modF.2-1}}}\plusendmoddef
static void beam2d_set_position(void* pself, mesh_t mesh)
\{
    beam2d_t* self = (beam2d_t*) pself;
    double* x1 = mesh_node(mesh, self->node[0])->x;
    double* x2 = mesh_node(mesh, self->node[1])->x;

    self->l  = (x2[0]-x1[0])*(x2[0]-x1[0]);
    self->l += (x2[1]-x1[1])*(x2[1]-x1[1]);
    self->l  = sqrt(self->l);
    self->Kaxial  = self->EA / self->l;
    compute_transformation(self->transform, x1, x2, self->l);
\}

\nwendcode{}\nwbegindocs{10}\nwdocspar

\nwenddocs{}\nwbegincode{11}\sublabel{NWmodF-getC-1}\nwmargintag{{\nwtagstyle{}\subpageref{NWmodF-getC-1}}}\moddef{get material~{\nwtagstyle{}\subpageref{NWmodF-getC-1}}}\endmoddef
material = mesh_param_material(mesh, mesh_param_byname(mesh, "material"));
\nwused{\\{NWmodF-modF.2-1}}\nwendcode{}\nwbegindocs{12}\nwdocspar

It might be smart to save the actual parameters instead of just the
derived quantities, so that we can later see what they are.  Right now,
though, we discard the parameters after we are finished computing
derived quantities.

\nwenddocs{}\nwbegincode{13}\sublabel{NWmodF-getM-1}\nwmargintag{{\nwtagstyle{}\subpageref{NWmodF-getM-1}}}\moddef{get process parameters~{\nwtagstyle{}\subpageref{NWmodF-getM-1}}}\endmoddef
rho = get_number(mesh, material, "density");
E = get_number(mesh, material, "Youngsmodulus");
self->w = get_number(mesh, material, "w");
self->h = get_number(mesh, material, "h");

A = self->w * self->h;
self->EA      = E*A;
self->EI      = E*A* self->w * self->w /12;
self->density = rho*A;
\nwused{\\{NWmodF-modF.2-1}}\nwendcode{}\nwbegindocs{14}\nwdocspar

\nwenddocs{}\nwbegincode{15}\sublabel{NWmodF-staG-1}\nwmargintag{{\nwtagstyle{}\subpageref{NWmodF-staG-1}}}\moddef{static functions~{\nwtagstyle{}\subpageref{NWmodF-staG-1}}}\endmoddef
static double get_number(mesh_t mesh, material_t* material,
                         const char* name)
\{
    mesh_param_t* param = mesh_param_byname(mesh, name);
    if (param == NULL && material != NULL && material->model != NULL &&
            material->model->param != NULL)
        param = (material->model->param)(material->data, name);
    return mesh_param_number(mesh, param);    
\}

\nwalsodefined{\\{NWmodF-staG-2}\\{NWmodF-staG-3}}\nwused{\\{NWmodF-modE.2-1}}\nwendcode{}\nwbegindocs{16}\nwdocspar

When we get the node identifier, we just check that the correct
number of nodes is specified.

\nwenddocs{}\nwbegincode{17}\sublabel{NWmodF-get9-1}\nwmargintag{{\nwtagstyle{}\subpageref{NWmodF-get9-1}}}\moddef{get nodes~{\nwtagstyle{}\subpageref{NWmodF-get9-1}}}\endmoddef
if (mesh_num_param_nodes(mesh) != 2)
    mesh_error(mesh, "Incorrect number of nodes for beam2d");
self->node[0] = mesh_param_node(mesh, 0);
self->node[1] = mesh_param_node(mesh, 1);
\nwused{\\{NWmodF-modF.2-1}}\nwendcode{}\nwbegindocs{18}\nwdocspar

Since this model is strictly in-plane, computing the desired
rotation is simple.  Note that the {\tt{}transform} matrix from
local to global coordinates is a 3-by-4 matrix.  The last
column represents a translation.  We are using
what the graphics folks would call ``homogeneous coordinates.''

\nwenddocs{}\nwbegincode{19}\sublabel{NWmodF-staG-2}\nwmargintag{{\nwtagstyle{}\subpageref{NWmodF-staG-2}}}\moddef{static functions~{\nwtagstyle{}\subpageref{NWmodF-staG-1}}}\plusendmoddef
static void compute_transformation(double *transform,
                                   double* x1, double* x2, double l)
\{
    double c = (x2[0] - x1[0])/l;
    double s = (x2[1] - x1[1])/l;
    memset(transform, 0, 12*sizeof(double));

    #define T(i,j) (transform[((j)-1)*3+((i)-1)])

    T(1,1) =  c;
    T(1,2) = -s;
    T(2,1) =  s;
    T(2,2) =  c;
    T(3,3) =  1;

    T(1,4) = x1[0];
    T(2,4) = x1[1];
    T(3,4) = x1[2];

    #undef T
\}

\nwendcode{}\nwbegindocs{20}\nwdocspar

We associate {\tt{}x}, {\tt{}y}, and {\tt{}rz} variables with each coordinate.

\nwenddocs{}\nwbegincode{21}\sublabel{NWmodF-modF.2-3}\nwmargintag{{\nwtagstyle{}\subpageref{NWmodF-modF.2-3}}}\moddef{model functions~{\nwtagstyle{}\subpageref{NWmodF-modF.2-1}}}\plusendmoddef
static void beam2d_vars(void* pself, vars_mgr_t vars)
\{
    beam2d_t* self = (beam2d_t*) pself;

    self->vars[0] = vars_node(vars, self->node[0], "x");
    self->vars[1] = vars_node(vars, self->node[0], "y");
    self->vars[2] = vars_node(vars, self->node[0], "rz");

    self->vars[3] = vars_node(vars, self->node[1], "x");
    self->vars[4] = vars_node(vars, self->node[1], "y");
    self->vars[5] = vars_node(vars, self->node[1], "rz");
\}

\nwendcode{}\nwbegindocs{22}\nwdocspar

\nwenddocs{}\nwbegincode{23}\sublabel{NWmodF-modF.2-4}\nwmargintag{{\nwtagstyle{}\subpageref{NWmodF-modF.2-4}}}\moddef{model functions~{\nwtagstyle{}\subpageref{NWmodF-modF.2-1}}}\plusendmoddef
static void beam2d_R(void* pself, assemble_matrix_t *R, 
                     assemble_matrix_t* x,
                     assemble_matrix_t* v,
                     assemble_matrix_t* a)
\{
    beam2d_t* self = (beam2d_t*) pself;
    \LA{}get parameters from \code{}self\edoc{}~{\nwtagstyle{}\subpageref{NWmodF-getS-1}}\RA{}

    double Rlocal[6];

    if (x) \{
        double xlocal[6];
        memset(xlocal, 0, 6 * sizeof(double));

        assemble_matrix_add(x, self->vars, 6, xlocal);
        affine_apply_AT(self->transform, xlocal + 0);
        affine_apply_AT(self->transform, xlocal + 3);

        Rlocal[0] = Kaxial * (xlocal[0] - xlocal[3]);
        
        Rlocal[1] = c * (12 * ( xlocal[1] - xlocal[4] ) + 
                         6*l* ( xlocal[2] + xlocal[5] ));

        Rlocal[2] = c * (6*l* ( xlocal[1] - xlocal[4] ) +
                         l*l*(4*xlocal[2] + 2*xlocal[5] ));


        Rlocal[3] = Kaxial * (xlocal[3] - xlocal[0]);

        Rlocal[4] = c * (12 *  (-xlocal[1] + xlocal[4] ) + 
                         -6*l* ( xlocal[2] + xlocal[5] ));

        Rlocal[5] = c * (6*l* ( xlocal[1] - xlocal[4] ) +
                         l*l*(2*xlocal[2] + 4*xlocal[5] ));
    \}

    if (a) \{
        double xlocal[6];
        memset(xlocal, 0, 6 * sizeof(double));

        assemble_matrix_add(x, self->vars, 6, xlocal);
        affine_apply_AT(self->transform, xlocal + 0);
        affine_apply_AT(self->transform, xlocal + 3);

        assert(0);
        /* TODO: Fill in the inertial force computation */
        Rlocal[0] = Kaxial * (xlocal[0] - xlocal[3]);
        
        Rlocal[1] = c * (12 * ( xlocal[1] - xlocal[4] ) + 
                         6*l* ( xlocal[2] + xlocal[5] ));

        Rlocal[2] = c * (6*l* ( xlocal[1] - xlocal[4] ) +
                         l*l*(4*xlocal[2] + 2*xlocal[5] ));


        Rlocal[3] = Kaxial * (xlocal[3] - xlocal[0]);

        Rlocal[4] = c * (12 *  (-xlocal[1] + xlocal[4] ) + 
                         -6*l* ( xlocal[2] + xlocal[5] ));

        Rlocal[5] = c * (6*l* ( xlocal[1] - xlocal[4] ) +
                         l*l*(2*xlocal[2] + 4*xlocal[5] ));
    \}

    affine_apply_A(self->transform, Rlocal + 0);
    affine_apply_A(self->transform, Rlocal + 3);
    assemble_matrix_add(R, self->vars, 6, Rlocal);
\}

\nwendcode{}\nwbegindocs{24}\nwdocspar

\nwenddocs{}\nwbegincode{25}\sublabel{NWmodF-modF.2-5}\nwmargintag{{\nwtagstyle{}\subpageref{NWmodF-modF.2-5}}}\moddef{model functions~{\nwtagstyle{}\subpageref{NWmodF-modF.2-1}}}\plusendmoddef
static void beam2d_dR(void* pself, assemble_matrix_t* dR, 
                      double cx, assemble_matrix_t* x,
                      double cv, assemble_matrix_t* v,
                      double ca, assemble_matrix_t* a)
\{
    beam2d_t* self = (beam2d_t*) pself;
    \LA{}get parameters from \code{}self\edoc{}~{\nwtagstyle{}\subpageref{NWmodF-getS-1}}\RA{}

    if (cx) \{
        double Klocal[36];
        memset(Klocal, 0, 36*sizeof(double));
        c *= cx;

        #define K(i,j) (Klocal[(j)*6+(i)-7])
        \LA{}compute axial stiffness~{\nwtagstyle{}\subpageref{NWmodF-comN-1}}\RA{}
        \LA{}compute transverse stiffness~{\nwtagstyle{}\subpageref{NWmodF-comS-1}}\RA{}
        #undef K

        beam2d_transform(self->transform, Klocal);
        assemble_matrix_add(dR, self->vars, 6, Klocal);
    \}

    if (ca) \{
        double Mlocal[36];
        memset(Mlocal, 0, 36*sizeof(double));
        crhoA *= ca;

        #define M(i,j) (Mlocal[(j)*6+(i)-7])
        \LA{}compute axial mass~{\nwtagstyle{}\subpageref{NWmodF-comI-1}}\RA{}
        \LA{}compute transverse mass~{\nwtagstyle{}\subpageref{NWmodF-comN.2-1}}\RA{}
        #undef M

        beam2d_transform(self->transform, Mlocal);
        assemble_matrix_add(dR, self->vars, 6, Mlocal);
    \}

\}

\nwendcode{}\nwbegindocs{26}\nwdocspar

\nwenddocs{}\nwbegincode{27}\sublabel{NWmodF-getS-1}\nwmargintag{{\nwtagstyle{}\subpageref{NWmodF-getS-1}}}\moddef{get parameters from \code{}self\edoc{}~{\nwtagstyle{}\subpageref{NWmodF-getS-1}}}\endmoddef
double Kaxial = self->Kaxial;
double EI     = self->EI;
double l      = self->l, l2 = l*l, l3 = l2*l;
double crhoA  = self->w * self->h * self->density;
double c      = EI/(l*l*l);
\nwused{\\{NWmodF-modF.2-4}\\{NWmodF-modF.2-5}}\nwendcode{}\nwbegindocs{28}\nwdocspar

\nwenddocs{}\nwbegincode{29}\sublabel{NWmodF-comN-1}\nwmargintag{{\nwtagstyle{}\subpageref{NWmodF-comN-1}}}\moddef{compute axial stiffness~{\nwtagstyle{}\subpageref{NWmodF-comN-1}}}\endmoddef
K(1,1) = K(4,4) =  cx * Kaxial;
K(4,1) = K(1,4) = -cx * Kaxial;

\nwused{\\{NWmodF-modF.2-5}}\nwendcode{}\nwbegindocs{30}\nwdocspar

\nwenddocs{}\nwbegincode{31}\sublabel{NWmodF-comI-1}\nwmargintag{{\nwtagstyle{}\subpageref{NWmodF-comI-1}}}\moddef{compute axial mass~{\nwtagstyle{}\subpageref{NWmodF-comI-1}}}\endmoddef
M(1,1) = M(4,4) =  crhoA * l / 3;
M(4,1) = M(1,4) =  crhoA * l / 6;

\nwused{\\{NWmodF-modF.2-5}}\nwendcode{}\nwbegindocs{32}\nwdocspar

\nwenddocs{}\nwbegincode{33}\sublabel{NWmodF-comS-1}\nwmargintag{{\nwtagstyle{}\subpageref{NWmodF-comS-1}}}\moddef{compute transverse stiffness~{\nwtagstyle{}\subpageref{NWmodF-comS-1}}}\endmoddef
/* Fill in the 11 block, first */
K(2,2) =           12*c;
K(3,3) =            4*c*l2;
K(2,3) = K(3,2) =   6*c*l;

/* Now fill in the 22 block */
K(5,5) =           12*c;
K(6,6) =            4*c*l2;
K(5,6) = K(6,5) =  -6*c*l;

/* Now the off-diagonal blocks */
K(5,2) = K(2,5) = -12*c;
K(6,3) = K(3,6) =   2*c*l2;
K(6,2) = K(2,6) =   6*c*l;
K(5,3) = K(3,5) =  -6*c*l;

\nwused{\\{NWmodF-modF.2-5}}\nwendcode{}\nwbegindocs{34}\nwdocspar

\nwenddocs{}\nwbegincode{35}\sublabel{NWmodF-comN.2-1}\nwmargintag{{\nwtagstyle{}\subpageref{NWmodF-comN.2-1}}}\moddef{compute transverse mass~{\nwtagstyle{}\subpageref{NWmodF-comN.2-1}}}\endmoddef
/* Fill in the 11 block, first */
M(2,2) =           crhoA * 12/l3;
M(3,3) =           crhoA * 4 /l;
M(2,3) = M(3,2) =  crhoA * 6 /l2;

/* Now fill in the 22 block */
M(5,5) =           crhoA * 12/l3;
M(6,6) =           crhoA * 4 /l;
M(5,6) = M(6,5) = -crhoA * 6 /l2;

/* Now the off-diagonal blocks */
M(5,2) = M(2,5) = -crhoA * 12/l3;
M(6,3) = M(3,6) =  crhoA * 2 /l;
M(6,2) = M(2,6) =  crhoA * 6 /l2;
M(5,3) = M(3,5) = -crhoA * 6 /l2;

\nwused{\\{NWmodF-modF.2-5}}\nwendcode{}\nwbegindocs{36}\nwdocspar

The {\tt{}beam2d{\char95}transform} function transforms a matrix from
the local coordinate frame to the global coordinate system.

\nwenddocs{}\nwbegincode{37}\sublabel{NWmodF-staG-3}\nwmargintag{{\nwtagstyle{}\subpageref{NWmodF-staG-3}}}\moddef{static functions~{\nwtagstyle{}\subpageref{NWmodF-staG-1}}}\plusendmoddef
static void beam2d_transform(double* transform, double* Klocal)
\{
    int i;

    \LA{}transform to local coordinates on right~{\nwtagstyle{}\subpageref{NWmodF-trad-1}}\RA{}
    \LA{}transform from local coordinates on left~{\nwtagstyle{}\subpageref{NWmodF-trae-1}}\RA{}
\}

\nwendcode{}\nwbegindocs{38}\nwdocspar

\nwenddocs{}\nwbegincode{39}\sublabel{NWmodF-trad-1}\nwmargintag{{\nwtagstyle{}\subpageref{NWmodF-trad-1}}}\moddef{transform to local coordinates on right~{\nwtagstyle{}\subpageref{NWmodF-trad-1}}}\endmoddef
for (i = 0; i < 6; ++i)
    affine_apply_A_stride(transform, Klocal + i,      6);
for (i = 0; i < 6; ++i)
    affine_apply_A_stride(transform, Klocal + i + 18, 6);
\nwused{\\{NWmodF-staG-3}}\nwendcode{}\nwbegindocs{40}\nwdocspar

\nwenddocs{}\nwbegincode{41}\sublabel{NWmodF-trae-1}\nwmargintag{{\nwtagstyle{}\subpageref{NWmodF-trae-1}}}\moddef{transform from local coordinates on left~{\nwtagstyle{}\subpageref{NWmodF-trae-1}}}\endmoddef
for (i = 0; i < 6; ++i)
    affine_apply_A_stride(transform, Klocal + i*6,     1);
for (i = 0; i < 6; ++i)
    affine_apply_A_stride(transform, Klocal + i*6 + 3, 1);
\nwused{\\{NWmodF-staG-3}}\nwendcode{}\nwbegindocs{42}\nwdocspar


\nwenddocs{}\nwbegincode{43}\sublabel{NWmodF-modF.2-6}\nwmargintag{{\nwtagstyle{}\subpageref{NWmodF-modF.2-6}}}\moddef{model functions~{\nwtagstyle{}\subpageref{NWmodF-modF.2-1}}}\plusendmoddef
static void beam2d_output(void* pself, netout_t* netout)
\{
    beam2d_t* self = (beam2d_t*) pself;

    netout_string        (netout, "model",     "beam2d");
    netout_int_matrix    (netout, "node",      self->node, 1, 2);
    netout_double        (netout, "density",   self->density);
    netout_double        (netout, "Kaxial",    self->Kaxial);
    netout_double        (netout, "EI",        self->EI);
    netout_double        (netout, "l",         self->l);
    netout_double        (netout, "w",         self->w);
    netout_double        (netout, "h",         self->h);
    netout_double_matrix (netout, "transform", self->transform, 3, 4);
    netout_int_matrix    (netout, "vars",      self->vars, 1, 6);
\}

\nwendcode{}\nwbegindocs{44}\nwdocspar

\nwenddocs{}\nwbegincode{45}\sublabel{NWmodF-modF.2-7}\nwmargintag{{\nwtagstyle{}\subpageref{NWmodF-modF.2-7}}}\moddef{model functions~{\nwtagstyle{}\subpageref{NWmodF-modF.2-1}}}\plusendmoddef
static void beam2d_display(void* pself, netdraw_gc_t* gc)
\{
    beam2d_t* self = (beam2d_t*) pself;
    int vars[12];
    int i;

    for (i = 0; i < 12; ++i)
        vars[i] = -1;

    vars[0]  = self->vars[0];
    vars[1]  = self->vars[1];
    vars[5]  = self->vars[2];
    vars[6]  = self->vars[3];
    vars[7]  = self->vars[4];
    vars[11] = self->vars[5];

    netdraw_beam(gc, self->transform, self->l, self->w, self->h, vars);
\}

\nwendcode{}\nwbegindocs{46}\nwdocspar

\nwenddocs{}\nwbegincode{47}\sublabel{NWmodF-regL-1}\nwmargintag{{\nwtagstyle{}\subpageref{NWmodF-regL-1}}}\moddef{registration function~{\nwtagstyle{}\subpageref{NWmodF-regL-1}}}\endmoddef
void model_beam2d_register(model_mgr_t model_mgr)
\{
    model_element_t model;
    memset(&model, 0, sizeof(model));

    model.init         = beam2d_init;
    model.set_position = beam2d_set_position;
    model.vars         = beam2d_vars;
    model.R            = beam2d_R;
    model.dR           = beam2d_dR;
    model.output       = beam2d_output;
    model.display      = beam2d_display;

    model_mgr_add_element(model_mgr, "beam2d", &model);
\}

\nwused{\\{NWmodF-modE.2-1}}\nwendcode{}

\nwixlogsorted{c}{{compute axial mass}{NWmodF-comI-1}{\nwixu{NWmodF-modF.2-5}\nwixd{NWmodF-comI-1}}}%
\nwixlogsorted{c}{{compute axial stiffness}{NWmodF-comN-1}{\nwixu{NWmodF-modF.2-5}\nwixd{NWmodF-comN-1}}}%
\nwixlogsorted{c}{{compute transverse mass}{NWmodF-comN.2-1}{\nwixu{NWmodF-modF.2-5}\nwixd{NWmodF-comN.2-1}}}%
\nwixlogsorted{c}{{compute transverse stiffness}{NWmodF-comS-1}{\nwixu{NWmodF-modF.2-5}\nwixd{NWmodF-comS-1}}}%
\nwixlogsorted{c}{{get material}{NWmodF-getC-1}{\nwixu{NWmodF-modF.2-1}\nwixd{NWmodF-getC-1}}}%
\nwixlogsorted{c}{{get nodes}{NWmodF-get9-1}{\nwixu{NWmodF-modF.2-1}\nwixd{NWmodF-get9-1}}}%
\nwixlogsorted{c}{{get parameters from \code{}self\edoc{}}{NWmodF-getS-1}{\nwixu{NWmodF-modF.2-4}\nwixu{NWmodF-modF.2-5}\nwixd{NWmodF-getS-1}}}%
\nwixlogsorted{c}{{get process parameters}{NWmodF-getM-1}{\nwixu{NWmodF-modF.2-1}\nwixd{NWmodF-getM-1}}}%
\nwixlogsorted{c}{{model functions}{NWmodF-modF.2-1}{\nwixu{NWmodF-modE.2-1}\nwixd{NWmodF-modF.2-1}\nwixd{NWmodF-modF.2-2}\nwixd{NWmodF-modF.2-3}\nwixd{NWmodF-modF.2-4}\nwixd{NWmodF-modF.2-5}\nwixd{NWmodF-modF.2-6}\nwixd{NWmodF-modF.2-7}}}%
\nwixlogsorted{c}{{model-beam2d.c}{NWmodF-modE.2-1}{\nwixd{NWmodF-modE.2-1}}}%
\nwixlogsorted{c}{{model-beam2d.h}{NWmodF-modE-1}{\nwixd{NWmodF-modE-1}}}%
\nwixlogsorted{c}{{registration function}{NWmodF-regL-1}{\nwixu{NWmodF-modE.2-1}\nwixd{NWmodF-regL-1}}}%
\nwixlogsorted{c}{{static functions}{NWmodF-staG-1}{\nwixu{NWmodF-modE.2-1}\nwixd{NWmodF-staG-1}\nwixd{NWmodF-staG-2}\nwixd{NWmodF-staG-3}}}%
\nwixlogsorted{c}{{transform from local coordinates on left}{NWmodF-trae-1}{\nwixu{NWmodF-staG-3}\nwixd{NWmodF-trae-1}}}%
\nwixlogsorted{c}{{transform to local coordinates on right}{NWmodF-trad-1}{\nwixu{NWmodF-staG-3}\nwixd{NWmodF-trad-1}}}%
\nwixlogsorted{c}{{types}{NWmodF-typ5-1}{\nwixu{NWmodF-modE.2-1}\nwixd{NWmodF-typ5-1}}}%
\nwbegindocs{48}\nwdocspar


\nwenddocs{}
