
\section{Introduction}

The {\tt{}force} module actually applies two model functions.
The {\tt{}force} model function applies forces to specified
nodal variables; the {\tt{}displace} model function applies
displacement boundary conditions.


\section{Interface}

\nwfilename{model-force.nw}\nwbegincode{1}\sublabel{NWmodE-modD-1}\nwmargintag{{\nwtagstyle{}\subpageref{NWmodE-modD-1}}}\moddef{model-force.h~{\nwtagstyle{}\subpageref{NWmodE-modD-1}}}\endmoddef
#ifndef MODEL_FORCE_H
#define MODEL_FORCE_H

#include "modelmgr.h"

void model_force_register(model_mgr_t model_mgr);

#endif /* MODEL_FORCE_H */
\nwnotused{model-force.h}\nwendcode{}\nwbegindocs{2}\nwdocspar


\section{Implementation}

\nwenddocs{}\nwbegincode{3}\sublabel{NWmodE-modD.2-1}\nwmargintag{{\nwtagstyle{}\subpageref{NWmodE-modD.2-1}}}\moddef{model-force.c~{\nwtagstyle{}\subpageref{NWmodE-modD.2-1}}}\endmoddef
#include <stdio.h>
#include <stdlib.h>
#include <string.h>
#include <assert.h>

#include "model-force.h"
#include "mesh.h"
#include "vars.h"
#include "assemble.h"
#include "netout.h"

\LA{}types~{\nwtagstyle{}\subpageref{NWmodE-typ5-1}}\RA{}
\LA{}model functions~{\nwtagstyle{}\subpageref{NWmodE-modF-1}}\RA{}
\LA{}registration function~{\nwtagstyle{}\subpageref{NWmodE-regL-1}}\RA{}
\nwnotused{model-force.c}\nwendcode{}\nwbegindocs{4}\nwdocspar

Both the {\tt{}force} model and the {\tt{}displace} model use the same
data structure, which keeps track of the node and of the
values (either of forces or displacements).

\nwenddocs{}\nwbegincode{5}\sublabel{NWmodE-typ5-1}\nwmargintag{{\nwtagstyle{}\subpageref{NWmodE-typ5-1}}}\moddef{types~{\nwtagstyle{}\subpageref{NWmodE-typ5-1}}}\endmoddef
typedef struct bc_t \{
    element_t     element;
    const char*   model;
    int           node;
    int*          vars;
    const char**  var_names;
    double*       var_values;
    int           num_vars;
\} bc_t;

\nwused{\\{NWmodE-modD.2-1}}\nwendcode{}\nwbegindocs{6}\nwdocspar

\nwenddocs{}\nwbegincode{7}\sublabel{NWmodE-modF-1}\nwmargintag{{\nwtagstyle{}\subpageref{NWmodE-modF-1}}}\moddef{model functions~{\nwtagstyle{}\subpageref{NWmodE-modF-1}}}\endmoddef
static element_t* bc_init(mesh_t mesh, const char* model, 
                          model_element_t* modelfunc)
\{
    int i, j;
    mempool_t pool = mesh_pool(mesh);
    bc_t*     self = (bc_t*) mempool_cget(pool, sizeof(bc_t));

    self->element.data = self;
    self->element.model = modelfunc;

    self->model = model;
    \LA{}get \code{}node\edoc{}~{\nwtagstyle{}\subpageref{NWmodE-getC-1}}\RA{}
    \LA{}get variable bindings~{\nwtagstyle{}\subpageref{NWmodE-getL-1}}\RA{}
    return &(self->element);
\}

\nwalsodefined{\\{NWmodE-modF-2}\\{NWmodE-modF-3}\\{NWmodE-modF-4}\\{NWmodE-modF-5}}\nwused{\\{NWmodE-modD.2-1}}\nwendcode{}\nwbegindocs{8}\nwdocspar

When we get the node identifier, we just check that the correct
number of nodes are specified.  We \emph{should} probably also
check that the node is in the valid range, but I have not done
that yet.

\nwenddocs{}\nwbegincode{9}\sublabel{NWmodE-getC-1}\nwmargintag{{\nwtagstyle{}\subpageref{NWmodE-getC-1}}}\moddef{get \code{}node\edoc{}~{\nwtagstyle{}\subpageref{NWmodE-getC-1}}}\endmoddef
if (mesh_num_param_nodes(mesh) != 1)
    mesh_error(mesh, "Incorrect number of nodes for anchor");
self->node = mesh_param_node(mesh, 0);

\nwused{\\{NWmodE-modF-1}}\nwendcode{}\nwbegindocs{10}\nwdocspar

We ignore the ``model'' argument, but all the rest of the arguments
must be variable names and associated values.

\nwenddocs{}\nwbegincode{11}\sublabel{NWmodE-getL-1}\nwmargintag{{\nwtagstyle{}\subpageref{NWmodE-getL-1}}}\moddef{get variable bindings~{\nwtagstyle{}\subpageref{NWmodE-getL-1}}}\endmoddef
self->num_vars   = mesh_num_params(mesh);
self->vars       = mempool_cget( pool, self->num_vars * sizeof(int)    );
self->var_names  = mempool_cget( pool, self->num_vars * sizeof(char*)  );
self->var_values = mempool_cget( pool, self->num_vars * sizeof(double) );

j = 0;
for (i = 0; i < self->num_vars; ++i) \{
    mesh_param_t* param = mesh_param(mesh, i);
    if (strcmp(param->name, "model") != 0) \{
        self->var_names[j]  = param->name;
        self->var_values[j] = mesh_param_number(mesh, param);
        ++j;
    \}
\}

self->num_vars = j;

\nwused{\\{NWmodE-modF-1}}\nwendcode{}\nwbegindocs{12}\nwdocspar

\nwenddocs{}\nwbegincode{13}\sublabel{NWmodE-modF-2}\nwmargintag{{\nwtagstyle{}\subpageref{NWmodE-modF-2}}}\moddef{model functions~{\nwtagstyle{}\subpageref{NWmodE-modF-1}}}\plusendmoddef
static void bc_vars(void* pself, vars_mgr_t vars)
\{
    int i;
    bc_t* self = (bc_t*) pself;

    for (i = 0; i < self->num_vars; ++i)
        self->vars[i] = vars_node(vars, self->node, self->var_names[i]);
\}

\nwendcode{}\nwbegindocs{14}\nwdocspar

The {\tt{}displace} model function uses {\tt{}bc{\char95}displace} but not
{\tt{}bc{\char95}R}; the {\tt{}force} model function is the opposite.

\nwenddocs{}\nwbegincode{15}\sublabel{NWmodE-modF-3}\nwmargintag{{\nwtagstyle{}\subpageref{NWmodE-modF-3}}}\moddef{model functions~{\nwtagstyle{}\subpageref{NWmodE-modF-1}}}\plusendmoddef
static void bc_displace(void* pself, assemble_matrix_t* assembler)
\{
    bc_t* self = (bc_t*) pself;

    assemble_matrix_add(assembler, self->vars, self->num_vars, 
                        self->var_values);
\}

\nwendcode{}\nwbegindocs{16}\nwdocspar

\nwenddocs{}\nwbegincode{17}\sublabel{NWmodE-modF-4}\nwmargintag{{\nwtagstyle{}\subpageref{NWmodE-modF-4}}}\moddef{model functions~{\nwtagstyle{}\subpageref{NWmodE-modF-1}}}\plusendmoddef
static void bc_R(void* pself, assemble_matrix_t* R, 
                 assemble_matrix_t* x,
                 assemble_matrix_t* v,
                 assemble_matrix_t* a)
\{
    int i;
    bc_t* self = (bc_t*) pself;

    for (i = 0; i < self->num_vars; ++i)
        assemble_matrix_add(R, self->vars, self->num_vars, 
                            self->var_values);
\}

\nwendcode{}\nwbegindocs{18}\nwdocspar

\nwenddocs{}\nwbegincode{19}\sublabel{NWmodE-modF-5}\nwmargintag{{\nwtagstyle{}\subpageref{NWmodE-modF-5}}}\moddef{model functions~{\nwtagstyle{}\subpageref{NWmodE-modF-1}}}\plusendmoddef
static void bc_output(void* pself, netout_t* netout)
\{
    int i;
    bc_t* self = (bc_t*) pself;

    netout_string     (netout, "model", self->model);
    netout_int_matrix (netout, "node",  &(self->node), 1, 1);
    netout_int_matrix (netout, "vars",  self->vars, 1, self->num_vars);

    for (i = 0; i < self->num_vars; ++i)
        netout_double(netout, self->var_names[i], self->var_values[i]);
\}

\nwendcode{}\nwbegindocs{20}\nwdocspar

\nwenddocs{}\nwbegincode{21}\sublabel{NWmodE-regL-1}\nwmargintag{{\nwtagstyle{}\subpageref{NWmodE-regL-1}}}\moddef{registration function~{\nwtagstyle{}\subpageref{NWmodE-regL-1}}}\endmoddef
void model_force_register(model_mgr_t model_mgr)
\{
    model_element_t model;

    memset(&model, 0, sizeof(model));
    model.init     = bc_init;
    model.vars     = bc_vars;
    model.R        = bc_R;
    model.output   = bc_output;
    model_mgr_add_element(model_mgr, "force", &model);

    memset(&model, 0, sizeof(model));
    model.init     = bc_init;
    model.vars     = bc_vars;
    model.displace = bc_displace;
    model.output   = bc_output;
    model_mgr_add_element(model_mgr, "displace", &model);
\}

\nwused{\\{NWmodE-modD.2-1}}\nwendcode{}

\nwixlogsorted{c}{{get \code{}node\edoc{}}{NWmodE-getC-1}{\nwixu{NWmodE-modF-1}\nwixd{NWmodE-getC-1}}}%
\nwixlogsorted{c}{{get variable bindings}{NWmodE-getL-1}{\nwixu{NWmodE-modF-1}\nwixd{NWmodE-getL-1}}}%
\nwixlogsorted{c}{{model functions}{NWmodE-modF-1}{\nwixu{NWmodE-modD.2-1}\nwixd{NWmodE-modF-1}\nwixd{NWmodE-modF-2}\nwixd{NWmodE-modF-3}\nwixd{NWmodE-modF-4}\nwixd{NWmodE-modF-5}}}%
\nwixlogsorted{c}{{model-force.c}{NWmodE-modD.2-1}{\nwixd{NWmodE-modD.2-1}}}%
\nwixlogsorted{c}{{model-force.h}{NWmodE-modD-1}{\nwixd{NWmodE-modD-1}}}%
\nwixlogsorted{c}{{registration function}{NWmodE-regL-1}{\nwixu{NWmodE-modD.2-1}\nwixd{NWmodE-regL-1}}}%
\nwixlogsorted{c}{{types}{NWmodE-typ5-1}{\nwixu{NWmodE-modD.2-1}\nwixd{NWmodE-typ5-1}}}%
\nwbegindocs{22}\nwdocspar

\nwenddocs{}
