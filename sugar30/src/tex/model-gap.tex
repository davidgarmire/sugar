
\section{Introduction}

The {\tt{}gap} model computes the force and moment between two
rigid plates using an ``infinitesimal parallel plate''
approximation with an ad-hoc correction for fringing fields.


\subsection{Infinitesimal parallel plates}

The attraction between two parallel plates at different voltage
levels is well approximated by
\[
  F = \frac{1}{2} \epsilon_0 tL \frac{V^2}{g^2}
\]
where $\epsilon_0$ is the permittivity of free space, $t$ and $L$
are the thickness and length of the electrodes (so $tL$ is the area
on the side facing the other electrode), $V$ is the voltage level,
and $g$ is the gap between two elements.  This approximation
somewhat underestimates the total force, since it neglects the effect of
fringing fields.

When two plates are only slightly tilted with respect to
each other, we can approximate the electrostatic forces between
them by pretending they are made up of infinitesimal parallel
plates:
\[
  F(x) dx = \frac{1}{2} \epsilon_0 t \frac{V^2}{g(x)^2}
\]
For the purposes of exposition, define 
$K_e = \frac{1}{2} \epsilon_0 V^2$, so that
\[
  F(x) dx = K_e g(x)^{-2}
\]
where the gap $g(x)$ is a linear function of the position:
\[
  g(x) = \alpha + \theta x
\]
This approach is used, for instance, in an exercise in Halliday
and Resnick's physics text~\cite{HallResn62b}.

\subsection{Corrections to the parallel plate approximation}

The SUGAR 2.0 model multiplies all forces by $1 + 1.12 g_0 / t$,
where $g_0$ is the initial gap.  Li and Aluru~\cite{LiAlur01} 
multiply their pressures by $1 + 0.65 g / t$.  
These fudge factors are intended to correct for the presence 
of fringing fields.  Frustratingly, there are no references in 
either place.

Pelesko recently recommended an approach for better approximating
electrostatic forces using boundary layer theory~\cite{Peles01}.
Pelesko's paper focuses on approximating the electrostatic pressure 
on a circular membrane, but his general approach should extend to 
other geometries.


\section{Interface}

\nwfilename{model-gap.nw}\nwbegincode{1}\sublabel{NWmodC-modB-1}\nwmargintag{{\nwtagstyle{}\subpageref{NWmodC-modB-1}}}\moddef{model-gap.h~{\nwtagstyle{}\subpageref{NWmodC-modB-1}}}\endmoddef
#ifndef MODEL_GAP_H
#define MODEL_GAP_H

#include "modelmgr.h"

void model_gap_register(model_mgr_t model_mgr);

#endif /* MODEL_GAP_H */
\nwnotused{model-gap.h}\nwendcode{}\nwbegindocs{2}\nwdocspar



\section{Implementation}

\nwenddocs{}\nwbegincode{3}\sublabel{NWmodC-modB.2-1}\nwmargintag{{\nwtagstyle{}\subpageref{NWmodC-modB.2-1}}}\moddef{model-gap.c~{\nwtagstyle{}\subpageref{NWmodC-modB.2-1}}}\endmoddef
#include <stdio.h>
#include <stdlib.h>
#include <string.h>
#include <math.h>
#include <assert.h>

#include "model-gap.h"
#include "affine.h"
#include "mesh.h"
#include "vars.h"
#include "assemble.h"
#include "netdraw.h"
#include "netout.h"

\LA{}types~{\nwtagstyle{}\subpageref{NWmodC-typ5-1}}\RA{}

\LA{}static functions~{\nwtagstyle{}\subpageref{NWmodC-staG-1}}\RA{}
\LA{}model functions~{\nwtagstyle{}\subpageref{NWmodC-modF-1}}\RA{}
\LA{}functions~{\nwtagstyle{}\subpageref{NWmodC-fun9-1}}\RA{}
\nwnotused{model-gap.c}\nwendcode{}\nwbegindocs{4}\nwdocspar


\subsection{Integrating the pressure load}

All the variants of the electrostatic pressure load from above have
the form
\[
  p(x) = K_{e1} g(x)^{-2} + K_{e2} g(x)^{-1}
\]

Let $g(x) = \alpha + \theta x$ for $x \in (-L/2, L/2)$.
We can cook up a potential corresponding to the pressure load:
\[
  G(\alpha, \theta, x) = -K_{e1} g(x)^{-1} + K_{e2} \log g(x)
\]
If we differentiate $G$ with respect to $\alpha$ and $\theta$,
respectively, we get $p(x)$ and $x p(x)$.

The total force magnitude and moment acting on one rigid plate
are given by
\begin{eqnarray*}
  F & = & \frac{\partial \psi}{\partial \alpha} 
          = \int_{-L/2}^{L/2} p(x) \, dx \\
  M & = & \frac{\partial \psi}{\partial \theta}
          = \int_{-L/2}^{L/2} x p(x) \, dx \\
  \psi & = & \int_{-L/2}^{L/2} G(x) \, dx
\end{eqnarray*}

With some algebra, we can compute
\[
  \psi = -K_{e1} \psi_1 + K_{e2} \psi_2
\]
where for $z = (\theta L)/(2 \alpha)$
\begin{eqnarray*}
  \psi_1 & = & \frac{1}{\theta} \log \frac{1+z}{1-z} \\
         & = & \frac{L}{2\alpha} \frac{1}{z} \log \frac{1+z}{1-z} \\
  \psi_2 & = & \frac{L}{2} \left(
                 \frac{1}{z} \log \frac{1+z}{1-z} +
                 \log (1-z^2) 
               \right) + L (log \alpha - 1)
\end{eqnarray*}

Two terms will occur with sufficient frequency in the following
that we give them names.  Let $r(z) = \log \frac{1+z}{1-z}$ and 
$s(z) = (1-z^2)^{-1}$.  Note that $dr/dz = 2s$ and $ds/dz = 2z s^2$.

Now we differentiate for a while to get
\begin{eqnarray*}
  \psi_{1,\alpha} 
    & = & -\frac{L}{\alpha^2} s 
\\
  \psi_{1,\alpha \alpha} 
    & = & \frac{2L}{\alpha^3} s^2
\\
  \psi_{1,\alpha \theta}
    & = & -\frac{L^2 z}{\alpha^3} s^2
\\
  \psi_{1,\theta}
    & = & -\frac{L^2}{4 \alpha^2 z^2} (2zs - r)
\\
  \psi_{1,\theta \theta}
    & = & -\frac{L^3}{4 \alpha^3 z^3} (r + (4z^2-2)z s^2)
\\
  \psi_{2,\alpha}
    & = & \frac{L}{2 \alpha z} r \\
    & = & \psi_1
\\
  \psi_{2,\theta}
    & = & \frac{L^2}{4 \alpha z^2} (2z - r)
\\
  \psi_{2,\theta \theta}
    & = & \frac{L^3}{4 \alpha^2 z^3} (r - (2-z^2)z s)
\end{eqnarray*}

We would be done and ready to start coding, but several of the formulas
above have removable singularities at $z = 0$ ($\theta = 0$).  In
particular, we need to compute $\psi_1$, $\psi_{1, \theta}$,
$\psi_{1, \theta \theta}$, $\psi_{2, \theta}$ and $\psi_{2, \theta \theta}$
using some different formulas near $z = 0$.  Our alternative formulas
come from Taylor expansion
\begin{eqnarray*}
  \psi_1 & = & \frac{L}{\alpha} \left\{
    1 + \frac{1}{3}z^2 + \frac{1}{5} z^4 + \frac{1}{7} z^6 + \ldots
  \right\}
\\
  \psi_{1,\theta} & = & \frac{L^2}{2 \alpha^2} \left\{
    \frac{2}{3} z + \frac{4}{5} z^3 + \frac{6}{7} z^5 + \ldots
  \right\}
\\
  \psi_{1,\theta \theta} & = & \frac{L^3}{4 \alpha^3} \left\{
    \frac{2}{3} + \frac{3 \cdot 4}{5} z^2 + \frac{5 \cdot 6}{7} z^4 + \ldots
  \right\}
\\
  \psi_{2,\theta} & = & -\frac{L^2}{2 \alpha} \left\{
    \frac{1}{3} z + \frac{1}{5} z^3 + \frac{1}{7} z^5 + \ldots
  \right\}
\\
  \psi_{2,\theta \theta} & = & -\frac{L^3}{4 \alpha^2} \left\{
    \frac{1}{3} + \frac{3}{5} z^2 + \frac{5}{7} z^4 + \ldots
  \right\}
\end{eqnarray*}

It turns out that if we use these expansions for $|z| < 0.1$ and the
original formulas for $|z| > 0.1$, we will get at least 12 digits of
accuracy up to the point where overflow sets in.  The details of the
analysis are tedious, but the idea is simple.  In evaluating $r$,
the dominant term in the linearized error analysis (for small $z$)
is a small absolute error on the order of the unit roundoff.
Except for $\psi_1$, all the formulas with removable singularities
involve a term that looks (to first order) like $r - 2z = O(z^3)$.
In each case, we can show that in fact that term will be 
$\gamma z^3 + O(z^5)$, where $\gamma = O(1)$.  Thus, the relative error
is $O(\epsilon) / (\gamma z^3 + O(z^5)) \approx O(\epsilon z^{-3})$.
The argument for $\psi_1$ is similar, but the relative error is only
$O(\epsilon z^{-1})$.

We use the first six terms in each of the expansions, which is plenty.
I will not copy the analysis here, but the idea is pretty simple --
just bound the tails of the expansions by a constant times
$z^k (1-z^2)^{-1} = z^k + z^{k+2} + z^{k+4} + \ldots$.

\nwenddocs{}\nwbegincode{5}\sublabel{NWmodC-comZ-1}\nwmargintag{{\nwtagstyle{}\subpageref{NWmodC-comZ-1}}}\moddef{compute terms via Taylor expansions~{\nwtagstyle{}\subpageref{NWmodC-comZ-1}}}\endmoddef
double z2 = z*z;

p1t = L2*z / (2*a2) *
(2/3 + z2*(4/5 + z2*(5/7 + z2*(7/9 + z2*(9/11 + z2*(11/13))))));

p1tt = L3 / (4*a3) *
(2/3 + z2*(12/5 + z2*(20/7 + z2*(42/9 + z2*(72/11 + z2*(110/13))))));

p2a = L/a *
(1 + z2*(1/3 + z2*(1/5 + z2*(1/7 + z2*(1/9 + z2*(1/11))))));

p2t = -L2*z/(2*a) * 
(1/3 + z2*(1/5 + z2*(1/7 + z2*(1/9 + z2*(1/11 + z2*(1/13))))));

p2tt = -L3/(4*a2) *
(1/3 + z2*(3/5 + z2*(5/7 + z2*(7/9 + z2*(9/11 + z2*(11/13))))));

\nwused{\\{NWmodC-staG-1}}\nwendcode{}\nwbegindocs{6}\nwdocspar

\nwenddocs{}\nwbegincode{7}\sublabel{NWmodC-staG-1}\nwmargintag{{\nwtagstyle{}\subpageref{NWmodC-staG-1}}}\moddef{static functions~{\nwtagstyle{}\subpageref{NWmodC-staG-1}}}\endmoddef
static void compute_forces(double* F, double* dF,
                           double Ke1, double Ke2, 
                           double a, double t, double L)
\{
    double z = (t*L)/(2*a);
    double s = 1/((1+z)*(1-z));

    double p1a, p1aa, p1at, p1t, p1tt;
    double p2a, p2aa, p2at, p2t, p2tt;

    double a2 = a*a, a3 = a2*a;
    double s2 = s*s;
    double z2 = z*z, z3 = z2*z;
    double L2 = L*L, L3 = L2*L;

    p1a  =   -L  / a2 * s;
    p1aa =  2*L  / a3 * s2;
    p1at =   -L2 / a3 * s2*z;

    if (abs(z) > 0.1) \{

        double r = log((1+z)/(1-z));

        p1t  =  L2 / (4*a2*z2) * (2*z*s - r);
        p1tt =  L3 / (4*a3*z3) * (r + (4*z2 - 2)*z*s2);
        p2a  =   L / (2*a*z) * r;
        p2t  =  L2 / (4*a*z2) * (2*z - r);
        p2tt =  L3 / (4*a2*z3) * (r - (2-z2)*z*s);

    \} else \{
        \LA{}compute terms via Taylor expansions~{\nwtagstyle{}\subpageref{NWmodC-comZ-1}}\RA{}
    \}

    p2aa = p1a;
    p2at = p1t;

    F[0]  = -Ke1*p1a  + Ke2*p2a;
    F[1]  = -Ke1*p1t  + Ke2*p2t;

    dF[0] = -Ke1*p1aa + Ke2*p2aa;
    dF[3] = -Ke1*p1tt + Ke2*p2tt;
    dF[1] = -Ke1*p1at + Ke2*p2at;
    dF[2] = dF[1];
\}

\nwalsodefined{\\{NWmodC-staG-2}}\nwused{\\{NWmodC-modB.2-1}}\nwendcode{}\nwbegindocs{8}\nwdocspar


\subsection{Model functions}

\nwenddocs{}\nwbegincode{9}\sublabel{NWmodC-typ5-1}\nwmargintag{{\nwtagstyle{}\subpageref{NWmodC-typ5-1}}}\moddef{types~{\nwtagstyle{}\subpageref{NWmodC-typ5-1}}}\endmoddef
typedef struct gap_t \{
    element_t element;
    int node[2];
    int vars[6];
    double relpos[3];
    double g0;
    double l, w, h;
    double V;
\} gap_t;

\nwused{\\{NWmodC-modB.2-1}}\nwendcode{}\nwbegindocs{10}\nwdocspar

\nwenddocs{}\nwbegincode{11}\sublabel{NWmodC-modF-1}\nwmargintag{{\nwtagstyle{}\subpageref{NWmodC-modF-1}}}\moddef{model functions~{\nwtagstyle{}\subpageref{NWmodC-modF-1}}}\endmoddef
static element_t* gap_init(mesh_t mesh, const char* model,
                           model_element_t* modelfunc)
\{
    gap_t* self = (gap_t*) mempool_cget(mesh_pool(mesh), sizeof(*self));
    material_t* material;

    self->element.data = self;
    self->element.model = modelfunc;

    if (mesh_num_param_nodes(mesh) != 2)
        mesh_error(mesh, "Incorrect number of nodes for gap");
    self->node[0] = mesh_param_node(mesh, 0);
    self->node[1] = mesh_param_node(mesh, 1);

    material = mesh_param_material(mesh, mesh_param_byname(mesh, "material"));
    
    self->l = get_number(mesh, material, "l");
    self->w = get_number(mesh, material, "w");
    self->h = get_number(mesh, material, "h");
    self->V = get_number(mesh, material, "V");

    return &(self->element);
\}

\nwalsodefined{\\{NWmodC-modF-2}\\{NWmodC-modF-3}\\{NWmodC-modF-4}\\{NWmodC-modF-5}\\{NWmodC-modF-6}}\nwused{\\{NWmodC-modB.2-1}}\nwendcode{}\nwbegindocs{12}\nwdocspar

\nwenddocs{}\nwbegincode{13}\sublabel{NWmodC-staG-2}\nwmargintag{{\nwtagstyle{}\subpageref{NWmodC-staG-2}}}\moddef{static functions~{\nwtagstyle{}\subpageref{NWmodC-staG-1}}}\plusendmoddef
static double get_number(mesh_t mesh, material_t* material,
                         const char* name)
\{
    mesh_param_t* param = mesh_param_byname(mesh, name);
    if (param == NULL && material != NULL && material->model != NULL &&
            material->model->param != NULL)
        param = (material->model->param)(material->data, name);
    return mesh_param_number(mesh, param);    
\}

\nwendcode{}\nwbegindocs{14}\nwdocspar

\nwenddocs{}\nwbegincode{15}\sublabel{NWmodC-modF-2}\nwmargintag{{\nwtagstyle{}\subpageref{NWmodC-modF-2}}}\moddef{model functions~{\nwtagstyle{}\subpageref{NWmodC-modF-1}}}\plusendmoddef
static void gap_set_position(void* pself, mesh_t mesh)
\{
    gap_t* self = (gap_t*) pself;
    double* x1 = mesh_node(mesh, self->node[0])->x;
    double* x2 = mesh_node(mesh, self->node[1])->x;

    double dx0 = x2[0] - x1[0];
    double dx1 = x2[1] - x1[1];
    double dx2 = x2[2] - x2[2];

    self->relpos[0] = dx0;
    self->relpos[1] = dx1;
    self->relpos[2] = dx2;

    self->g0 = sqrt(dx0*dx0 + dx1*dx1 + dx2*dx2);
\}

\nwendcode{}\nwbegindocs{16}\nwdocspar

\nwenddocs{}\nwbegincode{17}\sublabel{NWmodC-modF-3}\nwmargintag{{\nwtagstyle{}\subpageref{NWmodC-modF-3}}}\moddef{model functions~{\nwtagstyle{}\subpageref{NWmodC-modF-1}}}\plusendmoddef
static void gap_vars(void* pself, vars_mgr_t vars)
\{
    gap_t* self = (gap_t*) pself;

    self->vars[0] = vars_node(vars, self->node[0], "x");
    self->vars[1] = vars_node(vars, self->node[0], "y");
    self->vars[2] = vars_node(vars, self->node[0], "rz");

    self->vars[3] = vars_node(vars, self->node[0], "x");
    self->vars[4] = vars_node(vars, self->node[0], "y");
    self->vars[5] = vars_node(vars, self->node[0], "rz");
\}

\nwendcode{}\nwbegindocs{18}\nwdocspar

\nwenddocs{}\nwbegincode{19}\sublabel{NWmodC-modF-4}\nwmargintag{{\nwtagstyle{}\subpageref{NWmodC-modF-4}}}\moddef{model functions~{\nwtagstyle{}\subpageref{NWmodC-modF-1}}}\plusendmoddef
static void gap_R(void* pself, assemble_matrix_t* R,
                  assemble_matrix_t* x,
                  assemble_matrix_t* v,
                  assemble_matrix_t* a)
\{
/*
    gap_t* self = (gap_t*) pself;
    double u[6];
    double Rlocal[6];

    memset(u, 0, 6*sizeof(double));
    assemble_matrix_add(x, self->vars, 6, u);

    double Ke1 = self->e0 * self->h * self->V * self->V;
    double Ke2 = 0.65 * Ke1 / self->w;

    double rpx = self->relpos[0] + u[0] - u[3];
    double rpy = self->relpos[0] + u[1] - u[4];

    double a = sqrt(rpx*rpx + rpy*rpy);
    double t = u[2] - u[5];

    compute_forces(F, dF, Ke1, Ke2, a, t, self->l);

    Rlocal[0] =  F[0] * rpx / a;
    Rlocal[1] =  F[0] * rpy / a;
    Rlocal[2] =  F[1];

    Rlocal[3] = -Rlocal[0];
    Rlocal[4] = -Rlocal[1];
    Rlocal[5] = -F[1];

    assemble_matrix_add(R, self->vars, 6, Rlocal);
*/
\}

\nwendcode{}\nwbegindocs{20}\nwdocspar

\nwenddocs{}\nwbegincode{21}\sublabel{NWmodC-modF-5}\nwmargintag{{\nwtagstyle{}\subpageref{NWmodC-modF-5}}}\moddef{model functions~{\nwtagstyle{}\subpageref{NWmodC-modF-1}}}\plusendmoddef
static void gap_dR(void* pself, assemble_matrix_t* dR,
                   double cx, assemble_matrix_t* x,
                   double cv, assemble_matrix_t* v,
                   double ca, assemble_matrix_t* a)
\{
/*
    gap_t* self = (gap_t*) pself;
    double u[6];
    double dRlocal[36];

    memset(u, 0, 6*sizeof(double));
    assemble_matrix_add(x, self->vars, 6, u);

    double Ke1 = self->e0 * self->h * self->V * self->V;
    double Ke2 = 0.65 * Ke1 / self->w;

    double rpx = self->relpos[0] + u[0] - u[3];
    double rpy = self->relpos[0] + u[1] - u[4];

    double a = sqrt(rpx*rpx + rpy*rpy);
    double t = u[2] - u[5];

    compute_forces(F, dF, Ke1, Ke2, a, t, self->l);

    dRlocal[0]  =               dR[0] * rpx * rpx / a / a;
    dRlocal[7]  =               dR[0] * rpy * rpy / a / a;
    dRlocal[1]  = dRlocal[6]  = dR[0] * rpx * rpy / a / a;

    dRlocal[21] =               dR[3] * rpx * rpx / a / a;
    dRlocal[28] =               dR[3] * rpy * rpy / a / a;
    dRlocal[22] = dRlocal[27] = dR[3] * rpx * rpy / a / a;





    Rlocal[0] =  F[0] * rpx / a;
    Rlocal[1] =  F[0] * rpy / a;
    Rlocal[2] =  F[1];

    Rlocal[3] = -Rlocal[0];
    Rlocal[4] = -Rlocal[1];
    Rlocal[5] = -F[1];


    assemble_matrix_add(dR, self->vars, 6, dRlocal);
*/
\}

\nwendcode{}\nwbegindocs{22}\nwdocspar

\nwenddocs{}\nwbegincode{23}\sublabel{NWmodC-modF-6}\nwmargintag{{\nwtagstyle{}\subpageref{NWmodC-modF-6}}}\moddef{model functions~{\nwtagstyle{}\subpageref{NWmodC-modF-1}}}\plusendmoddef
static void gap_output(void* pself, netout_t* netout)
\{
    gap_t* self = (gap_t*) pself;

    netout_string        (netout, "model",  "gap");
    netout_int_matrix    (netout, "node",   self->node,   1, 2);
    netout_double        (netout, "l",      self->l);
    netout_double        (netout, "w",      self->w);
    netout_double        (netout, "h",      self->h);
    netout_double        (netout, "V",      self->V);
    netout_double_matrix (netout, "relpos", self->relpos, 1, 3);
    netout_int_matrix    (netout, "vars",   self->vars,   1, 6);
\}

\nwendcode{}\nwbegindocs{24}\nwdocspar

\subsection{Registration function}

\nwenddocs{}\nwbegincode{25}\sublabel{NWmodC-fun9-1}\nwmargintag{{\nwtagstyle{}\subpageref{NWmodC-fun9-1}}}\moddef{functions~{\nwtagstyle{}\subpageref{NWmodC-fun9-1}}}\endmoddef
void model_gap_register(model_mgr_t model_mgr)
\{
    model_element_t model;
    memset(&model, 0, sizeof(model));

    model.init         = gap_init;
    model.set_position = gap_set_position;
    model.vars         = gap_vars;
    model.R            = gap_R;
    model.dR           = gap_dR;
    model.output       = gap_output;

    model_mgr_add_element(model_mgr, "gap2d", &model);
\}

\nwused{\\{NWmodC-modB.2-1}}\nwendcode{}

\nwixlogsorted{c}{{compute terms via Taylor expansions}{NWmodC-comZ-1}{\nwixd{NWmodC-comZ-1}\nwixu{NWmodC-staG-1}}}%
\nwixlogsorted{c}{{functions}{NWmodC-fun9-1}{\nwixu{NWmodC-modB.2-1}\nwixd{NWmodC-fun9-1}}}%
\nwixlogsorted{c}{{model functions}{NWmodC-modF-1}{\nwixu{NWmodC-modB.2-1}\nwixd{NWmodC-modF-1}\nwixd{NWmodC-modF-2}\nwixd{NWmodC-modF-3}\nwixd{NWmodC-modF-4}\nwixd{NWmodC-modF-5}\nwixd{NWmodC-modF-6}}}%
\nwixlogsorted{c}{{model-gap.c}{NWmodC-modB.2-1}{\nwixd{NWmodC-modB.2-1}}}%
\nwixlogsorted{c}{{model-gap.h}{NWmodC-modB-1}{\nwixd{NWmodC-modB-1}}}%
\nwixlogsorted{c}{{static functions}{NWmodC-staG-1}{\nwixu{NWmodC-modB.2-1}\nwixd{NWmodC-staG-1}\nwixd{NWmodC-staG-2}}}%
\nwixlogsorted{c}{{types}{NWmodC-typ5-1}{\nwixu{NWmodC-modB.2-1}\nwixd{NWmodC-typ5-1}}}%
\nwbegindocs{26}\nwdocspar
\nwenddocs{}
