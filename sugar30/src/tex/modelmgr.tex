
\section{Introduction}

The model manager module defines the interfaces for element and
material model functions.  It also defines the \emph{model manager},
which manages a registry of available model functions.


\section{Interface}

\nwfilename{modelmgr.nw}\nwbegincode{1}\sublabel{NWmodB-modA-1}\nwmargintag{{\nwtagstyle{}\subpageref{NWmodB-modA-1}}}\moddef{modelmgr.h~{\nwtagstyle{}\subpageref{NWmodB-modA-1}}}\endmoddef
#ifndef MODELMGR_H
#define MODELMGR_H

#include <stdio.h>

struct mesh_struct;
struct mesh_param_t;
struct vars_mgr_struct;
struct assembler_struct;
struct assemble_matrix_t;
struct netdraw_gc_t;
struct netout_t;
typedef struct model_mgr_struct* model_mgr_t;

\LA{}exported types~{\nwtagstyle{}\subpageref{NWmodB-expE-1}}\RA{}
\LA{}exported functions~{\nwtagstyle{}\subpageref{NWmodB-expI-1}}\RA{}

#endif /* MODELMGR_H */
\nwnotused{modelmgr.h}\nwendcode{}\nwbegindocs{2}\nwdocspar


\subsection{The element interface}

The table of functions that implement a particular element are organized
in an object of type {\tt{}model{\char95}element{\char95}t}.  Not all the functions in the
table need to be defined; if one of the function pointers is NULL, some
default behavior will be assumed.

\nwenddocs{}\nwbegincode{3}\sublabel{NWmodB-expE-1}\nwmargintag{{\nwtagstyle{}\subpageref{NWmodB-expE-1}}}\moddef{exported types~{\nwtagstyle{}\subpageref{NWmodB-expE-1}}}\endmoddef
typedef struct model_element_t \{
    \LA{}element functions~{\nwtagstyle{}\subpageref{NWmodB-eleH-1}}\RA{}
\} model_element_t;

typedef struct element_t \{
    model_element_t* model;
    void* data;
\} element_t;

\nwalsodefined{\\{NWmodB-expE-2}}\nwused{\\{NWmodB-modA-1}}\nwendcode{}\nwbegindocs{4}\nwdocspar

The element {\tt{}init} function constructs the data structures
for a particular element.  The function can use the pool
associated with a mesh to allocate space; what it does with that
space is up to the element writer.  Typically, the init function
will initialize a structure to store the indices of the attached
nodes, whatever material models are used, parameters like beam
width, etc.

The initializer function gets a pointer to the current mesh,
which it can use to access the mesh memory pool, to access
the node and material tables, and to access the parameter lists.

\nwenddocs{}\nwbegincode{5}\sublabel{NWmodB-eleH-1}\nwmargintag{{\nwtagstyle{}\subpageref{NWmodB-eleH-1}}}\moddef{element functions~{\nwtagstyle{}\subpageref{NWmodB-eleH-1}}}\endmoddef
struct element_t* (*init)(struct mesh_struct* mesh, const char* model,
                          struct model_element_t* modelfunc);
\nwalsodefined{\\{NWmodB-eleH-2}\\{NWmodB-eleH-3}\\{NWmodB-eleH-4}\\{NWmodB-eleH-5}\\{NWmodB-eleH-6}\\{NWmodB-eleH-7}\\{NWmodB-eleH-8}}\nwused{\\{NWmodB-expE-1}}\nwendcode{}\nwbegindocs{6}\nwdocspar

\nwenddocs{}\nwbegincode{7}\sublabel{NWmodB-expI-1}\nwmargintag{{\nwtagstyle{}\subpageref{NWmodB-expI-1}}}\moddef{exported functions~{\nwtagstyle{}\subpageref{NWmodB-expI-1}}}\endmoddef
element_t* element_init(model_mgr_t model_mgr, 
                        struct mesh_struct* mesh, const char* model);
\nwalsodefined{\\{NWmodB-expI-2}\\{NWmodB-expI-3}\\{NWmodB-expI-4}\\{NWmodB-expI-5}\\{NWmodB-expI-6}\\{NWmodB-expI-7}\\{NWmodB-expI-8}\\{NWmodB-expI-9}\\{NWmodB-expI-A}\\{NWmodB-expI-B}\\{NWmodB-expI-C}\\{NWmodB-expI-D}\\{NWmodB-expI-E}\\{NWmodB-expI-F}\\{NWmodB-expI-G}}\nwused{\\{NWmodB-modA-1}}\nwendcode{}\nwbegindocs{8}\nwdocspar


If the model function performs any allocations outside the mesh
pool, it may need to perform deallocation on shutdown.

\nwenddocs{}\nwbegincode{9}\sublabel{NWmodB-eleH-2}\nwmargintag{{\nwtagstyle{}\subpageref{NWmodB-eleH-2}}}\moddef{element functions~{\nwtagstyle{}\subpageref{NWmodB-eleH-1}}}\plusendmoddef
void (*destroy)(void* self);
\nwendcode{}\nwbegindocs{10}\nwdocspar

\nwenddocs{}\nwbegincode{11}\sublabel{NWmodB-expI-2}\nwmargintag{{\nwtagstyle{}\subpageref{NWmodB-expI-2}}}\moddef{exported functions~{\nwtagstyle{}\subpageref{NWmodB-expI-1}}}\plusendmoddef
void element_destroy(element_t* self);
\nwendcode{}\nwbegindocs{12}\nwdocspar

The {\tt{}set{\char95}position} method can be used to compute or update any 
quantities that rely on undisplaced node positions.

\nwenddocs{}\nwbegincode{13}\sublabel{NWmodB-eleH-3}\nwmargintag{{\nwtagstyle{}\subpageref{NWmodB-eleH-3}}}\moddef{element functions~{\nwtagstyle{}\subpageref{NWmodB-eleH-1}}}\plusendmoddef
void (*set_position)(void* self, struct mesh_struct* mesh);
\nwendcode{}\nwbegindocs{14}\nwdocspar

\nwenddocs{}\nwbegincode{15}\sublabel{NWmodB-expI-3}\nwmargintag{{\nwtagstyle{}\subpageref{NWmodB-expI-3}}}\moddef{exported functions~{\nwtagstyle{}\subpageref{NWmodB-expI-1}}}\plusendmoddef
void element_set_position(element_t* self, struct mesh_struct* mesh);
\nwendcode{}\nwbegindocs{16}\nwdocspar

The {\tt{}vars} function is responsible for assigning variable indices.
It may be called multiple times, so it probably should not be doing
new allocations every time.

\nwenddocs{}\nwbegincode{17}\sublabel{NWmodB-eleH-4}\nwmargintag{{\nwtagstyle{}\subpageref{NWmodB-eleH-4}}}\moddef{element functions~{\nwtagstyle{}\subpageref{NWmodB-eleH-1}}}\plusendmoddef
void (*vars)(void* self, struct vars_mgr_struct* vars);
\nwendcode{}\nwbegindocs{18}\nwdocspar

\nwenddocs{}\nwbegincode{19}\sublabel{NWmodB-expI-4}\nwmargintag{{\nwtagstyle{}\subpageref{NWmodB-expI-4}}}\moddef{exported functions~{\nwtagstyle{}\subpageref{NWmodB-expI-1}}}\plusendmoddef
void element_vars(element_t* self, struct vars_mgr_struct* vars);
\nwendcode{}\nwbegindocs{20}\nwdocspar


The {\tt{}displace} function is responsible for assigning displacement
boundary conditions, if there are any such to be assigned.

\nwenddocs{}\nwbegincode{21}\sublabel{NWmodB-eleH-5}\nwmargintag{{\nwtagstyle{}\subpageref{NWmodB-eleH-5}}}\moddef{element functions~{\nwtagstyle{}\subpageref{NWmodB-eleH-1}}}\plusendmoddef
void (*displace)(void* self, struct assemble_matrix_t* assembler);
\nwendcode{}\nwbegindocs{22}\nwdocspar

\nwenddocs{}\nwbegincode{23}\sublabel{NWmodB-expI-5}\nwmargintag{{\nwtagstyle{}\subpageref{NWmodB-expI-5}}}\moddef{exported functions~{\nwtagstyle{}\subpageref{NWmodB-expI-1}}}\plusendmoddef
void element_displace(element_t* self, struct assemble_matrix_t* assembler);
\nwendcode{}\nwbegindocs{24}\nwdocspar


The {\tt{}R} and {\tt{}dR} routines contribute the local contributions
to the residual function and the tangent matrix, respectively.

\nwenddocs{}\nwbegincode{25}\sublabel{NWmodB-eleH-6}\nwmargintag{{\nwtagstyle{}\subpageref{NWmodB-eleH-6}}}\moddef{element functions~{\nwtagstyle{}\subpageref{NWmodB-eleH-1}}}\plusendmoddef
void (*R) (void* self, struct assemble_matrix_t* R, 
           struct assemble_matrix_t* x,
           struct assemble_matrix_t* v,
           struct assemble_matrix_t* a);
void (*dR)(void* self, struct assemble_matrix_t* dR,
           double cx, struct assemble_matrix_t* x,
           double cv, struct assemble_matrix_t* v,
           double ca, struct assemble_matrix_t* a);
\nwendcode{}\nwbegindocs{26}\nwdocspar

\nwenddocs{}\nwbegincode{27}\sublabel{NWmodB-expI-6}\nwmargintag{{\nwtagstyle{}\subpageref{NWmodB-expI-6}}}\moddef{exported functions~{\nwtagstyle{}\subpageref{NWmodB-expI-1}}}\plusendmoddef
void element_R (element_t* self, struct assemble_matrix_t* R,
                struct assemble_matrix_t* x,
                struct assemble_matrix_t* v,
                struct assemble_matrix_t* a);
void element_dR(element_t* self, struct assemble_matrix_t* dR,
                double cx, struct assemble_matrix_t* x,
                double cv, struct assemble_matrix_t* v,
                double ca, struct assemble_matrix_t* a);
\nwendcode{}\nwbegindocs{28}\nwdocspar


The {\tt{}output} function for an element is currently solely
responsible for dumping out a text representation of whatever
the element data structure might be.

\nwenddocs{}\nwbegincode{29}\sublabel{NWmodB-eleH-7}\nwmargintag{{\nwtagstyle{}\subpageref{NWmodB-eleH-7}}}\moddef{element functions~{\nwtagstyle{}\subpageref{NWmodB-eleH-1}}}\plusendmoddef
void (*output)(void* self, struct netout_t* netout);
\nwendcode{}\nwbegindocs{30}\nwdocspar

\nwenddocs{}\nwbegincode{31}\sublabel{NWmodB-expI-7}\nwmargintag{{\nwtagstyle{}\subpageref{NWmodB-expI-7}}}\moddef{exported functions~{\nwtagstyle{}\subpageref{NWmodB-expI-1}}}\plusendmoddef
void element_output(element_t* self, struct netout_t* netout);
\nwendcode{}\nwbegindocs{32}\nwdocspar


The {\tt{}display} function is responsible for drawing the element
to a {\tt{}netdraw} context.

\nwenddocs{}\nwbegincode{33}\sublabel{NWmodB-eleH-8}\nwmargintag{{\nwtagstyle{}\subpageref{NWmodB-eleH-8}}}\moddef{element functions~{\nwtagstyle{}\subpageref{NWmodB-eleH-1}}}\plusendmoddef
void (*display)(void* self, struct netdraw_gc_t* gc);
\nwendcode{}\nwbegindocs{34}\nwdocspar

\nwenddocs{}\nwbegincode{35}\sublabel{NWmodB-expI-8}\nwmargintag{{\nwtagstyle{}\subpageref{NWmodB-expI-8}}}\moddef{exported functions~{\nwtagstyle{}\subpageref{NWmodB-expI-1}}}\plusendmoddef
void element_display(element_t* self, struct netdraw_gc_t* gc);
\nwendcode{}\nwbegindocs{36}\nwdocspar


\subsection{The material interface}

The table of functions to implement a material model are stored in
a {\tt{}model{\char95}material{\char95}t} structure.

\nwenddocs{}\nwbegincode{37}\sublabel{NWmodB-expE-2}\nwmargintag{{\nwtagstyle{}\subpageref{NWmodB-expE-2}}}\moddef{exported types~{\nwtagstyle{}\subpageref{NWmodB-expE-1}}}\plusendmoddef
typedef struct model_material_t \{
    \LA{}material functions~{\nwtagstyle{}\subpageref{NWmodB-matI-1}}\RA{}
\} model_material_t;

typedef struct material_t \{
    model_material_t* model;
    void* data;
\} material_t;

\nwendcode{}\nwbegindocs{38}\nwdocspar

Like the element initializer function, the model initializer function
constructs the data structures used by the model and does basic
checking on the model parameters.

The {\tt{}mesh} argument provides a handle by which the initializer can
access most of the information it will need: the parameter list,
any other materials, etc.  The model name is useful since it may
be the case that a single model may go by multiple aliases,
or that a default model that calls through to Matlab (for instance)
might need the model name.

\nwenddocs{}\nwbegincode{39}\sublabel{NWmodB-matI-1}\nwmargintag{{\nwtagstyle{}\subpageref{NWmodB-matI-1}}}\moddef{material functions~{\nwtagstyle{}\subpageref{NWmodB-matI-1}}}\endmoddef
struct material_t* (*init)(struct mesh_struct* mesh, const char* model,
                           struct model_material_t* modelfunc);
\nwalsodefined{\\{NWmodB-matI-2}\\{NWmodB-matI-3}\\{NWmodB-matI-4}}\nwused{\\{NWmodB-expE-2}}\nwendcode{}\nwbegindocs{40}\nwdocspar

\nwenddocs{}\nwbegincode{41}\sublabel{NWmodB-expI-9}\nwmargintag{{\nwtagstyle{}\subpageref{NWmodB-expI-9}}}\moddef{exported functions~{\nwtagstyle{}\subpageref{NWmodB-expI-1}}}\plusendmoddef
struct material_t* material_init(model_mgr_t model_mgr, 
                                 struct mesh_struct* mesh, const char* model);
\nwendcode{}\nwbegindocs{42}\nwdocspar

If the material allocated anything outside the mesh pool,
we may need to do some cleanup.

\nwenddocs{}\nwbegincode{43}\sublabel{NWmodB-matI-2}\nwmargintag{{\nwtagstyle{}\subpageref{NWmodB-matI-2}}}\moddef{material functions~{\nwtagstyle{}\subpageref{NWmodB-matI-1}}}\plusendmoddef
void (*destroy)(void* self);
\nwendcode{}\nwbegindocs{44}\nwdocspar

\nwenddocs{}\nwbegincode{45}\sublabel{NWmodB-expI-A}\nwmargintag{{\nwtagstyle{}\subpageref{NWmodB-expI-A}}}\moddef{exported functions~{\nwtagstyle{}\subpageref{NWmodB-expI-1}}}\plusendmoddef
void material_destroy(material_t* self);
\nwendcode{}\nwbegindocs{46}\nwdocspar


One of the roles of the material objects is to keep track of any
parameters to be passed forward to the individual elements.
The {\tt{}param} function is a query method to get those parameters
by name.

\nwenddocs{}\nwbegincode{47}\sublabel{NWmodB-matI-3}\nwmargintag{{\nwtagstyle{}\subpageref{NWmodB-matI-3}}}\moddef{material functions~{\nwtagstyle{}\subpageref{NWmodB-matI-1}}}\plusendmoddef
struct mesh_param_t* (*param)(void* self, const char* name);
\nwendcode{}\nwbegindocs{48}\nwdocspar

\nwenddocs{}\nwbegincode{49}\sublabel{NWmodB-expI-B}\nwmargintag{{\nwtagstyle{}\subpageref{NWmodB-expI-B}}}\moddef{exported functions~{\nwtagstyle{}\subpageref{NWmodB-expI-1}}}\plusendmoddef
struct mesh_param_t* material_param(material_t* self, const char* name);
\nwendcode{}\nwbegindocs{50}\nwdocspar

The {\tt{}output} function for a material is responsible for printing
a text representation of the material data structure.

\nwenddocs{}\nwbegincode{51}\sublabel{NWmodB-matI-4}\nwmargintag{{\nwtagstyle{}\subpageref{NWmodB-matI-4}}}\moddef{material functions~{\nwtagstyle{}\subpageref{NWmodB-matI-1}}}\plusendmoddef
void (*output)(void* self, struct netout_t* netout);
\nwendcode{}\nwbegindocs{52}\nwdocspar

\nwenddocs{}\nwbegincode{53}\sublabel{NWmodB-expI-C}\nwmargintag{{\nwtagstyle{}\subpageref{NWmodB-expI-C}}}\moddef{exported functions~{\nwtagstyle{}\subpageref{NWmodB-expI-1}}}\plusendmoddef
void material_output(material_t* self, struct netout_t* netout);
\nwendcode{}\nwbegindocs{54}\nwdocspar


\subsection{Manager interface}

The {\tt{}model{\char95}mgr{\char95}t} type is an opaque pointer to a model manager
object.  Since the model manager object maintains its own memory
pool, it has a nontrivial destructor function associated with it.

\nwenddocs{}\nwbegincode{55}\sublabel{NWmodB-expI-D}\nwmargintag{{\nwtagstyle{}\subpageref{NWmodB-expI-D}}}\moddef{exported functions~{\nwtagstyle{}\subpageref{NWmodB-expI-1}}}\plusendmoddef
model_mgr_t model_mgr_init();
void        model_mgr_destroy(model_mgr_t self);

\nwendcode{}\nwbegindocs{56}\nwdocspar

To add a set of standard model functions, call the {\tt{}standard{\char95}register}.

\nwenddocs{}\nwbegincode{57}\sublabel{NWmodB-expI-E}\nwmargintag{{\nwtagstyle{}\subpageref{NWmodB-expI-E}}}\moddef{exported functions~{\nwtagstyle{}\subpageref{NWmodB-expI-1}}}\plusendmoddef
void        model_mgr_standard_register(model_mgr_t self);
\nwendcode{}\nwbegindocs{58}\nwdocspar

The {\tt{}add{\char95}element} and {\tt{}add{\char95}material} functions add function
tables to the registry.  Though the function table is passed in
by reference, the model manager makes its own copy for safe
keeping rather than just storing the pointer.  If the {\tt{}name}
argument is NULL, the model will be registered as the default model.

\nwenddocs{}\nwbegincode{59}\sublabel{NWmodB-expI-F}\nwmargintag{{\nwtagstyle{}\subpageref{NWmodB-expI-F}}}\moddef{exported functions~{\nwtagstyle{}\subpageref{NWmodB-expI-1}}}\plusendmoddef
void model_mgr_add_element (model_mgr_t self, const char* name, 
                            model_element_t*  funcs);
void model_mgr_add_material(model_mgr_t self, const char* name, 
                            model_material_t* funcs);

\nwendcode{}\nwbegindocs{60}\nwdocspar

The {\tt{}element} and {\tt{}material} functions retrieve function
tables from the registry.  If the table does not contain any models
matching the requested name, these functions will return NULL.

\nwenddocs{}\nwbegincode{61}\sublabel{NWmodB-expI-G}\nwmargintag{{\nwtagstyle{}\subpageref{NWmodB-expI-G}}}\moddef{exported functions~{\nwtagstyle{}\subpageref{NWmodB-expI-1}}}\plusendmoddef
model_element_t*  model_mgr_element (model_mgr_t self, const char* name);
model_material_t* model_mgr_material(model_mgr_t self, const char* name);
\nwendcode{}\nwbegindocs{62}\nwdocspar


\section{Implementation}

\nwenddocs{}\nwbegincode{63}\sublabel{NWmodB-modA.2-1}\nwmargintag{{\nwtagstyle{}\subpageref{NWmodB-modA.2-1}}}\moddef{modelmgr.c~{\nwtagstyle{}\subpageref{NWmodB-modA.2-1}}}\endmoddef
#include <stdio.h>
#include <stdlib.h>
#include <string.h>

#include "mempool.h"
#include "hash.h"
#include "mesh.h"
#include "modelmgr.h"

\LA{}model headers~{\nwtagstyle{}\subpageref{NWmodB-modD-1}}\RA{}

\LA{}types~{\nwtagstyle{}\subpageref{NWmodB-typ5-1}}\RA{}
\LA{}static functions~{\nwtagstyle{}\subpageref{NWmodB-staG-1}}\RA{}
\LA{}functions~{\nwtagstyle{}\subpageref{NWmodB-fun9-1}}\RA{}
\nwnotused{modelmgr.c}\nwendcode{}\nwbegindocs{64}\nwdocspar

\subsection{Material and element interface wrappers}

\nwenddocs{}\nwbegincode{65}\sublabel{NWmodB-fun9-1}\nwmargintag{{\nwtagstyle{}\subpageref{NWmodB-fun9-1}}}\moddef{functions~{\nwtagstyle{}\subpageref{NWmodB-fun9-1}}}\endmoddef
element_t* element_init(model_mgr_t model_mgr,
                        mesh_t mesh, const char* model)
\{
    model_element_t* modelfunc = model_mgr_element(model_mgr, model);
    if (modelfunc == NULL) \{
        mesh_error(mesh, "Unknown model");
        return NULL;
    \}

    return modelfunc->init(mesh, model, modelfunc);
\}

void element_destroy(element_t* self)
\{
    if (self->model->destroy)
        self->model->destroy(self->data);
\}

\nwalsodefined{\\{NWmodB-fun9-2}\\{NWmodB-fun9-3}\\{NWmodB-fun9-4}\\{NWmodB-fun9-5}\\{NWmodB-fun9-6}\\{NWmodB-fun9-7}\\{NWmodB-fun9-8}\\{NWmodB-fun9-9}\\{NWmodB-fun9-A}\\{NWmodB-fun9-B}\\{NWmodB-fun9-C}\\{NWmodB-fun9-D}\\{NWmodB-fun9-E}\\{NWmodB-fun9-F}}\nwused{\\{NWmodB-modA.2-1}}\nwendcode{}\nwbegindocs{66}\nwdocspar

\nwenddocs{}\nwbegincode{67}\sublabel{NWmodB-fun9-2}\nwmargintag{{\nwtagstyle{}\subpageref{NWmodB-fun9-2}}}\moddef{functions~{\nwtagstyle{}\subpageref{NWmodB-fun9-1}}}\plusendmoddef
void element_set_position(element_t* self, struct mesh_struct* mesh)
\{
    if (self->model->set_position)
        self->model->set_position(self->data, mesh);
\}

\nwendcode{}\nwbegindocs{68}\nwdocspar

\nwenddocs{}\nwbegincode{69}\sublabel{NWmodB-fun9-3}\nwmargintag{{\nwtagstyle{}\subpageref{NWmodB-fun9-3}}}\moddef{functions~{\nwtagstyle{}\subpageref{NWmodB-fun9-1}}}\plusendmoddef
void element_vars(element_t* self, struct vars_mgr_struct* vars)
\{
    if (self->model->vars)
        self->model->vars(self->data, vars);
\}

\nwendcode{}\nwbegindocs{70}\nwdocspar

\nwenddocs{}\nwbegincode{71}\sublabel{NWmodB-fun9-4}\nwmargintag{{\nwtagstyle{}\subpageref{NWmodB-fun9-4}}}\moddef{functions~{\nwtagstyle{}\subpageref{NWmodB-fun9-1}}}\plusendmoddef
void element_displace(element_t* self, struct assemble_matrix_t* assembler)
\{
    if (self->model->displace)
        self->model->displace(self->data, assembler);
\}

void element_R(element_t* self, struct assemble_matrix_t* R,
               struct assemble_matrix_t* x,
               struct assemble_matrix_t* v,
               struct assemble_matrix_t* a)
\{
    if (self->model->R)
        self->model->R(self->data, R, x, v, a);
\}

void element_dR(element_t* self, struct assemble_matrix_t* dR,
                double cx, struct assemble_matrix_t* x,
                double cv, struct assemble_matrix_t* v,
                double ca, struct assemble_matrix_t* a)
\{
    if (self->model->dR)
        self->model->dR(self->data, dR, cx, x, cv, v, ca, a);
\}

\nwendcode{}\nwbegindocs{72}\nwdocspar

\nwenddocs{}\nwbegincode{73}\sublabel{NWmodB-fun9-5}\nwmargintag{{\nwtagstyle{}\subpageref{NWmodB-fun9-5}}}\moddef{functions~{\nwtagstyle{}\subpageref{NWmodB-fun9-1}}}\plusendmoddef
void element_output(element_t* self, struct netout_t* netout)
\{
    if (self->model->output)
        self->model->output(self->data, netout);
\}

void element_display(element_t* self, struct netdraw_gc_t* gc)
\{
    if (self->model->display)
        self->model->display(self->data, gc);
\}

\nwendcode{}\nwbegindocs{74}\nwdocspar

\nwenddocs{}\nwbegincode{75}\sublabel{NWmodB-fun9-6}\nwmargintag{{\nwtagstyle{}\subpageref{NWmodB-fun9-6}}}\moddef{functions~{\nwtagstyle{}\subpageref{NWmodB-fun9-1}}}\plusendmoddef
material_t* material_init(model_mgr_t model_mgr,
                          mesh_t mesh, const char* model)
\{
    model_material_t* modelfunc = model_mgr_material(model_mgr, model);
    if (modelfunc == NULL) \{
        mesh_error(mesh, "Unknown model");
        return NULL;
    \}

    return modelfunc->init(mesh, model, modelfunc);
\}

void material_destroy(material_t* self)
\{
    if (self->model->destroy)
        self->model->destroy(self->data);
\}

\nwendcode{}\nwbegindocs{76}\nwdocspar

\nwenddocs{}\nwbegincode{77}\sublabel{NWmodB-fun9-7}\nwmargintag{{\nwtagstyle{}\subpageref{NWmodB-fun9-7}}}\moddef{functions~{\nwtagstyle{}\subpageref{NWmodB-fun9-1}}}\plusendmoddef
struct mesh_param_t* material_param(material_t* self, const char* name)
\{
    if (self->model->param)
        return self->model->param(self->data, name);
    return NULL;
\}

\nwendcode{}\nwbegindocs{78}\nwdocspar

\nwenddocs{}\nwbegincode{79}\sublabel{NWmodB-fun9-8}\nwmargintag{{\nwtagstyle{}\subpageref{NWmodB-fun9-8}}}\moddef{functions~{\nwtagstyle{}\subpageref{NWmodB-fun9-1}}}\plusendmoddef
void material_output(material_t* self, struct netout_t* netout)
\{
    if (self->model->output)
        self->model->output(self->data, netout);
\}

\nwendcode{}\nwbegindocs{80}\nwdocspar

\subsection{Model manager}

A model manager has an associated pool and two hash tables: one for
storing element function tables, and the other for material function
tables.

\nwenddocs{}\nwbegincode{81}\sublabel{NWmodB-typ5-1}\nwmargintag{{\nwtagstyle{}\subpageref{NWmodB-typ5-1}}}\moddef{types~{\nwtagstyle{}\subpageref{NWmodB-typ5-1}}}\endmoddef
struct model_mgr_struct \{
    mempool_t pool;
    hash_t element_table;
    hash_t material_table;
    model_element_t*  default_element;
    model_material_t* default_material;
\};

\nwalsodefined{\\{NWmodB-typ5-2}}\nwused{\\{NWmodB-modA.2-1}}\nwendcode{}\nwbegindocs{82}\nwdocspar

The entries in the hash tables are just (name, function table) pairs.

\nwenddocs{}\nwbegincode{83}\sublabel{NWmodB-typ5-2}\nwmargintag{{\nwtagstyle{}\subpageref{NWmodB-typ5-2}}}\moddef{types~{\nwtagstyle{}\subpageref{NWmodB-typ5-1}}}\plusendmoddef
typedef struct element_entry_t \{
    const char* name;
    model_element_t element;
\} element_entry_t;

typedef struct material_entry_t \{
    const char* name;
    model_material_t material;
\} material_entry_t;

\nwendcode{}\nwbegindocs{84}\nwdocspar


We allocate everything (including the model manager object) on
a locally allocated pool.  Note that, lacking a better size estimate,
I assume a ``small'' number of different element and material types 
(on the order of 100).  There is no {\tt{}copy} function defined;
all the copying is done by the wrapper functions in this module.

\nwenddocs{}\nwbegincode{85}\sublabel{NWmodB-fun9-9}\nwmargintag{{\nwtagstyle{}\subpageref{NWmodB-fun9-9}}}\moddef{functions~{\nwtagstyle{}\subpageref{NWmodB-fun9-1}}}\plusendmoddef
model_mgr_t model_mgr_init()
\{
    mempool_t pool = mempool_create(MEMPOOL_DEFAULT_SPAN);
    model_mgr_t self = 
        (model_mgr_t) mempool_cget(pool, sizeof(struct model_mgr_struct));

    self->pool = pool;

    self->element_table = hash_pcreate(pool, 100, 
                                       (hash_compare_fun_t) element_compare,
                                       (hash_hash_fun_t)    element_hash,
                                       NULL);

    self->material_table = hash_pcreate(pool, 100,
                                        (hash_compare_fun_t) material_compare,
                                        (hash_hash_fun_t)    material_hash,
                                        NULL);

    self->default_element = NULL;
    self->default_material = NULL;

    return self;
\}

\nwendcode{}\nwbegindocs{86}\nwdocspar

\nwenddocs{}\nwbegincode{87}\sublabel{NWmodB-fun9-A}\nwmargintag{{\nwtagstyle{}\subpageref{NWmodB-fun9-A}}}\moddef{functions~{\nwtagstyle{}\subpageref{NWmodB-fun9-1}}}\plusendmoddef
void model_mgr_destroy(model_mgr_t self)
\{
    mempool_destroy(self->pool);
\}

\nwendcode{}\nwbegindocs{88}\nwdocspar

When we add a new element or material, we just copy everything
(including the name) onto the pool and then insert the pointer
into the hash.  Note that we make no special check to see if a
model with the requested name already exists; if there is a conflict,
the old model just goes away.

\nwenddocs{}\nwbegincode{89}\sublabel{NWmodB-fun9-B}\nwmargintag{{\nwtagstyle{}\subpageref{NWmodB-fun9-B}}}\moddef{functions~{\nwtagstyle{}\subpageref{NWmodB-fun9-1}}}\plusendmoddef
void model_mgr_add_element(model_mgr_t self, const char* name,
                           model_element_t* element)
\{
    element_entry_t* entry = 
        (element_entry_t*) mempool_get(self->pool, sizeof(*entry));
    
    entry->name = mempool_strdup(self->pool, name);
    memcpy(&(entry->element), element, sizeof(*element));
    if (entry->name == NULL)
        self->default_element = &(entry->element);
    else
        hash_insert(self->element_table, entry);
\}

\nwendcode{}\nwbegindocs{90}\nwdocspar

\nwenddocs{}\nwbegincode{91}\sublabel{NWmodB-fun9-C}\nwmargintag{{\nwtagstyle{}\subpageref{NWmodB-fun9-C}}}\moddef{functions~{\nwtagstyle{}\subpageref{NWmodB-fun9-1}}}\plusendmoddef
void model_mgr_add_material(model_mgr_t self, const char* name,
                            model_material_t* material)
\{
    material_entry_t* entry = 
        (material_entry_t*) mempool_get(self->pool, sizeof(*entry));
    
    entry->name = mempool_strdup(self->pool, name);
    memcpy(&(entry->material), material, sizeof(*material));
    if (entry->name == NULL)
        self->default_material = &(entry->material);
    else
        hash_insert(self->material_table, entry);
\}

\nwendcode{}\nwbegindocs{92}\nwdocspar

When we retrieve an element or material, we first check to make
sure that it exists in the table.  If it does exist, we return
a pointer to the function table (not to the entire entry pair);
if it does not exist, we return the default.

\nwenddocs{}\nwbegincode{93}\sublabel{NWmodB-fun9-D}\nwmargintag{{\nwtagstyle{}\subpageref{NWmodB-fun9-D}}}\moddef{functions~{\nwtagstyle{}\subpageref{NWmodB-fun9-1}}}\plusendmoddef
model_element_t* model_mgr_element(model_mgr_t self, const char* name)
\{
    element_entry_t key;
    element_entry_t* entry = NULL;
    key.name = name;
    if (name != NULL)
        entry = hash_retrieve(self->element_table, &key);
    return (entry == NULL) ? self->default_element : &(entry->element);
\}

\nwendcode{}\nwbegindocs{94}\nwdocspar

\nwenddocs{}\nwbegincode{95}\sublabel{NWmodB-fun9-E}\nwmargintag{{\nwtagstyle{}\subpageref{NWmodB-fun9-E}}}\moddef{functions~{\nwtagstyle{}\subpageref{NWmodB-fun9-1}}}\plusendmoddef
model_material_t* model_mgr_material(model_mgr_t self, const char* name)
\{
    material_entry_t key;
    material_entry_t* entry = NULL;
    key.name = name;
    if (name != NULL)
        entry = hash_retrieve(self->material_table, &key);
    return (entry == NULL) ? self->default_material : &(entry->material);
\}

\nwendcode{}\nwbegindocs{96}\nwdocspar

The hash table helper functions are just wrappers around
the standard string hash and comparison functions.

\nwenddocs{}\nwbegincode{97}\sublabel{NWmodB-staG-1}\nwmargintag{{\nwtagstyle{}\subpageref{NWmodB-staG-1}}}\moddef{static functions~{\nwtagstyle{}\subpageref{NWmodB-staG-1}}}\endmoddef
static unsigned element_hash(element_entry_t* entry)
\{
    return hash_strhash(entry->name);
\}

static unsigned material_hash(material_entry_t* entry)
\{
    return hash_strhash(entry->name);
\}

\nwalsodefined{\\{NWmodB-staG-2}}\nwused{\\{NWmodB-modA.2-1}}\nwendcode{}\nwbegindocs{98}\nwdocspar

\nwenddocs{}\nwbegincode{99}\sublabel{NWmodB-staG-2}\nwmargintag{{\nwtagstyle{}\subpageref{NWmodB-staG-2}}}\moddef{static functions~{\nwtagstyle{}\subpageref{NWmodB-staG-1}}}\plusendmoddef
static int element_compare(element_entry_t* a, element_entry_t* b)
\{
    return strcmp(a->name, b->name);
\}

static int material_compare(material_entry_t* a, material_entry_t* b)
\{
    return strcmp(a->name, b->name);
\}

\nwendcode{}\nwbegindocs{100}\nwdocspar

\subsection{Registering standard models}

\nwenddocs{}\nwbegincode{101}\sublabel{NWmodB-modD-1}\nwmargintag{{\nwtagstyle{}\subpageref{NWmodB-modD-1}}}\moddef{model headers~{\nwtagstyle{}\subpageref{NWmodB-modD-1}}}\endmoddef
/* Standard model headers */
#include "model-anchor.h"
#include "model-force.h"
#include "model-beam2d.h"
#include "model-rigid.h"
#include "model-gap.h"

\nwused{\\{NWmodB-modA.2-1}}\nwendcode{}\nwbegindocs{102}\nwdocspar

\nwenddocs{}\nwbegincode{103}\sublabel{NWmodB-fun9-F}\nwmargintag{{\nwtagstyle{}\subpageref{NWmodB-fun9-F}}}\moddef{functions~{\nwtagstyle{}\subpageref{NWmodB-fun9-1}}}\plusendmoddef
void model_mgr_standard_register(model_mgr_t self)
\{
    model_anchor_register (self);
    model_force_register  (self);
    model_beam2d_register (self);
    model_rigid_register  (self);
    model_gap_register    (self);
\}

\nwendcode{}

\nwixlogsorted{c}{{element functions}{NWmodB-eleH-1}{\nwixu{NWmodB-expE-1}\nwixd{NWmodB-eleH-1}\nwixd{NWmodB-eleH-2}\nwixd{NWmodB-eleH-3}\nwixd{NWmodB-eleH-4}\nwixd{NWmodB-eleH-5}\nwixd{NWmodB-eleH-6}\nwixd{NWmodB-eleH-7}\nwixd{NWmodB-eleH-8}}}%
\nwixlogsorted{c}{{exported functions}{NWmodB-expI-1}{\nwixu{NWmodB-modA-1}\nwixd{NWmodB-expI-1}\nwixd{NWmodB-expI-2}\nwixd{NWmodB-expI-3}\nwixd{NWmodB-expI-4}\nwixd{NWmodB-expI-5}\nwixd{NWmodB-expI-6}\nwixd{NWmodB-expI-7}\nwixd{NWmodB-expI-8}\nwixd{NWmodB-expI-9}\nwixd{NWmodB-expI-A}\nwixd{NWmodB-expI-B}\nwixd{NWmodB-expI-C}\nwixd{NWmodB-expI-D}\nwixd{NWmodB-expI-E}\nwixd{NWmodB-expI-F}\nwixd{NWmodB-expI-G}}}%
\nwixlogsorted{c}{{exported types}{NWmodB-expE-1}{\nwixu{NWmodB-modA-1}\nwixd{NWmodB-expE-1}\nwixd{NWmodB-expE-2}}}%
\nwixlogsorted{c}{{functions}{NWmodB-fun9-1}{\nwixu{NWmodB-modA.2-1}\nwixd{NWmodB-fun9-1}\nwixd{NWmodB-fun9-2}\nwixd{NWmodB-fun9-3}\nwixd{NWmodB-fun9-4}\nwixd{NWmodB-fun9-5}\nwixd{NWmodB-fun9-6}\nwixd{NWmodB-fun9-7}\nwixd{NWmodB-fun9-8}\nwixd{NWmodB-fun9-9}\nwixd{NWmodB-fun9-A}\nwixd{NWmodB-fun9-B}\nwixd{NWmodB-fun9-C}\nwixd{NWmodB-fun9-D}\nwixd{NWmodB-fun9-E}\nwixd{NWmodB-fun9-F}}}%
\nwixlogsorted{c}{{material functions}{NWmodB-matI-1}{\nwixu{NWmodB-expE-2}\nwixd{NWmodB-matI-1}\nwixd{NWmodB-matI-2}\nwixd{NWmodB-matI-3}\nwixd{NWmodB-matI-4}}}%
\nwixlogsorted{c}{{model headers}{NWmodB-modD-1}{\nwixu{NWmodB-modA.2-1}\nwixd{NWmodB-modD-1}}}%
\nwixlogsorted{c}{{modelmgr.c}{NWmodB-modA.2-1}{\nwixd{NWmodB-modA.2-1}}}%
\nwixlogsorted{c}{{modelmgr.h}{NWmodB-modA-1}{\nwixd{NWmodB-modA-1}}}%
\nwixlogsorted{c}{{static functions}{NWmodB-staG-1}{\nwixu{NWmodB-modA.2-1}\nwixd{NWmodB-staG-1}\nwixd{NWmodB-staG-2}}}%
\nwixlogsorted{c}{{types}{NWmodB-typ5-1}{\nwixu{NWmodB-modA.2-1}\nwixd{NWmodB-typ5-1}\nwixd{NWmodB-typ5-2}}}%
\nwbegindocs{104}\nwdocspar
\nwenddocs{}
