
\section{MUMPS}

The Multi-User MEMS Process (MUMPS) is one common surface
micromachining process.  We provide information about
the POLY1 ({\tt{}p1}) and POLY2 ({\tt{}p2}) layers.

This file still needs work.

\nwfilename{process-net.nw}\nwbegincode{1}\sublabel{NWproE-mum9-1}\nwmargintag{{\nwtagstyle{}\subpageref{NWproE-mum9-1}}}\moddef{mumps.net~{\nwtagstyle{}\subpageref{NWproE-mum9-1}}}\endmoddef
-- Simplified MUMPS process information file --

polysi =
material \{
  Poisson = 0.3,            -- Poisson ratio = 0.3
  thermcond = 2.33,         -- Thermal conductivity Si = 2.33e-6/C
  viscosity = 1.78e-5,      -- Viscosity (of air) = 1,78e-5
  fluid = 2e-6,             -- Between the device and the substrate.
  density = 2300,           -- Material density = 2300 kg/m^3
  Youngsmodulus = 165e9,    -- Young modulus = 1.65e11 N/m^2
  permittivity = 8.854e-12, -- permittivity: C^2/(uN.um^2)=(C.s)^2/kg.um^3;
  sheetresistance = 20,     -- Poly-Si sheet resistance [ohm/square]
  stress = 0,
  straingradient=0,
  thermalexpansion=0,
  ambienttemperature=0
\}

p1 = 
material \{
  parent = polysi,
  h = 2u
\}

p2 =
material \{
  parent = polysi,
  h = 1.5u
\}

\nwnotused{mumps.net}\nwendcode{}

\nwixlogsorted{c}{{mumps.net}{NWproE-mum9-1}{\nwixd{NWproE-mum9-1}}}%
\nwbegindocs{2}\nwdocspar

\nwenddocs{}
