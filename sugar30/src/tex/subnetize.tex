\subsection{Introduction}

The {\tt{}subnetize} file includes Lua code related to turning an ordinary
function into a subnet.  Internally, the code segment
\begin{verbatim}
subnet foo(A, material, l)
  -- stuff --
end
\end{verbatim}
is equivalent to
\begin{verbatim}
function foo(args)
  local A        = args[1] or args.A or 
                   (args.material and args.material.A)
  local material = args[2] or args.material or 
                   (args.material and args.material.material)
  local l        = args[3] or args.l or 
                   (args.material and args.material.l)
end
foo = subnetize(foo)
\end{verbatim}

Thus, almost all the work of making a subnet be a subnet goes
into the {\tt{}subnetize} function.  This module defines the
standard version of {\tt{}subnetize}, which maintains a heirarchical
namespace for nodes declared in subnets and maintains the stack
of nested local coordinate systems defined in the {\tt{}xformstack}
module.


\subsection{Implementation}

\nwfilename{subnetize.nw}\nwbegincode{1}\sublabel{NWsubC-subD-1}\nwmargintag{{\nwtagstyle{}\subpageref{NWsubC-subD-1}}}\moddef{subnetize.lua~{\nwtagstyle{}\subpageref{NWsubC-subD-1}}}\endmoddef
use("xformstack.lua")

\LA{}initialize data structures~{\nwtagstyle{}\subpageref{NWsubC-iniQ-1}}\RA{}
\LA{}functions~{\nwtagstyle{}\subpageref{NWsubC-fun9-1}}\RA{}
\nwnotused{subnetize.lua}\nwendcode{}\nwbegindocs{2}\nwdocspar

The {\tt{}namestack} is a stack of prefixes for names declared in the
current scope.  For the global scope, the stack entry is an empty
string (when you say ``foo'', you mean ``foo'').  Inside of a subnet
instance ``bar,'' though, when you make a node named ``foo'' you actually
get ``bar.foo.''  Therefore, the prefix ``bar.'' is stored on the
stack while processing the subnet instance.

The {\tt{}localnames} is a table of names for local nodes (nodes in the current
scope).

\nwenddocs{}\nwbegincode{3}\sublabel{NWsubC-iniQ-1}\nwmargintag{{\nwtagstyle{}\subpageref{NWsubC-iniQ-1}}}\moddef{initialize data structures~{\nwtagstyle{}\subpageref{NWsubC-iniQ-1}}}\endmoddef
-- Stack of node name prefixes
_namestack = \{""; n = 1, nanon = 0\};

-- Table of node names in the current scope
_localnames = \{\};

\nwused{\\{NWsubC-subD-1}}\nwendcode{}\nwbegindocs{4}\nwdocspar

The {\tt{}subnetize} function transforms a function that creates a substructure
into a function that creates a substructure in a nested coordinate frame.

\nwenddocs{}\nwbegincode{5}\sublabel{NWsubC-fun9-1}\nwmargintag{{\nwtagstyle{}\subpageref{NWsubC-fun9-1}}}\moddef{functions~{\nwtagstyle{}\subpageref{NWsubC-fun9-1}}}\endmoddef
function subnetize(f)
  return function(p)
    \LA{}add name of current subnet instance to stack~{\nwtagstyle{}\subpageref{NWsubC-addi-1}}\RA{}

    local old_localnames = _localnames
    _localnames = \{\};

    \LA{}form $T$ from \code{}ox\edoc{}, \code{}oy\edoc{}, and \code{}oz\edoc{}~{\nwtagstyle{}\subpageref{NWsubC-fore-1}}\RA{}
    xform_push(T)

    %f(p)

    xform_pop()
    _localnames = old_localnames
    _namestack.n = _namestack.n - 1
  end
end

\nwalsodefined{\\{NWsubC-fun9-2}}\nwused{\\{NWsubC-subD-1}}\nwendcode{}\nwbegindocs{6}\nwdocspar

When we add a new subnet instance, we store its extended name on
the stack.  If it has no name, it is assigned a name beginning
with the tag ``anon.''

\nwenddocs{}\nwbegincode{7}\sublabel{NWsubC-addi-1}\nwmargintag{{\nwtagstyle{}\subpageref{NWsubC-addi-1}}}\moddef{add name of current subnet instance to stack~{\nwtagstyle{}\subpageref{NWsubC-addi-1}}}\endmoddef
local name = p.name
if not name then
  name = "anon" .. _namestack.nanon
  _namestack.nanon = _namestack.nanon + 1
end
_namestack[_namestack.n + 1] = _namestack[_namestack.n] .. name .. "."
_namestack.n = _namestack.n + 1

if p.name then
  p.name = _namestack[_namestack.n-1] .. name
end
\nwused{\\{NWsubC-fun9-1}}\nwendcode{}\nwbegindocs{8}\nwdocspar

When setting up coordinate transformations,
we rotate about the $y$ axis, then $z$, then $x$.  I am unsure why
that convention was chosen, but it is the same convention used in
SUGAR 2.0.

\nwenddocs{}\nwbegincode{9}\sublabel{NWsubC-fore-1}\nwmargintag{{\nwtagstyle{}\subpageref{NWsubC-fore-1}}}\moddef{form $T$ from \code{}ox\edoc{}, \code{}oy\edoc{}, and \code{}oz\edoc{}~{\nwtagstyle{}\subpageref{NWsubC-fore-1}}}\endmoddef
local T = xform_identity()

if p.oy then
  T = xform_compose(T, xform_oy(p.oy))
  p.oy = nil
end

if p.oz then
  T = xform_compose(T, xform_oz(p.oz))
  p.oz = nil
end

if p.ox then
  T = xform_compose(T, xform_ox(p.ox))
  p.ox = nil
end

\nwused{\\{NWsubC-fun9-1}}\nwendcode{}\nwbegindocs{10}\nwdocspar

\nwenddocs{}\nwbegincode{11}\sublabel{NWsubC-fun9-2}\nwmargintag{{\nwtagstyle{}\subpageref{NWsubC-fun9-2}}}\moddef{functions~{\nwtagstyle{}\subpageref{NWsubC-fun9-1}}}\plusendmoddef
-- Create a new node.  If something with the same name
-- already exists in the current scope, use that.
--
function node(args)
  if type(args) == "string" then
    args = \{name = args\}
  end

  local name = args.name
  if not name then
    name = "anon" .. _namestack.nanon
    _namestack.nanon = _namestack.nanon + 1
  end

  if not _localnames[name] then 
    args.name = _namestack[_namestack.n] .. name
    _localnames[name] = %node(args)
  end

  return _localnames[name]
end

\nwendcode{}

\nwixlogsorted{c}{{add name of current subnet instance to stack}{NWsubC-addi-1}{\nwixu{NWsubC-fun9-1}\nwixd{NWsubC-addi-1}}}%
\nwixlogsorted{c}{{form $T$ from \code{}ox\edoc{}, \code{}oy\edoc{}, and \code{}oz\edoc{}}{NWsubC-fore-1}{\nwixu{NWsubC-fun9-1}\nwixd{NWsubC-fore-1}}}%
\nwixlogsorted{c}{{functions}{NWsubC-fun9-1}{\nwixu{NWsubC-subD-1}\nwixd{NWsubC-fun9-1}\nwixd{NWsubC-fun9-2}}}%
\nwixlogsorted{c}{{initialize data structures}{NWsubC-iniQ-1}{\nwixu{NWsubC-subD-1}\nwixd{NWsubC-iniQ-1}}}%
\nwixlogsorted{c}{{subnetize.lua}{NWsubC-subD-1}{\nwixd{NWsubC-subD-1}}}%
\nwbegindocs{12}\nwdocspar

\nwenddocs{}
