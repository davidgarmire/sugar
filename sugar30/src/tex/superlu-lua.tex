
\section{Introduction}

The {\tt{}superlu-lua} module adds Lua support for calls to the SuperLU
interface defined in {\tt{}superlu.nw}.  Supported functions are
\begin{itemize}
  \item {\tt{}A.print()}:   print matrix
  \item {\tt{}A.factor()}:  factor matrix
  \item {\tt{}A.solve(x)}:  compute $x := A^{-1}x$
  \item {\tt{}A.free()}:    free matrix
\end{itemize}


\section{Interface}

\nwfilename{superlu-lua.nw}\nwbegincode{1}\sublabel{NWsupE-supD-1}\nwmargintag{{\nwtagstyle{}\subpageref{NWsupE-supD-1}}}\moddef{superlu-lua.h~{\nwtagstyle{}\subpageref{NWsupE-supD-1}}}\endmoddef
#ifndef SUPERLU_LUA_H
#define SUPERLU_LUA_H 

#include <lua.h>
#include "superlu.h"

void      lua_superlu_register(lua_State* L);
void      lua_superlu_push(lua_State* L, superlu_t matrix);
superlu_t lua_superlu_get(lua_State* L, int index);

#endif /* SUPERLU_LUA_H */
\nwnotused{superlu-lua.h}\nwendcode{}\nwbegindocs{2}\nwdocspar

The {\tt{}lua{\char95}superlu{\char95}register} function registers the functions of
the module with the Lua interpreter.  The {\tt{}lua{\char95}superlu{\char95}push}
and {\tt{}lua{\char95}superlu{\char95}get} functions add and get matrices
on the Lua stack.


\section{Implementation}

\nwenddocs{}\nwbegincode{3}\sublabel{NWsupE-supD.2-1}\nwmargintag{{\nwtagstyle{}\subpageref{NWsupE-supD.2-1}}}\moddef{superlu-lua.c~{\nwtagstyle{}\subpageref{NWsupE-supD.2-1}}}\endmoddef
#include <stdio.h>
#include <stdlib.h>
#include <string.h>
#include <assert.h>
#include "superlu-lua.h"
#include "matrix-lua.h"

\LA{}static data~{\nwtagstyle{}\subpageref{NWsupE-staB-1}}\RA{}

#ifdef HAS_SUPERLU
\LA{}static functions~{\nwtagstyle{}\subpageref{NWsupE-staG-1}}\RA{}
#endif

\LA{}functions~{\nwtagstyle{}\subpageref{NWsupE-fun9-1}}\RA{}
\nwnotused{superlu-lua.c}\nwendcode{}\nwbegindocs{4}\nwdocspar


\subsection{Matrix tag}

The {\tt{}lua{\char95}superlu{\char95}tag} is the tag for the Lua SuperLU sparse matrix type.
Note that you \emph{could} get into trouble
with this if multiple interpreters are simultaneously active,
since the different interpreters may not end up allocating the same
tag value.

\nwenddocs{}\nwbegincode{5}\sublabel{NWsupE-staB-1}\nwmargintag{{\nwtagstyle{}\subpageref{NWsupE-staB-1}}}\moddef{static data~{\nwtagstyle{}\subpageref{NWsupE-staB-1}}}\endmoddef
static int lua_superlu_tag;

\nwused{\\{NWsupE-supD.2-1}}\nwendcode{}\nwbegindocs{6}\nwdocspar


\subsection{Matrix {\tt{}print} method}

The {\tt{}matrix{\char95}print} routine (also known as the {\tt{}print} method
for a matrix object) uses {\tt{}print{\char95}matrix} to output a reasonably
pretty representation of the matrix.

\nwenddocs{}\nwbegincode{7}\sublabel{NWsupE-staG-1}\nwmargintag{{\nwtagstyle{}\subpageref{NWsupE-staG-1}}}\moddef{static functions~{\nwtagstyle{}\subpageref{NWsupE-staG-1}}}\endmoddef
static int lua_matrix_print(lua_State* L)
\{
    superlu_t A = lua_touserdata(L,-1);

    if (lua_gettop(L) != 1)
        lua_error(L, "Wrong number of arguments");

    superlu_print(A);

    lua_settop(L, 0);
    return 0;
\}

\nwalsodefined{\\{NWsupE-staG-2}\\{NWsupE-staG-3}\\{NWsupE-staG-4}\\{NWsupE-staG-5}}\nwused{\\{NWsupE-supD.2-1}}\nwendcode{}\nwbegindocs{8}\nwdocspar


\subsection{Matrix {\tt{}factor} method}

The {\tt{}matrix{\char95}factor} function calls the {\tt{}superlu{\char95}factor}
wrapper to access SuperLU's factorization routine {\tt{}dgstrf}. %'

\nwenddocs{}\nwbegincode{9}\sublabel{NWsupE-staG-2}\nwmargintag{{\nwtagstyle{}\subpageref{NWsupE-staG-2}}}\moddef{static functions~{\nwtagstyle{}\subpageref{NWsupE-staG-1}}}\plusendmoddef
static int lua_matrix_factor(lua_State* L)
\{
    superlu_t A = lua_touserdata(L,-1);

    superlu_factor(A);

    lua_settop(L,0);
    lua_pushusertag(L, A, lua_superlu_tag);
    return 1;
\}

\nwendcode{}\nwbegindocs{10}\nwdocspar


\subsection{Matrix {\tt{}solve} method}

The {\tt{}matrix{\char95}solve} function computes $x := A^{-1} x$ given a factored
matrix $A$.  If the matrix has not already been factored, it will
be automatically factored by the SuperLU wrapper.

\nwenddocs{}\nwbegincode{11}\sublabel{NWsupE-staG-3}\nwmargintag{{\nwtagstyle{}\subpageref{NWsupE-staG-3}}}\moddef{static functions~{\nwtagstyle{}\subpageref{NWsupE-staG-1}}}\plusendmoddef
static int lua_matrix_solve(lua_State* L)
\{
    superlu_t A = lua_touserdata(L,-1);
    lua_matrix_t B;

    if (lua_gettop(L) != 2)
        lua_error(L, "Wrong number of arguments");

    B = lua_matrix_get(L,1);

    if (superlu_n(A) != B->m)
        lua_error(L, "Dimension mismatch");

    superlu_solve(A, B->data);

    lua_settop(L,0);
    lua_pushusertag(L, B, lua_superlu_tag);
    return 1;
\}

\nwendcode{}\nwbegindocs{12}\nwdocspar


\subsection{Matrix {\tt{}free} method}

The {\tt{}matrix{\char95}free} function (also known as the {\tt{}free} method)
deallocates the object memory.  It should probably be called on
Lua garbage collection, but it isn't yet. %'

\nwenddocs{}\nwbegincode{13}\sublabel{NWsupE-staG-4}\nwmargintag{{\nwtagstyle{}\subpageref{NWsupE-staG-4}}}\moddef{static functions~{\nwtagstyle{}\subpageref{NWsupE-staG-1}}}\plusendmoddef
static int lua_matrix_free(lua_State* L)
\{
    superlu_t A = lua_touserdata(L,-1);

    if (lua_gettop(L) != 1)
        lua_error(L, "Too many arguments");

    superlu_destroy(A);

    lua_settop(L,0);
    return 0;
\}

\nwendcode{}\nwbegindocs{14}\nwdocspar


\subsection{Method recall}

When a matrix object is indexed by a method name string,
we return a Lua closure that implements the method.
So when the user requests {\tt{}A.print}, for instance,
the returned closure object will have {\tt{}A} as its final
argument when it is called.

On entry, the Lua stack contains the matrix object
and the name string.

\nwenddocs{}\nwbegincode{15}\sublabel{NWsupE-staG-5}\nwmargintag{{\nwtagstyle{}\subpageref{NWsupE-staG-5}}}\moddef{static functions~{\nwtagstyle{}\subpageref{NWsupE-staG-1}}}\plusendmoddef
static int lua_matrix_getmethod(lua_State* L)
\{
    const char* name = lua_tostring(L,2);

    lua_pop(L,1);
    if (strcmp(name, "print") == 0)
        lua_pushcclosure(L, lua_matrix_print, 1);
    else if (strcmp(name, "free") == 0)
        lua_pushcclosure(L, lua_matrix_free, 1);
    else if (strcmp(name, "factor") == 0)
        lua_pushcclosure(L, lua_matrix_factor, 1);
    else if (strcmp(name, "solve") == 0)
        lua_pushcclosure(L, lua_matrix_solve, 1);
    else
        lua_error(L, "Invalid method name");

    return 1;
\}

\nwendcode{}\nwbegindocs{16}\nwdocspar


\subsection{Setting and removing vectors}

The {\tt{}lua{\char95}superlu{\char95}push} function is a thin wrapper around
{\tt{}lua{\char95}pushusertag}.  Similarly, {\tt{}lua{\char95}superlu{\char95}get} is a
thin wrapper around {\tt{}lua{\char95}touserdata}.  The only reason we
don't want the user to directly use the {\tt{}lua{\char95}pushusertag} and 
{\tt{}lua{\char95}touserdata} functions is that then we would have
to expose the matrix tag value to the world.  That wouldn't
be a tragedy, but it would be nice to keep it private.

\nwenddocs{}\nwbegincode{17}\sublabel{NWsupE-fun9-1}\nwmargintag{{\nwtagstyle{}\subpageref{NWsupE-fun9-1}}}\moddef{functions~{\nwtagstyle{}\subpageref{NWsupE-fun9-1}}}\endmoddef
void lua_superlu_push(lua_State* L, superlu_t matrix)
\{
    lua_pushusertag(L, matrix, lua_superlu_tag);
\}

superlu_t lua_superlu_get(lua_State* L, int index)
\{
    if (index > lua_gettop(L))
        lua_error(L, "Index out of range");

    if (lua_tag(L,index) != lua_superlu_tag)
        lua_error(L, "Variable is not a SuperLU matrix");

    return lua_touserdata(L,index);
\}

\nwalsodefined{\\{NWsupE-fun9-2}}\nwused{\\{NWsupE-supD.2-1}}\nwendcode{}\nwbegindocs{18}\nwdocspar


\subsection{Registration functions}

\nwenddocs{}\nwbegincode{19}\sublabel{NWsupE-fun9-2}\nwmargintag{{\nwtagstyle{}\subpageref{NWsupE-fun9-2}}}\moddef{functions~{\nwtagstyle{}\subpageref{NWsupE-fun9-1}}}\plusendmoddef
void lua_superlu_register(lua_State* L)
\{
    lua_superlu_tag = lua_newtag(L);
#ifdef HAS_SUPERLU
    lua_pushcclosure(L, lua_matrix_getmethod, 0);
    lua_settagmethod(L, lua_superlu_tag, "gettable");
#endif
\}

\nwendcode{}

\nwixlogsorted{c}{{functions}{NWsupE-fun9-1}{\nwixu{NWsupE-supD.2-1}\nwixd{NWsupE-fun9-1}\nwixd{NWsupE-fun9-2}}}%
\nwixlogsorted{c}{{static data}{NWsupE-staB-1}{\nwixu{NWsupE-supD.2-1}\nwixd{NWsupE-staB-1}}}%
\nwixlogsorted{c}{{static functions}{NWsupE-staG-1}{\nwixu{NWsupE-supD.2-1}\nwixd{NWsupE-staG-1}\nwixd{NWsupE-staG-2}\nwixd{NWsupE-staG-3}\nwixd{NWsupE-staG-4}\nwixd{NWsupE-staG-5}}}%
\nwixlogsorted{c}{{superlu-lua.c}{NWsupE-supD.2-1}{\nwixd{NWsupE-supD.2-1}}}%
\nwixlogsorted{c}{{superlu-lua.h}{NWsupE-supD-1}{\nwixd{NWsupE-supD-1}}}%
\nwbegindocs{20}\nwdocspar
\nwenddocs{}
