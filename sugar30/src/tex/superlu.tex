
\section{Introduction}

This package provides a wrapper around the SuperLU solver package.
The module can only be used if the {\tt{}HAS{\char95}SUPERLU} macro is defined.


\section{Interface}

\nwfilename{superlu.nw}\nwbegincode{1}\sublabel{NWsupA-sup9-1}\nwmargintag{{\nwtagstyle{}\subpageref{NWsupA-sup9-1}}}\moddef{superlu.h~{\nwtagstyle{}\subpageref{NWsupA-sup9-1}}}\endmoddef
#ifndef SUPERLU_H
#define SUPERLU_H

#include "assemblem-sparse.h"

typedef struct superlu_struct* superlu_t;

superlu_t superlu_create  (assemble_matrix_t* assembler, int n);
void      superlu_destroy (superlu_t self);
void      superlu_print   (superlu_t self);
void      superlu_factor  (superlu_t self);
void      superlu_solve   (superlu_t self, double* b);
int       superlu_n       (superlu_t self);

#endif /* SUPERLU_H   */
\nwnotused{superlu.h}\nwendcode{}\nwbegindocs{2}\nwdocspar

The {\tt{}superlu{\char95}create} constructor takes a sparse matrix assembler and
builds from it a SuperLU solver object.  The memory used by the sparse
matrix assembler is currently copied, in order to avoid confusion
about who owns what space.  This may be changed later.  The {\tt{}destroy}
method destroys the internal copies along with everything else.

The {\tt{}print} method prints a text representation of the matrix to the
screen.  The output is a raw dump of the matrix data and indexing vectors,
and is not particularly pretty.  This is mostly for debugging purposes.

The {\tt{}factor} method calls the SuperLU factorization routine, and
the {\tt{}solve} method uses the factors to solve the system $Ax = b$.
The {\tt{}solve} routine overwrites the storage for $b$ with $x$.

This interface should eventually be extended so that elements of the
data structures (the indexing structure, row permutation, etc) can
be re-used.


\section{Implementation}

\nwenddocs{}\nwbegincode{3}\sublabel{NWsupA-sup9.2-1}\nwmargintag{{\nwtagstyle{}\subpageref{NWsupA-sup9.2-1}}}\moddef{superlu.c~{\nwtagstyle{}\subpageref{NWsupA-sup9.2-1}}}\endmoddef
#ifdef HAS_SUPERLU

#include <stdlib.h>
#include <assert.h>

#include <dsp_defs.h>
#include <util.h>

#include "superlu.h"

\LA{}types~{\nwtagstyle{}\subpageref{NWsupA-typ5-1}}\RA{}
\LA{}functions~{\nwtagstyle{}\subpageref{NWsupA-fun9-1}}\RA{}

#endif /* HAS_SUPERLU */
\nwnotused{superlu.c}\nwendcode{}\nwbegindocs{4}\nwdocspar

The structure of this module is mostly copied from the first example
in the SuperLU user's guide. %'

\nwenddocs{}\nwbegincode{5}\sublabel{NWsupA-typ5-1}\nwmargintag{{\nwtagstyle{}\subpageref{NWsupA-typ5-1}}}\moddef{types~{\nwtagstyle{}\subpageref{NWsupA-typ5-1}}}\endmoddef
struct superlu_struct \{
    SuperMatrix A;
    SuperMatrix L;
    SuperMatrix U;
    int  is_factored;

    int* perm_r;
    int* perm_c;
    int* etree;

    double* row_scales;
    double* col_scales;
    double rowcnd, colcnd, Amax;
    char equed;

    int  n;
\};

\nwused{\\{NWsupA-sup9.2-1}}\nwendcode{}\nwbegindocs{6}\nwdocspar

The matrix {\tt{}A} is the original column-compressed matrix, which
we initialize from the sparse matrix assembler data.  The {\tt{}A} matrix
has dimensions {\tt{}n} by {\tt{}n}.  Once the {\tt{}factor} method has been
called, {\tt{}is{\char95}factored} is set to true, the {\tt{}L} and {\tt{}U} matrices
are filled with the factorization data, and {\tt{}perm{\char95}r} and {\tt{}perm{\char95}c}
are filled with the row and column permutations.

\nwenddocs{}\nwbegincode{7}\sublabel{NWsupA-fun9-1}\nwmargintag{{\nwtagstyle{}\subpageref{NWsupA-fun9-1}}}\moddef{functions~{\nwtagstyle{}\subpageref{NWsupA-fun9-1}}}\endmoddef
superlu_t superlu_create(assemble_matrix_t* assembler, int n)
\{
    superlu_t self = calloc(1, sizeof(*self));

    int     nnz    = assembler_sparse_nnz(assembler);
    double* nzval  = doubleCalloc( nnz );
    int*    rowind = intCalloc   ( nnz );
    int*    colptr = intCalloc   ( n+1 );

    memcpy( nzval,  assembler_sparse_nzval(assembler),  nnz  * sizeof(double));
    memcpy( rowind, assembler_sparse_rowind(assembler), nnz  * sizeof(int) );
    memcpy( colptr, assembler_sparse_colptr(assembler), (n+1) * sizeof(int) );

    self->n = n;
    dCreate_CompCol_Matrix(&(self->A), n, n, nnz, nzval, rowind, colptr,
                           NC, _D, GE);

    self->perm_c = intMalloc( n );
    self->perm_r = intMalloc( n );
    self->etree  = intMalloc( n );

    self->col_scales = doubleMalloc( n );
    self->row_scales = doubleMalloc( n );

    return self;
\}

\nwalsodefined{\\{NWsupA-fun9-2}\\{NWsupA-fun9-3}\\{NWsupA-fun9-4}\\{NWsupA-fun9-5}\\{NWsupA-fun9-6}}\nwused{\\{NWsupA-sup9.2-1}}\nwendcode{}\nwbegindocs{8}\nwdocspar

\nwenddocs{}\nwbegincode{9}\sublabel{NWsupA-fun9-2}\nwmargintag{{\nwtagstyle{}\subpageref{NWsupA-fun9-2}}}\moddef{functions~{\nwtagstyle{}\subpageref{NWsupA-fun9-1}}}\plusendmoddef
void superlu_destroy(superlu_t self)
\{
    if (self->is_factored) \{
        Destroy_SuperNode_Matrix (&(self->L));
        Destroy_CompCol_Matrix   (&(self->U));
    \}
    SUPERLU_FREE(self->row_scales);
    SUPERLU_FREE(self->col_scales);
    SUPERLU_FREE(self->perm_r);
    SUPERLU_FREE(self->perm_c);
    SUPERLU_FREE(self->etree);
    Destroy_CompCol_Matrix(&(self->A));
    free(self);
\}

\nwendcode{}\nwbegindocs{10}\nwdocspar

\nwenddocs{}\nwbegincode{11}\sublabel{NWsupA-fun9-3}\nwmargintag{{\nwtagstyle{}\subpageref{NWsupA-fun9-3}}}\moddef{functions~{\nwtagstyle{}\subpageref{NWsupA-fun9-1}}}\plusendmoddef
void superlu_print(superlu_t self)
\{
    dPrint_CompCol_Matrix("A", &(self->A));
    if (self->is_factored) \{
        dPrint_CompCol_Matrix   ("U", &(self->U));
        dPrint_SuperNode_Matrix ("L", &(self->L));
    \}
\}

\nwendcode{}\nwbegindocs{12}\nwdocspar

The {\tt{}factor} and {\tt{}solve} methods were buily by cutting and copying
from the SuperLU {\tt{}dgssv} routine.  Once the error handling, timing,
and transposition code were removed, the resulting routines proved
not to be that long.

\nwenddocs{}\nwbegincode{13}\sublabel{NWsupA-fun9-4}\nwmargintag{{\nwtagstyle{}\subpageref{NWsupA-fun9-4}}}\moddef{functions~{\nwtagstyle{}\subpageref{NWsupA-fun9-1}}}\plusendmoddef
void superlu_factor(superlu_t self)
\{
    char* refact = "N";
    SuperMatrix AC;     /* Matrix postmultiplied by Pc */
    int lwork = 0;
    int info;
    int permc_spec = 2; /* Minimum degree on A + A^T */
    
    double diag_pivot_thresh = 1.0;
    double drop_tol          = 0;
    int    panel_size = sp_ienv(1); /* panel size */
    int    relax      = sp_ienv(2); /* no of columns in a relaxed snodes */

    if (self->is_factored)
        return;

    get_perm_c(permc_spec, &(self->A), self->perm_c);

    StatInit(panel_size, relax);
 
    sp_preorder(refact, &(self->A), self->perm_c, self->etree, &AC);

    /* Compute the LU factorization of A. */
    dgstrf(refact, &AC, diag_pivot_thresh, drop_tol, relax, panel_size,
           (self->etree), NULL, lwork, 
           (self->perm_r), (self->perm_c), &(self->L), &(self->U), 
           &info);

    Destroy_CompCol_Permuted(&AC);

    StatFree();
    assert(info == 0);

    self->is_factored = 1;
\}

\nwendcode{}\nwbegindocs{14}\nwdocspar

\nwenddocs{}\nwbegincode{15}\sublabel{NWsupA-fun9-5}\nwmargintag{{\nwtagstyle{}\subpageref{NWsupA-fun9-5}}}\moddef{functions~{\nwtagstyle{}\subpageref{NWsupA-fun9-1}}}\plusendmoddef
void superlu_solve(superlu_t self, double* b)
\{
    char* trans = "N";
    int info;
    
    int panel_size = sp_ienv(1); /* panel size */
    int relax      = sp_ienv(2); /* no of columns in a relaxed snodes */

    SuperMatrix B;
    dCreate_Dense_Matrix(&B, self->n, 1, b, self->n, DN, _D, GE);

    if (!self->is_factored)
        superlu_factor(self);

    StatInit(panel_size, relax);
    dgstrs(trans, &(self->L), &(self->U), self->perm_r, self->perm_c, 
           &B, &info);
    StatFree();

    Destroy_SuperMatrix_Store(&B);
    assert(info == 0);
\}

\nwendcode{}\nwbegindocs{16}\nwdocspar

The {\tt{}superlu{\char95}n} routine is a getter function to query the
dimension of the matrix.

\nwenddocs{}\nwbegincode{17}\sublabel{NWsupA-fun9-6}\nwmargintag{{\nwtagstyle{}\subpageref{NWsupA-fun9-6}}}\moddef{functions~{\nwtagstyle{}\subpageref{NWsupA-fun9-1}}}\plusendmoddef
int superlu_n(superlu_t self)
\{
    return self->n;
\}

\nwendcode{}

\nwixlogsorted{c}{{functions}{NWsupA-fun9-1}{\nwixu{NWsupA-sup9.2-1}\nwixd{NWsupA-fun9-1}\nwixd{NWsupA-fun9-2}\nwixd{NWsupA-fun9-3}\nwixd{NWsupA-fun9-4}\nwixd{NWsupA-fun9-5}\nwixd{NWsupA-fun9-6}}}%
\nwixlogsorted{c}{{superlu.c}{NWsupA-sup9.2-1}{\nwixd{NWsupA-sup9.2-1}}}%
\nwixlogsorted{c}{{superlu.h}{NWsupA-sup9-1}{\nwixd{NWsupA-sup9-1}}}%
\nwixlogsorted{c}{{types}{NWsupA-typ5-1}{\nwixu{NWsupA-sup9.2-1}\nwixd{NWsupA-typ5-1}}}%
\nwbegindocs{18}\nwdocspar

\nwenddocs{}
