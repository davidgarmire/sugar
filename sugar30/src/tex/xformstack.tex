
\section{Introduction}

As a convenience to users, we provide a mechanism for declaring
nested coordinate systems.  This is particularly useful
for creating \emph{substructures} or \emph{subnets}.  Different
instances of a subnet may have the same relative placements
of nodes, but at different translations and orientations.


\section{Interface}

We support nested coordinate systems via a \emph{transformation stack}.
The entries in this stack data structures are all affine transformations that
take the current coordinate system to the global system.  New transformations
are \emph{composed} with the previous top-of-stack, so that if
$T_0$ is the current transformation to global coordinates and a
transformation $S$ is pushed onto the stack, then $T_0 S$ will
be the new transformation from current to global coordinates.

The transformation $T = Ax + b$ is given by a table of twelve entries 
which represents the components of the matrix $[A; b]$ in column major 
order.  A vector $x$ is given by a table with coordinate entries at
indices 1 to 3.  Note that the table can contain additional information
(e.g. a node name).  Part of the reason the transformation functions
overwrite the input vector table is so that entries besides those
at indices 1 to 3 can be retained unchanged.

The functions provided by this module are
\begin{itemize}
  \item {\tt{}xform{\char95}push(T)}: compose a transformation onto the stack
  \item {\tt{}xform{\char95}pop}: pop a level off the transformation stack
  \item {\tt{}top{\char95}xform(x)}: overwrite $x$ with $T_\mathit{top} x$,
        where $T_\mathit{top}$ is the top transform on the stack
  \item {\tt{}xform{\char95}apply(T,\ x)}: overwrite $x$ with $Tx$.
  \item {\tt{}xform{\char95}applyA(t,\ x)}: overwrite $x$ with $A_T x$,
        where $A_T$ is the linear part of the affine transform $T$.
  \item {\tt{}xform{\char95}compose(T,\ S)}: return $TS$
  \item {\tt{}xform{\char95}identity}: return the identity transform
  \item {\tt{}xform{\char95}ox(r),\ xform{\char95}oy(r),\ xform{\char95}oz(r)}:
        return right-handed rotations about the coordinate axes
  \item {\tt{}xform{\char95}translate(z)}: return a translation by the
        vector $z$
  \item {\tt{}subnet(f)}: return a function which creates a nested
        coordinate system according to any {\tt{}ox}, {\tt{}oy},
        and {\tt{}oz} parameters, calls {\tt{}f}, and then pops the
        nested coordinate system.
\end{itemize}


\section{Implementation}

\nwfilename{xformstack.nw}\nwbegincode{1}\sublabel{NWxfoD-xfoE-1}\nwmargintag{{\nwtagstyle{}\subpageref{NWxfoD-xfoE-1}}}\moddef{xformstack.lua~{\nwtagstyle{}\subpageref{NWxfoD-xfoE-1}}}\endmoddef
\LA{}data~{\nwtagstyle{}\subpageref{NWxfoD-dat4-1}}\RA{}
\LA{}functions~{\nwtagstyle{}\subpageref{NWxfoD-fun9-1}}\RA{}
\nwnotused{xformstack.lua}\nwendcode{}\nwbegindocs{2}\nwdocspar

\subsection{The transform stack}

The size of the stack is maintained in the field $n$.

\nwenddocs{}\nwbegincode{3}\sublabel{NWxfoD-dat4-1}\nwmargintag{{\nwtagstyle{}\subpageref{NWxfoD-dat4-1}}}\moddef{data~{\nwtagstyle{}\subpageref{NWxfoD-dat4-1}}}\endmoddef
xform_stack = \{n = 0\};

\nwused{\\{NWxfoD-xfoE-1}}\nwendcode{}\nwbegindocs{4}\nwdocspar

Note that we compose new transformations onto the stack, interpreting
the transformation argument as a transformation into the current coordinates
instead of a transformation into the global coordinates.

\nwenddocs{}\nwbegincode{5}\sublabel{NWxfoD-fun9-1}\nwmargintag{{\nwtagstyle{}\subpageref{NWxfoD-fun9-1}}}\moddef{functions~{\nwtagstyle{}\subpageref{NWxfoD-fun9-1}}}\endmoddef
function xform_push(T)
  local n = xform_stack.n + 1
  if n == 1 then
    xform_stack[n] = T
  else
    xform_stack[n] = xform_compose(xform_stack[n-1], T);
  end
  xform_stack.n = n;
end

\nwalsodefined{\\{NWxfoD-fun9-2}\\{NWxfoD-fun9-3}\\{NWxfoD-fun9-4}\\{NWxfoD-fun9-5}\\{NWxfoD-fun9-6}\\{NWxfoD-fun9-7}\\{NWxfoD-fun9-8}\\{NWxfoD-fun9-9}\\{NWxfoD-fun9-A}\\{NWxfoD-fun9-B}\\{NWxfoD-fun9-C}}\nwused{\\{NWxfoD-xfoE-1}}\nwendcode{}\nwbegindocs{6}\nwdocspar

\nwenddocs{}\nwbegincode{7}\sublabel{NWxfoD-fun9-2}\nwmargintag{{\nwtagstyle{}\subpageref{NWxfoD-fun9-2}}}\moddef{functions~{\nwtagstyle{}\subpageref{NWxfoD-fun9-1}}}\plusendmoddef
function xform_pop()
  xform_stack.n = xform_stack.n - 1
end

\nwendcode{}\nwbegindocs{8}\nwdocspar

\nwenddocs{}\nwbegincode{9}\sublabel{NWxfoD-fun9-3}\nwmargintag{{\nwtagstyle{}\subpageref{NWxfoD-fun9-3}}}\moddef{functions~{\nwtagstyle{}\subpageref{NWxfoD-fun9-1}}}\plusendmoddef
function top_xform(x)
  local n = xform_stack.n
  if n > 0 then
    return xform_apply(xform_stack[n], x)
  else
    return x
  end
end 

\nwendcode{}\nwbegindocs{10}\nwdocspar

\subsection{Transformation functions}

\nwenddocs{}\nwbegincode{11}\sublabel{NWxfoD-fun9-4}\nwmargintag{{\nwtagstyle{}\subpageref{NWxfoD-fun9-4}}}\moddef{functions~{\nwtagstyle{}\subpageref{NWxfoD-fun9-1}}}\plusendmoddef
-- Overwrite x with Tx
function xform_apply(T, x)
  local y1 = T[1]*x[1] + T[4]*x[2] + T[7]*x[3] + T[10];
  local y2 = T[2]*x[1] + T[5]*x[2] + T[8]*x[3] + T[11];
  local y3 = T[3]*x[1] + T[6]*x[2] + T[9]*x[3] + T[12];
  x[1] = y1;
  x[2] = y2;
  x[3] = y3;
  return x;
end

\nwendcode{}\nwbegindocs{12}\nwdocspar

\nwenddocs{}\nwbegincode{13}\sublabel{NWxfoD-fun9-5}\nwmargintag{{\nwtagstyle{}\subpageref{NWxfoD-fun9-5}}}\moddef{functions~{\nwtagstyle{}\subpageref{NWxfoD-fun9-1}}}\plusendmoddef
-- Overwrite x with A_T * x
function xform_applyA(T, x)
  local y1 = T[1]*x[1] + T[4]*x[2] + T[7]*x[3]; 
  local y2 = T[2]*x[1] + T[5]*x[2] + T[8]*x[3]; 
  local y3 = T[3]*x[1] + T[6]*x[2] + T[9]*x[3];
  x[1] = y1;
  x[2] = y2;
  x[3] = y3;
  return x;
end

\nwendcode{}\nwbegindocs{14}\nwdocspar

Recall that 
\[
  T(S(x)) = A_T (A_S x + b_S) + b_T = (A_T A_S) x + T(b_T)
\]

\nwenddocs{}\nwbegincode{15}\sublabel{NWxfoD-fun9-6}\nwmargintag{{\nwtagstyle{}\subpageref{NWxfoD-fun9-6}}}\moddef{functions~{\nwtagstyle{}\subpageref{NWxfoD-fun9-1}}}\plusendmoddef
-- Return TS
function xform_compose(T, S)
  local TS = \{\};
  for k = 1,4 do
    local base = 3*k-3;
    TS[base+1] = T[1]*S[base+1] + T[4]*S[base+2] + T[7]*S[base+3];
    TS[base+2] = T[2]*S[base+1] + T[5]*S[base+2] + T[8]*S[base+3];
    TS[base+3] = T[3]*S[base+1] + T[6]*S[base+2] + T[9]*S[base+3];
  end
  TS[10] = TS[10] + T[10];
  TS[11] = TS[11] + T[11];
  TS[12] = TS[12] + T[12];
  return TS;
end

\nwendcode{}\nwbegindocs{16}\nwdocspar

\nwenddocs{}\nwbegincode{17}\sublabel{NWxfoD-fun9-7}\nwmargintag{{\nwtagstyle{}\subpageref{NWxfoD-fun9-7}}}\moddef{functions~{\nwtagstyle{}\subpageref{NWxfoD-fun9-1}}}\plusendmoddef
-- Return the identity
function xform_identity()
  return \{ 1, 0, 0,
           0, 1, 0,
           0, 0, 1,
           0, 0, 0 \};
end

\nwendcode{}\nwbegindocs{18}\nwdocspar

Getting right hand rotations correct is always a trick.
Important note -- remember that the transformation matrix is
interpreted as column-major, but is ``typographically'' row
major.  So put on your transposing hat before reading the
following three functions.

\nwenddocs{}\nwbegincode{19}\sublabel{NWxfoD-fun9-8}\nwmargintag{{\nwtagstyle{}\subpageref{NWxfoD-fun9-8}}}\moddef{functions~{\nwtagstyle{}\subpageref{NWxfoD-fun9-1}}}\plusendmoddef
-- Return a rotation about the x axis
function xform_ox(r)
  local c = cos(r)
  local s = sin(r)
  return \{ 1,   0,  0,
           0,   c,  s,
           0,  -s,  c,
           0,   0,  0 \}
end

\nwendcode{}\nwbegindocs{20}\nwdocspar

\nwenddocs{}\nwbegincode{21}\sublabel{NWxfoD-fun9-9}\nwmargintag{{\nwtagstyle{}\subpageref{NWxfoD-fun9-9}}}\moddef{functions~{\nwtagstyle{}\subpageref{NWxfoD-fun9-1}}}\plusendmoddef
-- Return a rotation about the y axis
function xform_oy(r)
  local c = cos(r)
  local s = sin(r)
  return \{ c,   0, -s,
           0,   1,  0,
           s,   0,  c,
           0,   0,  0 \}
end

\nwendcode{}\nwbegindocs{22}\nwdocspar

\nwenddocs{}\nwbegincode{23}\sublabel{NWxfoD-fun9-A}\nwmargintag{{\nwtagstyle{}\subpageref{NWxfoD-fun9-A}}}\moddef{functions~{\nwtagstyle{}\subpageref{NWxfoD-fun9-1}}}\plusendmoddef
-- Return a rotation about the z axis
function xform_oz(r)
  local c = cos(r)
  local s = sin(r)
  return \{ c,   s,  0,
          -s,   c,  0,
           0,   0,  1,
           0,   0,  0 \}
end

\nwendcode{}\nwbegindocs{24}\nwdocspar

\nwenddocs{}\nwbegincode{25}\sublabel{NWxfoD-fun9-B}\nwmargintag{{\nwtagstyle{}\subpageref{NWxfoD-fun9-B}}}\moddef{functions~{\nwtagstyle{}\subpageref{NWxfoD-fun9-1}}}\plusendmoddef
-- Return a translation 
function xform_translate(z)
  return \{ 1,    0,    0,
           0,    1,    0,
           0,    0,    1,
           z[1], z[2], z[3] \}
end

\nwendcode{}\nwbegindocs{26}\nwdocspar


\subsection{{\tt{}node} and {\tt{}subnetize} functions}

We assume that node positions are expressed in the current coordinate
system (at least, that's what we assume for most purposes).
So the {\tt{}nodex} (``node transformed'') function transforms
the input coordinates from local to global, and them makes a node.

We still want to leave the option of expressing node coordinates
directly in the global coordinate system, though.  For instance,
we may want to put a node halfway between two other nodes which
have already been transformed into global coordinates.  For this
reason, we leave the {\tt{}node} function alone.

\nwenddocs{}\nwbegincode{27}\sublabel{NWxfoD-fun9-C}\nwmargintag{{\nwtagstyle{}\subpageref{NWxfoD-fun9-C}}}\moddef{functions~{\nwtagstyle{}\subpageref{NWxfoD-fun9-1}}}\plusendmoddef
function nodex(p)
  if p[1] then
    top_xform(p)
  end
  return node(p)
end

\nwendcode{}\nwbegindocs{28}\nwdocspar



\subsection{Test code}

\nwenddocs{}\nwbegincode{29}\sublabel{NWxfoD-xfoC-1}\nwmargintag{{\nwtagstyle{}\subpageref{NWxfoD-xfoC-1}}}\moddef{xformtst.lua~{\nwtagstyle{}\subpageref{NWxfoD-xfoC-1}}}\endmoddef
use("xformstack.lua")

\LA{}test transform constructors~{\nwtagstyle{}\subpageref{NWxfoD-tesR-1}}\RA{}
\LA{}test composition~{\nwtagstyle{}\subpageref{NWxfoD-tesG-1}}\RA{}
\nwnotused{xformtst.lua}\nwendcode{}\nwbegindocs{30}\nwdocspar

\nwenddocs{}\nwbegincode{31}\sublabel{NWxfoD-tesR-1}\nwmargintag{{\nwtagstyle{}\subpageref{NWxfoD-tesR-1}}}\moddef{test transform constructors~{\nwtagstyle{}\subpageref{NWxfoD-tesR-1}}}\endmoddef
-- First check out the basics

rx45 = xform_ox(45)
ry45 = xform_oy(45)
rz45 = xform_oz(45)
trans = xform_translate \{1, 2, 3\}

x1 = xform_apply(rx45,  \{1, 0, 0\})
x2 = xform_apply(ry45,  \{1, 0, 0\})
x3 = xform_apply(rz45,  \{1, 0, 0\})
x4 = xform_apply(trans, \{1, 0, 0\})
x5 = xform_applyA(trans, \{1, 0, 0\})

print("Rotate e1 by rx45: ",     x1[1], x1[2], x1[3])
print("Rotate e1 by ry45: ",     x2[1], x2[2], x2[3])
print("Rotate e1 by rz45: ",     x3[1], x3[2], x3[3])
print("Translate e1 by 1,2,3: ", x4[1], x4[2], x4[3])
print("Apply A e1 by 1,2,3: ",   x5[1], x5[2], x5[3])

\nwused{\\{NWxfoD-xfoC-1}}\nwendcode{}\nwbegindocs{32}\nwdocspar

\nwenddocs{}\nwbegincode{33}\sublabel{NWxfoD-tesG-1}\nwmargintag{{\nwtagstyle{}\subpageref{NWxfoD-tesG-1}}}\moddef{test composition~{\nwtagstyle{}\subpageref{NWxfoD-tesG-1}}}\endmoddef
-- Now check out composition

T = xform_compose(trans, rz45)
x = xform_apply(T, \{1, 0, 0\})

print("Rotate then translate: ", x[1], x[2], x[3])

undoT = xform_compose(xform_oz(-45), xform_translate\{-1,-2,-3\})
xform_apply(undoT, x)

print("After undo operation: ", x[1], x[2], x[3])

\nwused{\\{NWxfoD-xfoC-1}}\nwendcode{}

\nwixlogsorted{c}{{data}{NWxfoD-dat4-1}{\nwixu{NWxfoD-xfoE-1}\nwixd{NWxfoD-dat4-1}}}%
\nwixlogsorted{c}{{functions}{NWxfoD-fun9-1}{\nwixu{NWxfoD-xfoE-1}\nwixd{NWxfoD-fun9-1}\nwixd{NWxfoD-fun9-2}\nwixd{NWxfoD-fun9-3}\nwixd{NWxfoD-fun9-4}\nwixd{NWxfoD-fun9-5}\nwixd{NWxfoD-fun9-6}\nwixd{NWxfoD-fun9-7}\nwixd{NWxfoD-fun9-8}\nwixd{NWxfoD-fun9-9}\nwixd{NWxfoD-fun9-A}\nwixd{NWxfoD-fun9-B}\nwixd{NWxfoD-fun9-C}}}%
\nwixlogsorted{c}{{test composition}{NWxfoD-tesG-1}{\nwixu{NWxfoD-xfoC-1}\nwixd{NWxfoD-tesG-1}}}%
\nwixlogsorted{c}{{test transform constructors}{NWxfoD-tesR-1}{\nwixu{NWxfoD-xfoC-1}\nwixd{NWxfoD-tesR-1}}}%
\nwixlogsorted{c}{{xformstack.lua}{NWxfoD-xfoE-1}{\nwixd{NWxfoD-xfoE-1}}}%
\nwixlogsorted{c}{{xformtst.lua}{NWxfoD-xfoC-1}{\nwixd{NWxfoD-xfoC-1}}}%
\nwbegindocs{34}\nwdocspar

\nwenddocs{}
