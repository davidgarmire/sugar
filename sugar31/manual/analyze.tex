
% \section{Types of analysis}
% \section{Viewing analysis results}
% \section{Static analysis}
% \section{Modal analysis}
% \section{Steady-state analysis}
% \section{Transient analysis}

\section{Types of analysis}

SUGAR supports three basic styles of analysis:
\begin{description}
\item[Static analysis]:
  In static analysis, we find the equilibrium state of a device.
Static analysis is sometimes called DC analysis by analogy to
the equilibrium analysis for direct current circuits.

\item[Linearized analysis]:
  A linearized approximation to a system near equilibrium can
provide valuable information about the stability of the system
and the nature of small oscillations about equilibrium.  SUGAR
provides two flavors of linearized analysis:
\begin{itemize}
\item In \emph{modal analysis}, the characteristic modes of the
      system (and their corresponding frequencies) are determined.
      SUGAR can display the shapes of the displacements corresponding
      to various modes.
\item In \emph{steady-state analysis}, SUGAR computes the frequency 
      response of a user-specified variable when another user-specified
      variable is sinusoidally excited.  The output of steady
      state analysis is Bode plots.
\end{itemize}

\item[Transient analysis]:
  In transient analysis (or dynamic analysis), the motion of the system
is integrated forward in time.  Transient analysis in SUGAR is still
somewhat unreliable; we hope to have better support for it soon.

\end{description}


%\section{Viewing analysis results}
%
%All the analyses except steady state analysis result in one or more
%\emph{displacement}, which represent how much the device is displaced
%from its original state as described in the netlist.  In static analysis,
%the displacement represents the equilibrium position; in modal analysis,
%displacements represent the shapes of the fundamental vibrating modes; 
%and in transient analysis, the displacements represent snapshots of the 
%device in time.


\section{Static analysis}

In static analysis, we attempt to find an equilibrium state for a MEMS
device.  In the most general case, the equilibrium may not be unique;
in this case, SUGAR will usually find the equilibrium position closest
to where it starts looking (which, by default, is the undisplaced position).

The equilibrium state is characterized by a collection of force and moment
balance equations (and their electrical and thermal analogues):
\[
  F(x) = 0
\]
where $x$ is a vector of displacements from the original positions
(and voltages, temperatures, etc) of the device.
We solve these equations using a standard Newton-Raphson
iteration.  For linear problems, a Newton-Raphson iteration will converge
in one steps; for nonlinear problems, the iteration may never converge.
Currently, SUGAR assumes the iteration has converged when the size of the
change between iterations is sufficently small in an appropriately scaled
norm.  If convergence has not set in after 40 iterations, the routine
exits with a diagnostic message.

The function to perform static analysis is \texttt{cho\_dc}:
\begin{center}
\begin{verbatim}
res = cho_dc(net, q0, is_sp)
\end{verbatim}
\end{center}
The first argument, \texttt{net}, is the netlist structure
returned from \texttt{cho\_load}.  The other arguments are optional.
The starting value for the iteration is given by \texttt{q0};
by default, the iteration starts at the undisplaced position 
($\mathtt{q0} = 0$).  The flag \texttt{is\_sp} tells the routine
whether it should use sparse solvers or not; by default, the flag is
true (sparse solvers are used).  The function returns a vector
of displacements to reach the computed equilibrium (\texttt{res.q}),
and a flag that indicates whether the iteration converged 
(\texttt{res.converged}).

In some cases, it is possible to find tricky equilibrium positions
by approaching them step-by-step.  For example, suppose we wanted
to determine the equilibrium position of a device near a pull-in
voltage.  As we approach the critical voltage, it becomes more
difficult to find the equilibrium position, and past the critical
voltage, no equilibrium exists.  If the commands
\begin{verbatim}
  params.V = Vfinal;
  net = cho_load('device.net', params);
  q = cho_dc(net)
\end{verbatim}
fail, we could try
\begin{verbatim}
  q = [];
  for V = 0:.5:Vfinal
    params.V = V
    net = cho_load('device.net', params);
    q = cho_dc(net, q);
  end
\end{verbatim}
Even if we were still unable to find the equilibrium position, we might
get useful information from seeing how nearly we were able to approach
the final voltage, and what the equilibrium was at the last point
where we were able to find it.

There are two ways to view the results of a static analysis:
\begin{enumerate}

\item 
We can view individual components of the displacement vector using
the command \texttt{cho\_dq\_view}:
\begin{verbatim}
% Find displacement of the y coordinate at node 'tip'
tipy = cho_dq_view(q, net, 'tip', 'y');
\end{verbatim}
Alternately, we could look up the index of the tip $y$ coordinate,
and then look at the corresponding entry of the \texttt{q} vector:
\begin{verbatim}
% Find displacement of the y coordinate at node 'tip'
tipy_index = lookup_coord(net, 'tip', 'y');
tipy = q(tipy_index);
\end{verbatim}

\item
We can display the shape of the displaced structure using the
\texttt{cho\_display} routine:
\begin{verbatim}
% Display the undisplaced structure in figure 1,
% and the displaced structure in figure 2.
q = cho_dc(net);
figure(1); cho_display(net);
figure(2); cho_display(net, q);
\end{verbatim}

\end{enumerate}


\section{Steady-state and Modal Analysis}

We determine system behavior near equilibrium by analyzing the linearized
system.  In modal analysis, we find the resonant behavior of the structure,
assuming no damping, by solving the eigenproblem
\[
  \det(\lambda^2 M - K) = 0
\]
The eigenvalues give the resonant frequencies, and the corresponding
eigenvectors give the resonant modes.  The routine \verb|cho_mode| returns
selected frequencies and mode shapes for a structure, along with the
operating point at which linearization took place.  Mode shapes can
be viewed graphically using the \verb|cho_modeshape| commands.  For small
problems, the default dense solvers are adequate; for larger problems,
users should select the number of modes they want, and those modes
will be computed using a less expensive iterative method.

To use the steady-state analysis routine, a user specifies a single
input degree of freedom and a single output degree of freedom,
usually by naming a nodal variable.  SUGAR then draws a Bode plot
illustrating the amplitude gain and phase shift between a harmonic
excitation at the input and a measured harmonic at the output.  Note
that, unlike the modal analysis routine, the steady state routine
does not discard damping terms.

%To determine the steady-state response, SUGAR first linearizes the system of
%ordinary differential equation at the point of static equilibrium. The second
%order system of ODEs is then converted into first order form given by  
%\begin{eqnarray}
%  \dot{x} & = & Ax + Bu \\
%  y       & = & Cx + Du
%\end{eqnarray}
%where $x$ is the system dynamic state variable, $u$ is the sinusoidal external
%excitation, and $y$ is the system dynamic response. $A$, $B$, $C$, and $D$ are
%the system, input coupling, output, and feed forward matrices respectively [1].
%The solution of equation provides Bode plots as well as modal analysis.  


\section{Transient Analysis}

\emph{Note: Note yet in SUGAR 3.0}

%The transient solver simulates the dynamic response of a MEMS device.
%The device model may contain nonlinear elements, non-harmonic excitations,
%and other features which would make linearized analysis suspect.

%This solver calculates the transient response of a MEMS device, which may
%contain nonlinear elements and excitations that are functions of time t and
%state vector q. Several ODE solvers are available, whereby speed may be traded
%for accuracy and long-term stability. These numerical methods include an
%implicit second order Rosenbrock solver for stiff problems where low accuracy
%is acceptable, an explicit Runge-Kutta 4th-5th order solver for non-stiff
%systems, an implicit multi-step integration method of varying order for stiff
%problems requiring higher accuracy, and a simple explicit Euler algorithm. The
%transient solvers require the system ODEs to be in first order form. We do this
%in the standardized way [2] by introducing a new state vector $Q$ where 
%\[
%  \dot{Q} = 
%    \frac{d}{dt} 
%      \left( 
%        \begin{array}{l} 
%          x \\ 
%          \dot{x} 
%        \end{array} 
%      \right) = f(t,x)
%\]


%\section{Future analysis routines}
%
%In the future, we plan to also support \emph{sensitivity analysis}.
%Sensitivity analysis is not an independent style of analysis as much
%as it is an extension to the forms of analysis listed above.
%For example, a static sensitivity analysis might tell how the
%equilibrium position would change due to variations from the nominal
%material properties, layer thicknesses, etc.  Similarly, sensitivity
%analysis used with the linearized analysis routines might tell how the
%fundamental frequencies of the device would change if the device
%properties were perturbed, and sensitivity analysis of transient results
%would tell how the dynamic response would be affected by perturbing
%device properties.  Some forms of sensitivity analysis were supported
%in SUGAR 1.01, but that code has not yet been integrated into SUGAR 2.0.
%

%\section{Sensitivity Analysis}

%The predicted performance of MEMS is subject to process variations that occur
%during the fabrication stage. Small variations in geometry may lead to
%performance, which is substantially different from the ideal. SUGAR models this
%effect by perturbing the system stiffness matrix K. 
%For static analysis 
%\[
%  F = [K(g) + \Delta K(d)] x
%\]
%where $g$ contains the ideal layout geometry and process file parameters. $d$
%is the variation in $g$ given by Gaussian, uniform, or uniform distribution of
%corners. The perturbation in stiffness, $\Delta K(d)$, is determined by both
%probabilistic and deterministic techniques. 

%A Monte-Carlo algorithm [3] evaluates the most likely outcome of equation
%(3.5) by drawing many samples from a random distribution of geometric
%variables.  Given the bounds on the geometric variables, an Ellipsoidal
%Calculus technique [4] is used to find the extreme bounds on performance
%parameters, i.e. the worst-case scenario. 

%  References 
% 
% [1]  Norman S.Nise, "Control System engineering,"  The Benjamin/Cummings 
%      Publishing, Inc, CA (1991) 
% [2]  S. Sastry, "Nonlinear Systems Analysis, Stabil ity, and Control," Springer (1999) 
% [3]   D.J.C MacKay, "Introduction to Monte Car lo Methods," M. I. Jordan, (1999) 
% [4]  G. Calafiore, L. El Ghaoui, "Confidence ellipsoids for uncertain linear equations 
%      with structure," In Proceedings of the IEEE CDC , (1999) 
