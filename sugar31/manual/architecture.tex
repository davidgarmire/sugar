
% Code architecture

There are three major components to SUGAR:
\begin{itemize}
 \item Model functions which describe how to check parameters, write
       equations, and display individual elements of a device.
 \item Routines to assemble and analyze device equations and display
       results; and
 \item A parser and preprocessor, written using C and the compiler tools
       bison and flex;
\end{itemize}

The three components are respectively in the \texttt{model},
\texttt{analysis}, and \texttt{compile} subdirectories of the SUGAR 
distribution.  We have listed the pieces in order from ``most likely to
be modified by a casual user'' to ``least likely to be modified by
a casual user.''  It seems likely that many users will want to add
their own model functions, and the code is organized so that it should
be possible to do this without learning about any of the rest of
the code base.  Some users may want to try out new forms of analysis,
or implement new solvers; in order to do this, they will need a working
knowledge of how to work with the Matlab representation of the device,
and how analysis routines interact with the model functions.  The
brave few who decide to extend the netlist language, or otherwise modify
SUGAR in a fundamental way will need to understand the parser and
preprocessor routines along with everything else.  However, we have tried
to keep the design modular, so that it is possible to change pieces of
the code without understanding the entirety.

(Still need to describe the communication between the major components
of the system, perhaps draw a pretty picture, and tell users which chapter
they need to look at to make which common extension.  But even with that,
this will be a short section.)
