
%\section{Dissecting an example netlist}
%\section{Units and metric suffixes}
%\section{Expressions}
%\section{Rotations and Euler angles}
%\section{Lexical notes}
%\section{Using library files}
%\section{Parameters and definitions}
%\section{Material and processing information}
%\section{Element lines}
%\section{Subnets}
%\section{Arrays}
%\section{Conditionals}

Devices in SUGAR are described by input files called \emph{netlists}.
In this chapter, we describe the features of the netlist language which is
a derivative of the Lua language.

\section{Units and metric suffixes}

By convention, the SUGAR model functions use the familiar MKS
(meter-kilogram-second) system of units.  This means that beam lengths, for
example, are measured in meters instead of micrometers.  In order to make it
easier to type lengths of microns and pressures of gigapascals, we adopt a
standard system of metric suffixes that can be appended to SUGAR numbers.  For
example, a hundred micron length in SUGAR could be represented as \texttt{100u}
as well as in scientific notation (\texttt{100e-6}) or as a simple decimal
(0.0001).  Similarly, a Young's modulus of 150 gigapascals might be written as
\texttt{150G}.  Both suffixes and ordinary scientific notation can be used
together, too; \texttt{0.1e7u} is a perfectly legitimate (if somewhat silly)
way of writing the number 1.

The standard suffixes are:
\begin{center}
\begin{tabular}{|l|ll||l|ll|}
\hline
  \texttt{d} & deci  & $10^{-1}$  &              &       &          \\
  \texttt{c} & centi & $10^{-2}$  &   \texttt{h} & hecto & $10^2$   \\
  \texttt{m} & milli & $10^{-3}$  &   \texttt{k} & kilo  & $10^3$   \\
  \texttt{u} & micro & $10^{-6}$  &   \texttt{M} & mega  & $10^6$   \\
  \texttt{n} & nano  & $10^{-9}$  &   \texttt{G} & giga  & $10^9$   \\
  \texttt{p} & pico  & $10^{-12}$ &   \texttt{T} & tera  & $10^{12}$  \\
  \texttt{f} & femto & $10^{-15}$ &   \texttt{P} & peta  & $10^{15}$  \\
  \texttt{a} & atto  & $10^{-18}$ &   \texttt{E} & exa   & $10^{18}$  \\
\hline
\end{tabular}
\end{center}


\section{Expressions}

It's often convenient to do simple calculations inside a netlist.
For example, suppose I have defined a variable \texttt{beamL} for
the length of a beam in a device.  Then we can define the length
of another beam in terms of \texttt{beamL}:
\begin{verbatim}
  use("mumps.net")
  use("stdlib.net")

  beamL = 100u   -- Make a beam one hundred microns long

  -- make a beamL by beamL/2 rectangle
  A = node{0, 0, 0; name = "A"}
  B = node{name = "B"}
  C = node{name = "C"}
  D = node{name = "D"}
  
  beam3d{ A, B ; material=p1, w=2u, l=beamL }
  beam3d{ C, D ; material=p1, w=2u, l=beamL }
  beam3d{ A, C ; material=p1, w=2u, l=beamL/2, oz=90 }
  beam3d{ B, D ; material=p1, w=2u, l=beamL/2, oz=90 }
\end{verbatim}

SUGAR's treatment of expressions are similar to Lisp's treatment of expressions: we can 
add, subtract, multiply, divide, exponentiate, and negate numbers, and also 
evaluate functions. All functionality described in Lua is also available in 
SUGAR. All operations are left associative, so \texttt{a + b + c} is 
evaluated as 
\texttt{(a + b) + c} rather than as \texttt{a + (b + c)}.  The order of operations, from highest precedence
to lowest precedence, is
\begin{center}
\begin{tabular}{|l|l|}
\hline
Level & Operators \\
7 & \verb+|+ (logical or)\\
6 & \verb+&+ (logical and)\\
5 & \texttt{==} (equality), \texttt{!=} (inequality), 
    \texttt{>} (greater), \texttt{<} (less), 
    \texttt{>=} (greater or equal), \texttt{<=} (less or equal)\\
4 & \texttt{-} (add), \texttt{+} (subtract)\\
3 & \texttt{*} (multiply), \texttt{/} (divide)\\
2 & \texttt{-} (negate), \texttt{!} (logical not)\\
1 & \verb+^+ (exponentiate) \\
\hline
\end{tabular}
\end{center}

As in C and MATLAB, there is no separate logical type.  Non-zero numbers
are interpreted as ``true,'' and zero is interpreted as ``false.''  When
a comparison or logical operation is true, it will evaluate to 1.

Besides numbers, SUGAR supports a string type.  String literals are denoted
by double quotes.  Unlike C and MATLAB, the backslash does not quote
characters in SUGAR expressions: \verb+"\t"+ is a backslash followed by
a t, not a tab.  Strings are values as well and can be used as operands
to arithmetic or logical operations. 

It is useful to note that a faster way of writing
\begin{verbatim}
  if a then
    x = a
  elseif b then
    x = b
  else
    x = c
  end
\end{verbatim}
is to form it as
\begin{verbatim}
  x = a or b or c
\end{verbatim}

Lua for SUGAR also contains the primitives {\tt addpath},
{\tt node}, {\tt material}, {\tt element}, and {\tt subnet}, which will be 
discussed more thoroughly later. {\tt addpath} allows the inclusion of other 
directories for the {\tt use} command. {\tt node} defines a node in 3 space 
for use later in defining elements. {\tt element} invokes a model defined on 
a set of nodal points with a set of parameters. {\tt subnet} uses a function 
like declaration to parameterize netlists of sub-components.


\section{Rotation angles}

In order to orient structures in SUGAR, users specify how each piece is rotated
from a model coordinate frame into its actual orientation in the structure.
This specification can be done according to a global reference frame by specifying
the exact positions of the nodes (and allowing SUGAR to figure out the internal
rotations) or by specifying rotations of each component (may be done hierarchically 
in the case of subnets) and having SUGAR position the nodes.
Rotations are specified by a sequence of rotations about the $x$, $z$, and $y$
axes, respectively.  The amount to rotate about each axis is given by angles
\texttt{ox}, \texttt{oz}, and \texttt{oy}, given in degrees.

%; these three
%numbers are known as \emph{Euler angles}, and can be used to describe any
%rotation. Note that we can give 

For many applications, most of the structure will lie initially
in a single plane, and the only rotations used to describe the device
will be rotations about the $z$ axis.  For example, the lines
\begin{verbatim}
  beam3d{ A, B ; material=p1, w=2u, l=beamL }
  beam3d{ C, D ; material=p1, w=2u, l=beamL }
  beam3d{ A, C ; material=p1, w=2u, l=beamL/2, oz=90 }
  beam3d{ B, D ; material=p1, w=2u, l=beamL/2, oz=90 }
\end{verbatim}
describe a rectangle; the first two beams run parallel to the $x$ axis,
and the latter two beams are rotated ninety degrees in the plane to run
parallel to the $y$ axis.

For more complicated examples, it is important to remember that order
matters.  To go from local coordinates to global coordinates, first the
rotation about the $x$ axis is applied, then the $z$ axis, and then the $y$
axis.  For instance, in the model coordinate system, a beam points in the
direction $(1,0,0)$, along the $x$ axis.  If we rotate the beam first $90$
degrees about the $x$ axis and then $90$ degrees about the $z$ axis, it would
point along the $y$ axis, in the $(0,1,0)$ direction.  If, however, we were to
rotate the beam first about the $z$ axis and then about the $x$ axis, it would
end up pointing along the $z$ axis.


\section{Lexical notes}

SUGAR 3.0 netlists are ``free form''; that is, white space characters
like tabs and carriage returns are not significant.

A comment in a netlist begins with \texttt{--} and extends to the end of
the line. 

A SUGAR identifier (like a C identifier) consists of a letter
followed by a string of letters, numbers, and underscores.  
The keywords \texttt{use}, \texttt{subnet}, \texttt{addpath}, 
\texttt{element}, and \texttt{node} are reserved (along with a slew of Lua
commands), and cannot be used as identifiers. See Lua documentation for
more details.

\section{\texttt{use} statements}

Netlists can contain \texttt{use} statements to include other files.
A \texttt{use} statement has the form
\begin{verbatim}
  use("filename")
\end{verbatim}
For example, many netlists use the data for MUMPS process layers defined in 
\texttt{mumps.net}, and begin with the line
\begin{verbatim}
  use("mumps.net")
\end{verbatim}
Files included by a \texttt{use} statement are not particularly
special. You can use \texttt{use} to include files of process parameters,
libraries of frequently-used subnets, etc.  

A file will only be used once in a netlist.  
For example, if the file \texttt{subnets.net} started with
\begin{verbatim}
  use("mumps.net")
\end{verbatim}
and a test netlist called \texttt{test.net} started with
\begin{verbatim}
  use("subnets.net")
  use("mumps.net")
\end{verbatim}
then \texttt{test.net} would only include \texttt{mumps.net} once, and 
would not complain about the contents of \texttt{mumps.net} being defined 
multiple times.


\section{\texttt{addpath} statements}

Netlists can specify directories so the \texttt{use} statement does not need
to specify the entire path, only a filename. For example,
\begin{verbatim}
  use("src/lua/base/mumps.net")
  use("src/lua/base/std.net")
\end{verbatim}
could be rewritten as
\begin{verbatim}
  addpath("src/lua/base")
  use("mumps.net")
  use("std.net")
\end{verbatim}


\section{\texttt{node} statements}

%Nodes are points in space that elements use to define their positions. 
%Unlike
%version 2.0, 3.0 requests the nodes to be declared outside of the element
%declarations. Both relative and absolute nodal positions are then possible to
%be achieved. Components can be separated by gaps.

Nodes are connection points which have associated variables shared
by attached elements.  Mechanical nodes also have positions.

An example of an absolute nodal declaration of node \texttt{A} at position 
$(0, 0, 0)$ is
\begin{verbatim}
  A = node{0, 0, 0; name = "A"}
\end{verbatim}

An example of a relative nodal declaration of node \texttt{B} is
\begin{verbatim}
  B = node{name = "B"}
\end{verbatim}
When relative node positioning is used, SUGAR automatically ensures 
all of the node positions are generated for each of the elements.

The syntax
\begin{verbatim}
  node "A"
\end{verbatim}
is used to refer to a node named ``A'' in the current scope.
If no such node exists yet, then a new one is created.  For example,
if we write
\begin{verbatim}
eground {node "ground"}
Vsrc    {node "ground", node "attach"; V=10}
\end{verbatim}
the node ``ground'' is created in the first line, and the reference to
\verb|node "ground"| in the following line refers to the same node.


\section{\texttt{element} statements}

The basic unit of a SUGAR netlist is an element line. For example,
\begin{verbatim}
  crossbeam = element{ A, B; model="beam2d", material=p1, l=100u, w=2u }
\end{verbatim}
is an element line describing a beam.  This line consists of several
fields:
\begin{itemize}

\item Before the equals sign, one may optionally put a name for that element;  
   in this case, the element is named \texttt{crossbeam}.

\item Before the semicolon is a list of nodes.
   In this case, the specified beam connects nodes \texttt{A} and \texttt{B}.
   Elements are connected together by sharing a common node.  For instance, to
   attach a wider 100 micron beam to the \texttt{B} end of the beam above, 
   we might write
\begin{verbatim}
  element { B, C; model="beam2d", material=p1, l=100u, w=5u }
\end{verbatim}

   Unlike in previous versions of SUGAR, node names in SUGAR 3.0 must
   begin with an alphabetic character. 

\item After the list of nodes comes a list of element parameters. Usually a
   process contains a model, a process, and a list of parameters.

   The name of the model for the element is specified as 
   \texttt{model="<model name>"}; in this case,
   it is the two-dimensional beam model \texttt{beam2d}.  There are models
   for beams, anchors, electrical devices, etc.; a complete list of
   models, along with information on how to build new models, can
   be found later in this document.

   The process parameter structure for the element is specified as
   \texttt{material=<process name>}; 
   in this case the beam is fabricated in the first layer of polysilicon
   in a MUMPS process, named \texttt{p1}.  By specifying the process layer 
   \texttt{p1},
   a user informs the model function of common material parameters such as 
   the Young's modulus for polysilicon and the thickness of the deposited
   layer.  For models which require no process information, such as models 
   for external forces, the process field may be omitted.

   In the example above, the model parameters consisted of the length and width
   of a beam; other models will require other parameters.  A parameter
   specification always has the form \texttt{identifier = expression}.
   All permutations of arguments connected by commas are equivalent.

\end{itemize}

All the basic models in SUGAR have corresponding \emph{subnet} wrappers.
For example, these wrappers allow you to rewrite
\begin{verbatim}
  element { B, C; model="beam2d", material=p1, l=100u, w=5u }
\end{verbatim}
as
\begin{verbatim}
  beam2d{ B, C; material=p1, l=100u, w=5u }
\end{verbatim}
The preferred way to invoke elements is with the latter syntax.


\section{Variables}

In order to allow users to experiment with variations on a simulation,
or to do parameter sweeps, SUGAR allows users to define variables in
the Lua environment at load time.  For example, a user might write
\begin{verbatim}
  > params.nfingers = 10;
  > net = cho_load('comb.net', params);
\end{verbatim}
to create an instance of \texttt{comb.net} with comb drives having
ten fingers.  To test inside the netlist whether a variable is defined,
use the \verb|if| statement
\begin{verbatim}
  if not nfingers then
    nfingers = 10  -- default is ten fingers
  end
\end{verbatim}
a shorter way to write the code above would be as
\begin{verbatim}
  nfingers = nfingers or 10
\end{verbatim}

SUGAR netlists may also include definitions, such as
\begin{verbatim}
  long_length = 200u
  short_length = 100u
  avg_length = (long_length + short_length)/2
\end{verbatim}
Netlist variables are scoped, so that a definition declared with the
keyword, \texttt{local}, made inside
a subnet (see below) will not affect top-level element statements.


\section{\texttt{material} statements - Process parameter structures}

Physical parameters associated with a particular layer of a particular
material are process parameters.  An example of the baseline process
information for the polysilicon layers in MUMPS (\texttt{default}) is provided
below
\begin{verbatim}
  default = material { 
      Poisson = 0.3,               --Poisson's Ratio = 0.3
      thermcond = 2.33,            --Thermal conductivity Si = 2.33e-6/C
      viscosity = 1.78e-5,         --Viscosity (of air) = 1.78e-5
      fluid = 2e-6,                --Between the device and the substrate.
      density = 2300,              --Material density = 2300 kg/m^3
      Youngsmodulus = 165e9,       --Young's modulus = 1.65e11 N/m^2
      permittivity = 8.854e-12,    --permittivity: C^2/(uN.um^2)=(C.s)^2/kg.um^3
      sheetresistance = 20         --Poly-Si sheet resistance [ohm/square]
  }
\end{verbatim}

In general a process definition has the form \texttt{name = material \{...\}},
where \texttt{material} is a keyword, \texttt{name} is the name to be given 
to the process information.

Process parameter structures may be derived from other process parameter
structures.  For example, a 2 micron poly layer named \texttt{p1} might be 
written
\begin{verbatim}
  p1 = material {
      parent = default,
      h = 2u
  }
\end{verbatim}
This layer automatically includes all the definitions made in the default
process parameter structure.


\section{\texttt{subnet} statements}

Subnets provide users with a means to extend the set of available models
without leaving SUGAR.  An example subnet for a single unit of a serpentine
structure is shown below:
\begin{verbatim}
subnet serpent( A, F, material, unitwid, unitlen, beamw )
  beamw = beamw or 2u;
  len2  = unitlen/2
  beam2d{ A, node "b";        material=material, l=unitwid/2, w=beamw, oz=-90 }
  beam2d{ node "b", node "c"; material=material, l=len2,      w=beamw }
  beam2d{ node "c", node "d"; material=material, l=unitwid,   w=beamw, oz=90 }
  beam2d{ node "d", node "e"; material=material, l=len2,      w=beamw }
  beam2d{ node "e", F;        material=material, l=unitwid/2, w=beamw, oz=-90 }
end
\end{verbatim}
Element lines using subnets are invoked in the same manner as element lines
using model functions built in MATLAB
\begin{verbatim}
  serp1 = serpent {X, Y; material=p1, unitwid=10u, unitlen=10u}
  serpent {y, z ; material=p1, unitwid=10u, unitlen=10u, w=3u}
\end{verbatim}
The \texttt{parent} process for a subnet is the process specified in creating
an instance of that subnet.  In the above example, the \texttt{p1} process
information would be used for the beams in the serpent subnet.

In general, a subnet definition consists of the keyword \texttt{subnet},
a name for the subnet model, a set of arguments, 
and a code block followed by \texttt{end}.  The code block
may include definitions, element lines, and array structures (see below).

Sometimes it may be necessary to access variables attached to nodes
internal to a netlist.  For example, in the above example we might be
interested in the version of node \texttt{b} for subnet instance
\texttt{serp1}.  In the analysis functions, that node would be
referred to as \texttt{serp1.b}.  It would not be valid to refer
to node \texttt{x} as \texttt{serp1.A}, since \texttt{x} already has
a name defined outside the subnet.

Subnet instances that are not explicitly named, like the second 
\texttt{serpent} element in the example above, are assigned names
consisting of \texttt{anon} followed by some number.  It is possible
to use a name like \texttt{anon1.b} to refer to the \texttt{b} node
in the second line, but it is not recommended since the internal
naming schemes for anonymous elements are subject to future change.


\section{Arrays}

SUGAR supports arrays of structures through the same syntax as Lua.
Arrays are built enclosed in braces and referenced using brackets.  
For example, the following code fragment creates a spring composed of 
twenty of the serpentine units from the subnet example and anchors it 
at one end:
\begin{verbatim}
link = { node {} };
anchor {p[1]; material=p1, l=5u, w=5u}
for k = 1,nlinks do
  link[k+1] = node {};
  serpent {link[k], link[k+1]; material=p1, unitwid=10u, unitlen=10u, w=2u}
end
\end{verbatim}
Note that \texttt{link} needs to first be initialized.  It is
possible to have names with multiple indices as well (eg \texttt{link[i][j])}).
The index variable is only valid within the loop body.

The general syntax of a for loop is
\begin{verbatim}
  for index = lowerbound, upperbound [, increment] do
    ... code lines ...
  end
\end{verbatim}
where \texttt{index} is the name of the index variable, 
\texttt{lowerbound} is an expression for the lower bound of the loop, and 
\texttt{upperbound} is an espression for the upper bound of a loop.


\section{Conditionals}

SUGAR supports \texttt{if} statements of the form
\begin{verbatim}
  if expression then
    ... code lines ...
  end
\end{verbatim}
and
\begin{verbatim}
  if expression then
    ... code lines ...
  else
    ... code lines ...
  end
\end{verbatim}

The main purpose of \texttt{if} statements is to give some flexibility
to subnet writers.  For example, suppose I wanted a beam that would
automatically compute its electrical resistance if one was not provided.
I could do that with the following subnet:
\begin{verbatim}
  subnet mybeam( A, B, l, w, h, R, resistivity )
    -- If R is nil, then the user didn't specify anything to override
    -- the default, so we'll help calculate the resistance.
    if not R then
      resistance = resistivity * l/(w*h)
      beam3de {A, B; material=parent, l=l, w=w, h=h, R=resistance}

    -- Otherwise, we'll just accept whatever the user wrote in
    else 
      beam3de {A, B; material=parent, l=l, w=w, h=h, R=R}
    end
  end
\end{verbatim}

There are a few caveats that go with this example:
\begin{enumerate}
 \item
       Usually, \texttt{resistivity} would be defined in the process
       information when an instance of \texttt{mybeam} was created.
 \item 
      A briefer way to write this example would be
\begin{verbatim}
  subnet mybeam(A, B, l, w, h, R, resistivity)
    beam3de{A, B; material=parent, l=l, w=w, h=h, 
            R=( R or (resistivity * l/(w*h)) )}
  end
\end{verbatim}
    The \texttt{local} statement restricts the variable to the given scope.
 \item
    The SUGAR netlist language is designed to be extensible, yet it is limited.
    By using the \texttt{if} statement, it is possible to write arbitrarily 
    complicated netlists, with constructs
    like recursive subnets or even more subtle beasts.  Exercise good taste
    when you write netlists, and try to relegate any subtle and complicated
    coding tasks to MATLAB instead of to the SUGAR netlist language.
\end{enumerate}
