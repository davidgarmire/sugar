
% Some examples.

% (From Jason's demo20.pdf -- figures removed.  Need to put them back in,
% and also add some more sophisticated examples (using parameters, etc))
 
This section describes how to use SUGAR 3.0 by examples. The netlists, commands
for running analysis, and output are shown. For convenience, all netlist files
given here are available in the SUGAR demo directory (converted from 2.0). Netlist 
format is defined in chapter 2.  

\section{Examples with explanations}

\subsection{cantilever example (demo\_dc1)}

This  demo  shows  how  to  simulate  the  deflection  of  a  beam  due  to  an
external  force, where the beam is fixed at one end. It shows to make the
netlist, how run static analysis, and obtain graphical output. 

To model the beam a 3D linear beam model called beam3d will be used. If a
planar beam is desired, simply replace the model b3dm with beam2d in line 2 of
following netlist. 

\subsubsection*{Netlist}

The following netlist is created by opening a text editor, entering the 3 lines
of netlist text shown below, and saved as cantilever.net. The 3 text lines
represent an anchor, beam, and force.  
\begin{verbatim}
-- `cantilever.net' 
use("mumps.net")
use("stdlib.net")

anchor {node "substrate"; material=p1, l=10u, w=10u, h=10u}
beam3d {node "substrate", node "tip"; material=p1, l=100u, w=2u, h=2u}
f3d    {node "tip"; F=2u, oz=90}
\end{verbatim}

The first line in the netlist includes a process file ``mumps.net''. All process
information such as layer thickness, Young's modulus etc. are defined in
``mumps.net''.   The second line includes the standard library file
``stdlib.net''.

The first line in the netlist represents the anchor element. Anchors are the
MEMS components that mechanically ground flexible structures to the substrate.
Without anchors, structures would be statically indeterminate.  

The material properties of the anchor are given in process file "mumps.net".
The fabrication layer of this process is p1.  

The anchor is attached to the substrate by the node labeled substrate. Notice
that both the anchor element and beam element (described below) contain the
node labeled substrate.  The anchor is coupled to the beam through node
substrate. 

The parameters section of this line of text provides the geometry and
orientation, where the length, width are 10 microns.  

The second element is a flexible beam. The model used for this beam is called
beam3d, which is described in appendix. The fabrication layer with which this
beam is composed of is the p1 layer (i.e. the first layer of polysilicon). It
is fixed on one end due to its connection to the anchor through the node
labeled substrate. The opposite end of the beam, labeled tip, is free to move.  

The last section of this line provides geometry and orientation. The beam
extends to the right, from node substrate to node tip. 

The final line is a force applied at the free end of the beam (node tip). The
magnitude of the force is given as 2 microNewtons. The orientation of the force
vector is in the y- direction since it was rotated from its default position
along the positive x-axis.  


\subsubsection*{Command}

Once the netlist text file is created, load it into Matlab with the
cho\_load command.
Then static analysis may be
performed, which finds a final equilibrium state of the system. Running static
analysis on the above netlist requires 3 commands.   Within the Matlab
workspace:
\begin{enumerate}
 \item load the netlist, 
 \item perform static analysis on it, and 
 \item display the results: 
\end{enumerate}

\begin{verbatim}
net = cho_load('cantilever.net'); 
dq = cho_dc(net); 
cho_display(net,dq); 
\end{verbatim}

The first command loads the text file called cantilever.net into a variable
called net. The net variable contains all of the important information in the
netlist file.

The second line performs the static analysis. The cho\_dc command takes net as
its input.  Using the parameter values given in the netlist and the
parameterized element models described in appendix, it calculates the
deflection of the structure. The displacement vector dq is the output of
cho\_dc.  

 Incidentally, cho stands for the basic building locks of sugar (i.e. carbon,
hydrogen, oxygen). 

 Using geometries and orientations from net and node displacements from dq as
input to cho\_display, SUGAR can graphically display the deflected structure in
Matlab.

To display original, non-deflected structure, simply type  
\begin{verbatim}
 cho_display(net);  
\end{verbatim}
After the structure is displayed, left clicking and dragging within the
display window may adjust the view. The magnification buttons in the display
window may be used to zoom in and out by first clicking on, say the zoom-in
(+), followed by pointing and clicking on the display window at the precise
position that is to be magnified. 


\subsection{multiple beam example (demo\_dc2)}

This demo is similar to above demo with the exception that it uses multiple
beams and it is deflected by a moment. Netlist is as following: 

\subsubsection*{Netlist}

\begin{verbatim}
-- 'multibeam.net' 
use("mumps.net")
use("stdlib.net")

anchor {node "substrate"; material=p1, l=10u, w=10u, oz=180}
beam3d {node "substrate", node "A"; material=p1, l=100u, w=10u}
beam3d {node "A", node "B"; material=p1, l=50u, w=4u, oz=45}
beam3d {node "A", node "C"; material=p1, l=50u, w=4u, oz=-45}
beam3d {node "C", node "D"; material=p1, l=50u, w=4u, oy=-45}
f3d    {node "D"; M=1n}
\end{verbatim}

As before, each element is connected at shared nodes.
The commands to load the netlist, do the analysis, and 
display the non-deflected and deflected figures are 

\begin{verbatim}
net = cho_load('multibeam.net'); 
dq = cho_dc(net); 
figure(1); cho_display(net);  
figure(2); cho_display(net,dq); 
\end{verbatim} 

 
\subsection{Beam gap structure (demo\_beamgap)}

This is a 2D coupled electrical and mechanical domain analysis. It contains
electrical voltage source, electrical ground, electro-mechanical anchors, beam
and gap. The netlist and structure of this demo are as following: 

\subsubsection*{Netlist}

\begin{verbatim}
-- `beamgap2e.net' 
use "mumps.net"
use "stdlib.net"

Vsrc    {node "A", node "f"; V = 10}
eground {node "f"}
anchor  {node "A", p1; l=5u, w=10u, oz=180}
beam2de {node "A", node "b", p1; 
         l=100u, w=2u, h=2u, R=100}
gap2de  {node "b", node "c", node "D", node "E", p1;
         l=100u, w1=10u, w2=5u, gap=2u, R1=100, R2=100}
eground {node "D"}
eground {node "E"}
anchor  {node "D", p1; l=5u, w=10u, oz=-90}
anchor  {node "E", p1; l=5u, w=10u, oz=-90}
\end{verbatim}


Equilibrium displacements have been calculated at an input voltage 10v. The
deflected structure is shown as following: 
                                                                                         
\subsection{Modal analysis (demo\_mirror)}

This is a 3D mechanical modal analysis for a mirror structure. 3D mechanical
anchors and beams are included. Resonant frequencies have been calculated and
the first to fourth mode shapes are displayed. Below are the netlist and demo
pictures  

\subsubsection*{Netlist}

\begin{verbatim}
-- `mirror.net' 
use "mumps.net"
use "stdlib.net"

anchor {node "b";           material=p1, l=10u,  w=10u, oz= 90, h=8u}
beam3d {node "b", node "c"; material=p1, l=80u,  w=2u,  oz=-90, h=2u}
beam3d {node "d", node "e"; material=p1, l=80u,  w=2u,  oz=-90, h=2u}
anchor {node "e";           material=p1, l=10u,  w=10u, oz=-90, h=8u}

-- outer frame:
beam3d {node "c",  node "f";  material=p1, l=100u, w=20u, oz=-90, h=4u}
beam3d {node "f",  node "d";  material=p1, l=100u, w=20u, oz=-90, h=4u}
beam3d {node "c1", node "f1"; material=p1, l=100u, w=20u, oz=-90, h=4u}
beam3d {node "f1", node "d1"; material=p1, l=100u, w=20u, oz=-90, h=4u}
beam3d {node "c",  node "c1"; material=p1, l=200u, w=20u, oz=  0, h=4u}
beam3d {node "d",  node "d1"; material=p1, l=200u, w=20u, oz=  0, h=4u}

-- inner torsion hinges:
beam3d {node "g3", node "f1"; material=p1, l=40u,  w=2u,   oz=0,  h=2u}
beam3d {node "f",  node "g6"; material=p1, l=40u,  w=2u,   oz=0,  h=2u}

-- inner solid "plate":
beam3d {node "g6", node "g3"; material=p1, l=120u, w=140u, oz=0,  h=4u}

-- rear lever:
beam3d {node "h",  node "f";  material=p1, l=75u,  w=80u,  oz=0,  h=4u}
\end{verbatim}  


\subsection{Steady state analysis (demo\_ss)}

This is a 2D steady state analysis for a resonator as following:  

\subsubsection*{Netlist}

\begin{verbatim}
-- 'multmode_m.net' 
use("mumps.net")
use("stdlib.net")

anchor {node "A";           material=p1, l=5u,   oz=0,   w=10u  }
beam2d {node "A", node "D"; material=p1, l=150u, oz=180, w=2u   }
anchor {node "B";           material=p1, l=5u,   oz=0,   w=10u  }
beam2d {node "B", node "E"; material=p1, l=150u, oz=180, w=2u   }
beam2d {node "D", node "F"; material=p1, l=50u,  oz=90,  w=5u   }
beam2d {node "D", node "E"; material=p1, l=50u,  oz=-90, w=5u   }
beam2d {node "E", node "G"; material=p1, l=50u,  oz=-90, w=5u   }
beam2d {node "F", node "H"; material=p1, l=50u,  oz=0,   w=2u   }
beam2d {node "G", node "L"; material=p1, l=150u, oz=0,   w=2u   }
beam2d {node "H", node "I"; material=p1, l=50u,  oz=0,   w=20u  }
beam2d {node "I", node "O"; material=p1, l=50u,  oz=0,   w=20u  }
beam2d {node "I", node "J"; material=p1, l=75u,  oz=-90, w=20u  }
beam2d {node "J", node "K"; material=p1, l=75u,  oz=-90, w=20u  }
beam2d {node "L", node "K"; material=p1, l=50u,  oz=0,   w=20u  }
beam2d {node "K", node "P"; material=p1, l=50u,  oz=0,   w=20u  }
beam2d {node "J", node "M"; material=p1, l=300u, oz=0,   w=2u   }
beam2d {node "M", node "N"; material=p1, l=196u, oz=0,   w=116u }
\end{verbatim}

%An input excitation of sinusoidal force is applied on the resonator as shown as
%following: 

%The bode plot of the y direction response at the mass is shown below. 

                                                                                            
\subsection{Transient analysis (demo\_ta1gap)}

\emph{Note: This demo has not yet been converted to SUGAR 3.0}

This is a 3D electromechanical transient analysis for a gap-closing actuator. A
piecewise linear voltage V(t) is applied across the page. The voltage V(t)
ramps from 5V at t=10usec to 12V at t=500usec, and then drops to 0V. The
displacement component of node C in the direction of force is observed below.
The initial voltage step causes the device to oscillate. At the voltage
increases at a linear rate, the gap decreases at a nonlinear rate due to the
electrostatic force increasing proportionally to $1/\mathit{gap}(q)^2$. This
force also causes the period of oscillation to increase. Once the voltage is
removed, the actuator exponentially decays back to equilibrium due to viscous
Couette air damping between the beams and the substrate. 

\subsubsection*{Netlist}

\begin{verbatim}
use("mumps.net")
a = node{0, 0, 0; name = "a"}
b = node{name = "b"}
c = node{name = "c"}
d = node{name = "d"}
e = node{name = "e"}

-- beam and it's anchor: 
anchor  {a; material=p1, l=5u, w=10u, oz=180, ox=0, oy=0, h=10u}
beam2de {a, b; material=p1, l=100u, w=4u, oz=0, ox=0, oy=0, h=4u}

-- electrostatic 
gapV2 {b, c, d, e; material=p1, V1=5, t1=10u,  V2=12, t2=500u, 
                   l=100u, w1=4u, w2=4u, oz=0, ox=0, oy=0, h=4u, gap=2u}
-- gap anchors 
anchor {d; material=p1, l=5u, w=10u, oz=-90, ox=0, oy=0, h=10u}
anchor {e; material=p1, l=5u, w=10u, oz=-90, ox=0, oy=0, h=10u}
\end{verbatim}
