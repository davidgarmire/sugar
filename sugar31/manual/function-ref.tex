
\section{Load netlist}

\subsection*{Calling sequence}

\begin{verbatim}
net = cho_load(name, params)
\end{verbatim}

Loads and processes a netlist.

\subsection*{Inputs}

\begin{description}
\item[\texttt{name}]
  String naming the netlist file to be loaded
\item[\texttt{params}]
  (Optional) - a structure whose entries are the values of the parameters
  to be overridden.
\end{description}

\subsection*{Example}

\begin{verbatim}
nfingers = 10;                       % Set the nfingers parameter
net = cho_load('comb.net', params);  % Load netlist 
\end{verbatim}



\section{Device display}

\subsection*{Calling sequence}

\begin{verbatim}
cho_display(net, q)
\end{verbatim}

Display the mechanical structure described by a netlist.

\subsection*{Inputs}

\begin{description}
\item[\texttt{net}]
  Netlist structure returned from calling cho\_load
\item[\texttt{q}]
  (Optional) - the displacement of the original structure,
  as returned by the static analysis routine or the transient
  analysis routine.
  If \texttt{q} is unspecified, the undisplaced structure will
  be displayed.
\end{description}

\subsection*{Example}

\begin{verbatim}
net = cho_load('beamgap.net');  % Load netlist for beam-gap system
cho_display(net);               % Display undisplaced structure
\end{verbatim}

\subsection*{Caveats}

There is no display of electrical components.


\section{Viewing displacements}

\subsection*{Calling sequence}

\begin{verbatim}
dqcoord = cho_dq_coord(dq, net, node, coord);
\end{verbatim}

Extract the displacement corresponding to a particular coordinate
from a displacement vector q (as output by \texttt{cho\_dc}), or an array 
of displacement vectors (as output by \texttt{cho\_ta}).

Note that a node ``node'' completely internal to an instance ``foo''
of some subnet would be named ``node.foo''.

\subsection*{Inputs}

\begin{description}
\item[\texttt{dq}]:    Displacement vector or array of displacement vectors.
\item[\texttt{net}]: 
  Netlist structure returned from calling cho\_load
\item[\texttt{node}]:  Name of node.
\item[\texttt{coord}]: Name of coordinate at indicated node.
\end{description}

\subsection*{Outputs}

\begin{description}
\item[\texttt{dqcoord}]: Value (or vector of values) from \texttt{dq}
   associated with the indicated coordinate.
\end{description}

\subsection*{Example}

\begin{verbatim}
net = cho_load('beamgap.net');        % Load netlist 
dq = cho_dc(net);                     % Perform static analysis
dy = cho_dq_view(dq, net, 'c', 'y');  % Get the y component at c
\end{verbatim}


\section{Static analysis}

\subsection*{Calling sequence}

\begin{verbatim}
q = cho_dc(net, q0, is_sp);
\end{verbatim}

Finds a solution of the equilibrium equations
\[
  Kx = F(x)
\]
using a Newton-Raphson method.

\subsection*{Inputs}

\begin{description}
\item[\texttt{net}]: 
  Netlist structure returned from calling cho\_load
\item[\texttt{q0}]: 
  (Optional) Starting guess at the displacements from starting position
  for an equilibrium position.
  If no \texttt{q0} is provided, or if \texttt{q0 = {}} 
  the search will start at \texttt{q0} of 0.
\item[\texttt{is\_sp}]: 
  (Optional) If true, the code will use sparse solvers.
  The current default is true (use sparse solvers).
\end{description}

\subsection*{Outputs}

\begin{description}
\item[\texttt{q}]: The computed equilibrium state, expressed as
  displacements from the initial position.
\end{description}

\subsection*{Example}

\begin{verbatim}
net = cho_load('beamgap.net');   % Load netlist for beam-gap system
dq = cho_dc(net);                % Perform static analysis
cho_display(net, dq);            % Display the deflected structure
\end{verbatim}

\subsection*{Caveats}

The zero-finder currently used by SUGAR is very simple, and may
fail to converge for some problems.  If the function fails to converge
after 40 iterations, it will exit and print a warning diagnostic:
\begin{verbatim}
 Warning: DC solution finder did not converge after 40 iterations
\end{verbatim}
In this case, the result \texttt{q} returned by the routine is likely
to be useless.  Failure to find equilibrium occurs in such cases as
an electrostatically actuated gap operating near pull-in voltage.

When SUGAR finds an equilibrium point, it may not always be the
equilibrium point desired.  In the case of an electrostatically actuated
gap, for example, there are two equilibria below pull-in voltage:
one stable and one unstable.  When the equilibria are close together,
especially with respect to the distance from the starting point \texttt{q0},
the solver may move to an unstable equilibrium.

Good initial guesses for an equilibrium point can often be attained
by finding the equilibrium point for a ``nearby'' problem.  For example,
in trying to find the equilibrium point for a electrostatically actuated
gap operating near pull-in, a good initial guess \texttt{q0} would
be the output of a static analysis for the same device at a lower voltage.


\section{Modal analysis}

\subsection*{Analysis routine}

\subsubsection*{Calling sequence}

\begin{verbatim}
[freq, egv, q0] = cho_mode(net, nmodes, q0, find_dc)
\end{verbatim}

Find the resonating frequencies and corresponding mode shapes
(eigenvalues and eigenvectors) for the linearized system about
an equilibrium point.

\subsubsection*{Inputs}

\begin{description}
\item[\texttt{net}]: 
  Netlist structure returned from calling cho\_load.
\item[\texttt{nmodes}]: 
  (Optional)  If \texttt{nmodes > 0}, use sparse solvers to get nmodes modes
  If \texttt{nmodes == 0}, use the usual dense solver to get all the modes
  If \texttt{nmodes < 0}, solve with \texttt{eig(K $\backslash$ M)} rather than 
  \texttt{eig(M,K)}.  This last option can potentially cause trouble (for 
  instance, if $K$ is singular), but it is faster.

\item[\texttt{q0}]: 
  (Optional) Equilibrium operating point, or initial guess for
  a search for an equilibrium operating point.  If not supplied,
  of if \texttt{q0 = {}}, 
  the routine will search for an equilibrium point near 0.
\item[\texttt{find\_dc}]: 
  (Optional)  If true, search for an equilibrium point
  near the supplied q0.
\end{description}

\subsubsection*{Outputs}

\begin{description}
\item[freq]: Vector of resonating frequencies (eigenvalues)
\item[egv]: Array of corresponding mode shapes (eigenvectors)
\item[q0]: Equilibrium point about which the system was linearized
\end{description}

\subsection*{Display routine}

\subsubsection*{Calling sequence}

\begin{verbatim}
cho_modeshape(net, freq, egv, q0, s, num)
\end{verbatim}

Display the shape of a resonating mode of the mechanical structure.

\subsubsection*{Inputs}

\begin{description}
\item[\texttt{net}]: 
  Netlist structure returned from calling cho\_load.
\item[\texttt{freq}]: Vector of resonant frequencies from \texttt{cho\_mode}.
\item[\texttt{egv}]: Array of mode shape vectors from \texttt{cho\_mode}.
\item[\texttt{q0}]: Equilibrium point from \texttt{cho\_mode}.
\item[\texttt{s}]: Scale factor.  Eigenvectors from \texttt{cho\_mode}
  are normalized to be unit length; for eigenvectors with significant
  components (within a few orders of magnitude of 1) in directions
  corresponding to components which normally move a few microns,
  a scale factor of $10^{-4}$ often makes the display of the mode
  more comprehensible.  The scale factor can be omitted, in which case
  SUGAR uses a heuristic to determine scaling.
\item[\texttt{num}]: Number of the mode to display.  Modes are numbered
  in order of decreasing frequency.
\end{description}

\subsection*{Example}

\begin{verbatim}
% Show the first (lowest-frequency) mode shape for the system,
% scaled by a factor of 0.1
net = cho_load('beamgap.net');
[f,e,q] = cho_mode(net);
cho_modeshape(net, f,e,q, 0.1, 1);
\end{verbatim}

\subsection*{Caveats}

While SUGAR will attempt to find an appropriate linearization point,
it is not guaranteed to converge to one.  See the caveats for
static analysis.

The modal analysis routine neglects damping forces. 

As noted above, using \texttt{cho\_modeshape} with a too-large scaling
factor often results in the displayed device being stretched to
incomprehensible proportions.  Currently, trial-and-error guesses at
an appropriate scale factor seem to work best.


\section{Steady state analysis}

\subsection*{Calling sequence}

\begin{verbatim}
find_ss(net, q0, in_node, in_var, out_node, out_var)
\end{verbatim}

Make Bode plots of the frequency response of the linearized system about
an equilibrium point \texttt{q0}. 

\subsection*{Inputs}

\begin{description}
\item[\texttt{net}]: 
  Netlist structure returned from calling cho\_load
\item[\texttt{q0}]: 
  Equilibrium position for the system, as determined via the
  static analysis routine \texttt{cho\_dc}
\item[\texttt{in\_node}]:
  Name of the node at which an input signal is to be applied.
\item[\texttt{in\_var}]:
  Name of the nodal variable to be excited.
\item[\texttt{out\_node}]:
  Name of the node where the response is to be observed.
\item[\texttt{out\_var}]:
  Name of the nodal variable to be observed.
\end{description}

\subsection*{Example}

\begin{verbatim}
net = cho_load('multimode.net'); 
dq = cho_dc(net);
find_ss(net, dq, 'node10', 'y', 'node14', 'y');
\end{verbatim}

\subsection*{Caveats}

Steady-state frequency response analysis currently fails for devices
involving purely algebraic constraint.  Such devices include electrical
resistor networks with no inductances or capacitances considered,
for example.

The steady-state analysis routines currently use functions from Matlab's
Control Toolbox, which may be unavailable to some Matlab users.


\section{Transient analysis}

\emph{Note: Transient analysis has not yet been implemented in SUGAR 3.0}

\subsection*{Calling sequence}

\begin{verbatim}
res = cho_ta(net,tspan,q0)
\end{verbatim}

Simulate the behavior of the device over some time period.

\subsection*{Inputs}

\begin{description}
\item[\texttt{net}]: 
  Netlist structure returned from calling cho\_load
\item[\texttt{tspan}]: 
  Two-element vector \texttt{[tstart tend]} indicating the start
  and end times for the simulation.
\item[\texttt{q0}]: 
  (Optional) Initial state at \texttt{tstart}.
  If \texttt{q0} is not provided, the default is zero.
\end{description}

\subsection*{Outputs}

\begin{description}
\item[\texttt{T}]: Time points where the solution was sampled
\item[\texttt{Q}]: Array of state vectors sampled at the times in 
  \texttt{T}
  (i.e. \texttt{Q(i,:)} is the state vector at time \texttt{T(i)})
\end{description}

\subsection*{Example}

\begin{verbatim}
net = cho_load('beamgap.net')        % Load the netlist
[T,Q[ = cho_ta(net,{0, 1e-3]);       % Simulate 1 ms behavior
dy = cho_dq_view(Q, net, 'c', 'y');  % Get the y component at c, and
plot(T, dy);                         %  plot how it moves over time
\end{verbatim}

\subsection*{Caveats}

The transient analysis routines currently take an impractically long
time to simulate even some simple examples over modest time spans
(like a millisecond).  Mixed electrical-mechanical simulations
are particularly problematic.

Like frequency-response analysis, the transient analysis routine fails
completely for devices involving purely algebraic constraints.

