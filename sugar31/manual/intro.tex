
\section{What is SUGAR?}

In less than a decade, the MEMS community has leveraged nearly all the
integrated-circuit community's fabrication techniques, but little of the wealth
of simulation capabilities.  A wide range of student and professional circuit
designers regularly use circuit simulation tools like SPICE, while MEMS
designers often resort to back-of-the-envelope calculations.  
For three decades, development of IC CAD tools has gone hand-in-hand
with the development of IC processes.  Tools for simulation will
play a similar role in future advances in the design of complicated
micro-electromechanical systems.

SUGAR inherits its name and philosophy from SPICE.  A MEMS designer can
describe a device in a compact netlist format, and very quickly simulate
the device's behavior.  Using simple simulations in SUGAR, a designer can
quickly find problems in a design or try out new ideas.
Later in the design process, a designer might run more detailed simulations
to check for subtle second-order effects; early in the design, a quick
approximate solution is key.  SUGAR provides that quick solution.

The main components of SUGAR are a netlist interpreter (based on a derivative 
of the Lua language), models written in MATLAB or C (possibly 
more languages in future revisions) describing the characteristics of the 
different components, a command-line and gui which allows interaction and 
visualization of the specified devices, and a SUGAR core that handles 
generated nodes and elements, mesh assembly, and analysis of devices.

\section{A first example}

Let's first look at a simple example of how SUGAR is used.
Perhaps we have just designed a simple test structure in the MUMPS process,
a long cantilever anchored at one end.  We want to know how much
it will deflect if we apply a small force to the free end.

First, we need an input file, called a \emph{netlist}, that describes
the device.  We save this netlist in the file \texttt{cantilever.net}:
\begin{verbatim}
  use("mumps.net")
  use("stdlib.net")

  A = node{0, 0, 0; name = "A"}
  B = node{name = "B"}
  
  anchor { A    ; material=p1, l=10u, w=10u }
  beam3d { A, B ; material=p1, l=100u, w=2u }
  f3d    { B    ; F=50u, oz=90 }
\end{verbatim}
%  element{ A    ; model="anchor", material=p1, l=10u, w=10u }
%  element{ A, B ; model="beam3d", material=p1, l=100u, w=2u }
%  element{ B    ; model="f3d", F=50u, oz=90 }
The first line tells SUGAR that our device is made using the standard
MUMPS process.  The second and third line define two nodes to be used 
later in describing the \emph{netlist}'s elements. The fourth and fifth 
lines tell SUGAR that the device
is made of an anchor at node \texttt{A}, which is ten microns on a side;
and a beam, two microns wide and one hundred long, which goes from the 
anchored end point \texttt{A} to the free end \texttt{B}.  Both the
anchor and the beam are made of the poly1 layer, or \texttt{p1}.
The sixth line describes a force of 50 micro-Newtons applied at a
right angle to the free end of the beam.

Using SUGAR, we only need to write three commands to see the effect 
of the force on the beam:
\begin{verbatim}
  > net = cho_load('cantilever.net');
  > q = cho_dc(net);
  > cho_display(net, q);
\end{verbatim}
The first command tells SUGAR to process the netlist description in
\texttt{cantilever.net}.  The result is a data structure stored in the
\texttt{net} variable that describes the device to other SUGAR routines.
The \texttt{cho\_dc} function does a static (DC) analysis of the device,
and returns a vector representing the equilibrium position.  Finally,
the third line causes SUGAR to display the displaced device.


\section{Installing SUGAR}

All the SUGAR software is available from the SUGAR home page at
\begin{verbatim}
  http://bsac.berkeley.edu/cadtools/sugar
\end{verbatim}
We update the software frequently; if you encounter problems, you may want to
download the latest version before devoting hours of your time (or ours!) to
debugging.

% TODO: Change the following when it isn't true any more!
There is also a web interface to SUGAR at:
\begin{verbatim}
  http://sugar.millennium.berkeley.edu/
\end{verbatim}

If you are building SUGAR for your particular system, note that you need to
first make the {\tt noweb} and Lua tools in the {\tt sugar30/tools} directory
of the source tree. {\tt noweb} (refer to {\tt http://www.eecs.harvard.edu/\~{}nr/noweb/USA.html}) enables literate programming. The modified form of Lua
(refer to {\tt http://www.lua.org/}) is for the netlist and 
command-line interpreter.


\subsection{System requirements}

To require SUGAR, you will need MATLAB release 5.2 or later. Version 6.0 is
recommended. Because the student edition of MATLAB 5 only handles matrices
of a limited size, users of the student edition will only be able
to simulate small devices.

SUGAR is regularly tested using MATLAB 5.3 on Windows, Sun, HP,
and Alpha systems, and is tested using MATLAB 6.0 on Linux systems.
We have not tested the software on other systems.
If you would like to use SUGAR on a different system,
you will need to compile the external routines for that system,
as described later in this section.

\subsection{Setting SUGAR paths}

To use SUGAR, make sure that your MATLAB path is set correctly.
In particular, make sure the analysis and model subdirectories are
included in your MATLAB path.  This can be done from within MATLAB, e.g.
\begin{verbatim}
 addpath /home/eecs/dbindel/sugar30/src/matlab
\end{verbatim}
or from the Unix shell, by setting the MATLABPATH environment variable.
In \texttt{csh}, for instance, this might be 
\begin{verbatim}
 setenv MATLABPATH /home/eecs/dbindel/sugar30/src/matlab
\end{verbatim}
In the Windows version of MATLAB, you can also set the path from 
the ``File'' menu.


\subsection{Compiling external routines}

You will need to compile the external routines \emph{only} if
pre-compiled versions are not already available on your system.
To compile the routines, follow the directions in the included
README file.
%change to the \texttt{src/c} subdirectory
%and from MATLAB type \texttt{makemex}.  Alternatively, on UNIX-based
%systems, change to the \texttt{src} subdirectory, modify the file
%\texttt{make.inc}, and type \texttt{make sugar-mex}.  Then copy all the files
%beginning with \texttt{sugarmex} to the \texttt{src/matlab} subdirectory.


\section{Getting (and giving) help}

If you have concerns or difficulties using SUGAR which are not addressed
in the manual sections, feel free to post to
\begin{verbatim}
  http://sourceforge.net/projects/mems
\end{verbatim}
We will try to respond promptly.

SUGAR \emph{is} research software; let us know if you would like to contribute
models, analysis routines, or examples to the SUGAR project.
