% ===> this file was generated automatically by noweave --- better not edit it
\section{Introduction}

The {\Tt{}dynarray\nwendquote} module implements a grow-able array type.
It behaves (mostly) like a normal array, but it can be expanded
when space is needed for more elements.  The index space is
assumed to be dense, and indexing starts at 0.

Note that objects stored in a dynamic array may change locations
when the array is forced to grow.  That means that pointer references
into the array are \emph{not} necessarily valid across calls to add
more elements.


\section{Interface}

\nwfilename{dynarray.nw}\nwbegincode{1}\sublabel{NW4UhJIQ-uvL9y-1}\nwmargintag{{\nwtagstyle{}\subpageref{NW4UhJIQ-uvL9y-1}}}\moddef{dynarray.h~{\nwtagstyle{}\subpageref{NW4UhJIQ-uvL9y-1}}}\endmoddef\nwstartdeflinemarkup\nwenddeflinemarkup
#ifndef DYNARRAY_H
#define DYNARRAY_H

typedef struct dynarray_t* dynarray_t;

\LA{}exported functions~{\nwtagstyle{}\subpageref{NW4UhJIQ-2gXoUc-1}}\RA{}

#endif /* DYNARRAY_H */
\nwnotused{dynarray.h}\nwendcode{}\nwbegindocs{2}\nwdocspar

Dynamic arrays are created and destroyed by the functions
{\Tt{}dynarray{\_}create\nwendquote} and {\Tt{}dynarray{\_}destroy\nwendquote}.

\nwenddocs{}\nwbegincode{3}\sublabel{NW4UhJIQ-2gXoUc-1}\nwmargintag{{\nwtagstyle{}\subpageref{NW4UhJIQ-2gXoUc-1}}}\moddef{exported functions~{\nwtagstyle{}\subpageref{NW4UhJIQ-2gXoUc-1}}}\endmoddef\nwstartdeflinemarkup\nwusesondefline{\\{NW4UhJIQ-uvL9y-1}}\nwprevnextdefs{\relax}{NW4UhJIQ-2gXoUc-2}\nwenddeflinemarkup
dynarray_t dynarray_create (int element_size, int capacity);
void       dynarray_destroy(dynarray_t self);

\nwalsodefined{\\{NW4UhJIQ-2gXoUc-2}\\{NW4UhJIQ-2gXoUc-3}}\nwused{\\{NW4UhJIQ-uvL9y-1}}\nwendcode{}\nwbegindocs{4}\nwdocspar

The {\Tt{}capacity\nwendquote} argument sets the initial capacity of the
array.  If the user writes to a location past the number
of elements indicated by {\Tt{}capacity\nwendquote}, the array will grow
to match the new requirements.  It is probably more efficient
to have a good guess at the capacity from the beginning, though.
It is an error to have a negative or zero initial capacity.

The {\Tt{}capacity\nwendquote} is \emph{not} the same as the number of
entries in the array.  Regardless of the capacity, the number
of entries in any newly-created array is zero.

\nwenddocs{}\nwbegincode{5}\sublabel{NW4UhJIQ-2gXoUc-2}\nwmargintag{{\nwtagstyle{}\subpageref{NW4UhJIQ-2gXoUc-2}}}\moddef{exported functions~{\nwtagstyle{}\subpageref{NW4UhJIQ-2gXoUc-1}}}\plusendmoddef\nwstartdeflinemarkup\nwusesondefline{\\{NW4UhJIQ-uvL9y-1}}\nwprevnextdefs{NW4UhJIQ-2gXoUc-1}{NW4UhJIQ-2gXoUc-3}\nwenddeflinemarkup
void*      dynarray_data  (dynarray_t self);
void*      dynarray_get   (dynarray_t self, int i);
void*      dynarray_set   (dynarray_t self, int i, const void* data);
void*      dynarray_append(dynarray_t self, const void* data);

\nwused{\\{NW4UhJIQ-uvL9y-1}}\nwendcode{}\nwbegindocs{6}\nwdocspar

The {\Tt{}dynarray{\_}set\nwendquote} and {\Tt{}dynarray{\_}get\nwendquote} routines respectively
set and get entries of the dynamic array.  The {\Tt{}dynarray{\_}append\nwendquote}
function is equivalent to {\Tt{}set\nwendquote}ting the element one past the
current end of the array.  The {\Tt{}dynarray{\_}data\nwendquote}
function returns a pointer to the current data array, which can
be used directly in routines like {\Tt{}qsort\nwendquote}.  Note that the
pointer returned by {\Tt{}dynarray{\_}data\nwendquote} will not necessarily remain
valid across calls to {\Tt{}dynarray{\_}get\nwendquote} or {\Tt{}dynarray{\_}append\nwendquote}.

If the index to {\Tt{}dynarray{\_}get\nwendquote} is out of bounds, the function will 
return NULL.  If the index to {\Tt{}dynarray{\_}set\nwendquote} is past the current end 
of the dynamic array, the array will be appropriately expanded to
accomodate the new element.  It is a checked error to pass a negative
index to {\Tt{}dynarray{\_}set\nwendquote}.

If the index to {\Tt{}dynarray{\_}set\nwendquote} is more than one past the current
end of the dynamic array, some new elements will be implicitly created.
These elements are initialized to zero.

The {\Tt{}data\nwendquote} argument to {\Tt{}dynarray{\_}set\nwendquote} can also be NULL,
in which case the desired element is left alone if it existed
before, or set to zero if it is a new element.

\nwenddocs{}\nwbegincode{7}\sublabel{NW4UhJIQ-2gXoUc-3}\nwmargintag{{\nwtagstyle{}\subpageref{NW4UhJIQ-2gXoUc-3}}}\moddef{exported functions~{\nwtagstyle{}\subpageref{NW4UhJIQ-2gXoUc-1}}}\plusendmoddef\nwstartdeflinemarkup\nwusesondefline{\\{NW4UhJIQ-uvL9y-1}}\nwprevnextdefs{NW4UhJIQ-2gXoUc-2}{\relax}\nwenddeflinemarkup
int        dynarray_count    (dynarray_t self);
void       dynarray_set_count(dynarray_t self, int n);
\nwused{\\{NW4UhJIQ-uvL9y-1}}\nwendcode{}\nwbegindocs{8}\nwdocspar

The {\Tt{}dynarray{\_}count\nwendquote} function returns the number of elements
currently in the array.  That number of elements can be increased
or reduced with calls to {\Tt{}dynarray{\_}set{\_}count\nwendquote}.  Elements newly
created by expanding the array with {\Tt{}dynarray{\_}set{\_}count\nwendquote} are
initialized to zero.


\section{Implementation}

\nwenddocs{}\nwbegincode{9}\sublabel{NW4UhJIQ-2z5EO6-1}\nwmargintag{{\nwtagstyle{}\subpageref{NW4UhJIQ-2z5EO6-1}}}\moddef{dynarray.c~{\nwtagstyle{}\subpageref{NW4UhJIQ-2z5EO6-1}}}\endmoddef\nwstartdeflinemarkup\nwenddeflinemarkup
#include <sugar.h>

#include <stdlib.h>
#include <string.h>
#include <assert.h>

#include <dynarray.h>

struct dynarray_t \{
    char*  data;
    int    element_size;
    int    count;
    int    capacity;
\};

\LA{}functions~{\nwtagstyle{}\subpageref{NW4UhJIQ-nRuDO-1}}\RA{}
\nwnotused{dynarray.c}\nwendcode{}\nwbegindocs{10}\nwdocspar

There are two pieces to a dynamic array.  The first is the ``handle''
structure, which resides at a fixed location of memory and keeps
track of where the array currently lives, how big it is, and how
many elements it has.  The second piece is the array itself, which
may be destroyed and reallocated during the course of normal
operation.

A dynamic array starts out with no elements, but with a user
specified capacity for storing elements before reallocation
is necessary.

\nwenddocs{}\nwbegincode{11}\sublabel{NW4UhJIQ-nRuDO-1}\nwmargintag{{\nwtagstyle{}\subpageref{NW4UhJIQ-nRuDO-1}}}\moddef{functions~{\nwtagstyle{}\subpageref{NW4UhJIQ-nRuDO-1}}}\endmoddef\nwstartdeflinemarkup\nwusesondefline{\\{NW4UhJIQ-2z5EO6-1}}\nwprevnextdefs{\relax}{NW4UhJIQ-nRuDO-2}\nwenddeflinemarkup
dynarray_t dynarray_create(int element_size, int capacity)
\{
    dynarray_t self;

    assert(element_size > 0 && capacity > 0);

    self = (dynarray_t) calloc(1, sizeof(struct dynarray_t));
    assert(self != NULL);

    self->data = (char*) calloc(capacity, element_size);
    assert(self->data != NULL);

    self->element_size = element_size;
    self->capacity = capacity;
    self->count = 0;
    return self;
\}

\nwalsodefined{\\{NW4UhJIQ-nRuDO-2}\\{NW4UhJIQ-nRuDO-3}\\{NW4UhJIQ-nRuDO-4}\\{NW4UhJIQ-nRuDO-5}\\{NW4UhJIQ-nRuDO-6}\\{NW4UhJIQ-nRuDO-7}}\nwused{\\{NW4UhJIQ-2z5EO6-1}}\nwendcode{}\nwbegindocs{12}\nwdocspar

When we free a dynamic array, we must free both the array and
the handle object.

\nwenddocs{}\nwbegincode{13}\sublabel{NW4UhJIQ-nRuDO-2}\nwmargintag{{\nwtagstyle{}\subpageref{NW4UhJIQ-nRuDO-2}}}\moddef{functions~{\nwtagstyle{}\subpageref{NW4UhJIQ-nRuDO-1}}}\plusendmoddef\nwstartdeflinemarkup\nwusesondefline{\\{NW4UhJIQ-2z5EO6-1}}\nwprevnextdefs{NW4UhJIQ-nRuDO-1}{NW4UhJIQ-nRuDO-3}\nwenddeflinemarkup
void dynarray_destroy(dynarray_t self)
\{
    free(self->data);
    free(self);
\}

\nwused{\\{NW4UhJIQ-2z5EO6-1}}\nwendcode{}\nwbegindocs{14}\nwdocspar

The {\Tt{}data\nwendquote} function and {\Tt{}count\nwendquote} function just retrieve the
corresponding fields from the handle object.

\nwenddocs{}\nwbegincode{15}\sublabel{NW4UhJIQ-nRuDO-3}\nwmargintag{{\nwtagstyle{}\subpageref{NW4UhJIQ-nRuDO-3}}}\moddef{functions~{\nwtagstyle{}\subpageref{NW4UhJIQ-nRuDO-1}}}\plusendmoddef\nwstartdeflinemarkup\nwusesondefline{\\{NW4UhJIQ-2z5EO6-1}}\nwprevnextdefs{NW4UhJIQ-nRuDO-2}{NW4UhJIQ-nRuDO-4}\nwenddeflinemarkup
void* dynarray_data(dynarray_t self)
\{
    assert(self != NULL);
    return self->data;
\}

int dynarray_count(dynarray_t self)
\{
    assert(self != NULL);
    return self->count;
\}

\nwused{\\{NW4UhJIQ-2z5EO6-1}}\nwendcode{}\nwbegindocs{16}\nwdocspar

The {\Tt{}set{\_}count\nwendquote} function expands or contracts the array,
adding new space if necessary.  When the count is increased,
the new slots are always empty, even if the expansion did
not require a reallocation.  We actually do the clear
operation on \emph{contraction}, though, in order to make
it simpler to catch errors involving pointers to invalid data
more quickly.

\nwenddocs{}\nwbegincode{17}\sublabel{NW4UhJIQ-nRuDO-4}\nwmargintag{{\nwtagstyle{}\subpageref{NW4UhJIQ-nRuDO-4}}}\moddef{functions~{\nwtagstyle{}\subpageref{NW4UhJIQ-nRuDO-1}}}\plusendmoddef\nwstartdeflinemarkup\nwusesondefline{\\{NW4UhJIQ-2z5EO6-1}}\nwprevnextdefs{NW4UhJIQ-nRuDO-3}{NW4UhJIQ-nRuDO-5}\nwenddeflinemarkup
void dynarray_set_count(dynarray_t self, int n)
\{
    assert(self != NULL && n >= 0);

    if (n > self->capacity) \{
        \LA{}increase capacity~{\nwtagstyle{}\subpageref{NW4UhJIQ-F1R72-1}}\RA{}
    \}

    if (n < self->count) \{
        \LA{}clear deallocated slots~{\nwtagstyle{}\subpageref{NW4UhJIQ-2CJieb-1}}\RA{}
    \}

    self->count = n;
\}

\nwused{\\{NW4UhJIQ-2z5EO6-1}}\nwendcode{}\nwbegindocs{18}\nwdocspar

Every time we run out of room, we try to double our capacity
(or we reluctantly increase more, if that's necessary).
Once we have the space, we copy over the data, free the old
buffer, and update our handle.

\nwenddocs{}\nwbegincode{19}\sublabel{NW4UhJIQ-F1R72-1}\nwmargintag{{\nwtagstyle{}\subpageref{NW4UhJIQ-F1R72-1}}}\moddef{increase capacity~{\nwtagstyle{}\subpageref{NW4UhJIQ-F1R72-1}}}\endmoddef\nwstartdeflinemarkup\nwusesondefline{\\{NW4UhJIQ-nRuDO-4}}\nwenddeflinemarkup
int   new_capacity = (n > self->capacity*2) ? n : (self->capacity*2);
char* new_data     = (char*) calloc(new_capacity, self->element_size);

assert(new_data != NULL);
memcpy(new_data, self->data, self->count * self->element_size);
free(self->data);

self->data = new_data;
self->capacity = new_capacity;
\nwused{\\{NW4UhJIQ-nRuDO-4}}\nwendcode{}\nwbegindocs{20}\nwdocspar

\nwenddocs{}\nwbegincode{21}\sublabel{NW4UhJIQ-2CJieb-1}\nwmargintag{{\nwtagstyle{}\subpageref{NW4UhJIQ-2CJieb-1}}}\moddef{clear deallocated slots~{\nwtagstyle{}\subpageref{NW4UhJIQ-2CJieb-1}}}\endmoddef\nwstartdeflinemarkup\nwusesondefline{\\{NW4UhJIQ-nRuDO-4}}\nwenddeflinemarkup
memset(&(self->data[n * self->element_size]), 0, 
        (self->count - n) * self->element_size);
\nwused{\\{NW4UhJIQ-nRuDO-4}}\nwendcode{}\nwbegindocs{22}\nwdocspar

The {\Tt{}get\nwendquote} function basically just does a check to make sure
the requested element is within range, and then some pointer
arithmetic.

\nwenddocs{}\nwbegincode{23}\sublabel{NW4UhJIQ-nRuDO-5}\nwmargintag{{\nwtagstyle{}\subpageref{NW4UhJIQ-nRuDO-5}}}\moddef{functions~{\nwtagstyle{}\subpageref{NW4UhJIQ-nRuDO-1}}}\plusendmoddef\nwstartdeflinemarkup\nwusesondefline{\\{NW4UhJIQ-2z5EO6-1}}\nwprevnextdefs{NW4UhJIQ-nRuDO-4}{NW4UhJIQ-nRuDO-6}\nwenddeflinemarkup
void* dynarray_get(dynarray_t self, int i)
\{
    assert(self != NULL);

    if (i >= self->count || i < 0)
        return NULL;

    return &(self->data[i * self->element_size]);
\}

\nwused{\\{NW4UhJIQ-2z5EO6-1}}\nwendcode{}\nwbegindocs{24}\nwdocspar

The {\Tt{}set\nwendquote} function is a little more complicated than its
{\Tt{}get\nwendquote} counterpart, since {\Tt{}set\nwendquote} might have to expand the
array.

\nwenddocs{}\nwbegincode{25}\sublabel{NW4UhJIQ-nRuDO-6}\nwmargintag{{\nwtagstyle{}\subpageref{NW4UhJIQ-nRuDO-6}}}\moddef{functions~{\nwtagstyle{}\subpageref{NW4UhJIQ-nRuDO-1}}}\plusendmoddef\nwstartdeflinemarkup\nwusesondefline{\\{NW4UhJIQ-2z5EO6-1}}\nwprevnextdefs{NW4UhJIQ-nRuDO-5}{NW4UhJIQ-nRuDO-7}\nwenddeflinemarkup
void* dynarray_set(dynarray_t self, int i, const void* data)
\{
    assert(self != NULL && i >= 0);

    if (i >= self->count)
        dynarray_set_count(self, i+1);

    if (data != NULL)
        memcpy(&(self->data[i * self->element_size]), data, 
               self->element_size);

    return &(self->data[i * self->element_size]);
\}

\nwused{\\{NW4UhJIQ-2z5EO6-1}}\nwendcode{}\nwbegindocs{26}\nwdocspar

Appending an element really is just setting the element after
the one that is currently last.

\nwenddocs{}\nwbegincode{27}\sublabel{NW4UhJIQ-nRuDO-7}\nwmargintag{{\nwtagstyle{}\subpageref{NW4UhJIQ-nRuDO-7}}}\moddef{functions~{\nwtagstyle{}\subpageref{NW4UhJIQ-nRuDO-1}}}\plusendmoddef\nwstartdeflinemarkup\nwusesondefline{\\{NW4UhJIQ-2z5EO6-1}}\nwprevnextdefs{NW4UhJIQ-nRuDO-6}{\relax}\nwenddeflinemarkup
void* dynarray_append(dynarray_t self, const void* data)
\{
    return dynarray_set(self, self->count, data);
\}

\nwused{\\{NW4UhJIQ-2z5EO6-1}}\nwendcode{}\nwbegindocs{28}\nwdocspar


\section{Test program}

\nwenddocs{}\nwbegincode{29}\sublabel{NW4UhJIQ-2mpMqd-1}\nwmargintag{{\nwtagstyle{}\subpageref{NW4UhJIQ-2mpMqd-1}}}\moddef{dynarray-test.c~{\nwtagstyle{}\subpageref{NW4UhJIQ-2mpMqd-1}}}\endmoddef\nwstartdeflinemarkup\nwenddeflinemarkup
#include <sugar.h>

#include <stdio.h>
#include <dynarray.h>

int main()
\{
    \LA{}test definitions~{\nwtagstyle{}\subpageref{NW4UhJIQ-49hYPk-1}}\RA{}
    \LA{}test code~{\nwtagstyle{}\subpageref{NW4UhJIQ-1IZWlr-1}}\RA{}
    return 0;
\}
\nwnotused{dynarray-test.c}\nwendcode{}\nwbegindocs{30}\nwdocspar

Our test array initially has the capacity to hold four
entries.

\nwenddocs{}\nwbegincode{31}\sublabel{NW4UhJIQ-1IZWlr-1}\nwmargintag{{\nwtagstyle{}\subpageref{NW4UhJIQ-1IZWlr-1}}}\moddef{test code~{\nwtagstyle{}\subpageref{NW4UhJIQ-1IZWlr-1}}}\endmoddef\nwstartdeflinemarkup\nwusesondefline{\\{NW4UhJIQ-2mpMqd-1}}\nwprevnextdefs{\relax}{NW4UhJIQ-1IZWlr-2}\nwenddeflinemarkup
dynarray_t array;
array = dynarray_create(sizeof(int), 4);
dynarray_destroy(array);
array = dynarray_create(sizeof(int), 4);
\nwalsodefined{\\{NW4UhJIQ-1IZWlr-2}\\{NW4UhJIQ-1IZWlr-3}\\{NW4UhJIQ-1IZWlr-4}\\{NW4UhJIQ-1IZWlr-5}\\{NW4UhJIQ-1IZWlr-6}\\{NW4UhJIQ-1IZWlr-7}\\{NW4UhJIQ-1IZWlr-8}}\nwused{\\{NW4UhJIQ-2mpMqd-1}}\nwendcode{}\nwbegindocs{32}\nwdocspar

The first thing we do is to populate those entries (and four more)
using {\Tt{}append\nwendquote}, then print out the results.

\nwenddocs{}\nwbegincode{33}\sublabel{NW4UhJIQ-49hYPk-1}\nwmargintag{{\nwtagstyle{}\subpageref{NW4UhJIQ-49hYPk-1}}}\moddef{test definitions~{\nwtagstyle{}\subpageref{NW4UhJIQ-49hYPk-1}}}\endmoddef\nwstartdeflinemarkup\nwusesondefline{\\{NW4UhJIQ-2mpMqd-1}}\nwenddeflinemarkup
int i;
\nwused{\\{NW4UhJIQ-2mpMqd-1}}\nwendcode{}\nwbegindocs{34}\nwdocspar

\nwenddocs{}\nwbegincode{35}\sublabel{NW4UhJIQ-1IZWlr-2}\nwmargintag{{\nwtagstyle{}\subpageref{NW4UhJIQ-1IZWlr-2}}}\moddef{test code~{\nwtagstyle{}\subpageref{NW4UhJIQ-1IZWlr-1}}}\plusendmoddef\nwstartdeflinemarkup\nwusesondefline{\\{NW4UhJIQ-2mpMqd-1}}\nwprevnextdefs{NW4UhJIQ-1IZWlr-1}{NW4UhJIQ-1IZWlr-3}\nwenddeflinemarkup
for (i = 0; i < 8; ++i)
    dynarray_append(array, &i);

printf("After appending eight times:");
\LA{}print array contents~{\nwtagstyle{}\subpageref{NW4UhJIQ-3W1qeg-1}}\RA{}
\nwused{\\{NW4UhJIQ-2mpMqd-1}}\nwendcode{}\nwbegindocs{36}\nwdocspar

\nwenddocs{}\nwbegincode{37}\sublabel{NW4UhJIQ-3W1qeg-1}\nwmargintag{{\nwtagstyle{}\subpageref{NW4UhJIQ-3W1qeg-1}}}\moddef{print array contents~{\nwtagstyle{}\subpageref{NW4UhJIQ-3W1qeg-1}}}\endmoddef\nwstartdeflinemarkup\nwusesondefline{\\{NW4UhJIQ-1IZWlr-2}\\{NW4UhJIQ-1IZWlr-3}\\{NW4UhJIQ-1IZWlr-4}\\{NW4UhJIQ-1IZWlr-5}\\{NW4UhJIQ-1IZWlr-7}}\nwenddeflinemarkup
for (i = 0; i < dynarray_count(array); ++i)
    printf(" %d", *(int*) dynarray_get(array, i));
printf("\\n");
\nwused{\\{NW4UhJIQ-1IZWlr-2}\\{NW4UhJIQ-1IZWlr-3}\\{NW4UhJIQ-1IZWlr-4}\\{NW4UhJIQ-1IZWlr-5}\\{NW4UhJIQ-1IZWlr-7}}\nwendcode{}\nwbegindocs{38}\nwdocspar

At this point, the array should be at capacity.  The border
case for the expansion code is between doubling the size
and increasing the size to the number of elements plus 1;
we test to make sure that we didn't screw up at that border.

\nwenddocs{}\nwbegincode{39}\sublabel{NW4UhJIQ-1IZWlr-3}\nwmargintag{{\nwtagstyle{}\subpageref{NW4UhJIQ-1IZWlr-3}}}\moddef{test code~{\nwtagstyle{}\subpageref{NW4UhJIQ-1IZWlr-1}}}\plusendmoddef\nwstartdeflinemarkup\nwusesondefline{\\{NW4UhJIQ-2mpMqd-1}}\nwprevnextdefs{NW4UhJIQ-1IZWlr-2}{NW4UhJIQ-1IZWlr-4}\nwenddeflinemarkup
i = 16;
dynarray_set(array, i, &i);

printf("After setting element sixteen:");
\LA{}print array contents~{\nwtagstyle{}\subpageref{NW4UhJIQ-3W1qeg-1}}\RA{}
\nwused{\\{NW4UhJIQ-2mpMqd-1}}\nwendcode{}\nwbegindocs{40}\nwdocspar

Now we shrink the array back to five elements using {\Tt{}set{\_}count\nwendquote}.

\nwenddocs{}\nwbegincode{41}\sublabel{NW4UhJIQ-1IZWlr-4}\nwmargintag{{\nwtagstyle{}\subpageref{NW4UhJIQ-1IZWlr-4}}}\moddef{test code~{\nwtagstyle{}\subpageref{NW4UhJIQ-1IZWlr-1}}}\plusendmoddef\nwstartdeflinemarkup\nwusesondefline{\\{NW4UhJIQ-2mpMqd-1}}\nwprevnextdefs{NW4UhJIQ-1IZWlr-3}{NW4UhJIQ-1IZWlr-5}\nwenddeflinemarkup
dynarray_set_count(array, 5);

printf("After decreasing to five elements:");
\LA{}print array contents~{\nwtagstyle{}\subpageref{NW4UhJIQ-3W1qeg-1}}\RA{}
\nwused{\\{NW4UhJIQ-2mpMqd-1}}\nwendcode{}\nwbegindocs{42}\nwdocspar

And immediately we shoot back up to sixteen elements to ensure
that things were properly cleared.

\nwenddocs{}\nwbegincode{43}\sublabel{NW4UhJIQ-1IZWlr-5}\nwmargintag{{\nwtagstyle{}\subpageref{NW4UhJIQ-1IZWlr-5}}}\moddef{test code~{\nwtagstyle{}\subpageref{NW4UhJIQ-1IZWlr-1}}}\plusendmoddef\nwstartdeflinemarkup\nwusesondefline{\\{NW4UhJIQ-2mpMqd-1}}\nwprevnextdefs{NW4UhJIQ-1IZWlr-4}{NW4UhJIQ-1IZWlr-6}\nwenddeflinemarkup
dynarray_set_count(array, 16);

printf("After increasing to 16 elements:");
\LA{}print array contents~{\nwtagstyle{}\subpageref{NW4UhJIQ-3W1qeg-1}}\RA{}
\nwused{\\{NW4UhJIQ-2mpMqd-1}}\nwendcode{}\nwbegindocs{44}\nwdocspar

Now let's see what happens when we try to read elements until we
starting hitting NULL returns from {\Tt{}get\nwendquote}.

\nwenddocs{}\nwbegincode{45}\sublabel{NW4UhJIQ-1IZWlr-6}\nwmargintag{{\nwtagstyle{}\subpageref{NW4UhJIQ-1IZWlr-6}}}\moddef{test code~{\nwtagstyle{}\subpageref{NW4UhJIQ-1IZWlr-1}}}\plusendmoddef\nwstartdeflinemarkup\nwusesondefline{\\{NW4UhJIQ-2mpMqd-1}}\nwprevnextdefs{NW4UhJIQ-1IZWlr-5}{NW4UhJIQ-1IZWlr-7}\nwenddeflinemarkup
for (i = 0; dynarray_get(array, i) != NULL; ++i);
printf("Ran into a NULL get at index %d\\n", i);
\nwused{\\{NW4UhJIQ-2mpMqd-1}}\nwendcode{}\nwbegindocs{46}\nwdocspar

Again, let's make sure that the expansion code does not have an
off-by-one problem by testing around the boundary of where an expansion
should occur.

\nwenddocs{}\nwbegincode{47}\sublabel{NW4UhJIQ-1IZWlr-7}\nwmargintag{{\nwtagstyle{}\subpageref{NW4UhJIQ-1IZWlr-7}}}\moddef{test code~{\nwtagstyle{}\subpageref{NW4UhJIQ-1IZWlr-1}}}\plusendmoddef\nwstartdeflinemarkup\nwusesondefline{\\{NW4UhJIQ-2mpMqd-1}}\nwprevnextdefs{NW4UhJIQ-1IZWlr-6}{NW4UhJIQ-1IZWlr-8}\nwenddeflinemarkup
*(int*) dynarray_set(array, 31, NULL) = 31;

printf("After setting element 31:");
\LA{}print array contents~{\nwtagstyle{}\subpageref{NW4UhJIQ-3W1qeg-1}}\RA{}
\nwused{\\{NW4UhJIQ-2mpMqd-1}}\nwendcode{}\nwbegindocs{48}\nwdocspar

Finally, clean up and call it a day.

\nwenddocs{}\nwbegincode{49}\sublabel{NW4UhJIQ-1IZWlr-8}\nwmargintag{{\nwtagstyle{}\subpageref{NW4UhJIQ-1IZWlr-8}}}\moddef{test code~{\nwtagstyle{}\subpageref{NW4UhJIQ-1IZWlr-1}}}\plusendmoddef\nwstartdeflinemarkup\nwusesondefline{\\{NW4UhJIQ-2mpMqd-1}}\nwprevnextdefs{NW4UhJIQ-1IZWlr-7}{\relax}\nwenddeflinemarkup
dynarray_destroy(array);
\nwused{\\{NW4UhJIQ-2mpMqd-1}}\nwendcode{}

\nwixlogsorted{c}{{clear deallocated slots}{NW4UhJIQ-2CJieb-1}{\nwixu{NW4UhJIQ-nRuDO-4}\nwixd{NW4UhJIQ-2CJieb-1}}}%
\nwixlogsorted{c}{{dynarray-test.c}{NW4UhJIQ-2mpMqd-1}{\nwixd{NW4UhJIQ-2mpMqd-1}}}%
\nwixlogsorted{c}{{dynarray.c}{NW4UhJIQ-2z5EO6-1}{\nwixd{NW4UhJIQ-2z5EO6-1}}}%
\nwixlogsorted{c}{{dynarray.h}{NW4UhJIQ-uvL9y-1}{\nwixd{NW4UhJIQ-uvL9y-1}}}%
\nwixlogsorted{c}{{exported functions}{NW4UhJIQ-2gXoUc-1}{\nwixu{NW4UhJIQ-uvL9y-1}\nwixd{NW4UhJIQ-2gXoUc-1}\nwixd{NW4UhJIQ-2gXoUc-2}\nwixd{NW4UhJIQ-2gXoUc-3}}}%
\nwixlogsorted{c}{{functions}{NW4UhJIQ-nRuDO-1}{\nwixu{NW4UhJIQ-2z5EO6-1}\nwixd{NW4UhJIQ-nRuDO-1}\nwixd{NW4UhJIQ-nRuDO-2}\nwixd{NW4UhJIQ-nRuDO-3}\nwixd{NW4UhJIQ-nRuDO-4}\nwixd{NW4UhJIQ-nRuDO-5}\nwixd{NW4UhJIQ-nRuDO-6}\nwixd{NW4UhJIQ-nRuDO-7}}}%
\nwixlogsorted{c}{{increase capacity}{NW4UhJIQ-F1R72-1}{\nwixu{NW4UhJIQ-nRuDO-4}\nwixd{NW4UhJIQ-F1R72-1}}}%
\nwixlogsorted{c}{{print array contents}{NW4UhJIQ-3W1qeg-1}{\nwixu{NW4UhJIQ-1IZWlr-2}\nwixd{NW4UhJIQ-3W1qeg-1}\nwixu{NW4UhJIQ-1IZWlr-3}\nwixu{NW4UhJIQ-1IZWlr-4}\nwixu{NW4UhJIQ-1IZWlr-5}\nwixu{NW4UhJIQ-1IZWlr-7}}}%
\nwixlogsorted{c}{{test code}{NW4UhJIQ-1IZWlr-1}{\nwixu{NW4UhJIQ-2mpMqd-1}\nwixd{NW4UhJIQ-1IZWlr-1}\nwixd{NW4UhJIQ-1IZWlr-2}\nwixd{NW4UhJIQ-1IZWlr-3}\nwixd{NW4UhJIQ-1IZWlr-4}\nwixd{NW4UhJIQ-1IZWlr-5}\nwixd{NW4UhJIQ-1IZWlr-6}\nwixd{NW4UhJIQ-1IZWlr-7}\nwixd{NW4UhJIQ-1IZWlr-8}}}%
\nwixlogsorted{c}{{test definitions}{NW4UhJIQ-49hYPk-1}{\nwixu{NW4UhJIQ-2mpMqd-1}\nwixd{NW4UhJIQ-49hYPk-1}}}%
\nwbegindocs{50}\nwdocspar




\nwenddocs{}
