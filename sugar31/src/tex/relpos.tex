% ===> this file was generated automatically by noweave --- better not edit it
\section{Introduction}

In previous versions of SUGAR, the preferred way to specify device 
geometry was by providing the dimensions and orientations of all 
the structural elements.  From that information, the node positions 
could be determined.  This module supports that form of 
implicit node positioning.

The current preferred method of specifying geometry is to specify
the node positions directly, and possibly infer parameters like beam
length (instead of vice-versa).  This module exists primarily to
ease migration.


\section{Interface}

Public functions for this module are
\begin{itemize}
  \item {\Tt{}relpos{\_}add(n1,\ n2,\ delta)\nwendquote}:
        indicate the position of node {\Tt{}n2\nwendquote} relative to {\Tt{}n1\nwendquote}
        is {\Tt{}delta\nwendquote}
  \item {\Tt{}relpos{\_}dump()\nwendquote}:
        dump the current table of relative positioning information
  \item {\Tt{}relpos{\_}treewalk()\nwendquote}:
        use relative positioning information to determine the
        actual node positions
\end{itemize}


\section{Implementation}

The {\Tt{}relpos{\_}table\nwendquote} data structure is an adjacency list
representation of the graph whose edges are known relative
positions.  For instance, we might have
\begin{verbatim}
 relpos_table[n1] = {{dx12, dy12, dz12, n2}, {dx13, dy13, dz13, n3}, ...}
\end{verbatim}
to represent known relative positions of node 2, 3, etc. with respect
to node 1.  The position of node 2 relative to an origin at node 1 is
$(dx_{12}, dy_{12}, dz_{12})$.

\nwfilename{relpos.nw}\nwbegincode{1}\sublabel{NW2I9Iuo-2auqn1-1}\nwmargintag{{\nwtagstyle{}\subpageref{NW2I9Iuo-2auqn1-1}}}\moddef{relpos.lua~{\nwtagstyle{}\subpageref{NW2I9Iuo-2auqn1-1}}}\endmoddef\nwstartdeflinemarkup\nwenddeflinemarkup
relpos_table = \{\};
relpos_first_attached = nil;

\LA{}functions~{\nwtagstyle{}\subpageref{NW2I9Iuo-nRuDO-1}}\RA{}
\nwnotused{relpos.lua}\nwendcode{}\nwbegindocs{2}\nwdocspar


\nwenddocs{}\nwbegincode{3}\sublabel{NW2I9Iuo-nRuDO-1}\nwmargintag{{\nwtagstyle{}\subpageref{NW2I9Iuo-nRuDO-1}}}\moddef{functions~{\nwtagstyle{}\subpageref{NW2I9Iuo-nRuDO-1}}}\endmoddef\nwstartdeflinemarkup\nwusesondefline{\\{NW2I9Iuo-2auqn1-1}}\nwprevnextdefs{\relax}{NW2I9Iuo-nRuDO-2}\nwenddeflinemarkup
-- Add edges from node 1 to 2 and vice-versa
--
function relpos_add(n1, n2, delta)
  local elist;

  relpos_first_attached = relpos_first_attached or n1;

  elist = relpos_get_elist(n1);
  elist.n = elist.n + 1;
  elist[elist.n] = \{  delta[1],  delta[2],  delta[3], n2\}

  elist = relpos_get_elist(n2);
  elist.n = elist.n + 1;
  elist[elist.n] = \{ -delta[1], -delta[2], -delta[3], n1\}

  return elist;
end

\nwalsodefined{\\{NW2I9Iuo-nRuDO-2}\\{NW2I9Iuo-nRuDO-3}\\{NW2I9Iuo-nRuDO-4}\\{NW2I9Iuo-nRuDO-5}\\{NW2I9Iuo-nRuDO-6}\\{NW2I9Iuo-nRuDO-7}}\nwused{\\{NW2I9Iuo-2auqn1-1}}\nwendcode{}\nwbegindocs{4}\nwdocspar

For the purposes of easing the translation of SUGAR 2.0 netlists
(where basically all the node positions are determined implicitly),
we drop the first node with relative position information at
the origin when {\Tt{}relpos{\_}position{\_}first{\_}node\nwendquote} is true.

\nwenddocs{}\nwbegincode{5}\sublabel{NW2I9Iuo-3Sy502-1}\nwmargintag{{\nwtagstyle{}\subpageref{NW2I9Iuo-3Sy502-1}}}\moddef{optionally place first node at origin~{\nwtagstyle{}\subpageref{NW2I9Iuo-3Sy502-1}}}\endmoddef\nwstartdeflinemarkup\nwenddeflinemarkup
if relpos_position_first_node then
  n1[1] = 0;
  n1[2] = 0;
  n1[3] = 0;
  n1.color = 1
  node_position(n1)
  relpos_position_first_node = nil;
end
\nwnotused{optionally place first node at origin}\nwendcode{}\nwbegindocs{6}\nwdocspar

We use the {\Tt{}relpos{\_}get{\_}elist\nwendquote} function to get an existing
adjacency list for a particular node, or to create a new list
if one does not already exist.

\nwenddocs{}\nwbegincode{7}\sublabel{NW2I9Iuo-nRuDO-2}\nwmargintag{{\nwtagstyle{}\subpageref{NW2I9Iuo-nRuDO-2}}}\moddef{functions~{\nwtagstyle{}\subpageref{NW2I9Iuo-nRuDO-1}}}\plusendmoddef\nwstartdeflinemarkup\nwusesondefline{\\{NW2I9Iuo-2auqn1-1}}\nwprevnextdefs{NW2I9Iuo-nRuDO-1}{NW2I9Iuo-nRuDO-3}\nwenddeflinemarkup
function relpos_get_elist(n)
  local elist = relpos_table[n];
  if not elist then
    elist = \{n = 0\};
    relpos_table[n] = elist;
  end
  return elist;
end

\nwused{\\{NW2I9Iuo-2auqn1-1}}\nwendcode{}\nwbegindocs{8}\nwdocspar

To figure out the node positions, we do a depth-first search
through the relative position graph.  When we finish the search,
we do a quick check to make sure that there are no nodes that
still need to be positioned.

\nwenddocs{}\nwbegincode{9}\sublabel{NW2I9Iuo-nRuDO-3}\nwmargintag{{\nwtagstyle{}\subpageref{NW2I9Iuo-nRuDO-3}}}\moddef{functions~{\nwtagstyle{}\subpageref{NW2I9Iuo-nRuDO-1}}}\plusendmoddef\nwstartdeflinemarkup\nwusesondefline{\\{NW2I9Iuo-2auqn1-1}}\nwprevnextdefs{NW2I9Iuo-nRuDO-2}{NW2I9Iuo-nRuDO-4}\nwenddeflinemarkup
function relpos_treewalk()
  local stack = \{n = 0\};

  \LA{}push all grey nodes onto the stack~{\nwtagstyle{}\subpageref{NW2I9Iuo-1UA7dE-1}}\RA{}

  while stack.n > 0 do
    local node = stack[stack.n]
    stack.n = stack.n - 1

    \LA{}visit neighbors~{\nwtagstyle{}\subpageref{NW2I9Iuo-3M5Ti8-1}}\RA{}

    node.color = 2
  end

  relpos_assert_all_done();
end

postload('relpos_treewalk()');

\nwused{\\{NW2I9Iuo-2auqn1-1}}\nwendcode{}\nwbegindocs{10}\nwdocspar

Recall that in DFS (or any other standard tree search), we can
color nodes according to how much they have been processed.
``White'' nodes ({\Tt{}color\ ==\ 0\nwendquote}) have not yet been visited.
``Grey'' nodes ({\Tt{}color\ ==\ 1\nwendquote}) have an absolute position,
but their neighbors have not yet been processed.  ``Black'' nodes
({\Tt{}colore\ ==\ 2\nwendquote}) are completely processed -- they have been
positioned, and so have all their neighbors.

The grey nodes at the start of the sweep are our initial roots.
We know their absolute positions from user specification.
If there are \emph{no} absolutely positioned nodes, we pick
a single root (the first node for which we saw relative position
information), put it at the origin, and work from there.

\nwenddocs{}\nwbegincode{11}\sublabel{NW2I9Iuo-1UA7dE-1}\nwmargintag{{\nwtagstyle{}\subpageref{NW2I9Iuo-1UA7dE-1}}}\moddef{push all grey nodes onto the stack~{\nwtagstyle{}\subpageref{NW2I9Iuo-1UA7dE-1}}}\endmoddef\nwstartdeflinemarkup\nwusesondefline{\\{NW2I9Iuo-nRuDO-3}}\nwenddeflinemarkup
for node, elist in relpos_table do
  if node.color == 1 then
    stack.n = stack.n + 1;
    stack[stack.n] = node;
  end
end

if stack.n == 0 and relpos_first_attached then
  stack.n = 1
  stack[1] = relpos_first_attached
  relpos_first_attached[1] = 0;
  relpos_first_attached[2] = 0;
  relpos_first_attached[3] = 0;
  relpos_first_attached.color = 1;
end

\nwused{\\{NW2I9Iuo-nRuDO-3}}\nwendcode{}\nwbegindocs{12}\nwdocspar

When we visit the neighbors of the current node, we only need
to do anything to the ones that have not already been positioned.

\nwenddocs{}\nwbegincode{13}\sublabel{NW2I9Iuo-3M5Ti8-1}\nwmargintag{{\nwtagstyle{}\subpageref{NW2I9Iuo-3M5Ti8-1}}}\moddef{visit neighbors~{\nwtagstyle{}\subpageref{NW2I9Iuo-3M5Ti8-1}}}\endmoddef\nwstartdeflinemarkup\nwusesondefline{\\{NW2I9Iuo-nRuDO-3}}\nwenddeflinemarkup
local elist = relpos_table[node];
for k = 1, elist.n do
  local e   = elist[k];
  local nbr = e[4]; 
  if nbr.color == 0 then
    \LA{}process uncolored neighbor~{\nwtagstyle{}\subpageref{NW2I9Iuo-1EA4a6-1}}\RA{}
  end
end
\nwused{\\{NW2I9Iuo-nRuDO-3}}\nwendcode{}\nwbegindocs{14}\nwdocspar

When we process a white node, we first assign it an absolute position,
then color it grey and put it on the stack.

\nwenddocs{}\nwbegincode{15}\sublabel{NW2I9Iuo-1EA4a6-1}\nwmargintag{{\nwtagstyle{}\subpageref{NW2I9Iuo-1EA4a6-1}}}\moddef{process uncolored neighbor~{\nwtagstyle{}\subpageref{NW2I9Iuo-1EA4a6-1}}}\endmoddef\nwstartdeflinemarkup\nwusesondefline{\\{NW2I9Iuo-3M5Ti8-1}}\nwenddeflinemarkup
nbr[1] = node[1] + e[1];
nbr[2] = node[2] + e[2];
nbr[3] = node[3] + e[3];
node_position(nbr);

nbr.color = 1;
stack.n = stack.n + 1;
stack[stack.n] = nbr;
\nwused{\\{NW2I9Iuo-3M5Ti8-1}}\nwendcode{}\nwbegindocs{16}\nwdocspar

We add a wrapper around the node command to mark nodes as 
grey if a position is specified, white otherwise.

\nwenddocs{}\nwbegincode{17}\sublabel{NW2I9Iuo-nRuDO-4}\nwmargintag{{\nwtagstyle{}\subpageref{NW2I9Iuo-nRuDO-4}}}\moddef{functions~{\nwtagstyle{}\subpageref{NW2I9Iuo-nRuDO-1}}}\plusendmoddef\nwstartdeflinemarkup\nwusesondefline{\\{NW2I9Iuo-2auqn1-1}}\nwprevnextdefs{NW2I9Iuo-nRuDO-3}{NW2I9Iuo-nRuDO-5}\nwenddeflinemarkup
function node(p)
  if p[1] then
    p.color = 1
  else
    p.color = 0
  end
  return %node(p)
end

\nwused{\\{NW2I9Iuo-2auqn1-1}}\nwendcode{}\nwbegindocs{18}\nwdocspar

The {\Tt{}relpos{\_}assert{\_}all{\_}done\nwendquote} function just makes sure that
none of the nodes that appear in the relative positioning graph
have no absolute position at the end of processing.  If there
is ever a component of the graph disconnected from one of the
initially-positioned roots, the netlist is in error.

\nwenddocs{}\nwbegincode{19}\sublabel{NW2I9Iuo-nRuDO-5}\nwmargintag{{\nwtagstyle{}\subpageref{NW2I9Iuo-nRuDO-5}}}\moddef{functions~{\nwtagstyle{}\subpageref{NW2I9Iuo-nRuDO-1}}}\plusendmoddef\nwstartdeflinemarkup\nwusesondefline{\\{NW2I9Iuo-2auqn1-1}}\nwprevnextdefs{NW2I9Iuo-nRuDO-4}{NW2I9Iuo-nRuDO-6}\nwenddeflinemarkup
function relpos_assert_all_done()
  for node, elist in relpos_table do
    if node.color == 0 then
      error("Unpositioned node " .. relpos_node_name(node))
    end
  end
end

\nwused{\\{NW2I9Iuo-2auqn1-1}}\nwendcode{}\nwbegindocs{20}\nwdocspar

The {\Tt{}relpos{\_}dump\nwendquote} command is useful for debugging.  It simply
prints a text representation of the relative positioning graph.

\nwenddocs{}\nwbegincode{21}\sublabel{NW2I9Iuo-nRuDO-6}\nwmargintag{{\nwtagstyle{}\subpageref{NW2I9Iuo-nRuDO-6}}}\moddef{functions~{\nwtagstyle{}\subpageref{NW2I9Iuo-nRuDO-1}}}\plusendmoddef\nwstartdeflinemarkup\nwusesondefline{\\{NW2I9Iuo-2auqn1-1}}\nwprevnextdefs{NW2I9Iuo-nRuDO-5}{NW2I9Iuo-nRuDO-7}\nwenddeflinemarkup
function relpos_dump()
  for node, elist in relpos_table do

    \LA{}print node name and position~{\nwtagstyle{}\subpageref{NW2I9Iuo-3OVMhn-1}}\RA{}
    \LA{}print adjacency list~{\nwtagstyle{}\subpageref{NW2I9Iuo-1lqfsC-1}}\RA{}

  end
end

\nwused{\\{NW2I9Iuo-2auqn1-1}}\nwendcode{}\nwbegindocs{22}\nwdocspar

If a node in the graph has an associated position, we might as 
well print that out, too.  But if the node is still colored white,
we just print the name of the node and skip the position.

\nwenddocs{}\nwbegincode{23}\sublabel{NW2I9Iuo-3OVMhn-1}\nwmargintag{{\nwtagstyle{}\subpageref{NW2I9Iuo-3OVMhn-1}}}\moddef{print node name and position~{\nwtagstyle{}\subpageref{NW2I9Iuo-3OVMhn-1}}}\endmoddef\nwstartdeflinemarkup\nwusesondefline{\\{NW2I9Iuo-nRuDO-6}}\nwenddeflinemarkup
if node.color == 0 then
  print("From " .. relpos_node_name(node) .. ":")
else
  print("From " .. relpos_node_name(node) .. " at ("
        .. node[1] .. ", "
        .. node[2] .. ", "
        .. node[3] .. "):")
end
\nwused{\\{NW2I9Iuo-nRuDO-6}}\nwendcode{}\nwbegindocs{24}\nwdocspar

When we print the adjacent nodes, we only print the node
name and the relative position information (and not the
associated position, if it has been determined).

\nwenddocs{}\nwbegincode{25}\sublabel{NW2I9Iuo-1lqfsC-1}\nwmargintag{{\nwtagstyle{}\subpageref{NW2I9Iuo-1lqfsC-1}}}\moddef{print adjacency list~{\nwtagstyle{}\subpageref{NW2I9Iuo-1lqfsC-1}}}\endmoddef\nwstartdeflinemarkup\nwusesondefline{\\{NW2I9Iuo-nRuDO-6}}\nwenddeflinemarkup
for k = 1, elist.n do
  local e = elist[k];
  print("  to " .. relpos_node_name(e[4]) .. ": (" ..
        e[1] .. ", " ..
        e[2] .. ", " .. 
        e[3] .. ")")
end
\nwused{\\{NW2I9Iuo-nRuDO-6}}\nwendcode{}\nwbegindocs{26}\nwdocspar

Some nodes have names; others may not.  We want to output the name
when it is available, and otherwise just output the node number.

\nwenddocs{}\nwbegincode{27}\sublabel{NW2I9Iuo-nRuDO-7}\nwmargintag{{\nwtagstyle{}\subpageref{NW2I9Iuo-nRuDO-7}}}\moddef{functions~{\nwtagstyle{}\subpageref{NW2I9Iuo-nRuDO-1}}}\plusendmoddef\nwstartdeflinemarkup\nwusesondefline{\\{NW2I9Iuo-2auqn1-1}}\nwprevnextdefs{NW2I9Iuo-nRuDO-6}{\relax}\nwenddeflinemarkup
function relpos_node_name(n)
  if n.name then
    return n.name .. " (" .. n["mesh index"] .. ")"
  else
    return "(" .. n["mesh index"] .. ")"
  end
end

\nwused{\\{NW2I9Iuo-2auqn1-1}}\nwendcode{}

\nwixlogsorted{c}{{functions}{NW2I9Iuo-nRuDO-1}{\nwixu{NW2I9Iuo-2auqn1-1}\nwixd{NW2I9Iuo-nRuDO-1}\nwixd{NW2I9Iuo-nRuDO-2}\nwixd{NW2I9Iuo-nRuDO-3}\nwixd{NW2I9Iuo-nRuDO-4}\nwixd{NW2I9Iuo-nRuDO-5}\nwixd{NW2I9Iuo-nRuDO-6}\nwixd{NW2I9Iuo-nRuDO-7}}}%
\nwixlogsorted{c}{{optionally place first node at origin}{NW2I9Iuo-3Sy502-1}{\nwixd{NW2I9Iuo-3Sy502-1}}}%
\nwixlogsorted{c}{{print adjacency list}{NW2I9Iuo-1lqfsC-1}{\nwixu{NW2I9Iuo-nRuDO-6}\nwixd{NW2I9Iuo-1lqfsC-1}}}%
\nwixlogsorted{c}{{print node name and position}{NW2I9Iuo-3OVMhn-1}{\nwixu{NW2I9Iuo-nRuDO-6}\nwixd{NW2I9Iuo-3OVMhn-1}}}%
\nwixlogsorted{c}{{process uncolored neighbor}{NW2I9Iuo-1EA4a6-1}{\nwixu{NW2I9Iuo-3M5Ti8-1}\nwixd{NW2I9Iuo-1EA4a6-1}}}%
\nwixlogsorted{c}{{push all grey nodes onto the stack}{NW2I9Iuo-1UA7dE-1}{\nwixu{NW2I9Iuo-nRuDO-3}\nwixd{NW2I9Iuo-1UA7dE-1}}}%
\nwixlogsorted{c}{{relpos.lua}{NW2I9Iuo-2auqn1-1}{\nwixd{NW2I9Iuo-2auqn1-1}}}%
\nwixlogsorted{c}{{visit neighbors}{NW2I9Iuo-3M5Ti8-1}{\nwixu{NW2I9Iuo-nRuDO-3}\nwixd{NW2I9Iuo-3M5Ti8-1}}}%
\nwbegindocs{28}\nwdocspar
\nwenddocs{}
